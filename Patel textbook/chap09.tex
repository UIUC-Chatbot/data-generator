\documentclass{patt}
\usepackage[section]{placeins}
\usepackage{xr}
\usepackage{alltt}
\usepackage{longtable}
\graphicspath{{../art/ch09/},{../art/designelements/},{../art/designelements/UnNumberedArt/}}

\setcounter{part}{1}
\setcounter{chapter}{8}
\setcounter{page}{219}

 \makeatletter
 \def\@makechapterhead#1{%
  \begingroup
  \parindent \z@%
  \vspace*{-8.5\p@}%
  \begin{picture}(0,0)
    \put(432,-577){\includegraphics[width=0.723333in,height=9.52667in]{PattChp.eps}}
    \put(428,-577){\rule{1\p@}{9.52667in}}
    \put(381.5,-181){\vbox{%
        \includegraphics{PattIcon1\ifnum\value{part}=0
            a\else b\fi.eps}\vspace{1pc}\par
        \includegraphics{PattIcon2\ifnum\value{part}=5
            a\else b\fi.eps}\vspace{1pc}\par
        \includegraphics{PattIcon3\ifnum\value{part}=5
            a\else b\fi.eps}\vspace{1pc}\par
        \includegraphics{PattIcon4\ifnum\value{part}=5
            a\else b\fi.eps}\vspace{1pc}\par
        \includegraphics{PattIcon5\ifnum\value{part}=5
            a\else b\fi.eps}\vspace{1pc}}}
  \end{picture}%
  \settowidth{\chapternumberwidth}{\fontsize{12}{12}\selectfont\industriasolid\trackonefifty{chapter}}%
  \hfill\parbox{\chapternumberwidth}{%
    \centering\industriasolid
    \centerline{\hss\fontsize{12}{12}\selectfont\trackonefifty{chapter}\hss}\par\vspace{1.5pc}
    \centerline{\sans\fontsize{72}{12}\selectfont\colour\thechapter}}
  \par
  \vspace{6.48pc}
  {\industriasolid\fontsize{30}{32}\selectfont\trackten{#1}\par}%
  \vspace{1.5pc}%
  {\colour\rule{36.4pc}{4\p@}}
  \vspace{4pt}
  \endgroup}
\makeatother

\begin{document}

\chapter{I/O}
\label{chapt:os}

Up to now we have completely ignored the details of input and output,
that is, how the computer actually gets information from the keyboard (input),
and how the computer actually delivers information to the monitor (output).  
Instead we have relied on the TRAP instruction (for example, TRAP x23 for input 
and TRAP x21 for output) to accomplish these tasks.  The TRAP instruction
enables us to tell the operating system what we need done by means of a 
trap vector, and we trust the operating system to do it for us.

The more generic term for our TRAP instruction is {\em system call} because 
the TRAP instruction is calling on the operating system to do something for 
us while allowing us to remain completely clueless as to how it gets done.
Now we are ready to examine how input and output actually work in the LC-3, 
what happens when the user program makes a system call by invoking the TRAP 
instruction, and how it all works under the control of the operating system.  

We will start with the actual physical structures that are required to 
cause input and output to occur.  But before we do that, it is useful to
say a few words about the operating system and understand a few basic 
concepts that have not been important so far, but become very important 
when considering what the operating system needs to do its job.

You may be familiar with Microsoft's various flavors of Windows, 
Apple's MacOS, and Linux.  These are all examples of operating systems.  
They all have the same goal: to optimize the use of all the resources of 
the computer system while making sure that no software does harmful things 
to any program or data that it has no right to mess with.  To better
understand their job, we need to understand the notions of privilege and 
priority and the layout of the memory address space (i.e., the regions of 
memory and the purpose of each).  

\section{Privilege, Priority, and the Memory Address Space}
\subsection{Privilege and Priority}

Two very different (we often say orthogonal) concepts associated with computer
processing are {\em privilege} and {\em priority}.  

\subsubsection{Privilege}
Privilege is all about the
right to do something, such as execute a particular instruction or access a
particular memory location.  Not all computer programs have the right to
execute all instructions.  For example, if a computer system is shared among
many users and the ISA contains a HALT instruction, we would not want any
random program to execute that HALT instruction and stop the computer.  If we
did, we  would have some pretty disgruntled users on our hands.  Similarly, 
some memory locations are only available to the operating system.  We would 
not want some random program to interfere with the data structures or code
that are part of the operating system, which would in all likelihood cause
the entire system to crash.  In order to make sure neither of these two 
things happen, we designate every computer program as either privileged or
unprivileged.  We often say {\em Supervisor privilege} to indicate privileged.  
Programs that have Supervisor privilege can execute all instructions and
can access all of memory.  Unprivileged programs can not.  We say a
program is executing in {\em Supervisor mode} to indicate privileged, or 
{\em User mode} to indicate unprivileged.

\subsubsection{Priority}
Priority is all about the urgency of a program to execute.  Every program is
assigned a priority, specifying its urgency as compared to all other programs.
This allows programs of greater urgency to interrupt programs of lesser
urgency.  For example, programs written by random users may be assigned a
priority of 0.  The keyboard may be asigned a priority of 4, and the fact
that the computer is plugged into a source of energy like a wall outlet may
be assigned a priority of 6.  If that is the case, a random user program
would be interrupted if someone sitting at a keybord wanted to execute a
program that caused data to be input into the computer.  And that program
would be interrupted if someone pulled the power cord out of the wall outlet
causing the computer to quickly lose its source of energy.  In such an event,
we would want the computer to execute some operating system program that is
provided specifically to handle that situation.

\subsubsection{Two orthogonal notions}
We said privilege and priority are two orthogoanal notions, meaning they have
nothing to do with each other.  We humans sometimes have a problem with that as
we think of fire trucks that have the privilege of ignoring traffic lights
because they have the urgency to reach the fire.  In our daily lives we often
are given privileges because of our greater sense of urgency.  Not the case
with computer systems.  

For example, we can have a user program that is tied
to a physics experiment that needs to interrupt the computer at a specific
instance of time to record information being generated by the physics 
experiment.  If the user program does not pre-empt the program running at that
instant of time, the data generated by the experiment may be lost.  This is
a user program so it does not have Supervisor privilege.  But it does have a 
greater urgency, so it does have a higher priority.

Another example, the system administrator who wants to execute diagnostic
programs that access all memory locations and execute all instructions as part 
of some standard preventive maintenance.  The diagnostic program needs
Supervisor privilege to execute all instructions and access all memory 
locations.  But it has no sense of urgency.  Whether this happens at 1am or 
2am is irrelevant, compared to the urgency of other programs that need access 
to the computer system exactly when they need it.  The diagnostic program has 
privilege but no priority.   
   
Finally, an example showing that even in human activity one can have priority
but not privilege.  Our friend Bob works in the basement of one of those 
New York City skyscapers.  He is about to go to the Men's Room when his 
manager tells him to take a message immediately to the Vice President on the 
88th floor, and bring back a response.  So Bob delays his visit to the 
Men's Room and takes the elevator to the 88th floor.  The Vice President 
keeps him waiting, causing Bob to be concerned he might have an accident.  
Finally, the Vice President gives his response, and Bob pushes the button 
to summon the elevator to take him back to the basement, in pain because he 
needs to go to the Men's Room.  While waiting for
the elevator, another Vice President appears, unlocks the Executive Men's Room
and enters.  Bob is in pain, but he can not enter the Executive Men's
Room.  Although he certainly has the priority, he does not have the privilege! 

\subsubsection{the Processor Status Register (PSR)}

Each program executing on the computer is assigned a privilege and a
priority.  Figure~\ref{fig:psr} shows the Processor Status Register (PSR),
which contains that assignment.  Each program has associated with it two very
importnat registers.  The Program Counter (PC) you are very familiar with. 
The PSR is shown below:

\begin{figure}
\begin{center}
\hbox{\begin{tabular}{@{\extracolsep{-2pt}}cccccccccccccccc}
15 & 14 & 13 & 12 & 11 & 10 & 9 & 8 & 7 & 6 & 5 & 4 & 3 & 2 & 1 & 0\\
\hline
\multicolumn{1}{|c|}{\rule{0pt}{10pt}Pr\rule{0pt}{10pt}}&
\multicolumn{3}{c}{} & \multicolumn{1}{c|}{}
& \multicolumn{1}{c}{}
& \multicolumn{1}{c}{PL n} & \multicolumn{1}{c|}{} &
\multicolumn{3}{c}{} & \multicolumn{2}{c}{} & \multicolumn{1}{|c}{N} &
 \multicolumn{1}{|c|}{Z} & \multicolumn{1}{c|}{P}\\
\hline
\multicolumn{2}{l}{Priv} & \multicolumn{4}{c}{ } &
\multicolumn{2}{c}{Priority} & \multicolumn{5}{c}{ } &
\multicolumn{3}{c}{cond codes}\\
\end{tabular} \quad\bfseries PSR}
\end{center}
\caption{Processsor Status Register (PSR)}
\label{fig:psr}
\end{figure}

Bit[15] specifies the privilege, where PSR[15]=0 means Supervisor privilege, 
and PSR[15]=1 means unprivileged.  Bits[10:8] specifies the priority level (PL)
of the program.  The highest priority level is 7 (PL7), the lowest is PL0.

The PSR also contains the current values of the condition codes, as shown
in Figure~\ref{fig:psr}.  We will see in Section 9.4 why it is important
that the condition codes are included in the PSR.

\subsection{Organization of Memory}

Figure~\ref{fig:memory_layout} shows the layout of the LC-3 memory.

\begin{figure}[h!]
\centerline{\includegraphics{memory_layout.eps}}
\caption{Regions of Memory}
\label{fig:memory_layout}
\end{figure}

You know that the LC-3 has a 16 bit address space; ergo, memory locations
from x0000 to xFFFF.  Locations x0000 to x2FFF are privileged memory
locations.  They contain the various data structures and code of the operating
system.  They require Supervisor privilege to access.  They are referred to as
{\em System Space}.

Locations x3000 to xFDFF are unprivileged memory locations.  Supervisor 
privilege is not required to access these memory locations.  All user programs 
and data use this region of memory.  The region is often referred to as 
{\em User Space}.

Addresses xFE00 to xFFFF do not correspond to memory locations at all.  
That is, the last address of a memory location is xFDFF. Addresses xFE00
to xFFFF are used to identify registers that take part in input and output
functions, and some special registers associated with the processor.  
For example, the PSR is assigned address xFFFC, and the processor's Master 
Control Register (MCR) is assigned address xFFFE.  The benefit of assigning
addresses from the memory address space will be discussed in Section 9.2.1.2.
The set of addresses from xFE00 to xFFFF is usually referred to as the I/O Page
since most of the addresses are used for identifying registers that take part 
in input or output functions.  Access to those registers requires 
Supervisor privilege.

Finally, note that Figure~\ref{fig:memory_layout} shows two stacks, a 
{\em Supervisor Stack} in system space and a {\em User Stack} in user space.  
The Supervisor Stack is controlled by the operating system, and requires 
Supervisor privilege to access.  The User Stack is controlled by the 
user program and does not require privilege to access.  

Each has a stack pointer, Supervisor Stack Pointer (SSP) and User Stack Pointer
(USP), to indicate the top of the stack.  Since a program can only execute
in Supervisor mode or User mode at any one time, only one of the two stacks
is active at any one time.  Register 6 is generally used as the 
stack pointer (SP) for the active stack.  Two registers, Saved\_SSP and 
Saved\_USP are provided to save the SP not in use.  When privilege changes,
for example, from Supervisor mode to User mode, the SP is stored in 
Saved\_SSP, and the SP is loaded from Saved\_USP.

\section{Input/Output}

Input and output devices (keyboards, monitors, disks, or kiosks at the
shopping mall) all handle input or output data using registers that are 
tailored to the needs of each particular input or output device.  
Even the simplest I/O devices usually need at least two registers: one to 
hold the data being transferred between the device and the computer, and one 
to indicate status information about the device.  An example of status 
information is whether the device is available or is it still busy processing 
the most recent I/O task.
\index{device register} 

\subsection{Some basic characteristics of I/O}

All I/O activity is controlled by instructions in the computer's ISA.  Does
the ISA need special instructions for dealing with I/O?  Does the I/O device
execute at the same speed as the computer, and if not what manages the
difference in speeds?  Is the transfer of information between the
computer and the I/O device initiated by a program executing in the computer
or is it initiated by the I/O device?  Answers to these questions form some
of the basic characteristics of I/O activity.

\subsubsection{Memory-Mapped I/O vs Special I/O Instructions}

An instruction that interacts with an input or output device register
must identify the particular input or output device register with
which it is interacting. Two schemes have been used in the past.  Some
computers use special input and output instructions. Most computers
prefer to use the same data movement instructions that are used to
move data in and out of memory.

The very old PDP-8 (from Digital Equipment Corporation, more than 60 years
ago---1965) is an example of a computer that used special input and
output instructions. The 12-bit PDP-8 instruction contained a 3-bit
opcode. If the opcode was 110, an I/O instruction was indicated. The
remaining nine bits of the PDP-8 instruction identified which I/O
device register and what operation was to be performed.

Most computer designers prefer not to specify an additional set of
instructions for dealing with input and output. They use the same data
movement instructions that are used for loading and storing data
between memory and the general purpose registers.  For example, a load
instruction (LD, LDI, or LDR), in which the source address is that of an 
input device register, is an input instruction.  Similarly, a store 
instruction (ST, STI, or STR) in which the destination address is that of 
an output device register is an output instruction.

Since programmers use the same data movement instructions that are
used for memory, every input device register and every output device
register must be uniquely identified in the same way that memory
locations are uniquely identified.  Therefore, each device register is
assigned an address from the memory address space of the ISA. That is,
the I/O device registers are {\em mapped} to a set of addresses that
are allocated to I/O device registers rather than to memory locations.
Hence the name {\em memory-mapped I/O}.  \index{memory-mapped I/O}

The original PDP-11 ISA had a 16-bit address space.  All addresses
wherein bits [15:13] $=$ 111 were allocated to I/O device registers.
That is, of the $2^{16}$ addresses, only 57,344 corresponded to memory
locations.  The remaining $2^{13}$ were memory-mapped I/O addresses.

The LC-3 uses memory-mapped I/O.  As we discussed in Section~9.1.2, 
addresses x0000 to xFDFF refer to actual memory locations.  Addresses 
xFE00 to xFFFF are reserved for input/output device registers.  Table~A.3 
lists the memory-mapped addresses of the LC-3 device registers that have 
been assigned so far.  Future uses and future sales of LC-3 microprocessors 
may require the expansion of device register address assignments as new and 
exciting applications emerge!

\subsubsection{Asynchronous vs Synchronous}

Most I/O is carried out at speeds very much slower than the speed of
the processor.  A typist, typing on a keyboard, loads an input device
register with one ASCII code every time he/she types a character.  A
computer can read the contents of that device register every time it
executes a load instruction, where the operand address is the
memory-mapped address of that input device register.

Many of today's microprocessors execute instructions under the control
of a clock that operates well in excess of 2 GHz.  Even for a
microprocessor operating at only 2 GHz, a clock cycle lasts only 0.5
nanoseconds.  Suppose a processor executed one instruction at a time,
and it took the processor 10 clock cycles to execute the instruction
that reads the input device register and stores its contents.  At that
rate, the processor could read the contents of the input device
register once every 5 nanoseconds.  Unfortunately, people do not type
fast enough to keep this processor busy full-time reading characters.
{\em Question:} How fast would a person have to type to supply input
characters to the processor at the maximum rate the processor can
receive them? 

We could mitigate this speed disparity by designing hardware that
would accept typed characters at some slower fixed rate.  For example,
we could design a piece of hardware that accepts one character every
200 million cycles. This would require a typing speed of 100
words/minute, assuming words on average consisted of 5 letters, which is 
certainly doable.  Unfortunately, it would also require that the typist 
work in lockstep with the computer's clock.  That is not acceptable 
since the typing speed (even of the same typist) varies from moment to moment.

What's the point?  The point is that I/O devices usually operate at
speeds very different from that of a microprocessor, and not in
lockstep.  We call this latter characteristic  {\em asynchronous}.
Most interaction between a processor \index{asynchronous I/O} and I/O
is asynchronous. To control processing in an asynchronous world
requires some protocol or {\em handshaking} \index{protocol}
\index{handshaking} mechanism.  So it is with our keyboard and monitor.
In the case of the keyboard, we will need a 1-bit status
register, called a {\em flag,} to indicate if someone has or has not
typed a character.  In the case of the monitor, we will need a 1-bit
status register to indicate whether or not the most recent character
sent to the monitor has been displayed, and so the monitor can be given
another character to display.

These flags are the simplest form of {\em synchronization}.  A single
\index{synchronization} flag, called the {\em Ready bit}, is enough to
synchronize the output of the typist who can type characters at the
rate of 100 words/minute with the input to a processor that can accept
these characters at the rate of 200 million characters/second.  Each
time the typist types a character, the Ready bit is set to 1. Each time the
computer reads a character, it clears the Ready bit.  By examining the
Ready bit before reading a character, the computer can tell whether it
has already read the last character typed.  If the Ready bit is clear,
no characters have been typed since the last time the computer read a
character, and so no additional read would take place. When the
computer detects that the Ready bit is set, it could only have been
caused by a {\bf new} character being typed, so the computer would
know to again read a character.

The single Ready bit provides enough handshaking to ensure that the
asynchronous transfer of information between the typist and the
microprocessor can be carried out accurately.

If the typist could type at a constant speed, and we did have a piece
of hardware that would accept typed characters at precise intervals
(for example, one character every 200 million cycles), then we would
not need the Ready bit.  The computer would simply know, after 200
million cycles of doing other stuff, that the typist had typed exactly
one more character, and the computer would read that character.  In
this hypothetical situation, the typist would be typing in lockstep
with the processor, and no additional synchronization would be needed.
We would say the computer and typist were operating {\em
  synchronously}.  That is, the input activity was synchronous.
\index{synchronous I/O}

\subsubsection{Interrupt-driven vs Polling}
\label{subsubsec:interruptVSpolling}

The processor, which is computing, and the typist, who is typing, are
two separate entities.  Each is doing its own thing.  Still, they need
to interact, that is, the data that is typed has to get into the
computer. The issue of {\em interrupt-driven} versus {\em polling} is
the issue of who controls the interaction.  Does the processor do its
own thing until being interrupted by an announcement from the
keyboard, ``Hey, a key has been struck. The ASCII code is in the input
device register.  You need to read it.'' This is called {\em
  interrupt-driven I/O}, where the keyboard controls the interaction.
Or, does the processor control the interaction, specifically by
interrogating (usually, again and again) the Ready bit until it (the
processor) detects that the Ready bit is set.  At that point, the
processor knows it is time to read the device register.  This second
type of interaction when the processor is in charge is called {\em polling}, 
since the Ready bit is polled by the processor, asking if any key has been 
struck.

Section~\ref{subsubsec:basic.input.routine} describes how polling
works. Section~\ref{sec:interruptsIO} explains interrupt-driven I/O.

\subsection{Input from the Keyboard}
\label{subsec:input.from.keyboard}

\subsubsection{Basic Input Registers (KBDR and KBSR)}

We have already noted that in order to handle character input from the
keyboard, we need two things: a data register that contains the
character to be input, and a synchronization mechanism to let the
processor know that input has occurred.  The synchronization mechanism
is contained in the status register associated with the keyboard.

These two registers are called the {\em keyboard data register} (KBDR)
and the \index{KBDR} \index{Keyboard Data Register} {\em keyboard
  status register} (KBSR).  They are assigned addresses from the
memory \index{KBSR} \index{Keyboard Status Register} address space.
As shown in Table~A.3, address xFE02 is assigned to the KBDR; address xFE00
is assigned to the KBSR.

\begin{figure}[h!]
\centerline{\includegraphics{pat67509_0801.eps}}
\caption{Keyboard device registers}
\label{fig:kbdr}
\end{figure}

Even though a character needs only eight bits and the synchronization
mechanism needs only one bit, it is easier to assign 16 bits (like all
memory addresses in the LC-3) to each.  In the case of KBDR, bits
[7:0] are used for the data, and bits [15:8] contain x00. In the case
of KBSR, bit [15] contains the synchronization mechanism, that is, the
Ready bit.  Figure~\ref{fig:kbdr} shows the two device registers
needed by the keyboard.

\subsubsection{The Basic Input Service Routine}
\label{subsubsec:basic.input.routine}
\index{basic input service routine}

KBSR[15] controls the synchronization of the slow keyboard and the
fast processor.  When a key on the keyboard is struck, the ASCII code
for that key is loaded into KBDR[7:0] and the electronic circuits
associated with the keyboard automatically set KBSR[15] to 1.  When
the LC-3 reads KBDR, the electronic circuits associated with the
keyboard automatically clear KBSR[15], allowing another key to be
struck.  If KBSR[15] $=$ 1, the ASCII code corresponding to the last
key struck has not yet been read, and so the keyboard is disabled; that is,
no key can be struck until the last key is read.

If input/output is controlled by the processor (i.e., via polling),
then a program can repeatedly test KBSR[15] until it notes that the
bit is set. At that point, the processor can load the ASCII code
contained in KBDR into one of the LC-3 registers.  Since the processor
only loads the ASCII code if KBSR[15] is 1, there is no danger of
reading a single typed character multiple times. Furthermore, since
the keyboard is disabled until the previous code is read, there is no
danger of the processor missing characters that were typed. In this
way, KBSR[15] provides the mechanism to guarantee that each key typed
will be loaded exactly once.

The following input routine loads R0 with the ASCII code that has been
entered through the keyboard and then moves on to the NEXT\_TASK in
the program.

\begin{colorverbatim}
01    START  LDI     R1, A        ; Test for
02           BRzp    START        ; character input
03           LDI     R0, B
04           BRnzp   NEXT_TASK    ; Go to the next task
05    A      .FILL   xFE00        ; Address of KBSR
06    B      .FILL   xFE02        ; Address of KBDR
\end{colorverbatim}

\noindent
As long as KBSR[15] is 0, no key has been struck since the last time
the processor read the data register.  Lines 01 and 02 comprise a loop
that tests bit [15] of KBSR.  Note the use of the LDI instruction,
which loads R1 with the contents of xFE00, the memory-mapped address
of KBSR. If the Ready bit, bit [15], is clear, BRzp will branch to
START and another iteration of the loop.  When someone strikes a key,
KBDR will be loaded with the ASCII code of that key and the Ready bit
of KBSR will be set.  This will cause the branch to fall through and
the instruction at line 03 will be executed.  Again, note the use of the
LDI instruction, which this time loads R0 with the contents of xFE02,
the memory-mapped address of KBDR.  The input routine is now done, so
the program branches unconditionally to its NEXT\_TASK.


\subsubsection{Implementation of Memory-Mapped Input}

Figure~\ref{fig:memmap.input} shows the additional data path required
to implement memory-mapped input.  You are already familiar, from
Chapter~5, with the data path required to carry out the EXECUTE phase
of the load instructions.  Essentially three steps are required:

\begin{figure}[h!]
\centerline{\includegraphics{pat67509_0802.eps}}
\caption{Memory-mapped input}
\label{fig:memmap.input}
\end{figure}

\begin{enumerate}
\item The MAR is loaded with the address of the memory location to be
  read.
\item Memory is read, resulting in MDR being loaded with the contents
  at the specified memory location.
\item The destination register (DR) is loaded with the contents of
  MDR.
\end{enumerate}
In the case of memory-mapped input, the same set of steps are carried
out, {\bf except} instead of MAR being loaded with the address of a
memory location, MAR is loaded with the address of a device register.
Instead of the {address control logic} enabling memory to read, the
address control logic selects the corresponding device register to
provide input to the MDR.

\FloatBarrier
\subsection{Output to the Monitor}
\label{subsec:output.to.monitor}

\subsubsection{Basic Output Registers (DDR and DSR)}

 Output works in a way very similar to input, with DDR and
DSR replacing the roles of KBDR and KBSR, respectively.  DDR stands
for Display Data Register, which drives the monitor display.  DSR
stands for Display Status Register.  In the LC-3, DDR is assigned
address xFE06.  DSR is assigned address xFE04.

\begin{figure}[h!]
\centerline{\includegraphics{pat67509_0803.eps}}
\caption{Monitor device registers}
\label{fig:crtdr}
\end{figure}

As is the case with input, even though an output character needs only
eight bits and the synchronization mechanism needs only one bit, it is
easier to assign 16 bits (like all memory addresses in the LC-3) to
each output device register.  In the case of DDR, bits [7:0] are used
for data, and bits [15:8] contain x00. In the case of DSR, bit [15]
contains the synchronization mechanism, that is, the Ready bit.
Figure~\ref{fig:crtdr} shows the two device registers needed by the
monitor.

\FloatBarrier
\subsubsection{The Basic Output Service Routine}
\label{subsubsec:basic.output.routine}
\index{basic output service routine}

DSR[15] controls the synchronization of the fast processor and the
slow monitor display.  When the LC-3 transfers an ASCII code to
DDR[7:0] for outputting, the electronics of the monitor automatically
clear DSR[15] as the processing of the contents of DDR[7:0] begins.
\index{DSR} \index{Display Status Register} \index{DDR} \index{Display
  Data Register} When the monitor finishes processing the character on
the screen, it (the monitor) automatically sets DSR[15]. This is a
signal to the processor that it (the processor) can transfer another
ASCII code to DDR for outputting. As long as DSR[15] is clear, the
monitor is still processing the previous character, so the monitor is
disabled as far as additional output from the processor is concerned.

If input/output is controlled by the processor (i.e., via polling),
a program can repeatedly test DSR[15] until it notes that the bit
is set, indicating that it is OK to write a character to the screen.
At that point, the processor can store the ASCII code for the
character it wishes to write into DDR[7:0], setting up the transfer of
that character to the monitor's display.

The following routine causes the ASCII code contained in R0 to be
displayed on the monitor:

\begin{colorverbatim}
01     START   LDI     R1, A         ; Test to see if
02             BRzp    START         ; output register is ready
03             STI     R0, B
04             BRnzp   NEXT_TASK
05     A       .FILL   xFE04         ; Address of DSR
06     B       .FILL   xFE06         ; Address of DDR
\end{colorverbatim}

\noindent Like the routine for KBDR and KBSR in
Section~\ref{subsubsec:basic.input.routine}, lines 01 and 02
repeatedly poll DSR[15] to see if the monitor electronics is finished
with the last character shipped by the processor.  Note the use of
LDI and the indirect access to xFE04, the memory-mapped address of
DSR. As long as DSR[15] is clear, the monitor electronics is still
processing this character, and BRzp branches to START for another
iteration of the loop.  When the monitor electronics finishes with the
last character shipped by the processor, it automatically sets DSR[15]
to 1, which causes the branch to fall through and the instruction at
line 03 to be executed.  Note the use of the STI instruction, which
stores R0 into xFE06, the \hbox{memory-mapped} address of DDR.  The
write to DDR also clears DSR[15], disabling for the moment DDR from
further output.  The monitor electronics takes over and writes the
character to the screen.  Since the output routine is now done, the
program unconditionally branches (line 04) to its NEXT\_TASK.

\subsubsection{Implementation of Memory-Mapped Output}

Figure~\ref{fig:memmap.output} shows the additional data path required
to implement memory-mapped output.  As we discussed previously with
respect to memory-mapped input, the mechanisms for handling the device
registers provide very little additional complexity to what already
exists for handling memory accesses.

\begin{figure}[h!]
\centerline{\includegraphics{pat67509_0804.eps}}
\caption{Memory-mapped output}
\label{fig:memmap.output}
\end{figure}


In Chapter~5, you became familiar with the process of carrying out the
EXECUTE phase of the store instructions.

\begin{enumerate}
\item The MAR is loaded with the address of the memory location to be
  written.
\item The MDR is loaded with the data to be written to memory.
\item Memory is written, resulting in the contents of MDR being stored
  in the specified memory location.
\end{enumerate}
In the case of memory-mapped output, the same steps are carried out,
{\bf except} instead of MAR being loaded with the address of a memory
location, MAR is loaded with the address of a device register.
Instead of the {address control logic} enabling memory to write, the
address control logic asserts the load enable signal of DDR.

Memory-mapped output also requires the ability to {\bf read} output
device registers.  You saw in
Section~\ref{subsubsec:basic.output.routine} that before the DDR could
be loaded, the Ready bit had to be in state 1, indicating that the
previous character had already finished being written to the screen.  The LDI
and BRzp instructions on lines 01 and 02 perform that test.  To do
this the LDI reads the output device register DSR, and BRzp tests bit
[15].  If the MAR is loaded with xFE04 (the memory-mapped address of
the DSR), the address control logic selects DSR as the input to the
MDR, where it is subsequently loaded into R1 and the condition codes
are set.

\FloatBarrier
\subsubsection{Example: Keyboard Echo}
\label{subsec:keyboard.echo}

When we type at the keyboard, it is helpful to know exactly what
characters we have typed. We can get this echo capability easily
(without any sophisticated electronics) by simply combining the two
routines we have discussed.  The result: the key typed at the keyboard is
displayed on the monitor.

\begin{colorverbatim}
01     START   LDI     R1, KBSR    ; Test for character input
02             BRzp    START
03             LDI     R0, KBDR
04     ECHO    LDI     R1, DSR     ; Test output register ready
05             BRzp    ECHO
06             STI     R0, DDR
07             BRnzp   NEXT_TASK
08    KBSR    .FILL   xFE00       ; Address of KBSR
09    KBDR    .FILL   xFE02       ; Address of KBDR
0A    DSR     .FILL   xFE04       ; Address of DSR
0B    DDR     .FILL   xFE06       ; Address of DDR
\end{colorverbatim}

\subsection{A More Sophisticated Input Routine}
\label{subsec:sophisticated.routine}

In the example of Section~\ref{subsubsec:basic.input.routine}, the
input routine would be a part of a program being executed by the
computer.  Presumably, the program requires character input from the
keyboard.  But how does the person sitting at the keyboard know when
to type a character?  Sitting there, the person may wonder whether or
not the program is actually running, or if perhaps the computer is
busy doing something else.

To let the person sitting at the keyboard know that the program is
waiting for input from the keyboard, the computer typically prints a
message on the monitor. Such a message is often referred to as a {\em
  prompt}. The symbol that is \index{prompt} displayed by your
operating system (for example, {\bf \%} or {\bf C:}) or by your editor
(for example, {\bf :}) are examples of prompts.

The program fragment shown in Figure~\ref{fig:sophisticated.routine}
obtains keyboard input via polling as we have shown in
Section~\ref{subsubsec:basic.input.routine}. It also includes
a prompt to let the person sitting at the keyboard know when it is
time to type a key.  Let's examine this program fragment.

\begin{figure}[h!]
\begin{Verbatim}[fontsize=\fontsize{9}{11}\selectfont]
01  START   ST      R1,SaveR1   ; Save registers needed
02          ST      R2,SaveR2   ; by this routine
03          ST      R3,SaveR3
04  ;
05          LD      R2,Newline
06  L1      LDI     R3,DSR
07          BRzp    L1          ; Loop until monitor is ready
08          STI     R2,DDR      ; Move cursor to new clean line
09  ;
0A          LEA     R1,Prompt   ; Starting address of prompt string
0B  Loop    LDR     R0,R1,#0    ; Write the input prompt
0C          BRz     Input       ; End of prompt string
0D  L2      LDI     R3,DSR
0E          BRzp    L2          ; Loop until monitor is ready
0F          STI     R0,DDR      ; Write next prompt character
10          ADD     R1,R1,#1    ; Increment prompt pointer
11          BRnzp   Loop        ; Get next prompt character
12  ;
13  Input   LDI     R3,KBSR
14          BRzp    Input       ; Poll until a character is typed
15          LDI     R0,KBDR     ; Load input character into R0
16  L3      LDI     R3,DSR
17          BRzp    L3          ; Loop until monitor is ready
18          STI     R0,DDR      ; Echo input character
19  ;
1A  L4      LDI     R3,DSR
1B          BRzp    L4          ; Loop until monitor is ready
1C          STI     R2,DDR      ; Move cursor to new clean line
1D          LD      R1,SaveR1   ; Restore registers
1E          LD      R2,SaveR2   ; to original values
1F          LD      R3,SaveR3
20          BRnzp   NEXT_TASK   ; Do the program's next task
21  ;
22  SaveR1  .BLKW   1           ; Memory for registers saved
23  SaveR2  .BLKW   1
24  SaveR3  .BLKW   1
25  DSR     .FILL   xFE04
26  DDR     .FILL   xFE06
27  KBSR    .FILL   xFE00
28  KBDR    .FILL   xFE02
29  Newline .FILL   x000A       ; ASCII code for newline
2A  Prompt  .STRINGZ ``Input a character>''
\end{Verbatim}
\caption{The more sophisticated input routine}
\label{fig:sophisticated.routine}
\end{figure}

You are already familiar with lines 13 through 19 and lines 25 through
28, which correspond to the code in Section~\ref{subsec:keyboard.echo}
for inputting a character via the keyboard and echoing it on the
monitor. 

You are also familiar with the need to save and restore registers if those
registers are needed by instructions in the input routine.  Lines 01 through 
03 save R1, R2, and R3, lines 1D through 1F restore R1, R2, and R3, and 
lines 22 through 24 set aside memory locations for those register values.

This leaves lines 05 through 08, 0A through 11, 1A through 1C, 29 and
2A.  These lines serve to alert the person sitting at the keyboard
that it is time to type a character.

Lines 05 through 08 write the ASCII code x0A to the monitor.  This is
the ASCII code for a {\em new line}.  Most ASCII codes correspond to
characters that are visible on the screen.  A few, like x0A, are
control characters.  They cause an action to occur. Specifically, the
ASCII code x0A causes the cursor to move to the far left of the next
line on the screen.  Thus the name {\em Newline}.  Before attempting
to write x0A, however, as is always the case, DSR[15] is tested (line
6) to see if DDR can accept a character.  If DSR[15] is clear, the
monitor is busy, and the loop (lines 06 and 07) is repeated.  When
DSR[15] is 1, the conditional branch (line 7) is not taken, and (line 8)
x0A is written to DDR for outputting.

Lines 0A through 11 cause the prompt
\verb|Input a character>|
to be written to the screen.  The prompt is specified by the
  .STRINGZ pseudo-op on line 2A and is stored in 19 memory
locations---18 ASCII codes, one per memory location, corresponding to
the 18 characters in the prompt, and the terminating sentinel x0000.

Line 0C iteratively tests to see if the end of the string has been
reached (by detecting x0000), and if not, once DDR is free, line 0F
writes the next character in the input prompt into DDR.  When x0000 is
detected, the entire input prompt has been written to the screen and 
the program branches to the code that handles the actual keyboard input 
(starting at line 13).

After the person at the keyboard types a character and it has been
echoed (lines 13 to 19), the program writes one more new line (lines
1A through 1C) before branching to its NEXT\_TASK.

\FloatBarrier
\subsection{Implementation of Memory-Mapped I/O, Revisited}

We showed in Figures~\ref{fig:memmap.input}
and~\ref{fig:memmap.output} partial implementations of the data path
to handle (separately) memory-mapped input and memory-mapped output.
We have also learned that in order to support interrupt-driven I/O,
the two status registers must be writeable as well as readable.

Figure~\ref{fig:memmap.io} (also shown as Figure~C.3 of Appendix~C)
shows the data path necessary to support the full range of features we
have discussed for the I/O device registers.  The Address Control
Logic block controls the input or output operation.  Note that there
are three inputs to this block.  MIO.EN indicates whether a data
movement from/to memory or I/O is to take place this clock cycle.  
MAR contains the address of the memory location or the memory-mapped
address of an I/O device register.  R.W indicates whether a load or a
store is to take place.  Depending on the values of these three
inputs, the Address Control Logic does nothing (MIO.EN $=$ 0), or
provides the control signals to direct the transfer of data between
the MDR and the memory or between the MDR and one of the I/O registers.

\begin{figure}[h!]
\centerline{\includegraphics{pat67509_0809.eps}}
\caption{Relevant data path implementation of memory-mapped I/0}
\label{fig:memmap.io}
\end{figure}

If R.W indicates a load, the transfer is from memory or I/O device to
the MDR.  The Address Control Logic block provides the select lines to
INMUX to source the appropriate I/O device register or memory
(depending on MAR) and also enables the memory if MAR contains the
address of a memory location.

If R.W indicates a store, the contents of the MDR are written either
to memory or to one of the device registers.  The Address Control
Logic either enables a write to memory or it asserts the load enable
line of the device register specified by the contents of the MAR.

\FloatBarrier
\section{Operating System Service Routines (LC-3 Trap Routines)}

\subsection{Introduction}

Recall Figure~\ref{fig:sophisticated.routine} of the previous section.
In order for the program to successfully obtain input from the
keyboard, it was necessary for the programmer to know several things:

\begin{enumerate}
\item The hardware data registers for both the monitor and the keyboard: the
monitor so a prompt could be displayed, and the keyboard so the program
would know where to get the input character.

\item The hardware status registers for both the monitor and the keyboard: the
monitor so the program would know when it was OK to display the next
character in the input prompt, and the keyboard so the program would
know when someone had struck a key.

\item The asynchronous nature of keyboard input relative to the executing program.
\end{enumerate}

This is beyond the knowledge of most application programmers.  In
fact, in the real world, if application programmers (or user
programmers, as they are sometimes called) had to understand I/O at
this level, there would be much less I/O and far fewer programmers in
the business.

There is another problem with allowing user programs to perform I/O
activity by directly accessing KBDR and KBSR.  I/O activity involves
the use of device registers that are shared by many programs.  This
means that if a user programmer were allowed to access the hardware
registers, and he/she messed up, it could create havoc for other user
programs.  Thus, in general it is ill-advised to give user programmers access 
to these registers.  That is why the the addresses of hardware registers are 
part of the privileged memory address space, 
and accessible only to programs that have Supervisor privilege.

The simpler solution as well as the safer solution to the problem of
user programs requiring I/O involves the TRAP instruction and the
operating system, which of course has Supervisor privilege.

We were first introduced to the TRAP instruction in Chapter~4 as a way to
get the operating system to halt the computer.  In Chapter~5 we saw that
a user program could use the TRAP instruction to get the operating system 
to do I/O tasks for it (the user program).  In fact a great benefit of the TRAP
instruction, which we have already pointed out, is that it allows the user 
programmer to not have to know the gory details of I/O discussed earlier 
in this chapter.  In addition, it protects user programs from the 
consequences of other inept user programmers.

Figure~\ref{fig:os_call} shows a user program that, upon reaching
location x4000, needs an I/O task performed.  The user program uses the 
TRAP instruction to
request the operating system to perform the task on behalf of the
user program.  The operating system takes control of the computer,
handles the request specified by the TRAP instruction, and then
returns control back to the user program at location x4001.  As we said
at the start of this chapter, we usually
refer to the request made by the user program as a {\em system call}
or a {\em service call}.\index{service call}\index{system call}

\begin{figure}[h!]
\centerline{\includegraphics{pat67509_0901.eps}}
\caption{Invoking an OS service routine using the TRAP instruction}
\label{fig:os_call}
\end{figure}

\FloatBarrier
\subsection{The trap mechanism}

The trap mechanism involves several elements:
\begin{enumerate}
\item {\bf A set of service routines} executed on behalf of user
  programs by the operating system.  These are part of the operating
  system and start at arbitrary addresses in System space. The LC-3 was
  designed so that up to 256 service routines can be specified.
  Table~A.2 in Appendix A contains the LC-3's current complete list of
  operating system service routines.

\item {\bf A table of the starting addresses} of these 256 service
  routines.  This table is stored in memory locations x0000 to x00FF.
  The table is referred to by various names by various companies.  One
  company calls this table the System Control Block.
  Another\index{System Control Block}\index{Trap Vector Table} company
  calls it the Trap Vector Table.  Figure~\ref{fig:trap_vector_table}
  shows the Trap Vector Table of the LC-3, with
  specific starting addresses highlighted.  Among the starting
  addresses are the one for the character output service routine
  (memory location x0420), which is stored in memory location x0021, 
  the one for the keyboard input service routine (location x04A0), 
  stored in location x0023, and the one for the machine halt service 
  routine (location x0520), stored in location x0025.

\begin{figure}[h!]
\centerline{\includegraphics{pat67509_0902.eps}}
\caption{The Trap Vector Table}
\label{fig:trap_vector_table}
\end{figure}

\item {\bf The TRAP instruction.} When a user program wishes to have
  the operating system execute a specific service routine on behalf of
  the user program, and then return control to the user program, the
  user program uses the TRAP instruction (as we have been doing since
  chapter 4).

\item {\bf A linkage} back to the user program.  The service routine
  must have a mechanism for returning control to the user program.
\end{enumerate}

\FloatBarrier
\subsection{The TRAP Instruction}
\index{TRAP instruction}

The TRAP instruction causes the service routine to execute by (1) changing
the PC to the starting address of the relevant service routine on the basis 
of its trap vector, and (2) providing a way to get back to the program that 
executed the TRAP instruction. The ``way back'' is referred to as 
a {\em linkage}.

As you know, the {\bf TRAP} instruction is made up of two parts: the TRAP 
opcode 1111 and the trap vector (bits [7:0]), which  identifies the service 
routine the user program wants the operating system to execute on its behalf.  
Bits [11:8] must be zero.  

In the following example, the trap vector is x23.
\begin{center}
\begin{tabular}{cccccccccccccccc}
15 & 14 & 13 & 12 & 11 & 10 & 9 & 8 & 7 & 6 & 5 & 4 & 3 & 2 & 1 & 0\\
\hline
\multicolumn{1}{|c}{\raisebox{-1pt}{1}}&\multicolumn{1}{c}{\raisebox{-1pt}{1}}&
\multicolumn{1}{c}{\raisebox{-1pt}{1}}&\multicolumn{1}{c}{\raisebox{-1pt}{1}}&
\multicolumn{1}{|c}{\raisebox{-1pt}{0}}&\multicolumn{1}{c}{\raisebox{-1pt}{0}}&
\multicolumn{1}{c}{\raisebox{-1pt}{0}}&\multicolumn{1}{c}{\raisebox{-1pt}{0}}&
\multicolumn{1}{|c}{\raisebox{-1pt}{0}}&\multicolumn{1}{c}{\raisebox{-1pt}{0}}&
\multicolumn{1}{c}{\raisebox{-1pt}{1}}&\multicolumn{1}{c}{\raisebox{-1pt}{0}}&
\multicolumn{1}{c}{\raisebox{-1pt}{0}}&\multicolumn{1}{c}{\raisebox{-1pt}{0}}&
\multicolumn{1}{c}{\raisebox{-1pt}{1}}&\multicolumn{1}{c|}{\raisebox{-1pt}{1}}\\
\hline
\multicolumn{4}{c}{TRAP} &\multicolumn{4}{c}{  } &
\multicolumn{8}{c}{trap vector}\\
\end{tabular}
\end{center}

The EXECUTE phase of the TRAP instruction's instruction cycle does
three things:
\begin{enumerate} 
\item The PSR and PC are both pushed onto the system stack.  Since the PC was 
  incremented during the FETCH phase of the TRAP instruction's instruction 
  cycle, the return linkage is automatically saved in the PC.  When control
  returns to the user program, the PC will automatically be pointing to the
  instruction following the TRAP instruction.


  Note that the program requesting the trap service routine
  can be running either in Supervisor mode or in User mode.  If in User mode, 
  R6, the stack pointer, is pointing to the user stack.  Before the PSR and PC 
  can be pushed onto the system stack, the current contents of R6 must be 
  stored in Saved\_USP, and the contents of Saved\_SSP loaded into R6.

\item PSR[15] is set to 0, since the service routine is going to require 
  Supervisor privilege to execute.  PSR[10:8] are left unchanged since the 
  priority of the TRAP routine is the same as the priority of the program 
  that requested it.

\item The 8-bit trap vector is zero-extended to 16 bits to form an
  address which corresponds to a location in the Trap Vector Table.  
  For the trap vector x23, that address is x0023.  Memory location x0023
  contains x04A0, the starting address of the TRAP x23 service routine. 
  The PC is loaded with x04A0, completing the instruction cycle.
\end{enumerate}

Since the PC contains x04A0, processing continues at memory
address x04A0.

Location x04A0 is the starting address of the operating system service
routine to input a character from the keyboard.  We say the trap
vector ``points'' to the starting address of the TRAP routine.  Thus,
TRAP x23 causes the operating system to start executing the keyboard
input service routine.

\subsection{The RTI Instruction: to return control to the calling program}

The only thing left to show is a mechanism for returning control to the 
calling program, once the trap service routine has finished execution.

This is accomplished by the Return from Trap or Interrupt (RTI) instruction:

\begin{center}
\begin{tabular}{cccccccccccccccc}
15 & 14 & 13 & 12 & 11 & 10 & 9 & 8 & 7 & 6 & 5 & 4 & 3 & 2 & 1 & 0 \\
\hline
\multicolumn{1}{|c}{\raisebox{-1pt}{1}}&\multicolumn{1}{c}{\raisebox{-1pt}{0}}&
\multicolumn{1}{c}{\raisebox{-1pt}{0}}&\multicolumn{1}{c}{\raisebox{-1pt}{0}}&
\multicolumn{1}{|c}{\raisebox{-1pt}{0}}&\multicolumn{1}{c}{\raisebox{-1pt}{0}}&
\multicolumn{1}{c}{\raisebox{-1pt}{0}}&\multicolumn{1}{ c}{\raisebox{-1pt}{0}}&
\multicolumn{1}{c}{\raisebox{-1pt}{0}}&\multicolumn{1}{c}{\raisebox{-1pt}{0}}&
\multicolumn{1}{ c}{\raisebox{-1pt}{0}}&\multicolumn{1}{c}{\raisebox{-1pt}{0}}&
\multicolumn{1}{c}{\raisebox{-1pt}{0}}&\multicolumn{1}{c}{\raisebox{-1pt}{0}}&
\multicolumn{1}{c}{\raisebox{-1pt}{0}}&\multicolumn{1}{c|}{\raisebox{-1pt}{0}}\\
\hline
\multicolumn{4}{c}{RTI} &\multicolumn{3}{c}{ } &
\multicolumn{3}{c}{ } &\multicolumn{6}{c}{ }\\
\end{tabular}
\end{center}

The RTI instruction (opcode = {\bf 1000}, with no operands) pops the the top 
two values on the system stack into the 
PC and PSR.  Since the PC contains the address following the address of the 
TRAP instruction, control returns to the user program at the correct address.

Finally, once the PSR has been popped off the system stack, PSR[15] must be
examined to see whether the processor was running in User mode or Supervisor
mode when the TRAP instruction was executed.  If in User mode, the stack 
pointers need to be adjusted to reflect that now back in User mode, the 
relevant stack in use is the User Stack.  This is done by loading the
Saved\-SSP with the current contents of R6, and loading R6 with the contents
of Saved\_USP.

Figure~\ref{fig:trap} shows the LC-3 using the TRAP instruction and
the RTI instruction to implement the example of
Figure~\ref{fig:os_call}.  The flow of control goes from (A) within a
user program that needs a character input from the keyboard, to (B)
the operating system service routine that performs that task on behalf
of the user program, back to the user program (C) that presumably uses
the information contained in the input character.

\begin{figure}[h!]
\centerline{\includegraphics{pat67509_0903.eps}}
\caption{Flow of control from a user program to an OS service routine and back}
\label{fig:trap}
\end{figure}

\subsection{A Summary of the trap service routine process}

As we know, the computer continually executes its instruction cycle
(FETCH, DECODE, etc.) on sequentially located instructions until the flow of 
control is changed by changing the contents of the PC during the EXECUTE 
phase of the current instruction.  In that way, the next FETCH will be at 
a redirected address.

The TRAP instruction with trap vector x23 in our user program does exactly
that.  Execution of TRAP x23 causes the PSR and incremented PC to be pushed 
onto the system stack and the contents of memory location x0023 
(which, in this case, contains x04A0) to be loaded into the PC.
The dashed line on Figure~\ref{fig:trap} show the use of the
trap vector x23 to obtain the starting address of the trap service routine
from the Trap Vector Table.

The next instruction cycle starts with the FETCH of the contents of
x04A0, which is the first instruction of the relevant operating system service
routine.  The trap service routine executes to completion, ending with the 
RTI instruction, which loads the PC and PSR with the top two elements on the
system stack, i.e., the PSR and incremented PC that were pushed during 
execution of the TRAP instruction.  Since the PC was incremented prior to being
pushed onto the system stack, it contains the address of the 
instruction following the TRAP instruction in the calling program, and
the user program resumes execution by fetching the instruction following the
TRAP instruction.

The following program is provided to illustrate the use of
the TRAP instruction.  It can also be used to amuse the average four-year-old!

\begin{example}

  Write a game program to do the following: A person is sitting at a
  keyboard. Each time the person types a capital letter, the program
  outputs the lowercase version of that letter.  If the person types a
  7, the program terminates.

The following LC-3 assembly language program will do the
job.

\begin{minipage}{\textwidth}
\begin{colorverbatim}
01           .ORIG x3000
02           LD    R2,TERM  ; Load -7
03           LD    R3,ASCII ; Load ASCII difference
04    AGAIN  TRAP  x23      ; Request keyboard input
05           ADD   R1,R2,R0 ; Test for terminating
06           BRz   EXIT     ; character
07           ADD   R0,R0,R3 ; Change to lowercase
08           TRAP  x21      ; Output to the monitor
09           BRnzp AGAIN    ; ... and do it again!
0A    TERM   .FILL xFFC9    ; FFC9 is negative of ASCII 7
0B    ASCII  .FILL x0020
0C    EXIT   TRAP  x25      ; Halt
0D           .END
\end{colorverbatim}
\end{minipage}

The program executes as follows: The program first loads constants
xFFC9 and x0020 into R2 and R3.  The constant xFFC9, which is the
negative of the ASCII code for 7, is used to test the character
typed at the keyboard to see if the four-year-old wants to
continue playing. The constant x0020 is the zero-extended
difference between the ASCII code for a capital letter and the
ASCII code for that same letter's lowercase representation.  For
example, the ASCII code for {\bf A} is x41; the ASCII code for
{\bf a} is x61.  The ASCII codes for {\bf Z} and {\bf z} are x5A
and x7A, respectively.

Then TRAP x23 is executed, which invokes the keyboard input service
routine.  When the service routine is finished, control returns to the
application program (at line 05), and R0 contains the ASCII code of
the character typed.  The ADD and BRz instructions test for the
terminating character 7.  If the character typed is not a 7, the ASCII
uppercase/lowercase difference (x0020) is added to the input ASCII
code, storing the result in R0.  Then a TRAP to the monitor output
service routine is called.  This causes the lowercase representation
of the same letter to be displayed on the monitor.  When control
returns to the application program (this time at line 09), an
unconditional BR to AGAIN is executed, and another request for
keyboard input appears.
\end{example}

The correct operation of the program in this example assumes that the
person sitting at the keyboard only types capital letters and the
value 7.  What if the person types a \$?  A better solution to
Example~9.1 would be a program that tests the character typed to be
sure it really is a capital letter from among the 26 capital letters
in the alphabet or the single digit 7, and if it is not, takes corrective 
action.

\questionmark {\em Question:} Augment this program to add the test for
bad data. That is, write a program that will type the lowercase
representation of any capital letter typed and will terminate if
anything other than a capital letter is typed. See Exercise~9.6.

\subsection{Trap Routines for Handling I/O}

 With the constructs just provided, the input routine described in
Figure~\ref{fig:sophisticated.routine} can be slightly modified to be
the input service routine shown in Figure~\ref{fig:char.input}. Two
changes are needed: (1) We add the appropriate .ORIG and .END
pseudo-ops.  .ORIG specifies the starting address of the input service
routine---the address found at location x0023 in the Trap Vector
Table.  And (2) we terminate the input service routine with the RTI
instruction rather than the BR NEXT\_TASK, as is
done on line 20 in Figure~\ref{fig:sophisticated.routine}. We use RTI
because the service routine is invoked by TRAP x23. It is not part
of the user program, as was the case in
Figure~\ref{fig:sophisticated.routine}.


\begin{figure}[h!]
\begin{minipage}{36pc}
\begin{Verbatim}[fontsize=\fontsize{9}{10}\selectfont]
01     ;  Service Routine for Keyboard Input
02     ;
03             .ORIG   x04A0
04     START   ST      R1,SaveR1       ; Save the values in the registers
05             ST      R2,SaveR2       ; that are used so that they
06             ST      R3,SaveR3       ; can be restored before RET
07     ;
08             LD      R2,Newline
09     L1      LDI     R3,DSR          ; Check DDR --  is it free?
0A             BRzp    L1
0B             STI     R2,DDR          ; Move cursor to new clean line
0C     ;
0D             LEA     R1,Prompt       ; Prompt is starting address
0E                                     ; of prompt string
1F     Loop    LDR     R0,R1,#0        ; Get next prompt character
10             BRz     Input           ; Check for end of prompt string
11     L2      LDI     R3,DSR
12             BRzp    L2
13             STI     R0,DDR          ; Write next character of
14                                     ; prompt string
15             ADD     R1,R1,#1        ; Increment prompt pointer
16             BRnzp   Loop
17     ;
18     Input   LDI     R3,KBSR         ; Has a character been typed?
19             BRzp    Input
1A             LDI     R0,KBDR         ; Load it into R0
1B     L3      LDI     R3,DSR
1C             BRzp    L3
1D             STI     R0,DDR          ; Echo input character
1E                                     ; to the monitor
1F     ;
20     L4      LDI     R3,DSR
21             BRzp    L4
22             STI     R2,DDR          ; Move cursor to new clean line
23             LD      R1,SaveR1       ; Service routine done, restore
24             LD      R2,SaveR2       ; original values in registers.
25             LD      R3,SaveR3
26             RTI                     ; Return from Trap
27     ;
28     SaveR1  .BLKW   1
29     SaveR2  .BLKW   1
2A     SaveR3  .BLKW   1
2B     DSR     .FILL   xFE04
2C     DDR     .FILL   xFE06
2D     KBSR    .FILL   xFE00
2E     KBDR    .FILL   xFE02
2F     Newline .FILL   x000A           ; ASCII code for newline
30     Prompt  .STRINGZ "Input a character>"
31             .END
\end{Verbatim}
\caption{Character input service routine}
\label{fig:char.input}
\end{minipage}
\end{figure}

The output routine of Section~\ref{subsubsec:basic.output.routine} can
be modified in a similar way, as shown in
Figure~\ref{fig:char.output}.  The results are input
(Figure~\ref{fig:char.input}) and output
(Figure~\ref{fig:char.output}) service routines that can be invoked
simply and safely by the TRAP instruction with the appropriate trap
vector. In the case of input, upon completion of TRAP x23, R0 contains
the ASCII code of the keyboard character typed.  In the case of
output, the initiating program must load R0 with the ASCII code of the
character it wishes displayed on the monitor and then invoke TRAP x21.

\begin{figure}[h!]
\begin{Verbatim}[fontsize=\fontsize{9}{10}\selectfont]
01             .ORIG   x0420         ; System call starting address
02              ST     R1, SaveR1    ; R1 will be used to poll the DSR
03                                   ; hardware
04    ; Write the character
05    TryWrite  LDI    R1, DSR       ; Get status
06              BRzp   TryWrite      ; Bit 15 on says display is ready
07    WriteIt   STI    R0, DDR       ; Write character
08
09    ; return from trap
0A    Return    LD     R1, SaveR1    ; Restore registers
0B              RTI                  ; Return from trap 
0C    DSR       .FILL  xFE04         ; Address of display status register
0D    DDR       .FILL  xFE06         ; Address of display data register
0E    SaveR1    .BLKW  1
0F              .END
\end{Verbatim}
\caption{Character output service routine}
\label{fig:char.output}
\end{figure}

\FloatBarrier
\subsection{A Trap Routine for Halting the Computer}

Recall from Section~4.5 that the RUN latch is ANDed with the crystal
oscillator to produce the clock that controls the operation of the
computer.  We noted that if that 1-bit latch was cleared, the output
of the AND gate would be 0, stopping the clock.

Years ago, most ISAs had a HALT instruction for stopping the clock.
Given how infrequently that instruction is executed, it seems wasteful
to devote an opcode to it. In many modern computers, the RUN latch is
cleared by a TRAP routine.  In the LC-3, the RUN latch is bit [15] of
the Master Control Register (MCR), which is memory-mapped to location
xFFFE.  Figure~\ref{fig:halt.routine} shows the trap service routine
for halting the processor, i.e., for stopping the clock.

\begin{figure}[h!]
\begin{lstlisting}[style=infigure]
01                   .ORIG   x0520     ; Where this routine resides
02                  ST      R1, SaveR1 ; R1: a temp for MC register
03                  ST      R0, SaveR0 ; R0 is used as working space
04
05  ; print message that machine is halting
06
07                  LD      R0, ASCIINewLine
08                  TRAP    x21
09                  LEA     R0, Message
0A                  TRAP    x22
0B                  LD      R0, ASCIINewLine
0C                  TRAP    x21
0D  ;
0E  ; clear bit 15 at xFFFE to stop the machine
0F  ;
10                  LDI     R1, MCR    ; Load MC register into R1
11                  LD      R0, MASK   ; R0 = x7FFF
12                  AND     R0, R1, R0 ; Mask to clear the top bit
13                  STI     R0, MCR    ; Store R0 into MC register
14  ;
15  ; return from HALT routine.
16  ; (how can this routine return if the machine is halted above?)
17  ;
18                  LD      R1, SaveR1 ; Restore registers
19                  LD      R0, SaveR0
1A                  RTI               
1B  ;
1C  ; Some constants
1D  ;
1E  ASCIINewLine    .FILL   x000A
1F  SaveR0          .BLKW   1
20  SaveR1          .BLKW   1
21  Message         .STRINGZ   "Halting the machine."
22  MCR             .FILL   xFFFE      ; Address of MCR
23  MASK            .FILL   x7FFF      ; Mask to clear the top bit
24                  .END
\end{lstlisting}
\caption{HALT service routine for the LC-3}
\label{fig:halt.routine}
\end{figure}


First (lines 02 and 03), registers R1, and R0 are saved.  R1
and R0 are saved because they are needed by the service routine.  
Then (lines 07 through 0C), the banner {\em
Halting the machine} is displayed on the monitor.  Finally (lines 10
through 13), the RUN latch (MCR[15]) is cleared by ANDing the MCR
with\questionmark[-2.3pc] 0111111111111111.  That is, MCR[14:0]
remains unchanged, but MCR[15] is cleared. {\em Question}: What
instruction (or trap service routine) can be used to start the clock?
Hint: This is a trick question!  :-)

\FloatBarrier
\subsection{The Trap routine for character input (one last time)}

Let's look again at the keyboard input service routine of
Figure~\ref{fig:char.input}.  In particular, let's look at the
three-line sequence that occurs at symbolic addresses L1, L2, L3, and
L4:

\begin{colorverbatim}
                 LABEL   LDI     R3,DSR
                         BRzp    LABEL
                         STI     Reg,DDR
\end{colorverbatim}
Can the JSR/RET mechanism enable us to replace these four occurrences
of the same sequence with a single subroutine?  {\em Answer:} Yes,
{\bf almost.}

Figure~\ref{fig:char.input.revisited}, our ``improved'' keyboard input
service routine, contains

\begin{colorverbatim}
                         JSR     WriteChar
\end{colorverbatim}
at lines 04, 0A, 10, and 13, and the four-instruction subroutine

\begin{colorverbatim}
           WriteChar     LDI     R3,DSR
                         BRzp    WriteChar
                         STI     R2,DDR
                         RET
\end{colorverbatim}
at lines 1A through 1D. Note the RET instruction (aka JMP R7)
that is needed to terminate the subroutine.

\begin{figure}[h!]
\begin{Verbatim}[fontsize=\fontsize{8}{9}\selectfont]
01                .ORIG   x04A0
02   START        JSR     SaveReg
03                LD      R2,Newline
04                JSR     WriteChar
05                LEA     R1,PROMPT
06   ;
07   ;
08   Loop         LDR     R2,R1,#0    ; Get next prompt char
09                BRz     Input
0A                JSR     WriteChar
0B                ADD     R1,R1,#1
0C                BRnzp   Loop
0D   ;
0E   Input        JSR     ReadChar
0F                ADD     R2,R0,#0    ; Move char to R2 for writing
10                JSR     WriteChar   ; Echo to monitor
11   ;
12                LD      R2, Newline
13                JSR     WriteChar
14                JSR     RestoreReg
15                RTI                 ; RTI terminates the trap routine
16   ;                                  
17   Newline      .FILL   x000A
18   PROMPT       .STRINGZ  "Input a character>"
19   ;
1A   WriteChar    LDI     R3,DSR
1B                BRzp    WriteChar
1C                STI     R2,DDR
1D                RET                 ; JMP R7 terminates subroutine
1E   DSR          .FILL   xFE04
1F   DDR          .FILL   xFE06
20    ;
21   ReadChar     LDI     R3,KBSR
22                BRzp    ReadChar
23                LDI     R0,KBDR
24                RET
25   KBSR         .FILL   xFE00
26   KBDR         .FILL   xFE02
27    ;
28    SaveReg       ST      R1,SaveR1
29                  ST      R2,SaveR2
2A                  ST      R3,SaveR3
2B                  ST      R4,SaveR4
2C                  ST      R5,SaveR5
2D                  ST      R6,SaveR6
2E                  RET
2F    ;
30    RestoreReg    LD      R1,SaveR1
31                  LD      R2,SaveR2
32                  LD      R3,SaveR3
33                  LD      R4,SaveR4
34                  LD      R5,SaveR5
35                  LD      R6,SaveR6
36                  RET
37    SaveR1        .FILL   x0000
38    SaveR2        .FILL   x0000
39    SaveR3        .FILL   x0000
3A    SaveR4        .FILL   x0000
3B    SaveR5        .FILL   x0000
3C    SaveR6        .FILL   x0000
3D                  .END
\end{Verbatim}
\caption{The LC-3 trap service routine for character input (our final answer!)}
\label{fig:char.input.revisited}
\end{figure}

Note the hedging: {\em almost}. In the original sequences starting at
L2 and L3, the STI instruction forwards the contents of R0 (not R2) to
the DDR.  We can fix that easily enough, as follows: In line 08 of
Figure~\ref{fig:char.input.revisited}, we use

\begin{colorverbatim}
                         LDR     R2,R1,#0
\end{colorverbatim}
instead of
\begin{colorverbatim}
                         LDR     R0,R1,#0
\end{colorverbatim}
This causes each character in the prompt to be loaded into R2.  The
subroutine Writechar forwards each character from R2 to the DDR.

In line 0F of Figure~\ref{fig:char.input.revisited}, we insert the
instruction

\begin{colorverbatim}
                         ADD     R2,R0,#0
\end{colorverbatim}
in order to move the keyboard input (which is in R0) into R2.  The
subroutine Writechar forwards it from R2 to the DDR.  Note that R0
still contains the keyboard input.  Furthermore, since no subsequent
instruction in the service routine loads R0, R0 still contains the
keyboard input after control returns to the user program.

In line 12 of Figure~\ref{fig:char.input.revisited}, we insert the
instruction

\begin{colorverbatim}
                         LD     R2,Newline
\end{colorverbatim}
in order to move the ``newline'' character into R2.  The subroutine
Writechar forwards it from R2 to the DDR.

Figure~\ref{fig:char.input.revisited} is the actual LC-3 trap service
routine provided for keyboard input.

\FloatBarrier
\subsection{PUTS: Writing a Character String to the Monitor}

Before we leave the example of Figure \ref{fig:char.input.revisited},
note the code on lines 08 through 0C. This fragment of the service
routine is used to write the sequence of characters {\em Input a
character} to the monitor.  A sequence of characters is often
\index{strings} referred to as a {\em string of characters} or a {\em
character string}.  This fragment is also present in
Figure~\ref{fig:halt.routine}, with the result that {\em Halting the
machine} is written to the monitor.  In fact, it is so often the case
that a user program needs to write a string of characters to the
monitor that this function is given its own trap service routine in the
\hbox{LC-3} operating system.  Thus, if a user program requires a
character string to be written to the monitor, it need only provide
(in R0) the starting address of the character string, and then invoke
TRAP x22.  In LC-3 assembly language this TRAP is called {\em PUTS}.

PUTS (or TRAP x22) causes control to be passed to the operating
system, and the trap routine shown in Figure~\ref{fig:puts.routine} is
executed.  Note that PUTS is the code of lines 08 through 0C of
Figure~\ref{fig:char.input.revisited}, with a few minor adjustments.

\begin{figure}[h!]
\begin{minipage}{36pc}
\begin{Verbatim}[fontsize=\fontsize{9}{10}\selectfont]
01     ; This service routine writes a NULL-terminated string to the console.
02     ; It services the PUTS service call (TRAP x22).
03     ; Inputs: R0 is a pointer to the string to print.
04     ;
05                  .ORIG   x0460           
06                  ST      R0, SaveR0      ; Save registers that 
07                  ST      R1, SaveR1      ; are needed by this
08                  ST      R3, SaveR3      ; trap service routine
09     ;
0A     ; Loop through each character in the array
0B     ;
0C     Loop         LDR     R1, R0, #0      ; Retrieve the character(s)
0D                  BRz     Return          ; If it is 0, done
0E     L2           LDI     R3,DSR
0F                  BRzp    L2
10                  STI     R1, DDR         ; Write the character
11                  ADD     R0, R0, #1      ; Increment pointer
12                  BRnzp   Loop            ; Do it all over again
13     ;
14     ; Return from the request for service call
15     Return       LD      R3, SaveR3
16                  LD      R1, SaveR1
17                  LD      R0, SaveR0
18                  RTI
19     ;
1A     ; Register locations
1B     DSR          .FILL   xFE04
1C     DDR          .FILL   xFE06
1D     SaveR0       .FILL   x0000
1E     SaveR1       .FILL   x0000
1F     SaveR3       .FILL   x0000
20                  .END
\end{Verbatim}
\caption{The LC-3 PUTS service routine}
\label{fig:puts.routine}
\end{minipage}
\end{figure}

\FloatBarrier 
\section{Interrupts and Interrupt-Driven I/O}
\label{sec:interruptsIO}
\index{interrupt-driven I/O}

In Section~\ref{subsubsec:interruptVSpolling}, we noted that
interaction between the processor and an I/O device can be controlled
by the processor (i.e., polling) or it can be controlled by the I/O
device (i.e., interrupt driven). In Sections
\ref{subsec:input.from.keyboard}, \ref{subsec:output.to.monitor}, and
\ref{subsec:sophisticated.routine}, we have studied several examples
of polling. In each case, the processor tested the Ready bit of the
status register again and again, and when the Ready bit was finally 1, the
processor branched to the instruction that did the input or output
operation.

We are now ready to study the case where the interaction is controlled
by the I/O device.

\subsection{What Is Interrupt-Driven I/O?}

The essence of interrupt-driven I/O is the notion that an I/O device
that may or may not have anything to do with the program that is
running can (1) force the running program to stop, (2) have the processor
execute a program that carries
out the needs of the I/O device, and then (3) have the stopped
program resume execution as if nothing had happened.  These three
stages of the instruction execution flow are shown in
Figure~\ref{fig:flow_interrupt_IO}.

\begin{figure}[h!]
\setlength{\topsep}{0pt}
\setlength{\partopsep}{0pt}
\setlength{\parskip}{0pt}
\begin{Verbatim}[fontsize=\fontsize{9}{13}\selectfont]
                             .
                             .
                             .
          Program A is executing instruction n
          Program A is executing instruction n+1
          Program A is executing instruction n+2
\end{Verbatim}
\setlength{\topsep}{0pt}
\setlength{\partopsep}{0pt}
\setlength{\parskip}{0pt}
\begin{Verbatim}[fontsize=\fontsize{9}{11}\selectfont,formatcom=\colour]
        1: Interrupt signal is detected
        1: Program A is put into suspended animation
	1: PC is loaded with the starting address of Program B
        2: Program B starts satisying I/O device's needs 
        2: Program B continues satisfying I/O device's needs
        2: Program B continues satisfying I/O device's needs
        2: Program B finishes satisfying I/O device's needs
        3: Program A is brought back to life
\end{Verbatim}
\setlength{\topsep}{0pt}
\setlength{\partopsep}{0pt}
\setlength{\parskip}{0pt}
\begin{Verbatim}[fontsize=\fontsize{9}{11}\selectfont]
          Program A is executing instruction n+3
          Program A is executing instruction n+4
                             .
                             .
                             .
\end{Verbatim}
\caption{Instruction execution flow for interrupt-driven I/O}
\label{fig:flow_interrupt_IO}
\end{figure}

As far as Program A is concerned, the work carried out and the results
computed are no different from what would have been the case if the
interrupt had never happened; that is, as if the instruction execution
flow had been the following:

\begin{Verbatim}[fontsize=\fontsize{9}{10}\selectfont,formatcom=\color{seventyblack}]
                             .
                             .
                             .
             Program A is executing instruction n
             Program A is executing instruction n+1
             Program A is executing instruction n+2
             Program A is executing instruction n+3
             Program A is executing instruction n+4
                             .
                             .
                             .
\end{Verbatim}

\FloatBarrier
\subsection{Why Have Interrupt-Driven I/O?}

As is undoubtedly clear, polling requires the processor to waste a lot
of time spinning its wheels, re-executing again and again the LDI and
BR instructions until the Ready bit is set.  With interrupt-driven
I/O, none of that testing and branching has to go on.
Interrupt-driven I/O allows the processor to spend its time doing what
is hopefully useful work, executing some other program perhaps, until
it is notified that some I/O device needs attention.

\begingroup
\makeatletter
\renewenvironment{example}{%
  \refstepcounter{example}%
  \begin{colourframed}%
    \def\@mathmargin{\z@}
    \let\enumargs\exenumargs
    \vspace*{-2\p@}%
    \rlap{\hspace*{29pc}\hbox{\hspace*{-\fboxsep}%
      \setlength{\fboxsep}{\z@}%
      {\colorbox{SPOThundred}{\vbox to 13\p@{%
          \vss\hbox to 7pc{\hss
          \color{white}\sansbold\fontsize{10}{12}\selectfont
          Example~\theexample\hss}\vss}}}}}%
  \normalfont\fontsize{9.5}{11}\selectfont
  \rightskip7.5pc\advance\rightskip by \fboxsep
  \leftskip3\p@
  \parindent1.5pc\@afterheading\@afterindentfalse
  \vspace*{-14.5\p@}}%
{\par\unskip\removelastskip\offinterlineskip\vspace{4\p@}\end{colourframed}}
\makeatother
\begin{example}
Suppose we are asked to write a program that takes a sequence of 100
characters typed on a keyboard and processes the information contained
in those 100 characters.  Assume the characters are typed at the rate
of 80 words/minute, which corresponds to one character every 0.125
seconds.  Assume the processing of the 100-character sequence takes
12.49999 seconds, and that our program is to perform this process on
1,000 consecutive sequences.  How long will it take our program to
complete the task?  (Why did we pick 12.49999? To make the numbers
come out nice, of course.)  :-)

We could obtain each character input by polling, as in
Section~\ref{subsec:input.from.keyboard}.  If we did, we would waste a
lot of time waiting for the ``next'' character to be typed.  It would
take $100\cdot 0.125$ or 12.5 seconds to get a 100-character sequence.

On the other hand, if we use interrupt-driven I/O, the processor does
not waste any time re-executing the LDI and BR instructions while
waiting for a character to be typed.  Rather, the processor can be
busy working on the previous 100-character sequence that was typed,
{\bf except} for those very small fractions of time when it is
interrupted by the I/O device to read the next character typed.  Let's
say that to read the next character typed requires executing a
10-instruction program that takes on the average 0.00000001 seconds to
execute each instruction. That means 0.0000001 seconds for each
character typed, or 0.00001 seconds for the entire 100-character
sequence.  That is, with interrupt-driven I/O, since the processor is
only needed when characters are actually being read, the time required
for each 100-character sequence is 0.00001 seconds, instead of
12.50000 seconds.  The remaining 12.49999 of every 12.50000 seconds,
the processor is available to do useful work. For example, it can
process the previous 100-character sequence.

The bottom line: With polling, the time to complete the entire task
for each sequence is 24.9999 seconds, 12.5 seconds to obtain the 100
characters + 12.49999 seconds to process them.  With interrupt-driven
I/O, the time to complete the entire task for each sequence after the
first is 12.5 seconds, 0.00001 seconds to obtain the characters +
12.49999 seconds to process them.  For 1,000 sequences, that is the
difference between 7 hours and 3 ${\scriptstyle\frac{1}{2}}$ hours.
\end{example}
\endgroup

\subsection{Two parts to the process}

There are two parts to interrupt-driven I/O: 

\begin{enumerate}
\item the mechanism that enables an I/O device to interrupt the processor, and 
\item the mechanism that handles the interrupt request.
\end{enumerate}

\subsection{Part I: Causing the interrupt to occur}

\noindent
Several things must be true for an I/O device to
actually interrupt the program that is running:

\begin{enumerate}
\item The I/O device {\bf must want} service.
\item The device {\bf must have the right} to request the service.
\item The device request {\bf must be more urgent} than what the processor
  is currently doing.
\end{enumerate}
If all three elements are present, the processor stops executing the
program that is running and takes care of the interrupt.

\subsubsection{The Interrupt Signal from the Device}

For an I/O device to generate an interrupt request, the device must want
service, and it must have the right to request that service.

\paragraph{The device must want service}  We have discussed that already in the 
study of polling.  It is the Ready bit of the KBSR or the DSR.  That is, 
if the I/O device is the keyboard, it wants service if someone has typed a
character.  If the I/O device is the monitor, it wants service (i.e.,
the next character to output) if the associated electronic circuits
have successfully completed the display of the last character.  In
both cases, the I/O device wants service when the corresponding Ready
bit is set.

\paragraph{The device must have the right to request that service}  This is 
the interrupt enable bit, which can be set or cleared by the processor
(usually by the operating system), depending on whether or not the processor 
wants to give the I/O device the right to request service.  In most
I/O devices, this interrupt enable (IE) bit is part of the device
status register.  In the KBSR and DSR shown in
Figure~\ref{fig:interrupt.enable.bits}, the IE bit is bit [14].
\index{interrupt enable} The {\bf interrupt request signal from the I/O
device} is the logical AND of the IE bit and the Ready bit, as is also
shown in Figure~\ref{fig:interrupt.enable.bits}.

\begin{figure}[h!]
\centerline{\includegraphics{pat67509_0807.eps}}
\caption{Interrupt enable bits and their use}
\label{fig:interrupt.enable.bits}
\end{figure}


If the interrupt enable bit (bit [14]) is clear, it does not matter
whether the Ready bit is set; the I/O device will not be able to
interrupt the processor because it (the I/O device) has not been given
the right to interrupt the processor.  In that case, the program will 
have to poll the I/O device to determine if it is ready.

If bit [14] is set, then interrupt-driven I/O is enabled.  In that
case, as soon as someone types a key (or as soon as the monitor has
finished processing the last character), bit [15] is set.  In this case,
the device wants service and it has been given the right to request service.
The AND gate is asserted, causing an interrupt request to be generated from 
the I/O device.

\FloatBarrier
\subsubsection{The urgency of the request}

The third element in the list of things that must be true for an I/O
device to actually interrupt the processor is that the request must be
more urgent than the program that is currently executing.  Recall from
Section 9.1.1.2 that each program runs at a specified level of urgency called
its priority level.  To interrupt the running program, the device must have a
higher priority than the program that is currently running.  Actually, 
there may be many devices that want to interrupt the processor at a specific 
time.  To succeed, the device must have a higher priority level than all other 
demands for use of the processor.  

Almost all computers have a set of priority levels that programs can
run at.  As we have already noted, the LC-3 has eight priority levels, 
PL0, .. PL7.  The higher
the number, the more urgent the program.  The PL of a program is
usually the same as the PL (i.e., urgency) of the request to run that
program. If a program is running at one PL, and a higher-level PL
request wants the computer, the lower-priority program
suspends processing until the higher-PL program executes and satisfies
its more urgent request.  For example, a computer's payroll program
may run overnight, and at PL0.  It has all night to finish---not
terribly urgent.  A program that corrects for a nuclear plant current
surge may run at PL6.  We are perfectly happy to let the payroll wait
while the nuclear power correction keeps us from being blown to bits.

For our I/O device to successfully stop the processor and start an
interrupt-driven I/O request, the priority of the request must be
higher than the priority of the program it wishes to interrupt.  For
example, we would not normally want to allow a keyboard interrupt from
a professor checking e-mail to interrupt the nuclear power correction
program.

\subsubsection{The INT signal}

To stop the processor from continuing execution of its currently running
program and service an interrupt request, the INT signal must be 
asserted.  Figure~\ref{fig:generation.interrupt.signal} shows what is
required to assert the INT signal.
Figure~\ref{fig:generation.interrupt.signal}
shows the status registers of several devices operating at various
priority levels (PL).  Any device that has bits [14] and [15] both set
asserts its interrupt request signal.  The interrupt request signals
are input to a priority encoder, a combinational logic structure that
selects the highest priority request from all those asserted.  If the
PL of that request is higher than the PL of the currently executing
program, the INT signal is asserted.

\begin{figure}[h!]
\centerline{\includegraphics{pat67509_0808.eps}}
\caption{Generation of the INT signal}
\label{fig:generation.interrupt.signal}
\end{figure}

\FloatBarrier
\subsubsection{The Test for INT} 

Finally, the test to enable the processor to stop and handle the interrupt.  
Recall
from Chapter~4 that the instruction cycle continually sequences through the 
phases of the instruction cycle (FETCH, DECODE, EVALUATE ADDRESS, FETCH 
OPERAND, EXECUTE, and STORE RESULT). Each instruction changes the state of
the computer, and that change is completed at the end of the instruction
cycle for that intruction.  That is, in the last clock cycle before the 
computer returns to the FETCH phase for the next instruction, the computer 
is put in the state caused by the complete execution of the current intruction.  

Interrupts can happen at any time.  They are asynchronous to the synchronous 
finite state machine controlling the computer.  For example, the interrupt 
signal could occur when the instruction cycle is in its FETCH OPERAND
phase.  If we stopped the currently executing program when the instruction
cycle was in its FETCH OPERAND phase, we would have to keep track of what
part of the current instruction has executed and what part of the current
instruction still has work to do.  It makes much more sense to ignore 
interrupt signals except when we are at an instruction boundary; that is, 
the current instruction has completed, and the next instruction has not yet 
started.  Doing that means we do not have to worry about partially executed 
instructions, since the state of the computer is the state created by the 
completion of the current instruction, period!

The additional logic to test for the interrupt signal is to augment
the last state of the instruction cycle for each instruction with a test.
Instead of {\bf always} going from the last state of one instruction cycle
to the first state of
the FETCH phase of the next instruction, the next state depends on the INT 
signal.  If INT is not asserted, then it is business as usual, with the 
control unit returning to the FETCH phase to start processing the next 
instruction.  If INT is asserted, then the next state is the first state 
of Part II, handling the interrupt request.

\subsection{Part II: Handling the interrupt request}

Handling the interrupt request goes through three
stages, as shown in Figure~\ref{fig:flow_interrupt_IO}:

\begin{enumerate}
\item[1.] Initiate the interrupt (three lines numbered 1 in Figure~\ref{fig:flow_interrupt_IO}.
\item[2.] Service the interrupt (four lines numbered 2 in Figure~\ref{fig:flow_interrupt_IO}.
\item[3.] Return from the interrupt (one line numbered 3 in Figure~\ref{fig:flow_interrupt_IO}.
\end{enumerate}

We will discuss each.

\subsubsection{Initiate the Interrupt} 

Since the INT signal was asserted, the processor does not return to
the first state of the FETCH phase of the next instruction cycle, but
rather begins a sequence of actions to initiate the 
interrupt.  The processor must do two things, (1) save the state of the
interrupted program so it can pick up where it left off after the
requirements of the interrupt have been completed, and (2) load the state of 
the higher priority interrupting program so it can start satisfying its request.
 
\paragraph{Save the State of the Interrupted Program}

The state of a program is a snapshot of the contents of all the
program's resources.  It includes the contents of the
memory locations that are part of the program and the contents of all
the general purpose registers. It also includes the PC and PSR.

Recall from Figure~9.1 in Section 9.1.1.4 that a program's PSR specifies 
the privilege level and priority level of that program.
PSR[15] indicates whether the program is running in privileged
(Supervisor) or unprivileged (User) mode.  PSR[10:8] specifies the 
program's priority level (PL), from PL0 (lowest) to PL7 (highest). 
Also, PSR[2:0] is used to store the condition codes.
PSR[2] is the N bit, PSR[1] is the Z bit, and PSR[0] is the P bit.

The first step in initiating the interrupt is to save enough of the
state of the program that is running so it can continue where it left
off after the I/O device request has been satisfied. That means, in
the case of the LC-3, saving the PC and the PSR. The PC must be saved
since it knows which instruction should be executed next when the
interrupted program resumes execution. The condition codes (the N, Z,
and P flags) must be saved since they may be needed by a subsequent
conditional branch instruction after the program resumes execution.
The priority level of the interrupted program must be saved because it
specifies the urgency of the interrupted program with respect to all
other programs. When the interrupted program resumes execution, it is
important to know what priority level programs can interrupt it 
and which ones can not. Finally, the privilege level of the program
must be saved since it specifies what processor
resources the interrupted program can and can not access.

Although many computers save the contents of the general purpose
registers, we will not since we will assume that the service routine 
will always save the contents of any general purpose register that
it needs before using it, and then restore it before returning to the 
interrupted program.  The only state information the LC-3 saves are
the PC and PSR.

The LC-3 saves this state information on the 
Supervisor Stack in the same way the PC and PSR are saved when a TRAP
instruction is executed.  That is, before the interrupt service routine starts, 
if the interrupted program is in User mode, the User Stack Pointer (USP) is
stored in Saved\_USP, and R6 is loaded with the Supervisor Stack Pointer (SSP)
from Saved\_SSP.  Then the PSR and PC of
the interrupted program are pushed onto the Supervisor Stack, where
they remain unmolested while the service routine executes.

\paragraph{Load the State of the Interrupt Service Routine}
Once the state of the interrupted program has been safely saved on the
Supervisor Stack, the second step is to load the PC and PSR of the
interrupt service routine. Interrupt service routines are similar to
the trap service routines we have already discussed.  They are program
fragments stored in System Space.  They service interrupt requests.

Most processors use the mechanism of \textit{vectored interrupts}. You
are familiar with this notion from your study of the trap vector
contained in the TRAP instruction. In the case of interrupts, the
8-bit vector is provided by the device that is requesting the
processor be interrupted. That is, the I/O device transmits to the
processor an 8-bit interrupt vector along with its interrupt request
signal and its priority level. The interrupt vector corresponding to
the highest priority interrupt request is the one supplied to the
processor. It is designated INTV. 

If the interrupt is taken, the
processor expands the 8-bit interrupt vector (INTV) to form a 16-bit
address, which is an entry into the Interrupt Vector Table.  You know 
that the Trap Vector Table consists of memory locations x0000 to x00FF, 
each containing the starting address of a trap service
routine. The Interrupt Vector Table consists of memory locations x0100
to x01FF, each containing the starting address of an interrupt service
routine. The processor loads the PC with the contents of the location in
the Interrupt Vector Table corresponding to the address formed by expanding 
the interrupt vector INTV.

For example, the LC-3 keyboard could interrupt the processor every
time a key is pressed by someone sitting at the keyboard. The
keyboard interrupt vector would indicate the location in the interrupt
vector table that contains the starting address of the keyboard interupt
service routine.

The PSR is loaded as follows: Since no instructions in the service
routine have yet executed, PSR[2:0] contains no meaningful information.
We arbitratily initially load it with 010.
Since the interrupt service routine runs in privileged mode, PSR[15]
is set to 0.  PSR[10:8] is set to the priority level associated with
the interrupt request.

This completes the initiation phase and the interrupt service routine
is ready to execute.

\subsubsection{Service the Interrupt}

Since the PC contains the starting address of the interrupt
service routine, the service routine will execute, and the
requirements of the I/O device will be serviced.

\subsubsection{Return from the Interrupt}

The last instruction in every interrupt service routine is RTI, return
from trap or interrupt.  When the processor finally accesses the RTI
instruction, all the requirements of the I/O device have been taken
care of.

Like the return from a trap routine discussed in Section~9.3.4, 
execution of the {\bf RTI} instruction (opcode $=$ 1000) for an interrupt
service routine consists
simply of popping the PC and the PSR from the Supervisor Stack (where
they have been resting peacefully) and restoring them to their
rightful places in the processor. The condition codes are now restored
to what they were when the program was interrupted, in case they are
needed by a subsequent BR instruction in the interrupted program.  PSR[15] and
PSR[10:8] now reflect the privilege level and priority level of the
about-to-be-resumed program.  If the privilege level of the interrupted
program is unprivileged, the stack pointers must be adjusted, i.e., the 
Supervisor Stack Pointer saved, and the User Stack Pointer loaded into R6.
The PC is restored to the address of the instruction that would have 
been executed next if the program had not been interrupted.

With all these things as they were before the interrupt occurred, the
program can resume as if nothing had happened.

\subsection{An Example}

We complete the discussion of interrupt-driven I/O with an example.

Suppose program A is executing when I/O device B, having a PL higher
than that of A, requests service. During the execution of the service
routine for I/O device B, a still more urgent device C requests
service.

Figure~\ref{fig:execution.flow.interrupt.IO} shows the execution flow
that must take place.

\begin{figure}[h!]
\centerline{\includegraphics{pat67509_1006.eps}}
\caption{Execution flow for interrupt-driven I/O}
\label{fig:execution.flow.interrupt.IO}
\end{figure}

Program A consists of instructions in locations x3000 to x3010 and was
in the middle of executing the ADD instruction at x3006, when device B
sent its interrupt request signal and accompanying interrupt vector
xF1, causing INT to be asserted.

Note that the interrupt service routine for device B is stored in
locations x6200 to x6210; x6210 contains the RTI instruction. Note
that the service routine for B was in the middle of executing the AND
instruction at x6202, when device C sent its interrupt request signal and
accompanying interrupt vector xF2. Since the request associated with
device C is of a higher priority than that of device B, INT is again
asserted.

Note that the interrupt service routine for device C is stored in
locations x6300 to x6315; x6315 contains the RTI instruction.

Let us examine the order of execution by the
processor. Figure~\ref{fig:stack.pc.interrupt.IO} shows several
snapshots of the contents of the Supervisor Stack and the PC during
the execution of this example.

\begin{figure}[h!]
\centerline{\includegraphics{pat67509_1007.eps}}
\caption{Snapshots of the contents of the Supervisor Stack and the PC during interrupt-driven I/O}
\label{fig:stack.pc.interrupt.IO}
\end{figure}

The processor executes as follows: Figure~\ref{fig:stack.pc.interrupt.IO}a
shows the Supervisor Stack and the PC before program A fetches the 
instruction at x3006.  Note that the stack pointer is shown as Saved\_SSP,
not R6.  Since the interrupt has not yet occurred, R6 is pointing to the
current contents of the User Stack, which are not shown!  The INT signal 
(caused by an interrupt from device B) is detected at the end of execution 
of the instruction in
x3006.  Since the state of program A must be saved on the Supervisor Stack,
the first step is to start using the Supervisor Stack. This is done by 
saving R6 in the Saved.UPC register, and loading R6 with the contents of the
Saved\_SSP register.  The PSR of program A, which includes the condition codes
produced by the ADD instruction, is pushed on the Supervisor Stack.  Then the
address x3007, the PC for the next instruction to be executed in program A
is pushed on the stack.  The interrupt vector associated with device B is
expanded to 16 bits x01F1, and the contents of x01F1 (x6200) is loaded into
the PC.  Figure~\ref{fig:stack.pc.interrupt.IO}b shows the stack and PC at
this point.

The service routine for device B executes until a higher priority interupt
is detected at the end of execution of the instruction at x6202.  The PSR
of the service routine for B, which includes the condition codes produced by
the AND instruction at x6202, and the address x6203 are pushed on the stack.
The interrupt vector associated with device C is expanded to 16 bits (x01F2),
and the contents of x01F2 
(x6300) are loaded into the
PC. Figure~\ref{fig:stack.pc.interrupt.IO}c shows the Supervisor Stack
and PC at this point.

Assume the interrupt service routine for device C executes to completion,
finishing with the RTI instruction in x6315. The Supervisor Stack is
popped twice, restoring the PC to x6203 and the PSR of the service routine 
for device B, including the condition codes produced by the AND instruction in
x6202.  Figure~\ref{fig:stack.pc.interrupt.IO}d shows the stack and PC
at this point.

The interrupt service routine for device B resumes execution at x6203
and runs to completion, finishing with the RTI instruction in x6210.
The Supervisor Stack is popped twice, restoring the PC to x3007 and
the PSR of program A, including the condition codes produced by the ADD 
instruction in x3006.  Finally, since program A is in
User Mode, the contents of R6 are stored in Saved\_SSP and R6 is loaded
with the contents of
Saved\_USP. Figure~\ref{fig:stack.pc.interrupt.IO}e shows the
Supervisor Stack and PC at this point.

Program A resumes execution with the instruction at x3007.

\FloatBarrier
\subsection{Not just I/O devices}

We have discussed the processing of interrupts in the context of I/O
devices that have higher priority than the program that is running and
therefore can stop that program to enable its interrupt service routine
to execute.  

We must point out that not all interrupts deal with I/O devices.  Any
event that has a higher priority and is external to the program that is
running can interrupt the computer.
It does so by supplying its INT signal, its INTV vector and its priority
level.  If it is the highest priority event that wishes to interrupt the
computer, it does so in the same way that I/O devices do as described above.

There are many examples of such events that have nothing to do with I/O 
devices.  For example, a {\em timer interrupt} interrupts the program that
is running in order to note the passage of a unit of time.  The {\em machine 
check} interrupt calls attention to the fact that some part of the computer 
system is not functioning properly.  The {\em power failure} interrupt notifies
the computer that for example someone has yanked the power cord out of its
receptacle.  Unfortunately, we will have to put off dealing with all of these 
until later in your coursework.

\section{Polling revisted, now that we know about interrupts}

\subsection{The problem}

Recall our discussion of polling: We continually test the Ready bit in the
relevant status register, and if it is not set, we branch back to again
test the Ready bit.  For example, suppose we are writing a character string to
the monitor, and we are using polling to determine when the monitor has
successfully written the current characer so we can dispatch the next
character.  We take it for granted that the three instruction sequence  
LDI (to load the ready bit of the DSR), BRzp (to test it and fall through 
if the device is ready), and STI (to store the next character in the DDR) 
acts as an atomic unit.  But what if we interrupts enabled 
at the same time.  That is, if an interrupt occurred {\bf within} that 
LDI, BRzp, 
STI sequence (say, just before the STI instruction), it could easily 
be the case that the LDI instruction indicated 
the DDR was ready, the BRzp instruction did not branch back, but by the time 
the interrupt service routine completed so the STI could write to the DDR,
the DDR may no longer be ready.  The computer would execute the STI but the
write would not happen.

A simple, but somewhat contrived example :-), will illustrate the problem.  
Suppose you are executing a "for" loop ten times, where each time the loop 
body prints to the monitor a particualr character.  Polling is used to
determine that the monitor is ready before writing the next character to the 
DDR.
Since the loop body executes ten times, this should result in the character 
printed on the monitor ten times.  Suppose you also have keyboard interrupts 
enabled, and the keyboard service routine echoes the character typed.  

Suppose the loop body executes as follows: LDI loads the ready bit, BRzp 
falls through since the monitor is ready, and STI stores the character 
in DDR.  In the middle of this sequence, before the STI can execute, someone 
types a key.  The keyboard interrupt occurs, the character typed is echoed, 
i.e., written to the DDR, and the keyboard interrupt service routine completes.

The interrupted loop body then takes over and "knows" the monitor is ready so 
it executes the STI.  ...except the monitor is not ready because it has not 
completed the write of the keyboard service routine!  The STI of the loop 
body writes but since DDR is not ready, the write does not occur.  The final
result: only nine characers get written, not ten.

The problem becomes more serious if the string written is in code, and the
missing write prevents the code from being decyphered.

A simple way to handle this would be to disable all interrupts while
polling was going on.  But consider the consequences.  
Suppose the polling was required for a long time. If we disable 
interrupts while polling is going on, interrupts would be disabled for
that very long time, unacceptable in an environment where one is concerned
about the time between a higher priority interrupt occurring and it
getting service.

\subsection{The solution}

A better solution is shown in Figure~\ref{fig:Lumetta}.

The sequence we want to make non-interruptable is shown on lines 0F to 11.
We accomplish this by first loading R1 with the PSR in line 09 and R2 
with the PSR having interrupts disabled in line 0A.  PSR[14] is 
the interrupt enable 
bit for all interrupts associated with this program.  Note that PSR is 
memory mapped to xFFFC.  We enable interupts by storing R1 in PSR (line 0D),
followed immediately by disabling interrupts by storing R2 in PSR (line 0E).
With interrupts disabled we execute the three instruction sequence LDI,
BRzp, and LDI (lines 0F, 10, and 11) if the status register indicates that
the device is ready.  If the device is not ready, BRzp (line 10) takes the
computer back to line 0D where interrupts are again enabled.

In this way, interrupts are disabled again and again, but each time only  
long enough to execute the three instruction sequence LDI, BRzp, STI
(in lines 0F, 10, 0D), 
afterwhich interrupts are enabled again.  The result: an interrupt would
have to wait for the three instruction sequence LDI, BRzp, STI to execute, 
rather than for the entire polling process to complete.  

\begin{figure}[h]
\begin{Verbatim}[fontsize=\fontsize{9}{11}\selectfont]
01	    .ORIG x0420          
02          ADD   R6,R6,#-1
03          STR   R1,R6,#0
04          ADD   R6,R6,#-1
05          STR   R2,R6,#0
06          ADD   R6,R6,#-1
07          STR   R3,R6,#0    ; Save R1,R2,R3 on the stack 
08 ;         
09          LDI   R1, PSR 
0A          LD    R2,INTMASK
0B          AND   R2,R1,R2    ; R1=original PSR, R2=PSR with interrupts disabled
0C 
0D  POLL    STI   R1,PSR    ; enable interrupts (if they were enabled to begin
0E          STI   R2,PSR    ; disable interrupts
0F          LDI   R3,DSR    
10          BRzp  POLL      ; Poll the DSR
11          STI   R0,DDR   ; Store the character into the DDR
12          STI   R1,PSR   ; Restore original PSR
13
14          LDR   R3,R6,#0
15          ADD   R6,R6,#1
16          LDR   R2,R6,#0
17          ADD   R6,R6,#1         
18          LDR   R1,R6,#0      
19          ADD   R6,R6,#1  ; Restore R3,R2,and R1 from the stack
1A  
1B          RTI          
1C          
1D INTMASK .FILL   xBFFF
1E PSR     .FILL   xFFFC
1F DSR     .FILL   xFE04
20 DDR     .FILL   xFE06
21 
22         .END
\end{Verbatim}
\FloatBarrier
\caption{Polling AND allowing interrupts}
\label{fig:Lumetta}
\end{figure}

\newpage
\clearpage

\begin{exercises}

\item[9.1]
\begin{enumerate}[a.]
\item[a.] What is a device register?
\item[b.] What is a device data register?
\item[c.] What is a device status register?
\end{enumerate}

\item[9.2] Why is a Ready bit not needed if synchronous I/O is used?

\item[9.3] In Section~9.2.1.3, the statement is made that a typist would
  have trouble supplying keyboard input to a 300-MHz processor at the
  maximum rate (one character every 33 nanoseconds) that the processor
  can accept it.  Assume an average word (including spaces between
  words) consists of six characters. How many words/minute would the
  typist have to type in order to exceed the processor's ability to
  handle the input?

\item[9.4] Are the following interactions usually synchronous or
  asynchronous?

\begin{enumerate}[a.]
\item[a.] Between a remote control and a television set
\item[b.] Between the mailcarrier and you, via a mailbox
\item[c.] Between a mouse and your PC
\end{enumerate}

Under what conditions would each of them be synchronous?  Under what
conditions would each of them be asynchronous?

\item[9.5] What is the purpose of bit [15] in the KBSR?

\item[9.6] What problem could occur if a program does not check the
  Ready bit of the KBSR before reading the KBDR?

\item[9.7] Which of the following combinations describe the system
  described in Section~9.2.2.2?

\begin{enumerate}[a.]
\item[a.] Memory mapped and interrupt driven
\item[b.] Memory mapped and polling
\item[c.] Special opcode for I/O and interrupt driven
\item[d.] Special opcode for I/O and polling
\end{enumerate}

\item[9.8] Write a program that checks the initial value in memory
  location x4000 to see if it is a valid ASCII code and if it is a
  valid ASCII code, prints the character. If the value in x4000 is not
  a valid ASCII code, the program prints nothing.

\item[9.9] What problem is likely to occur if the keyboard hardware
  does not check the KBSR before writing to the KBDR?

\item[9.10] What problem could occur if the display hardware does not
  check the DSR before writing to the DDR?

\item[9.11] Which is more efficient, interrupt-driven I/O or polling?
  Explain.

\item[9.12] Adam H. decided to design a variant of the LC-3 that did
  not need a keyboard status register.  Instead, he created a
  readable/writable keyboard data and status register (KBDSR), which
  contains the same data as the KBDR.  With the KBDSR, a program
  requiring keyboard input would wait until a nonzero value appeared
  in the KBDSR. The nonzero value would be the ASCII value of the last
  key press.  Then the program would write a zero into the KBDSR
  indicating that it had read the key press.  Modify the basic input
  service of Section~8.2.2 to implement Adam's scheme.

\item[9.13] Some computer engineering students decided to revise the
  LC-3 for their senior project. In designing the LC-4, they decided
  to conserve on device registers by combining the KBSR and the DSR
  into one status register: the IOSR (the input/output status
  register). IOSR[15] is the keyboard device Ready bit and IOSR[14] is
  the display device Ready bit. What are the implications for programs
  wishing to do I/O? Is this a poor design decision?


\item[9.14] An LC-3 Load instruction specifies the address xFE02. How
  do we know whether to load from the KBDR or from memory location
  xFE02?


\item[9.15] Name some of the advantages of doing I/O through a TRAP routine instead of
writing the routine yourself each time you would like your program to perform I/O.

\item[9.16]
\begin{enumerate}[a.]
\item[a.] How many  trap service routines can be implemented in the LC-3? Why?

\item[b.] Why must a RET instruction be used to return from a TRAP routine?
Why won't a BR (Unconditional Branch) instruction work instead?

\item[c.] How many accesses to memory are made during the processing of a TRAP
instruction?  Assume the TRAP is already in the IR.
\end{enumerate}

\item[9.17] Refer to Figure~\ref{fig:halt.routine}, the HALT service
routine.
\begin{enumerate}[a.]
\item[a.] What starts the clock after the machine is HALTed?
Hint: How can the HALT service routine return
after bit [15] of the machine control register is cleared?

\item[b.] Which instruction actually halts the machine?

\item[c.] What is the first instruction executed when the machine is started again?

\item[d.] Where will the RET of the HALT routine return to?
\end{enumerate}

\item[9.18] Consider the following LC-3 assembly language program:
\begin{colorverbatim}
                .ORIG   x3000
        L1      LEA     R1, L1
                AND     R2, R2, x0
                ADD     R2, R2, x2
                LD      R3, P1
        L2      LDR     R0, R1, xC
                OUT
                ADD     R3, R3, #-1
                BRz     GLUE
                ADD     R1, R1, R2
                BR      L2
        GLUE    HALT
        P1      .FILL   xB
                .STRINGZ "HBoeoakteSmtHaotren!s"
                .END
\end{colorverbatim}
\begin{enumerate}[a.]
\item[a.] After this program is assembled and loaded, what binary
  pattern is stored in memory location x3005?

\item[b.] Which instruction (provide a memory address) is executed
  after instruction x3005 is executed?

\item[c.] Which instruction (provide a memory address) is executed
  prior to instruction x3006?

\item[d.] What is the output of this program?
\end{enumerate}

\item[9.19] The following LC-3 program is assembled and then executed. There are
no assemble time or run-time errors.  What is the output of this
program? Assume all registers are initialized to 0 before the program executes.
\begin{colorverbatim}
                .ORIG    x3000
                LEA      R0, LABEL
                STR      R1, R0, #3
                TRAP     x22
                TRAP     x25
        LABEL   .STRINGZ "FUNKY"
        LABEL2  .STRINGZ "HELLO WORLD"
                .END
\end{colorverbatim}

\item[9.6] The correct operation of the program in Example~9.1 assumes
  that the person sitting at the keyboard only types capital letters
  and the value 7.  What if the person types a \$? A better program
  would be one that tests the character typed to be sure it really is
  a capital letter from among the 26 capital letters in the alphabet,
  and if it is not, takes corrective action.  Your job: Augment the
  program of Example~9.1 to add a test for bad data.  That is, write a
  program that will type the lowercase representation of any capital
  letter typed and will terminate if anything other than a capital
  letter is typed.

\item[9.19] Assume that an integer greater than 2 and less than 32,768 is deposited
in memory location A by another module before the program below is executed.
\begin{colorverbatim}
                .ORIG  x3000
                AND    R4, R4, #0
                LD     R0, A
                NOT    R5, R0
                ADD    R5, R5, #2
                ADD    R1, R4, #2
                ;
       REMOD    JSR    MOD
                BRz    STORE0
                ;
                ADD    R7, R1, R5
                BRz    STORE1
                ADD    R1, R1, #1
                BR     REMOD
                ;
       STORE1   ADD    R4, R4, #1
       STORE0   ST     R4, RESULT
                TRAP   x25
                ;
       MOD      ADD    R2, R0, #0
                NOT    R3, R1
                ADD    R3, R3, #1
       DEC      ADD    R2, R2, R3
                BRp    DEC
                RET
                ;
       A        .BLKW 1
       RESULT   .BLKW 1
                .END
\end{colorverbatim}

\noindent In 20 words or fewer, what does the above program do?

\item[9.20] Recall the machine busy example. Suppose the bit pattern indicating which
machines are busy and which are free is stored in memory location x4001.
Write subroutines that do the following.
\begin{enumerate}[a.]
\item[a.] Check if no machines are busy, and return 1 if none are
  busy.
\item[b.] Check if all machines are busy, and return 1 if all are
  busy.
\item[c.] Check how many machines are busy, and return the number of
  busy machines.
\item[d.] Check how many machines are free, and return the number of
  free machines.
\item[e.] Check if a certain machine number, passed as an argument in
  R5, is busy, and return 1 if that machine is busy.
\item[f.] Return the number of a machine that is not busy.
\end{enumerate}

\item[9.21] The starting address of the trap routine is stored at the
  address specified in the TRAP instruction. Why isn't the first
  instruction of the trap routine stored at that address instead?
  Assume each trap service routine requires at most 16 instructions.
  Modify the semantics of the LC-3 TRAP instruction so that the trap
  vector provides the starting address of the service routine.

\item[9.22] Following is part of a program that was fed to the LC-3
  assembler. The program is supposed to read a series of input lines
  from the console into a buffer, search for a particular character,
  and output the number of times that character occurs in the text.
  The input text is terminated by an EOT and is guaranteed to be no
  more than 1,000 characters in length.  After the text has been
  input, the program reads the character to count.

  The subroutine labeled COUNT that actually does the counting was
  written by another person and is located at address x3500. When
  called, the subroutine expects the address of the buffer to be in R5
  and the address of the character to count to be in R6. The buffer
  should have a NULL to mark the end of the text. It returns the count
  in R6.

  The OUTPUT subroutine that converts the binary count to ASCII digits
  and displays them was also written by another person and is at
  address x3600. It expects the number to print to be in~R6.

  Here is the code that reads the input and calls COUNT:

\begin{colorverbatim}
        .ORIG  x3000
        LEA    R1, BUFFER
G_TEXT  TRAP   x20          ; Get input text
        ADD    R2, R0, x-4
        BRz    G_CHAR
        STR    R0, R1, #0
        ADD    R1, R1, #1
        BRnzp  G_TEXT
G_CHAR  STR    R2, R1, #0   ; x0000 terminates buffer
        TRAP   x20          ; Get character to count
        ST     R0, S_CHAR
        LEA    R5, BUFFER
        LEA    R6, S_CHAR
        LD     R4, CADDR
        JSRR   R4           ; Count character
        LD     R4, OADDR
        JSRR   R4           ; Convert R6 and display
        TRAP   x25
CADDR   .FILL  x3500        ; Address of COUNT
OADDR   .FILL  x3600        ; Address of OUTPUT
BUFFER  .BLKW  1001
S_CHAR  .FILL  x0000
        .END
\end{colorverbatim}
There is a problem with this code. What is it, and how might it be
fixed? (The problem is {\em not} that the code for \verb+COUNT+ and
\verb+OUTPUT+ is missing.)

\item[9.23] Consider the following LC-3 assembly language program:

\begin{colorverbatim}
                .ORIG   x3000
                LEA     R0,DATA
                AND     R1,R1,#0
                ADD     R1,R1,#9
        LOOP1   ADD     R2,R0,#0
                ADD     R3,R1,#0
        LOOP2   JSR     SUB1
                ADD     R4,R4,#0
                BRzp    LABEL
                JSR     SUB2
        LABEL   ADD     R2,R2,#1
                ADD     R3,R3,#-1
                BRP     LOOP2
                ADD     R1,R1,#-1
                BRp     LOOP1
                HALT
        DATA    .BLKW   10 x0000
        SUB1    LDR     R5,R2,#0
                NOT     R5,R5
                ADD     R5,R5,#1
                LDR     R6,R2,#1
                ADD     R4,R5,R6
                RET
        SUB2    LDR     R4,R2,#0
                LDR     R5,R2,#1
                STR     R4,R2,#1
                STR     R5,R2,#0
                RET
                .END
\end{colorverbatim}

Assuming that the memory locations at DATA get filled in before the
program executes, what is the relationship between the final values at
DATA and the initial values at DATA?

\item[9.24] The following program is supposed to print the number 5 on the
screen. It does not work. Why? Answer in no more than ten words, please.

\begin{colorverbatim}
                .ORIG   x3000
                JSR     A
                OUT
                BRnzp   DONE
        A       AND     R0,R0,#0
                ADD     R0,R0,#5
                JSR     B
                RET
        DONE    HALT
        ASCII   .FILL   x0030
        B       LD      R1,ASCII
                ADD     R0,R0,R1
                RET
                .END

\end{colorverbatim}

\item[9.25] Figure~\ref{fig:halt.routine} shows a service routine to
stop the computer by clearing the RUN latch, bit [15] of the Machine
Control Register.  The latch is cleared by the instruction in line 14,
and the computer stops.  What purpose is served by the instructions on
lines 19 through 1C?

\item[9.26] Suppose we define a new service routine starting at memory location
x4000. This routine reads in a character and echoes it to the
screen. Suppose memory location x0072 contains the value x4000. The
service routine is shown below.

\begin{colorverbatim}
                .ORIG x4000
                ST R7, SaveR7
                GETC
                OUT
                LD R7, SaveR7
                RET
        SaveR7  .FILL x0000
\end{colorverbatim}

\begin{enumerate}[a.]
\item[a.] Identify the instruction that will invoke this routine.
\item[b.] Will this service routine work? Explain.
\end{enumerate}

\item[9.27] The two code sequences $a$ and $b$ are assembled
  separately. There are two errors that will be caught at assemble
  time or at link time. Identify the bugs and describe why the bug
  will cause an error, and whether it will be detected at assemble 
  time or link time.

\begin{enumerate}[a.]
\item[a.]
\begin{colorverbatim}
                .ORIG x3500
         SQRT   ADD   R0, R0, #0
                ; code to perform square
                ; root function and
                ; return the result in R0
                RET
                .END
\end{colorverbatim}

\item[b.]
\begin{colorverbatim}
                .EXTERNAL SQRT
                .ORIG   x3000
                LD      R0,VALUE
                JSR     SQRT
                ST      R0,DEST
                HALT
        VALUE   .FILL   x30000
        DEST    .FILL   x0025
                .END













\end{colorverbatim}
\end{enumerate}

\item[9.28] Shown below is a partially constructed program. The
  program asks the user his/her name and stores the sentence ``Hello,
  name'' as a string starting from the memory location indicated by
  the symbol HELLO. The program then outputs that sentence to the
  screen. The program assumes that the user has finished entering
  his/her name when he/she presses the Enter key, whose ASCII code is
  x0A. The name is restricted to be not more than 25 characters.

  Assuming that the user enters Onur followed by a carriage return
  when prompted to enter his/her name, the output of the program looks
  exactly like:

\begin{colorverbatim}
Please enter your name: Onur
Hello, Onur
\end{colorverbatim}

Insert instructions at (a)--(d) that will complete the program.

\begin{colorverbatim}
                .ORIG x3000
                LEA   R1,HELLO
      AGAIN     LDR   R2,R1,#0
                BRz   NEXT
                ADD   R1,R1,#1
                BR    AGAIN
      NEXT      LEA   R0,PROMPT
                TRAP  x22          ; PUTS
                ------------ (a)
      AGAIN2    TRAP  x20          ; GETC
                TRAP  x21          ; OUT
                ADD   R2,R0,R3
                BRz   CONT
                ------------ (b)
                ------------ (c)
                BR    AGAIN2
      CONT      AND   R2,R2,#0
                ------------ (d)
                LEA   R0, HELLO
                TRAP  x22          ; PUTS
                TRAP  x25          ; HALT
      NEGENTER  .FILL xFFF6        ; -x0A
      PROMPT    .STRINGZ "Please enter your name: "
      HELLO     .STRINGZ "Hello, "
                .BLKW #25
                .END













\end{colorverbatim}

\item[9.29] The program below, when complete, should print the following to the monitor:
\begin{center}
ABCFGH
\end{center}

Insert instructions at (a)--(d) that will complete the program.
\begin{colorverbatim}
                .ORIG x3000
                LEA   R1, TESTOUT
      BACK_1    LDR   R0, R1, #0
                BRz   NEXT_1
                TRAP  x21
                ------------ (a)
                BRnzp BACK_1
                ;
      NEXT_1    LEA   R1, TESTOUT
      BACK_2    LDR   R0, R1, #0
                BRz   NEXT_2
                JSR   SUB_1
                ADD   R1, R1, #1
                BRnzp BACK_2
                ;
      NEXT_2    ------------ (b)
                ;
      SUB_1     ------------ (c)

      K         LDI   R2, DSR
                ------------ (d)

                STI   R0, DDR
                RET
      DSR       .FILL xFE04
      DDR       .FILL xFE06
      TESTOUT   .STRINGZ "ABC"
                .END


\end{colorverbatim}

\item[9.30] A local company has decided to build a real LC-3 computer.
  In order to make the computer work in a network, four
  interrupt-driven I/O devices are connected. To request service, a
  device asserts its interrupt request signal (IRQ). This causes a bit
  to get set in a special LC-3 memory-mapped interrupt control
  register called INTCTL which is mapped to address xFF00. The INTCTL
  register is shown below. When a device requests service, the INT
  signal in the LC-3 data path is asserted.  The LC-3 interrupt
  service routine determines which device has requested service and
  calls the appropriate subroutine for that device.  If more than one
  device asserts its IRQ signal at the same time, only the subroutine
  for the highest priority device is executed.  During execution of
  the subroutine, the corresponding bit in INTCTL is cleared.

\begin{figure}[!h]
\centerline{\includegraphics{pat67509_un0901.eps}}
\end{figure}

The following labels are used to identify the first instruction of
each device subroutine:
\begin{center}
HARDDISK ETHERNET PRINTER CDROM
\end{center}

For example, if the highest priority device requesting service is the
printer, the interrupt service routine will call the printer
subroutine with the following instruction:
\begin{center}
JSR  PRINTER
\end{center}

Finish the code in the LC-3 interrupt service routine for the
following priority scheme by filling in the spaces labeled (a)--(k).
The lower the number, the higher the priority of the device.
\begin{enumerate}
\item[{\rm 1.}] Hard disk
\item[{\rm 2.}] Ethernet card
\item[{\rm 3.}] Printer
\item[{\rm 4.}] CD-ROM
\end{enumerate}

\begin{Verbatim}[fontsize=\fontsize{9}{10}\selectfont,formatcom=\color{seventyblack}]
                LDI   R1, INTCTL
      DEV0      LD    R2, ------ (a)
                AND   R2, R2, R1
                BRnz  DEV1
                JSR   ---------- (b)
                ---------------- (c)
         ;
      DEV1      LD    R2, ------ (d)
                AND   R2, R2,  R1
                BRnz  DEV2
                JSR   ---------- (e)
                ---------------- (f)
         ;
      DEV2      LD    R2, ------ (g)
                AND   R2, R2, R1
                BRnz  DEV3
                JSR   ---------- (h)
                ---------------- (i)
         ;
      DEV3      JSR  ----------- (j)
         ;
      END       ---------------- (k)

      INTCTL    .FILL   xFF00
      MASK8     .FILL   x0008
      MASK4     .FILL   x0004
      MASK2     .FILL   x0002
      MASK1     .FILL   x0001
\end{Verbatim}

\item[9.31] Interrupt-driven I/O:
\begin{enumerate}[a.]
\item[a.] What does the following LC-3 program do?
\begin{colorverbatim}
            .ORIG     x3000
            LD        R3, A
            STI       R3, KBSR
    AGAIN   LD        R0, B
            TRAP      x21
            BRnzp     AGAIN
    A      .FILL      x4000
    B      .FILL      x0032
    KBSR   .FILL      xFE00
           .END

\end{colorverbatim}

\item[b.] If someone strikes a key, the program will be interrupted and the
keyboard interrupt service routine will be executed as shown below.
What does the keyboard interrupt
  service routine do?
\begin{colorverbatim}
            .ORIG     x1000
            LDI       R0, KBDR
            TRAP      x21
            TRAP      x21
            RTI
    KBDR    .FILL     xFE02
            .END
\end{colorverbatim}
Note: RTI is an instruction that enables the computer to return to executing 
the program that was interrupted. It will be studied in Chapter 10. The only 
thing you need to know about it now is that it loads the PC with the address 
of the instruction that was about to be fetched when the interrupt occurred.
\item[c.] Finally, suppose the program of part $a$ started executing, and
  someone sitting at the keyboard struck a key. What would you see on
  the screen?
\item[d.] In part c, how many times is the digit typed shown on the screen? 
  Why is the correct answer: "I can not say for sure."
\end{enumerate}

\item[9.32] What does the following LC-3 program do?
  \begin{colorverbatim}
            .ORIG   x3000
            LD      R0,ASCII
            LD      R1,NEG
    AGAIN   LDI     R2,DSR
            BRzp    AGAIN
            STI     R0,DDR
            ADD     R0,R0,#1
            ADD     R2,R0,R1
            BRnp    AGAIN
            HALT
    ASCII   .FILL   x0041
    NEG     .FILL   xFFB6   ; -x004A
    DSR     .FILL   xFE04
    DDR     .FILL   xFE06
            .END
  \end{colorverbatim}

\item[9.33] During the initiation of the interrupt service routine, the N, Z,
  and P condition codes are saved on the stack. Show by means of a
  simple example how incorrect results would be generated if the
  condition codes were not saved.

\item[9.34] In the example of Section~9.4.n, what are the contents of locations
  x01F1 and x01F2? They are part of a larger structure. Provide a name
  for that structure. ({\it Hint:} See Table~A.3.)

\item[9.35] Expand the example of Section~9.4.n to include an interrupt by a
  still more urgent device D while the service routine of device C is
  executing the instruction at x6310. Assume device D's interrupt
  vector is xF3.  Assume the interrupt service routine is stored in
  locations x6400 to x6412. Show the contents of the stack and PC at
  each relevant point in the execution flow.

\item[9.36] Suppose device D in Exercise~9.35 has a lower priority than device C
  but a higher priority than device B. Rework Exercise~10.12 with this
  new wrinkle.

\item[9.37] Write an interrupt handler to accept keyboard input as follows: A
  buffer is allocated to memory locations x4000 through x40FE. The
  interrupt handler must accept the next character typed and store it
  in the next ``empty'' location in the buffer. Memory location x40FF
  is used as a pointer to the next available empty buffer location. If
  the buffer is full (i.e., if a character has been stored in location
  x40FE), the interrupt handler must display on the screen:
  ``Character cannot be accepted; input buffer full.''

\item[9.38] Consider the interrupt handler of Exercise~9.37. The buffer is
  modified as follows: The buffer is allocated to memory locations
  x4000 through x40FC. Location x40FF contains, as before, the address
  of the next available empty location in the buffer. Location x40FE
  contains the address of the oldest character in the buffer. Location
  x40FD contains the number of characters in the buffer. Other
  programs can remove characters from the buffer. Modify the interrupt
  handler so that, after x40FC is filled, the next location filled is
  x4000, assuming the character in x4000 has been previously removed.
  As before, if the buffer is full, the interrupt handler must display
  on the screen: ``Character cannot be accepted; input buffer full.''

\item[9.39] Consider the modified interrupt handler of Exercise~9.38, used in
  conjunction with a program that removes characters from the buffer.
  Can you think of any problem that might prevent the interrupt
  handler that is adding characters to the buffer and the program that
  is removing characters from the buffer from working correctly
  together?

\item[9.40] Suppose the keyboard interrupt vector is x34 and the keyboard
interrupt service routine starts at location x1000. What can you
infer about the contents of any memory location from the above
statement?

\item[9.41] Two students wrote interrupt service routines for an assignment.  Both
service routines did exactly the same work, but the first student
accidentally used RET at the end of his routine, while the second
student correctly used RTI. There are three errors that arose in the
first student's program due to his mistake.  Describe any two of them.

\item[9.42]STAR Since ASCII codes consist of 8 bits each, we can store two ASCII codes in one
word of LC-3 memory.  If a user types 2n characters on the keyboard, followed
by the \textless ENTER\textgreater  key, the subroutine PACK on the next page
will store the corresponding ASCII codes into n sequential memory locations,
two per memory location, starting at location A. \\

\noindent
You may assume that a user never enters an odd number of characters. \\

\noindent
Your job: Fill in the blanks in the program.\\

If a user types the string \textbf{Please help!} followed by the
$<$ENTER$>$ key, what does the program do? \\


%\begin{center}
\begin{minipage}[t]{0.5\linewidth}
\begin{flushleft}
{
\begin{alltt}
          .ORIG x7020
  PACK    ST R7, SAVER7
          ST R6, SAVER6
          ST R4, SAVER4
          ST R3, SAVER3
          LEA R6, A      ; R6 is the pointer
          AND R4, R4, #0
          ADD R4, R4, #8 ; R4 is our counter
          AND R3, R3, #0
          LEA R0, PROMPT
          TRAP x22
   POLL   --------------- (a)
          BRzp POLL
		  --------------- (b)
          LD R0, NEG_LF
          ADD R0, R7, R0
          --------------- (c)
          ADD R4, R4, #0
          BRz NOSHIFT
  SHIFT   ADD R7, R7, R7
          ADD R4, R4, #-1
          BRp SHIFT
          ADD R3, R7, #0
          BRnzp POLL
  NOSHIFT ADD R3, R3, R7
  		  --------------- (d)
          ADD R6, R6, #1
          ADD R4, R4, #8
          BRnzp POLL
  DONE    LD R7, SAVER7
          LD R6, SAVER6
          LD R4, SAVER4
          LD R3, SAVER3
          LEA R0, A     ; Returns a pointer to the characters
          RET
  KBSR    .FILL xFE00
  KBDR    .FILL xFE02
  NEG_LF  .FILL xFFF6
  PROMPT  .STINGZ "Please enter a string: "
  A       .BLKW #5
  SAVER7  .BLKW #1
  SAVER6  .BLKW #1
  SAVER4  .BLKW #1
  SAVER3  .BLKW #1
          .END
\end{alltt}
}
\end{flushleft}
\end{minipage}

\item[9.43]STAR We want to support 8 input keyboards instead of 1.  To do this we need 8 ready bits in KBSR, and 8 separate KBDRs.  We will use the 8 odd-numbered bits in the KBSR as ready bits for the 8 keyboards, as shown below. We will set the other 8 bits in the KBSR to 0.\\

\begin{figure}[h]
\begin{center}
\includegraphics[scale=0.5]{943_kbsr.eps}
\end{center}
\end{figure}

\noindent
The 8 memory-mapped keyboard data registers and their corresponding ready bits
are as follows: \\

\begin{center}
\begin{tabular}{l l l} 
FE04: & KBSR   & \\  
FE06: & KBDR1, & Ready bit is KBSR[1]\\  
FE08: & KBDR2, & Ready bit is KBSR[3]\\  
FE0A: & KBDR3, & Ready bit is KBSR[5]\\  
FE0C: & KBDR4, & Ready bit is KBSR[7]\\  
FE0E: & KBDR5, & Ready bit is KBSR[9]\\  
FE10: & KBDR6, & Ready bit is KBSR[11]\\  
FE12: & KBDR7, & Ready bit is KBSR[13]\\  
FE14: & KBDR8, & Ready bit is KBSR[15]\\  
\end{tabular}
\end{center}

\noindent
We wish to write a program that polls the keyboards and loads the ASCII code
typed by the highest priority keyboard into R0.  That is, if someone had
previously typed a key on keyboard 1, we want to load the ASCII code in KBDR1
into R0.  If no key was typed on keyboard 1, but a key had been typed on
keyboard 2, we want to load the ASCII code in KBDR2 into R0.  ...and so on.
That is, KB1 has higher priority than KB2, which has higher priority than KB3,
which has higher priority than KB4, etc.  KB8 has the lowest priority. \\

\noindent
The following program will do the job AFTER you fill in the missing 
instructions. Your job: fill in the missing instructions.

\begin{verbatim}
        .ORIG X3000
        LD    R0, KBDR1
POLL    LDI   R1, KBSR
        BRz   POLL
        AND   R2, R2, #0
        ADD   R2, R2, #2
AGAIN   ----------------- (a)
        BRnp    FOUND
        ADD     R0, R0, #2
        ----------------- (b)
        ----------------- (c)
        BRnp    AGAIN
        HALT
FOUND   ----------------- (d)
        HALT
KBSR    .FILL   xFE04
KBDR1   .FILL   xFE06
        .END
\end{verbatim}

\item[9.44]STAR The following program pushes elements onto a stack with JSR PUSH and pops elements off of the stack with JSR POP. \\

\begin{center}
\begin{minipage}[t]{0.5\linewidth}
\begin{flushleft}
{
\begin{alltt}
            .ORIG X3000
            LEA  R6, STACK\_BASE

X           TRAP  x20          ;GETC
            TRAP  x21          ;OUT
            ADD  R1, R0, x-0A  ;x0A is ASCII code for line feed, 
                               ;x-0A is the negative of x0A
            BRz  Y
            JSR  PUSH
            BRnzp  X

Y           LEA  R2, STACK\_BASE
            NOT  R2, R2
            ADD  R2, R2, #1
            ADD  R3, R2, R6
            BRz  DONE
            JSR  POP
            TRAP  x21          ;OUT
            BRnzp  Y

DONE        TRAP  x25          ;HALT
STACK       .BLKW  5
STACK\_BASE .FILL  x0FFF

PUSH        ADD  R6, R6, #-1
            STR  R0, R6, #0
            RET

POP         LDR  R0, R6, #0
            ADD  R6, R6, #1
            RET

            .END

\end{alltt}
}

\end{flushleft}
\end{minipage}
\end{center}

\noindent
What will appear on the screen if a user, sitting at a keyboard,
typed the three keys a, b, c, followed by the \textless Enter\textgreater { }key? \\ 

What will happen if a user, sitting at a keyboard, typed the eight keys a, b, c, d, e, f, g, h, followed by the \textless Enter\textgreater { }key? 

\item[9.45] We wish to add a new TRAP service routine, which will be called
by the instruction {\tt TRAP~x9A}.  The new trap routine will wait for someone
to type a lower case letter, then echo on the screen the corresponding capital
letter. Assume the user will not type anything except a lower case letter.  The
assembly language code for this trap service routine is shown below: \\

\begin{alltt}
        .ORIG 		x2055
        ----------------- (a)
        ST R1, SaveR1
        ST R0, SaveR0
        TRAP x20
        LD R1, A
        ----------------- (b)
        TRAP x21
        ----------------- (c)
        LD R1, SaveR1
        LD R0, SaveR0
        JMP R7
SaveR1 .BLKW 1
SaveR0 .BLKW 1
A      .FILL ------------ (d)
____   .BLKW           ;  (e) a missing label 

       .END
\end{alltt}


\noindent In order for {\tt TRAP~x9A} to call this service routine, what
memory location must contain what value.
\\

\vspace{.25in}

\noindent Address:
\hspace{1in
Value:} 

\vspace{.25in}

\noindent Fill in the missing information in the assembly language program.
i.e, the three missing instructions, the one missing label, and the operand
of the {\tt .FILL} pseudo-op.

\item[9.46] A programmer wrote the following program which was assembled and executed. Execution started with PC at x3000. 

\begin{alltt}
         .ORIG x3000

         LEA R0, Message	
         TRAP x01
         TRAP x22	; What is the output here?
         TRAP x25

Message	.STRINGZ "Cat in the hat."

        .END
\end{alltt}

\vspace{3mm}
\noindent Assume that the Trap Vector Table includes the following entries in addition to the ones we have previously used:

\begin{alltt}
Memory Address    Memory Contents
    x0000	             x0100
    x0001	             x0102
    x0002	             x0107
    x0003	             x010A
\end{alltt}

\noindent Assume further that additional trap service routines have been loaded previously in memory as specified below:

\begin{alltt}
        .ORIG x0100

        LD R7, SaveR7
        RET
        ST R7, SaveR7
        TRAP x02
        AND R1, R1, #0
        STR R1, R0, #3
        RET
        AND R1, R1, #0
        STR R1, R0, #5
        TRAP x00
        RET
SaveR7	.BLKW #1

        .END
\end{alltt}

\noindent What is the result of execution of this program? 

\item[9.47]STAR The state machine shown below will produce an output sequence if it receives an input sequence.  The intial state is S0.

\begin{figure}[h]
\centering
\vspace{0.3in}
\includegraphics[scale=0.5]{947_state.eps}
\end{figure}

\noindent For example, the input sequence 100 produces the output sequence bac.

\noindent We have written a program that simulates this state machine.  Inputs are requested from the keyboard, and the corresponding outputs are shown on the screen.  For example, for the input sequence shown above, the monitor would display
\begin{alltt}

INPUT (either 0 or 1): 1
OUTPUT: b
INPUT (either 0 or 1): 0
OUTPUT: a
INPUT (either 0 or 1): 0
OUTPUT: c
\end{alltt}

Complete the program that simulates the state machine, by filling in each blank with {\bf one} missing line of LC-3 assembly language code. You can assume the person at the keyboard can type a 1 or a 0 without error (i.e., you do not have to test for wrong input).
\newpage
\begin{alltt}
        .ORIG 		x3000
        LEA 		R6, S0
Loop    ------------------ (a)
        TRAP		x22
        TRAP		x20			; inputs a character
        TRAP 		x21
		
        LD		R1, NEGASCII
        ADD		R0, R0, R1
        ------------------ (b)
        LDR		R6, R6,#0
        LD		R0, NEWLINE
        TRAP		x21
        LEA 		R0, OUTPUT
        TRAP		x22
        ------------------ (c)
        TRAP		x21
        LD		R0, NEWLINE
        TRAP		x21
        BRnzp		LOOP

S0      .FILL		S2
        .FILL		S1
        .FILL		x0061

S1      .FILL		S0
        .FILL		S2
        .FILL		x0062
S2      .FILL   	______ (d)
		.FILL 		______ (e)
		.FILL 		______ (f)

NEGASCII .FILL		xFFD0				; the value -48
OUTPUT   .STRINGZ       "OUTPUT:"
INPUT    .STRINGZ       "INPUT (either 0 or 1):"
NEWLINE  .FILL		x000A

.END
\end{alltt}

\item[9.48]STAR Up to now, we have only had one output device, the monitor, with xFE04 and xFE06 used to address its two device registers. We now introduce a second output device, a light that requires a single device register, to which we assign the address xFE08.  Storing a 1 in xFE08 turns the light on, storing a 0 in xFE08 turns the light off. \\

\noindent
An Aggie decided to write a program which would control this light by a keyboard interrupt as follows: Pressing the key 0 would turn the light off.
Pressing the key 1 would cause the light to flash on and off repeatedly.  Shown
below is the Aggie's code, and his Keyboard interrupt service routine.

\begin{verbatim}
The User Program:
             .ORIG x3000
0             LEA  R7, LOOP
1     LOOP    LDI  R0, ENABLE
2             LD   R1, NEG_OFF
3             ADD  R0, R0, R1   ; check if switch is on
4             BRnp BLINK
      ;
5             AND R0, R0, #0
6             STI R0, LIGHT     ; turn light off
7             RET
      ;
8     BLINK   ST  R7, SAVE_R7   ; save linkage
9             LDI R0, LIGHT   
A             ADD R0, R0, #1
B             AND R0, R0, #1    ; toggle LIGHT between 0 and 1
C             STI R0, LIGHT
D             JSR DELAY         ; 1 second delay
E             LD R7, SAVE_R7
F             RET               ; <-- Breakpoint here
      ;
      LIGHT   .FILL xFE08
      ENABLE  .FILL x4000
      NEG_OFF .FILL x-30
      SAVE_R7 .BLKW #1
              .END

The Keyboard Interrupt Routine:
              .ORIG   x1500
0             ADD R6, R6, #-1   ; <-- Breakpoint here
1             STR R0, R6, #0    ; save R0 on stack
2             ADD R6, R6, #-1
3             STR R7, R6, #0    ; save R7 on stack
      ;
4             TRAP x20
5             STI  R0, ENABLE2
      ;
6             RTI               ; <-- Breakpoint here
7     ENABLE2 .FILL x4000
              .END
\end{verbatim}

\noindent
The DELAY subroutine was inserted in his program in order to separate the
turning on and off of the light by one second in order to make the on-off
behavior visible to the naked eye.  The DELAY subroutine does not modify any
registers. \\

\noindent
Unfortunately, per usual, the Aggie made a mistake in his program, and things
do not work as he intended.  So, he decided to debug his program (see the next
page). \\

\noindent
He set three breakpoints, at x1500, at x1506, and at x300F.  He initialized the
PC to x3000, the keyboard IE bit to 1, and memory location x0180 to x1500. \\

\noindent
Then he hit the Run button, which stopped executing when the PC reached x1500.
He hit the Run button three more times, each time the computer stopping when
the PC reached a breakpoint.  While the program was running, he pressed a key
on the keyboard EXACTLY ONCE. \\

\noindent
The table below shows the data in various registers and memory locations each
time a breakpoint was encountered. Note: Assume, when an interrupt is
initiated, the PSR is pushed onto the system stack before the PC.\\

\noindent
Complete the table. \\

\noindent
\begin{tabular}{|c|c|c|c|c|c|}
\hline
 & \textbf{Initial} & \textbf{Breakpoint 1} & \textbf{Breakpoint 2} &
 \textbf{Breakpoint 3} & \textbf{Breakpoint 4} \\
 \hline \hline
\textbf{PC} & x3000 & x1500 & x1506 & x1506 & x300F \\ \hline
\textbf{R0} & x1234 & & x0030 & &\\ \hline
\textbf{R6} & x3000 & & & &\\ \hline
\textbf{R7} & x1234 & & & &\\ \hline
\textbf{M[x2FFC]} & x0000 & & & &\\ \hline
\textbf{M[x2FFD]} & x0000 & & & &\\ \hline
\textbf{M[x2FFE]} & x0000 & x300D & & &\\ \hline
\textbf{M[x2FFF]} & x0000 & x8001 & & &\\ \hline
\textbf{M[x4000]} & x0031 & & & &\\ \hline
\textbf{M[xFE00]} & x4000 & & & &\\ \hline
\end{tabular}

\item[9.49]STAR The following user program (priority 0) is assembled and loaded
into memory.\\

\begin{verbatim}
       .ORIG x8000
       LD R0, Z
AGAIN  ADD R0, R0, #-1
       BRnp AGAIN
       LD R0, W
       BRp L1
       LD R0, X
       TRAP x21
       BRnzp DONE
L1     LEA R0, Y
       TRAP x22
DONE   HALT

X     .FILL x34
Y     .STRINGZ "OOOOPS!"
Z     .FILL x100
W     .BLKW #1
      .END
\end{verbatim}

\noindent
Before this code executes, two things happen: (a) another program loads a
value into W, and (b) a breakpoint is set at the address DONE. \\

\noindent
Then the run switch is hit and the program starts executing. Before the
computer stops due to the breakpoint, several interrupts occur and their
corresponding service routines are executed. Finally, the LC-3 stops due to
the breakpoint. We examine the memory shown, and R6, the supervisor stack
pointer. \\

\begin{center}
\begin{tabular}{c|c|}
\cline{2-2}
& { }{ }{ }{ }{ }Memory{ }{ }{ }{ }{ }   \\
  \cline{2-2}
 x2FF8& x0601 \\ \cline{2-2}
 x2FF9& x0601 \\ \cline{2-2}
 x2FFA& x0500 \\ \cline{2-2}
 x2FFB& x0504 \\ \cline{2-2}
 x2FFC& x0204 \\ \cline{2-2}
 x2FFD& x0201 \\ \cline{2-2}
 x2FFE& x8004 \\ \cline{2-2}
 x2FFF& x8002 \\ \cline{2-2}
 x3000& x8010 \\ \cline{2-2}
 x3001& x8012 \\ \cline{2-2}
 \end{tabular}
\end{center}

\begin{center}
\begin{tabular}{c|c|}
\cline{2-2}
  \cline{2-2}
R6 & x3000 \\ \cline{2-2}
 \end{tabular}
\end{center}

\noindent
What does the user program write to the monitor? How do you know that? \\

\item[9.50]STAR Your job in this problem will be to add the missing
instructions to a program that detects palindromes.  Recall a palindrome is a
string of characters that are identical when read from left to right or from
right to left. For example, racecar and 112282211. In this program, we will
have no spaces and no capital letters in our input string -- just a string of
lower case letters.\\

The program will make use of both a stack and a queue. The subroutines for
accessing the stack and queue are shown below.  Recall that elements are
PUSHed (added) and POPped (removed) from the stack. Elements are ENQUEUEd
(added) to the back of a queue, and DEQUEUEd (removed) from the front of
the queue.

\vspace{.2in}
\begin{figure}[h]
\begin{minipage}[c]{.6\textwidth}
\vspace{0pt}
\begin{alltt}
                 .ORIG x3050
         PUSH    ADD R6, R6, #-1
                 STR R0, R6, #0
                 RET
         POP     LDR R0, R6, #0
                 ADD R6, R6, #1
                 RET
         STACK   .BLKW #20
                 .END
\end{alltt}
\end{minipage}\hfill
\begin{minipage}[c]{.4\textwidth}
\begin{alltt}
        .ORIG x3080
ENQUEUE ADD R5, R5, #1
        STR R0, R5, #0  
        RET
DEQUEUE LDR R0, R4, #0
        ADD R4, R4, #1
        RET
QUEUE   .BLKW #20
        .END
\end{alltt}
\end{minipage}
\end{figure}

\vspace{.2in}
\noindent
The program is carried out in two phases. Phase 1 enables a user to input a
character string one keyboard character at a time. The character string is
terminated when the user types the enter key (line feed). In Phase 1, the
ASCII code of each character input is pushed on a stack, and its negative
value is inserted at the back of a queue. Inserting an element at the back
of a queue we call enqueuing. \\

In Phase 2, the characters on the stack and in the queue are examined by
removing them, one by one from their respective data structures (i.e., stack
and queue). If the string is a palindrome, the program stores a 1 in memory
location RESULT. If not, the program stores a zero in memory location RESULT.
The PUSH and POP routines for the stack as well as the ENQUEUE and DEQUEUE
routines for the queue are shown below. You may assume the user never inputs
more than 20 characters.

\begin{center}
\begin{minipage}[t]{0.5\linewidth}
\begin{flushleft}
{
\begin{alltt}
        .ORIG X3000
        LEA   R4, QUEUE
        LEA   R5, QUEUE
        ADD   R5, R5, #-1
        LEA   R6, ENQUEUE        ; Initialize SP
        LD    R1, ENTER
        AND   R3, R3, #0     
        ------------------ (a)
        TRAP x22
PHASE1  TRAP x20
		------------------ (b)
        BRz PHASE2
        JSR PUSH
        ------------------ (c)
        ------------------ (d)
        JSR ENQUEUE
        ADD R3, R3, #1
        BRnzp PHASE1
;
PHASE2  JSR POP
		------------------ (e)
        JSR DEQUEUE
        ADD R1, R0, R1   
        BRnp FALSE
        ------------------ (f)
        ------------------ (g)
        BRnzp PHASE2
;
TRUE    AND R0, R0, #0
        ADD R0, R0, #1
        ST R0, RESULT
        HALT
FALSE   AND R0, R0, #0
        ST R0, RESULT
        HALT
RESULT  .BLKW #1
ENTER   .FILL x-0A
PROMPT  .STRINGZ "Enter an input string: "
        .END
\end{alltt}
}
\end{flushleft}
\end{minipage}
\end{center}


\item[9.51]STAR Now that the keyboard interrupt is old stuff for you, it is time to introduce two interrupts for the LC-3: INTA and INTB.  The necessary hardware has been
added to allow them to happen.  INTA has priority 2 and an interrupt vector of
x50.  INTB has priority 4 and an interrupt vector of x60. \\

\noindent
Recall that the priority is specified in bits[10:8] of the PSR.  In fact,
the full PSR specification is:
\begin{center}
\begin{tabular}{rc|c|c|c|c|c|c|c|c|c|c|c|c|c|c|c|}
& \multicolumn{1}{c}{15} & \multicolumn{1}{c}{14} & \multicolumn{1}{c}{13} & \multicolumn{1}{c}{12} & \multicolumn{1}{c}{11} & \multicolumn{1}{c}{10} & \multicolumn{1}{c}{9} & \multicolumn{1}{c}{8} & \multicolumn{1}{c}{7} & \multicolumn{1}{c}{6} & \multicolumn{1}{c}{5} & \multicolumn{1}{c}{4} & \multicolumn{1}{c}{3} & \multicolumn{1}{c}{2} & \multicolumn{1}{c}{1} & \multicolumn{1}{c}{0} \\
 \cline{2-17}
PSR: & \multicolumn{1}{|c|}{Pr} & 0 & 0 & 0 & 0 & \multicolumn{3}{ c| }{Priority} & 0 & 0 & 0 & 0 & 0 & N & Z & P \\ 
\cline{2-17}

\end{tabular}
%\end{center}
%                 -----------------------------------------------
%       PSR:   |Pr| 0 0 0 0 | Priority | 0 0 0 0 0 | N | Z | P |
%                 -----------------------------------------------

\begin{tabular}{llll}
& \\
where & PSR[15] &=& 0 (supervisor mode), 1 (user mode). \\
      & PSR[14:11] &=& 0000 \\
      & PSR[10:8] &=& priority, 0 (lowest) to 7 (highest). \\
      & PSR[7:3] &=& 00000 \\
      & PSR[2:0] &=& condition codes for N,Z,P\\
\end{tabular}
\end{center}

\noindent
In this problem you are given the user program and the two interrupt service
routines. The user program starts executing at cycle 1 and runs at priority 0.\\

\begin{tabular}{p{5cm} p{5cm} p{5cm}}
User program:
\begin{alltt}
        .ORIG x3000
        AND R0,R0,#0
        ADD R0,R0,#5
        LD R1,COUNT
        NOT R0,R0
        ADD R0,R0,#1
AGAIN   ADD R2,R0,R1
        BRz DONE
        ADD R1,R1,#-1
        BRnzp AGAIN
DONE    TRAP x25
COUNT   .FILL x000F
        .END
\end{alltt}
&
INTA Service Routine:
\begin{alltt}
        .ORIG x1000
        AND R5,R4,#0
        ADD R5,R5,#-1
        LD R3,VAL
        ADD R3,R3,R5
        ST R3,VAL
        RTI
VAL     .BLKW 1
        .END
\end{alltt}
&
INTB Service Routine:
\begin{alltt}
        .ORIG x2000
        LDI R4,VAL2
        NOT R4,R4
        ADD R4,R4,#1
        STI R4,VAL2
        RTI
VAL2    .FILL xFE08
        .END
\end{alltt}

\end{tabular} \\
\noindent
Assume both interrupts are enabled.  Assume 22 cycles are needed to initiate an interrupt when you are in user mode; that is, from the time the test is taken until the interrupt service routine starts executing. Assume it takes 21 cycles if you are in privileged (supervisor) mode. You already know from problem 1 the number of cycles
individual instructions take. \\

\noindent
In order to support INTA and INTB, the interrupt vector table must have entries.  Show the addresses of these entries and the contents of those memory locations.
\begin{center}
\begin{tabular}{|p{3cm}|p{3cm}|}
\hline
 Memory address & Content \\ \hline \hline
 & \\
 & \\ \hline
 & \\
 & \\ \hline

\end{tabular}
\end{center}

\noindent
Suppose INTA requests service at cycle 30 and INTB requests service at cycle 68. In which cycle does each \textbf{service routine} start executing? \\

\noindent
The table below shows the contents of a section of memory (locations
x2FFA to x3002) before the user program starts executing.  Show the
contents of these locations and the contents of the Stack Pointer
in cycle 100. \\

\begin{center}
\begin{tabular}{c|c|c|}
\cline{2-3}
& { }{ }{ }{ }{ }Initial{ }{ }{ }{ }{ } & At the end of  \\
&         & cycle 100 \\
  \cline{2-3}
 x2FFA& x0001 &\\
 & &\\ \cline{2-3}
 x2FFB& x0010&\\
 & &\\ \cline{2-3}
 x2FFC& x0100&\\
 & &\\ \cline{2-3}
 x2FFD& x1000&\\
 & &\\ \cline{2-3}
 x2FFE& x1100&\\
 & &\\ \cline{2-3}
 x2FFF& x1110&\\
 & &\\ \cline{2-3}
 x3000& x5020&\\
 & &\\ \cline{2-3}
 x3001& x1025&\\
 & &\\ \cline{2-3}
 x3002& x2207&\\
 & &\\ \cline{2-3}
Stack& x3000 &\\
Pointer & &\\ \cline{2-3}
\end{tabular} 
\end{center}

\item[9.52]STAR Consider a two player game where the players must think quickly
each time it is their turn to make a move.  Each player has a total allotted
amount of time to make all his/her moves.  Two clocks display the remaining
time for each player.  While a player is thinking of his/her move, his clock
counts down.  If time runs out, the other player wins.  As soon as a player
makes his/her move, he hits a button, which serves to stop counting down his
clock and start counting down the other player's clock. \\

\noindent
The program on the next page implements this mechanism.  The main program keeps
track of the time remaining for each player by decrementing the proper counter
once per second while the player is thinking.  When a player's counter reaches
zero, a message is printed on the screen declaring the winner.  When a player
hits the button, an interrupt is taken.  The interrupt service routine takes
such action as to enable the main program (after returning from the interrupt)
to start decrementing the other counter. \\

\noindent
The interrupt vector for the button is x35.  The priority level of the button
is \#2.  Assume that the operating system has set the Interrupt Enable bit of the button to enable it to interrupt.  Assume the main program runs at priority \#1 and executes in user mode. \\

\noindent
In order for the interrupt service routine to be executed when
the button is pushed, what memory location must contain what value? \\

\vspace{.15in}
\noindent
Assume a player hits the button while the instruction at line 16
is being executed.  What two values (in hex) will be pushed on the stack? \\

\vspace{.15in}
\noindent
Fill in the missing instructions in the user program. \\

\noindent
This program has a bug that will only occur if an interrupt is taken at an inappropriate time. Write down the line number of an instruction such that if the button is pressed while that instruction is executing, unintended behavior will result.
\\

\noindent
How could we fix this bug?

\begin{alltt}
; Interrupt Service Routine
       .ORIG x1550
       NOT   R0, R0
       RTI
       .END

; User Program
       .ORIG x3000
       AND   R0, R0, #0      ; Line 1
       LD    R1, TIME        ; Line 2
       LD    R2, TIME        ; Line 3

NEXT   -------------------- (a)
	   -------------------- (b)
       BRn   P2_DEC          ; Line 6
       ADD   R1, R1, #-1     ; Line 7
       ------------------- (c)
       LEA   R0, P2WINS      ; Line 9
       BRnzp END             ; Line 10
P2_DEC ADD   R2, R2, #-1     ; Line 11
	   ------------------- (d)
       LEA   R0, P1WINS      ; Line 13
END    PUTS                  ; Line 14
       HALT                  ; Line 15
COUNT  LD    R3, SECOND      ; Line 16
LOOP   ADD   R3, R3, #-1     ; Line 17
       BRp   LOOP            ; Line 18
       ------------------- (e)
TIME   .FILL    #300
SECOND .FILL    #25000     ; 1 second
P1WINS .STRINGZ "Player 1 Wins."
P2WINS .STRINGZ "Player 2 Wins."
       .END
\end{alltt}

\item[9.53]STAR A program is running in privilege mode (PSR[1]] = 0). We set a breakpoint at location x2000. The operator immediately pushes the run button. What are the subsequent MAR/MDR values? 

\begin{table}[h]
\centering
\begin{tabular}{|@{\hspace{0.2in}}c@{\hspace{0.2in}}||@{\hspace{0.2in}}c@{\hspace{0.2in}}|}
\hline
{\bf MAR}  & {\bf MDR}   \\[0.2in] \hline  \hline
      & x8000 \\[0.2in] \hline
      & x1050 \\[0.2in] \hline
      & x0004 \\[0.2in] \hline
      & xBCAE \\[0.2in] \hline
      &       \\[0.2in] \hline
x2800 & x2C04 \\[0.2in] \hline
      &       \\[0.2in] \hline
x1052 & x3C4D \\[0.2in] \hline
      & x2C0A \\[0.2in] \hline 
\end{tabular}
\end{table}

\end{exercises}
\end{document}
