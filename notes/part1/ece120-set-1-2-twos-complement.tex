\classtitle

\subsection{The 2's Complement Representation}

This set of notes explains the rationale for using the 2's~complement
representation for signed integers and derives the representation 
based on equivalence of the addition function to that of addition
using the unsigned representation with the same number of bits.\\

\subsubsection{Review of Bits and the Unsigned Representation}

In modern digital systems, we represent all types of information
using binary digits, or {\bf bits}\label{one:tcr:bits}.  Logically, a bit is either~0 or~1.
Physically, a bit may be a voltage, a magnetic field, or even the
electrical resistance of a tiny sliver of glass.
%
Any type of information can be represented with an ordered set of
bits, provided that 
{\em any given pattern of bits corresponds to only
one value} and that {\em we agree in advance on which pattern of bits
represents which value}.

For unsigned integers---that is, whole numbers greater or equal to
zero---we chose to use the base~2 representation already familiar to
us from mathematics.  We call this representation the {\bf unsigned
representation}\label{one:tcr:unsigned}.  For example, in a \mbox{4-bit} unsigned representation,
we write the number~0 as~0000, the number~5 as~0101, and the number~12
as~1100.  Note that we always write the same number of bits for any
pattern in the representation: {\em in a digital system, there is no 
``blank'' bit value}.\\

\subsubsection{Picking a Good Representation}

In class, we discussed the question of what makes one representation
better than another.  The value of the unsigned representation, for
example, is in part our existing familiarity with the base~2 analogues
of arithmetic.  For base~2 arithmetic, we can use nearly identical
techniques to those that we learned in elementary school for adding, 
subtracting, multiplying, and dividing base~10 numbers.

Reasoning about the relative merits of representations from a
practical engineering perspective is (probably) currently beyond 
your ability.
%
Saving energy, making the implementation simple, and allowing the
implementation to execute quickly probably all sound attractive, but 
a quantitative comparison between two representations on any of these
bases requires knowledge that you will acquire in the
next few years.

We can sidestep such questions, however, by realizing that if a
digital system has hardware to perform operations such as addition
on unsigned values, using the same piece of hardware to operate
on other representations incurs little or no additional cost.
In this set of notes, we discuss the 2's~complement representation,
which allows reuse of the unsigned add unit (as well as a basis for
performing subtraction of either representation using an add unit!).
In discussion section and in your homework, you will 
use the same idea to perform operations on other representations, such as
changing an upper case letter in ASCII to a lower case one, or converting
from an ASCII digit to an unsigned representation of the same number.  \\

\subsubsection{The Unsigned Add Unit}

In order to define a representation for signed integers that allows
us to reuse a piece of hardware designed for unsigned integers, we
must first understand what such a piece of hardware actually does (we
do not need to know how it works yet---we'll explore that question 
later in our class).

The unsigned representation using~\mbox{$N$} bits is not closed
under addition.  In other words, for any value of~$N$, we can easily
find two~\mbox{$N$-bit} unsigned numbers that, when added together,
cannot be represented as an~\mbox{$N$-bit} unsigned number.  With~$N=4$, 
for example, we can add~$12$~(1100) and~$6$~(0110) to obtain~18.
Since~18 is outside of the range~$[0,2^4-1]$ representable using
the~\mbox{4-bit} unsigned representation, our representation breaks
if we try to represent the sum using this representation.  We call
this failure an {\bf overflow}\label{one:tcr:overflow} condition: the representation cannot
represent the result of the operation, in this case addition.

\begin{minipage}{5.4in}
Using more bits to represent the answer is not an attractive solution, 
since we might then want to use more bits for the inputs, which in turn
requires more bits for the outputs, and so on.  We cannot build 
something supporting an infinite number of bits.  Instead, we 
choose a value for~$N$ and build an add unit that adds two~\mbox{$N$-bit}
numbers and produces an~\mbox{$N$-bit} sum (and some overflow 
indicators, which we discuss in the next set of notes).  The diagram
to the right shows how we might draw such a device, with two~\mbox{$N$-bit}
numbers entering at from the top, and the~\mbox{$N$-bit} sum coming out
from the bottom.
\end{minipage}\hspace{.25in}%
\begin{minipage}{0.85in}
\epsfig{file=part1/figs/addunit.eps,width=0.85in}
\end{minipage}

\begin{minipage}{5.55in}
The function used for \mbox{$N$-bit} unsigned addition is addition 
modulo~$2^N$.  In a practical sense, you can think of this function
as simply keeping the last~$N$ bits of the answer; other bits 
are simply discarded.  In the example to the right,
we add~12 and~6 to obtain~18, but then discard the extra bit on the
left, so the add unit produces~2 (an overflow).
\end{minipage}\hspace{.25in}%
\begin{minipage}{0.7in}
\epsfig{file=part1/figs/adding.eps,width=0.7in}
\end{minipage}

\begin{minipage}{3.75in}
{\bf Modular arithmetic} defines a way of performing arithmetic for
a finite number of possible values, usually integers.  
As a concrete example, let's use modulo~16, which corresponds to
the addition unit for our~\mbox{4-bit} examples.\mpline

Starting with the full range of integers, we break the number
line into contiguous groups of 16 integers, as shown to the right.\linebreak
\end{minipage}\hspace{.25in}%
\begin{minipage}{2.5in}
\epsfig{file=part1/figs/modulo.eps,width=2.5in}
\end{minipage}\mpdone

The numbers~0 to~15 form one group.  The numbers~-16 to~-1 form a
second group, and the numbers from~16 to~31 form a third group. 
An infinite number of groups are defined in this manner.

We then define 16~{\bf equivalence classes} consisting of the first numbers
from all groups, the second numbers from all groups, and so forth.
For example, the numbers $\ldots, -32, -16, 0, 16, 32, \ldots$ form
one such equivalence class.
%
Mathematically, we say that two numbers~$A$ and~$B$ are equivalent modulo~16,
which we write as
%
\begin{eqnarray*}
(A &=& B)~\mbox{mod}~16, \protect{\hspace{0.25in}}\mbox{or sometimes as}\\
A &\equiv& B~\mbox{(mod 16)}
\end{eqnarray*}
%
if and only if $A=B+16k$ for some integer~$k$.

Equivalence as defined by a particular modulus
distributes over addition and multiplication.  If, for example,
we want to find the equivalence class for~$(A + B)~\mbox{mod}~16$,
we can find the equivalence classes for~$A$ (call it~$C$) and~$B$ 
(call it~$D$) and then calculate the equivalence class 
of~$(C + D)~\mbox{mod}~16$.
As a concrete example of distribution over multiplication, 
given~$(A = 1,083,102,112 \times 7,323,127)~\mbox{mod}~10$,
find~$A$.
%
For this problem, we note that the first number is equivalent 
to~$2~\mbox{mod}~10$, while the second number is equivalent 
to~$7~\mbox{mod}~10$.  We then write~$(A = 2 \times 7)~\mbox{mod}~10$,
and, since $2 \times 7 = 14$, we have~$(A = 4)~\mbox{mod}~10$.
\\

\subsubsection{Deriving 2's Complement}

\begin{minipage}{2.6in}
Given these equivalence classes, we might instead choose to draw a circle
to identify the equivalence classes and to associate each class with one
of the sixteen possible \mbox{4-bit} patterns, as shown to the right.
Using this circle representation, we can add by counting clockwise around
the circle, and we can subtract by counting in a counterclockwise direction
around the circle.  With an unsigned representation, we choose to use the
group from $[0,15]$ (the middle group in the diagram markings to the right)
as the number represented by each of the patterns.  Overflow occurs
with unsigned addition (or subtraction) because we can only choose one
value for each binary pattern.
\end{minipage}\hspace{.25in}%
\begin{minipage}{3.65in}
\epsfig{file=part1/figs/circle.eps,width=3.65in}
\end{minipage}

In fact, we can choose any single value for each pattern to create a 
representation, and our add unit will always produce results that
are correct modulo~16.  Look back at our overflow example, where
we added~12 and~6 to obtain~2, and notice that $(2=18)~\mbox{mod}~16$.
Normally, only a contiguous sequence of integers makes a useful
representation, but we do not have to restrict ourselves to 
non-negative numbers.

The 2's complement representation can then be defined by choosing a 
set of integers balanced around zero from the groups.  In the circle 
diagram, for example, we might choose to represent numbers
in the range $[-7,7]$ when using 4~bits.  What about the last pattern, 1000?
We could choose to represent either~-8 or~8.  The number of arithmetic
operations that overflow is the same with both choices (the choices
are symmetric around 0, as are the combinations of input operands that 
overflow), so we gain nothing in that sense from either choice.
If we choose to represent~-8, however, notice that all patterns starting
with a~1~bit then represent negative numbers.  No such simple check
arises with the opposite choice, and thus an~\mbox{$N$-bit}~2's complement 
representation is defined to represent the range~$[-2^{N-1},2^{N-1}-1]$,
with patterns chosen as shown in the circle.\\

\subsubsection{An Algebraic Approach}

Some people prefer an algebraic approach to understanding the
definition of~2's complement, so we present such an approach next.
Let's start by writing $f(A,B)$ for the result of our add unit:
%
\begin{eqnarray*}
f(A,B) = (A + B)~\mbox{mod}~2^N
\end{eqnarray*}

We assume that we want to represent a set of integers balanced around~0
using our signed representation, and that we will use the same binary
patterns as we do with an unsigned representation to represent
non-negative numbers.  Thus, with an~\mbox{$N$-bit} representation,
the patterns in the range~$[0,2^{N-1}-1]$ are the same as those
used with an unsigned representation.  In this case, we are left with
all patterns beginning with a~1~bit.

The question then is this: given an integer~$k$, $2^{N-1}>k>0$, for which we 
want to find a pattern to represent~$-k$, and any integer~$m\geq{0}$
that we might want to add to~$-k$, 
can we find another integer~$p>0$
such that 
%
\begin{eqnarray}
(-k + m = p + m)~\mbox{mod}~2^N~~~?\label{eqn:basealg}
\end{eqnarray}

If we can, we can use $p$'s representation to represent~$-k$ and our
unsigned addition unit~$f(A,B)$ will work correctly.

To find the value~$p$, start by subtracting~$m$ from both sides of
Equation~(\ref{eqn:basealg}) to obtain:
%
\begin{eqnarray}
(-k = p)~\mbox{mod}~2^N\label{eqn:stepone}
\end{eqnarray}

Note that $(2^N=0)~\mbox{mod}~2^N$, and add this equation to 
Equation~(\ref{eqn:stepone}) to obtain
%
\begin{eqnarray*}
(2^N-k = p)~\mbox{mod}~2^N
\end{eqnarray*}

Let~$p=2^N-k$.  
%
For example, if~$N=4$,~$k=3$ gives~$p=16-3=13$, which is the pattern~$1101$.
With~$N=4$ and~$k=5$, we obtain~$p=16-5=11$, which is the pattern~$1011$.
In general, since
$2^{N-1}>k>0$, 
we have $2^{N-1}<p<2^N$.  But these patterns are all unused---they all
start with a~1 bit!---so the patterns that we have defined for negative
numbers are disjoint from those that we used for positive numbers, and
the meaning of each pattern is unambiguous.
%
The algebraic definition of bit patterns for negative numbers
also matches our circle diagram from the last
section exactly, of course.\\

\pagebreak

\subsubsection{Negating 2's Complement Numbers}

The algebraic approach makes understanding negation of an integer
represented using~2's complement fairly straightforward, and gives 
us an easy procedure for doing so.
Recall that given an integer~$k$ in an~\mbox{$N$-bit}~2's complement
representation, the \mbox{$N$-bit} pattern for~$-k$ is given by~$2^N-k$ 
(also true for~$k=0$ if we keep only the low~$N$ bits of the result).  
But $2^N=(2^N-1)+1$.  Note that $2^N-1$ is the pattern of
all~1~bits.  Subtracting any value~$k$ from this value is equivalent
to simply flipping the bits, changing~0s to~1s and~1s to~0s.
(This operation is called a~{\bf 1's~complement}, by the way.)
We then add~1 to the result to find the pattern for~$-k$.

Negation can overflow, of course.  Try finding the negative pattern for~-8 
in~\mbox{4-bit}~2's~complement.

Finally, be aware that people often overload the term~2's complement
and use it to refer to the operation of negation in a~2's complement
representation.  In our class, we try avoid this confusion: 2's complement
is a representation for signed integers, and negation is an operation
that one can apply to a signed integer (whether the representation used
for the integer is 2's complement or some other representation for signed
integers).

\vfill

\pagebreak

