\classtitle

\subsection{Finite State Machine Design Examples, Part I}

This set of notes uses a series of examples to illustrate design principles 
for the implementation of finite state machines (FSMs) using digital logic.
We begin with an overview of the design process for a digital FSM, from
the development of an abstract model through the implementation of
functions for the next-state variables and output signals.
Our first few examples cover only the concrete aspects:
we implement several counters, which illustrate the basic 
process of translating a concrete and complete state transition diagram
into an implementation based on flip-flops and logic gates.
We next consider a counter with a number of states that is not a power of
two, with which we illustrate the need for FSM initialization.
% As part of solving the initialization problem, we also introduce 
% a general form of selection logic called a multiplexer.

We then consider the design process as a whole through a more general
example of a counter with multiple inputs to control its behavior. 
We work from
an abstract model down to an implementation, illustrating how semantic
knowledge from the abstract model can be used to simplify the 
implementation.  Finally, we illustrate how the choice of representation
for the FSM's internal state affects the complexity of the implementation.
Fortunately, designs that are more intuitive and easier for humans to
understand also typically make the best designs in terms of 
other metrics, such as logic complexity.\\


\subsubsection{Steps in the Design Process}

Before we begin exploring designs, let's talk briefly about the general
approach that we take when designing an FSM.  We follow a six-step
process:\vspace{-8pt}
%
\begin{enumerate}{\setlength{\itemsep}{0pt}\setlength{\parskip}{0pt}%
\setlength{\topsep}{0pt}\setlength{\partopsep}{0pt}\setlength{\parsep}{0pt}
\item{develop an abstract model}\label{step-abs}
\item{specify I/O behavior}\label{step-io}
\item{complete the specification}\label{step-complete}
\item{choose a state representation}\label{step-repn}
\item{calculate logic expressions}\label{step-logic}
\item{implement with flip-flops and gates}\label{step-gates}
}
\end{enumerate}\vspace{-8pt}
%
In Step~\ref{step-abs}, we translate our description in human language
into a model with states and desired behavior.  At this stage, we 
simply try to capture the intent of the description and are not
particularly thorough nor exact.

Step~\ref{step-io} begins to formalize the model, starting with its
input and output behavior.  If we eventually plan to develop an
implementation of our FSM as a digital system (which is not the 
only choice, of course!), all input and output
must consist of bits.  Often, input and/or output specifications
may need to match other digital systems to which we plan to connect
our FSM.  In fact, {\em most problems in developing large digital systems
today arise because of incompatibilities when composing two or more
separately designed pieces} (or {\bf modules}) into an integrated system.

Once we know the I/O behavior for our FSM, in Step~\ref{step-complete}
we start to make
any implicit assumptions clear and to make any other decisions
necessary to the design.  Occasionally, we may choose to leave
something undecided in the hope of simplifying the design with
``don't care'' entries in the logic formulation.

In Step~\ref{step-repn}, we select an internal representation
for the bits necessary to encode the state of our FSM.  In practice,
for small designs, this representation can be selected by a computer 
in such a way as to optimize the implementation.  However, for large
designs, such as the LC-3 instruction set architecture that we
study later in this class, humans do most of the work by hand.
%
In the later examples in this set of notes, we show how even a 
small design can
leverage meaningful information from the design when selecting
the representation, leading to an implementation that is simpler
and is easier to build correctly.
%
We also show how one can
use abstraction to simplify an implementation.

By Step~\ref{step-logic}, our design is a complete specification in
terms of bits, and we need merely derive logic expressions for the
next-state variables and the output signals.  This process is no
different than for combinational logic, and should already be fairly 
familiar to you.

\pagebreak

Finally, in Step~\ref{step-gates}, we translate our logic expressions
into gates and use flip-flops (or registers) to hold the internal
state bits of the FSM.  In later notes, we use more complex
building blocks when implementing an FSM, building up abstractions
in order to simplify the design process in much the same way that
we have shown for combinational logic.\\


\subsubsection{Example: A Two-Bit Gray Code Counter}

Let's begin with a two-bit Gray code counter with no inputs.
As we mentioned in Notes~Set~2.1, a Gray code is a cycle over all
bit patterns of a certain length in which consecutive patterns differ
in exactly one bit.
%
For simplicity, our first few examples are based on counters and
use the internal state
of the FSM as the output values.  You should already know
how to design combinational logic for the outputs if it were necessary.
%
The inputs to a counter, if any, are typically limited to functions
such as starting and stopping the counter, controlling the counting 
direction, and resetting the counter to a particular state.

A fully-specified transition diagram for 
a two-bit Gray code counter appears below.
With no inputs, the states simply form a loop, with
the counter moving from one state to the next each cycle.
%
Each state in the diagram is marked with the internal state value~$S_1S_0$ 
(before the ``/'') and the output $Z_1Z_0$ (after the ``/''), which are 
always equal for this counter.
%
Based on the transition diagram, we can fill in the K-maps for the 
next-state values $S_1^+$ and $S_0^+$ as shown to the right of the
transition diagram, then 
derive algebraic expressions in the usual way to obtain
$S_1^+=S_0$ and $S_0^+=\overline{\mbox{$S_1$}}$.
%
We then use the next-state logic to develop the implementation
shown on the far right, completing our first counter design.

\hspace{0.25in}\begin{minipage}{3in}
\epsfig{file=part3/figs/gctwo.eps,width=3in}
\end{minipage}\hspace{.5in}%
\begin{minipage}{0.625in}
\epsfig{file=part3/figs/gc2S1.eps,width=0.625in}\vspace{20pt}

\epsfig{file=part3/figs/gc2S0.eps,width=0.625in}
\end{minipage}\hspace{.325in}\begin{minipage}{1.6in}
\epsfig{file=part3/figs/gc2impl.eps,width=1.6in}
\end{minipage}\vspace{12pt}

\subsubsection{Example: A Three-Bit Gray Code Counter}

\begin{minipage}{3.25in}
Now we'll add a third bit to our counter, but again use a Gray code
as the basis for the state sequence.
%
A fully-specified transition diagram for such a counter appears to 
the right.  As before, with no inputs, the states simply form a loop, 
with the counter moving from one state to the next each cycle.
%
Each state in the diagram is marked with the internal state value~$S_2S_1S_0$ 
(before ``/'') and the output $Z_2Z_1Z_0$ (after ``/''). 
\end{minipage}\hspace{.25in}%
\begin{minipage}{3in}
\epsfig{file=part3/figs/gcthree.eps,width=3in}
\end{minipage}

\begin{minipage}{2.75in}
Based on the transition diagram, we can fill in the K-maps for the 
next-state values $S_2^+$, $S_1^+$, and $S_0^+$ as shown to the right, then 
derive algebraic expressions.  The results are more complex this 
time.
\end{minipage}\hspace{.25in}%
\begin{minipage}{1.00in}
\epsfig{file=part3/figs/gc3S2.eps,width=1.00in}
\end{minipage}\hspace{.25in}\begin{minipage}{1.00in}
\epsfig{file=part3/figs/gc3S1.eps,width=1.00in}
\end{minipage}\hspace{.25in}\begin{minipage}{1.00in}
\epsfig{file=part3/figs/gc3S0.eps,width=1.00in}
\end{minipage}

For our next-state logic, we obtain:
\begin{eqnarray*}
S_2^+ &=& S_2~S_0 + S_1~\overline{\mbox{$S_0$}} \\
S_1^+ &=& \overline{\mbox{$S_2$}}~S_0 + S_1~\overline{\mbox{$S_0$}} \\
S_0^+ &=& \overline{\mbox{$S_2$}}~\overline{\mbox{$S_1$}} + S_2~S_1
\end{eqnarray*}

\begin{minipage}{3.25in}
Notice that the equations for~$S_2^+$ and~$S_1^+$ share a common term,
$S_1\overline{\mbox{$S_0$}}$.
%
This design does not allow much choice in developing good equations for
the next-state logic, but some designs may enable you to reduce 
the design complexity by explicitly identifying and making use of 
common algebraic terms and sub-expressions for different outputs.
In modern design processes, identifying such opportunities is generally
performed by a computer program, but it's important to understand
how they arise.  Note that the common term becomes a single AND gate
in the implementation of our counter, as shown to the right.\mpline

Looking at the counter's implementation diagram, notice that the vertical
lines carrying the current state values and their inverses back to the
next state
logic inputs have been carefully ordered to simplify
understanding the diagram.  In particular, they are ordered from
left to right (on the left side of the figure) as 
$\overline{\mbox{$S_0$}}S_0\overline{\mbox{$S_1$}}S_1\overline{\mbox{$S_2$}}S_2$.
When designing any logic diagram, be sure to make use of a reasonable
order so as to make it easy for someone (including yourself!) to read 
and check the correctness of the logic.
\end{minipage}\hspace{.25in}%
\begin{minipage}{3in}
\epsfig{file=part3/figs/gc3impl.eps,width=3in}\vspace{12pt}
\end{minipage}\vspace{12pt}


\subsubsection{Example: A Color Sequencer}

\begin{minipage}{4.75in}
Early graphics systems used a three-bit red-green-blue~(RGB) 
encoding for colors.  The color mapping for such a system is shown to
the right.\mpline

Imagine that you are charged with creating a counter to drive a light
through a sequence of colors.  The light takes an RGB input as just
described, and the desired pattern is\mpline

\centerline{off (black)~~~~~yellow~~~~~violet~~~~~green~~~~~blue}\vspace{6pt}

You immediately recognize that you merely need a counter with five
states.  How many flip-flops will we need?  At least three, since
$\lceil\log_2~(5)\rceil=3$.  Given that we need three flip-flops, 
and that the colors we need to produce as\linebreak
\end{minipage}\hspace{.25in}%
\begin{minipage}{1.5in}
\begin{tabular}{c|l}
$RGB$& color\\ \hline
000& black\\
001& blue\\
010& green\\
011& cyan\\
100& red\\
101& violet\\
110& yellow\\
111& white
\end{tabular}\vspace{12pt}
\end{minipage}\mpdone

outputs are all unique
bit patterns, we can again choose to use the counter's internal 
state directly as our output values.

\begin{minipage}{2.6in}
A fully-specified transition diagram for our color sequencer
appears to the right.  The states again form a loop,
and are marked with the internal state value~$S_2S_1S_0$ 
and the output $RGB$.
\end{minipage}\hspace{.25in}%
\begin{minipage}{3.65in}
\epsfig{file=part3/figs/colors.eps,width=3.65in}
\end{minipage}

\begin{minipage}{2.6in}
As before, we can use the transition diagram to fill in K-maps for the 
next-state values $S_2^+$, $S_1^+$, and $S_0^+$, as shown to the right.
For each of the three states not included in our transition diagram,
we have inserted x's\linebreak
\end{minipage}\hspace{.3in}%
\begin{minipage}{1.00in}
\epsfig{file=part3/figs/colS2.eps,width=1.00in}\vspace{8pt}
\end{minipage}\hspace{.25in}\begin{minipage}{1.00in}
\epsfig{file=part3/figs/colS1.eps,width=1.00in}\vspace{8pt}
\end{minipage}\hspace{.25in}\begin{minipage}{1.00in}
\epsfig{file=part3/figs/colS0.eps,width=1.00in}\vspace{8pt}
\end{minipage}\mpdone

into the K-maps to indicate ``don't care.'' 
As you know, we can treat each~x as either a~0 or a~1, whichever
produces better results (where ``better'' usually means simpler 
equations).  The terms that we have chosen for our algebraic 
equations are illustrated in the K-maps.  The x's within the ellipses
become 1s in the implementation, and the x's outside of the ellipses
become 0s.

\begin{minipage}{3.25in}
For our next-state logic, we obtain:
\begin{eqnarray*}
S_2^+ &=& S_2~S_1 + \overline{\mbox{$S_1$}}~\overline{\mbox{$S_0$}} \\
S_1^+ &=& S_2~S_0 + \overline{\mbox{$S_1$}}~\overline{\mbox{$S_0$}} \\
S_0^+ &=& S_1
\end{eqnarray*}

Again our equations for~$S_2^+$ and~$S_1^+$ share a common term,
which becomes a single AND gate in the implementation shown to the
right.\vspace{100pt}
\end{minipage}\hspace{.25in}%
\begin{minipage}{3in}
\epsfig{file=part3/figs/colimpl.eps,width=3in}
\end{minipage}

\subsubsection{Identifying an Initial State}

Let's say that you go the lab and build the implementation above, 
hook it up
to the light, and turn it on.  Does it work?  Sometimes.
Sometimes it works perfectly, but sometimes
the light glows cyan or red briefly first.
At other times, the light is an
unchanging white.

\begin{minipage}{2.75in}
What could be going wrong?\mpline

Let's try to understand.  We begin by deriving
K-maps for the implementation, as shown to the right.  In these
K-maps, each of the x's in our design has been replaced by either a~0
or a~1.  These entries are highlighted with green italics.
\end{minipage}\hspace{.25in}%
\begin{minipage}{1.00in}
\epsfig{file=part3/figs/colS2-bad.eps,width=1.00in}
\end{minipage}\hspace{.25in}\begin{minipage}{1.00in}
\epsfig{file=part3/figs/colS1-bad.eps,width=1.00in}
\end{minipage}\hspace{.25in}\begin{minipage}{1.00in}
\epsfig{file=part3/figs/colS0-bad.eps,width=1.00in}
\end{minipage}

Now let's imagine what might happen if somehow our FSM got into the
$S_2S_1S_0=111$~state.  In such a state, the light would appear white,
since $RGB=S_2S_1S_0=111$.
%
What happens in the next cycle?
%
Plugging into the equations or looking into the K-maps gives (of
course) the same answer: the next state is the
$S_2^+S_1^+S_0^+=111$~state.
In other words, the light stays white indefinitely!
%
As an exercise, you should check what happens 
if the light is red or cyan.

We can extend the transition diagram that we developed for our design
with the extra states possible in the implementation, as shown below.
As with the five states in the design, the extra states are named with
the color of light that they produce.\\

\centerline{\epsfig{file=part3/figs/colors-full.eps,width=5.8in}}

Notice that the FSM does not move out of the~WHITE state (ever).  
%
You may at this point wonder whether more careful decisions 
in selecting our next-state expressions might address this issue.
To some extent, yes.  For example, if we replace the 
$S_2S_1$ term in the equation for $S_2^+$ with $S_2\overline{\mbox{$S_0$}}$, 
a decision allowed
by the ``don't care'' boxes in the K-map for our design,
the resulting transition diagram does not suffer from the problem
that we've found.
%
However, even if we do change our implementation slightly, we need
to address another aspect of the problem:
%
how can the FSM ever get into the unexpected states?

\begin{minipage}{2.25in}
What is the initial state of the three flip-flops in our implementation?
%
{\em The initial state may not even be~0s and~1s unless we have an 
explicit mechanism for initialization.} 
%
Initialization can work in two ways.  
%
The first approach makes use of the flip-flop design.
As you know, a flip-flop is built from a pair of latches, and
we can 
make use of the internal reset lines on these latches
to force each flip-flop into the 0~state (or the 1~state) using an
additional input. \mpline

Alternatively, we can add some extra logic to our design.
%
Consider adding a few AND gates and a $\overline{RESET}$ input
(active low), as shown in the dashed box in the figure to the right.
In this case, when we assert $\overline{RESET}$ by setting it to~0,
the FSM moves to state~000 in the next cycle, putting it into
the BLACK~state.  The approach taken here is for clarity; one can
optimize the design, if desired.  For example, we could simply connect
$\overline{RESET}$ as an extra input into the three AND gates on the
left rather than adding new ones, with the same effect.\mpline

We may sometimes want a more powerful initialization mechanism---one
that allows us to force the FSM into any specific state in the next
cycle.  In such a case, we can add multiplexers to each of our 
flip-flop inputs, allowing us to use the~$INIT$ input to choose between
normal operation~($INIT=0$) of the FSM and forcing the FSM into the
next state given by~$I_2I_1I_0$ (when $INIT=1$).
\end{minipage}\hspace{0.25in}%
\begin{minipage}{4in}
\epsfig{file=part3/figs/colinit.eps,width=4in}\vspace{12pt}
\centerline{\epsfig{file=part3/figs/colmux.eps,width=3.6in}}
\end{minipage}\vspace{8pt}



\subsubsection{Developing an Abstract Model}

\begin{minipage}{3.25in}
We are now ready to discuss the design process for an FSM from start
to finish.
%
For this first abstract FSM example, we build upon something
that we have already seen: a two-bit Gray code counter.
We now want a counter that allows us to start and stop the\linebreak
\end{minipage}\hspace{.25in}%
\begin{minipage}{3in}
\begin{tabular}{c|ccc}
state&    no input&  halt button& go button\\ \hline
counting& counting&      halted& \\
halted&   halted&              & counting
\end{tabular}
\end{minipage}\mpdone

count.
%
What is the mechanism for stopping and starting?  To
begin our design, we could sketch out an abstract next-state
table such as the one shown to the right above.  In this form of the table,
the first column lists the states, while each of the other columns lists
states to which the FSM transitions after a clock cycle for a particular
input combination. 
%
The table contains two states, counting and halted, and specifies
that the design uses two distinct buttons to move between the
states.
%The table further implies that if the counter is halted,
%the ``halt'' button has no additional effect, and if the counter
%is counting, the ``go'' button has no additional effect.

\begin{minipage}{2.9in}
A counter with a single counting state, of course, does not provide
much value.  We extend the table with four counting states and four
halted states, as shown to the right.  This version of the
table also introduces more formal state names, for which these notes 
use all capital letters.\mpline

The upper four states represent uninterrupted counting, in which 
the counter cycles through these states indefinitely.
%
A user can stop the counter in any state by pressing the ``halt''
button, causing the counter to retain its current value until the
user presses the ``go'' button.\mpline

Below the state table is an abstract transition diagram, which provides
exactly the same information in graphical form.  Here circles represent
states (as labeled) and arcs represent transitions from one state
to another based on an input combination (which is used to label the
arc).\mpline

We have already implicitly made a few choices about our counter design.
%
First, the counter\linebreak
\end{minipage}\hspace{.25in}%
\begin{minipage}{3.35in}
\begin{tabular}{c|ccc}
state&    no input&  halt button& go button\\ \hline
{\st COUNT~A}& {\st COUNT~B}& {\st HALT~A}& \\
{\st COUNT~B}& {\st COUNT~C}& {\st HALT~B}& \\
{\st COUNT~C}& {\st COUNT~D}& {\st HALT~C}& \\
{\st COUNT~D}& {\st COUNT~A}& {\st HALT~D}& \\
{\st HALT~A}&  {\st HALT~A}&              & {\st COUNT~B}\\
{\st HALT~B}&  {\st HALT~B}&              & {\st COUNT~C}\\
{\st HALT~C}&  {\st HALT~C}&              & {\st COUNT~D}\\
{\st HALT~D}&  {\st HALT~D}&              & {\st COUNT~A}
\end{tabular}\vspace{20pt}
\centerline{\epsfig{file=part3/figs/ssfsm.eps,width=3in}}\vspace{12pt}
\end{minipage}\mpdone

shown retains the current state of the system when
``halt'' is pressed.
We could instead reset the counter state whenever it
is restarted, in which case we need only five states: four for
counting and one more for a halted counter.
%
Second, we've designed the counter to stop
when the user presses ``halt'' and to resume counting 
when the user presses ``go.''  We could instead choose to delay these 
effects by a cycle.  For example, pressing ``halt'' in state~{\st COUNT~B}
could take the counter to state~{\st HALT~C}, and pressing ``go'' 
in state~{\st HALT~C} could take the system to state~{\st COUNT~C}.
%
In these notes, we implement only the diagrams shown.\\

\subsubsection{Specifying I/O Behavior}

\begin{minipage}{3.25in}
We next start to formalize our design by specifying its input and 
output behavior digitally.  Each of the two control buttons provides
a single bit of input.  The ``halt'' button we call~$H$, and the
``go'' button we call~$G$.
%
For the output, we use a two-bit 
Gray code.  With these choices, we can redraw the transition diagram 
as show to the right.\mpline

In this figure, the states are marked with output values~$Z_1Z_0$ and
transition arcs are labeled in terms of our two input buttons, $G$ and~$H$.  
The uninterrupted counting cycle is labeled with $\overline{H}$
to indicate that it continues until we press~$H$.
\end{minipage}\hspace{.25in}%
\begin{minipage}{3in}
\epsfig{file=part3/figs/ssfsmdigone.eps,width=3in}
\end{minipage}\\ \\

\subsubsection{Completing the Specification}

Now we need to think about how the system should behave if something 
outside of our initial expectations occurs.  Having drawn out a partial
transition diagram can help with this process, since we can use the
diagram to systematically consider all possible input conditions from
all possible states.  The state table form can make the missing
parts of the specification even more obvious.



\begin{minipage}{2.09in}
For our counter, the symmetry between counting states makes the problem 
substantially simpler.  Let's write out part of a list of states and
part of a state table with one 
counting state and one halt state, as shown to the right.
Four values of the inputs~$HG$ 
are possible (recall that $N$ bits allow $2^N$ possible patterns).
We list the columns in Gray code order, since we may want to
transcribe this table into K-maps later.
\end{minipage}\hspace{.25in}%
\begin{minipage}{4.16in}
\centerline{
\begin{tabular}{rcl}
\multicolumn{2}{c}{state}&\multicolumn{1}{c}{description}\\ \hline
first counting state& {\st COUNT~A}& counting, output $Z_1Z_0=00$\\
  first halted state&  {\st HALT~A}& halted, output $Z_1Z_0=00$
\end{tabular}}\vspace{24pt}

\begin{tabular}{c|cccc}
&\multicolumn{4}{c}{$HG$}\\
        state&            00&            01&          11&           10\\ \hline
{\st COUNT~A}& {\st COUNT~B}&   unspecified& unspecified& {\st HALT~A}\\
 {\st HALT~A}&  {\st HALT~A}& {\st COUNT~B}& unspecified&  unspecified
\end{tabular}\vspace{12pt}
\end{minipage}

Let's start with the~{\st COUNT~A} state.  
%
We know that if neither button is pressed ($HG=00$), we want 
the counter to move to the~{\st COUNT~B} state.  And, if we press the
``halt'' button ($HG=10$), we want the counter to move to the~{\st HALT~A}
state.  What should happen if a user presses the ``go'' button ($HG=01$)?
Or if the user presses both buttons ($HG=11$)?
%
Answering these questions is part of fully specifying our design.  We
can choose to leave some parts unspecified, but {\em any implementation of
our system will imply answers}, and thus we must be careful.
%
We choose to ignore the ``go'' button while counting, and to have the
``halt'' button override the ``go'' button.  Thus, if $HG=01$ when the
counter is in state~{\st COUNT~A}, the counter moves to state~{\st COUNT~B}.
And, if $HG=11$, the counter moves to state~{\st HALT~A}.

Use of explicit bit patterns for the inputs~$HG$ may help you to check 
that all four possible input values are covered from each state.  If 
you choose to use a transition diagram instead of a state table,
you might even want to add four arcs from each state, each labeled 
with a specific
value of~$HG$.  When two arcs connect the same two states, we can either 
use multiple labels or can indicate bits that do not matter using a
{\bf don't-care} symbol, $x$.  For example, the arc from state~{\st COUNT~A}
to state~{\st COUNT~B} could be labeled~$HG=00,01$ or $HG=0x$.  The
arc from state~{\st COUNT~A} to state~{\st HALT~A} could be labeled
$HG=10,11$ or $HG=1x$.  We can also use logical expressions as labels,
but such notation can obscure unspecified transitions.

Now consider the state~{\st HALT~A}.  The transitions specified so far
are that when we press ``go'' ($HG=01$), the counter moves to 
the~{\st COUNT~B} state, and that the counter remains halted in 
state~{\st HALT~A} if no buttons are pressed ($HG=00$).
What if the ``halt'' button is pressed ($HG=10$), or
both buttons are pressed ($HG=11$)?  For consistency, we decide that
``halt'' overrides ``go,'' but does nothing special if it alone is pressed
while the counter is halted.  Thus, input patterns $HG=10$ and $HG=11$ also 
take state~{\st HALT~A} back to itself.
Here the arc could be labeled $HG=00,10,11$ or, equivalently,
$HG=00,1x$ or $HG=x0,11$.

\begin{minipage}{3.25in}
To complete our design, we apply the same decisions that we made for 
the~{\st COUNT~A} state to all of the other counting states, and the 
decisions that we made for the~{\st HALT~A} state to all of the other 
halted states.  If we had chosen not to specify an answer, an implementation
could produce different behavior from the different counting
and/or halted states, which might confuse a user.\mpline

The resulting design appears to the right.
\end{minipage}\hspace{.25in}%
\begin{minipage}{3in}
\epsfig{file=part3/figs/ssfsmdigfull.eps,width=3in}
\end{minipage}\\ 


\subsubsection{Choosing a State Representation}

Now we need to select a representation for the states.  Since our counter
has eight states, we need at least three ($\lceil\log_2~(8)\rceil=3$)
state bits $S_2S_1S_0$ to keep track of the current state.
%
As we show later, {\em the choice of representation for an FSM's states
can dramatically affect the design complexity}.  For a design as simple as 
our counter, you could just let a computer implement all possible 
representations (there aren't more than~840, if we consider simple 
symmetries) and select one according to whatever metrics are interesting.
%
For bigger designs, however, the number of possibilities quickly becomes
impossible to explore completely.

Fortunately, {\em use of abstraction in selecting a representation 
also tends to produce better designs} for a wide variety of metrics
(such as design complexity, area, power consumption, and performance).
%
The right strategy is thus often to start by selecting a representation 
that makes sense to a human, even if it requires more bits than are
strictly necessary.  The
resulting implementation will be easier to
design and to debug than an implementation in which only the global 
behavior has any meaning.

\begin{minipage}{3.25in}
Let's return to our specific example, the counter.  We can use one bit, 
$S_2$, to record whether or not our counter is counting~($S_2=0$) or
halted~($S_2=1$).  The other two bits can then record the counter state
in terms of the desired output.  Choosing this representation
implies that only wires will be necessary to compute outputs~$Z_1$ 
and~$Z_0$ from the internal state: $Z_1=S_1$ and $Z_0=S_0$.  The resulting
design, in which states are now labeled with both internal state and
outputs ($S_2S_1S_0/Z_1Z_0$) appears to the right.  In this version,
we have changed the arc labeling to use logical expressions, which
can sometimes help us to think about the implementation.
\end{minipage}\hspace{.25in}%
\begin{minipage}{3in}
\epsfig{file=part3/figs/ssfsmdiggood.eps,width=3in}
\end{minipage}

The equivalent state listing and state table appear below.  We have ordered
the rows of the state table in Gray code order to simplify transcription
of K-maps.

\begin{tabular}{rcl}
\\
\multicolumn{1}{c}{state}& $S_2S_1S_0$&\multicolumn{1}{c}{description}\\ \hline
{\st COUNT~A}& 000& counting, output $Z_1Z_0=00$\\
{\st COUNT~B}& 001& counting, output $Z_1Z_0=01$\\
{\st COUNT~C}& 011& counting, output $Z_1Z_0=11$\\
{\st COUNT~D}& 010& counting, output $Z_1Z_0=10$\\
 {\st HALT~A}& 100& halted, output $Z_1Z_0=00$\\
 {\st HALT~B}& 101& halted, output $Z_1Z_0=01$\\
 {\st HALT~C}& 111& halted, output $Z_1Z_0=11$\\
 {\st HALT~D}& 110& halted, output $Z_1Z_0=10$
\end{tabular}
\hfill
\begin{tabular}{rc|cccc}
&&\multicolumn{4}{c}{$HG$}\\
\multicolumn{1}{c}{state}&$S_2S_1S_0$& 00& 01& 11& 10\\ \hline
{\st COUNT~A}&000& 001& 001& 100& 100\\
{\st COUNT~B}&001& 011& 011& 101& 101\\
{\st COUNT~C}&011& 010& 010& 111& 111\\
{\st COUNT~D}&010& 000& 000& 110& 110\\
 {\st HALT~D}&110& 110& 000& 110& 110\\
 {\st HALT~C}&111& 111& 010& 111& 111\\
 {\st HALT~B}&101& 101& 011& 101& 101\\
 {\st HALT~A}&100& 100& 001& 100& 100
\end{tabular}

\begin{minipage}{2.09in}
Having chosen a representation, we can go ahead and implement our
design in the usual way.  As shown to the right, K-maps for the 
next-state logic are complicated, since we have five variables
and must consider implicants that are not contiguous in the K-maps.
The $S_2^+$ logic is easy enough: we only need two terms, 
as shown.\mpline

Notice that we have used color and
line style to distinguish different\linebreak
\end{minipage}\hspace{.25in}%
\begin{minipage}{1.22in}
\epsfig{file=part3/figs/goodS2.eps,width=1.22in}\vspace{12pt}
\end{minipage}\hspace{.25in}\begin{minipage}{1.22in}
\epsfig{file=part3/figs/goodS1.eps,width=1.22in}\vspace{12pt}
\end{minipage}\hspace{.25in}\begin{minipage}{1.22in}
\epsfig{file=part3/figs/goodS0.eps,width=1.22in}\vspace{12pt}
\end{minipage}\mpdone

implicants in the K-maps.  Furthermore, the symmetry of the design
produces symmetry in the $S_1^+$ and $S_0^+$ formula, so we have
used the same color and line style for analogous terms in these
two K-maps.
%
For $S_1^+$, we need four terms.  The green 
ellipses in the $HG=01$ column are part of the same term, as are
the two halves of the dashed blue circle.  In $S_0^+$, we still
need four terms, but three of them are split into two pieces 
in the K-map.  As you can see, the utility of the K-map is starting
to break down with five variables.\\


\subsubsection{Abstracting Design Symmetries}

Rather than implementing the design as two-level logic, let's try to
take advantage of our design's symmetry to further simplify the
logic (we reduce gate count at the expense of longer, slower paths).

Looking back to the last transition diagram, in which the arcs
were labeled with logical expressions, let's calculate an expression
for when the counter should retain its current value in the next
cycle.  We call 
this variable $HOLD$.  In the counting states, when $S_2=0$, 
the counter stops (moves into a halted state without changing value) 
when $H$ is true.
In the halted states, when $S_2=1$, the counter stops (stays in 
a halted state) when $H+\overline{G}$ is true.  We can thus write

\begin{eqnarray*}
HOLD &=& \overline{S_2} \cdot H + S_2 \cdot ( H + \overline{G} )\\
HOLD &=& \overline{S_2} H + S_2 H + S_2 \overline{G}\\
HOLD &=& H + S_2 \overline{G}
\end{eqnarray*}

In other words, the counter should hold its current 
value (stop counting) if we press the ``halt'' button or if the counter
was already halted and we didn't press the ``go'' button.  As desired,
the current value of the counter ($S_1S_0$) has no impact on this 
decision.  You may have noticed that the expression we derived for
$HOLD$ also matches $S_2^+$, the next-state value of $S_2$ in the 
K-map on the previous page.

Now let's re-write our state transition table in terms of $HOLD$.  The
left version uses state names for clarity; the right uses state values
to help us transcribe K-maps.\\

\centerline{
\begin{tabular}{rc|cc}
&&\multicolumn{2}{c}{$HOLD$}\\
\multicolumn{1}{c}{state}&$S_2S_1S_0$& 0& 1\\ \hline
{\st COUNT~A}&000& {\st COUNT~B}& {\st HALT~A}\\
{\st COUNT~B}&001& {\st COUNT~C}& {\st HALT~B}\\
{\st COUNT~C}&011& {\st COUNT~D}& {\st HALT~C}\\
{\st COUNT~D}&010& {\st COUNT~A}& {\st HALT~D}\\
 {\st HALT~A}&100& {\st COUNT~B}& {\st HALT~A}\\
 {\st HALT~B}&101& {\st COUNT~C}& {\st HALT~B}\\
 {\st HALT~C}&111& {\st COUNT~D}& {\st HALT~C}\\
 {\st HALT~D}&110& {\st COUNT~A}& {\st HALT~D}\\
\end{tabular}\hspace{.5in}%
\begin{tabular}{rc|cc}
&&\multicolumn{2}{c}{$HOLD$}\\
\multicolumn{1}{c}{state}&$S_2S_1S_0$& 0& 1\\ \hline
{\st COUNT~A}&000& 001& 100\\
{\st COUNT~B}&001& 011& 101\\
{\st COUNT~C}&011& 010& 111\\
{\st COUNT~D}&010& 000& 110\\
 {\st HALT~A}&100& 001& 100\\
 {\st HALT~B}&101& 011& 101\\
 {\st HALT~C}&111& 010& 111\\
 {\st HALT~D}&110& 000& 110\\
\end{tabular}}

\begin{minipage}{2.6in}
The K-maps based on the $HOLD$ abstraction are shown to the right.
As you can see, the necessary logic has been simplified substantially,
requiring only two terms each for both $S_1^+$ and $S_0^+$.  Writing
the next-state logic algebraically, we obtain
%
\begin{eqnarray*}
S_2^+ &=& HOLD\\
S_1^+ &=& \overline{HOLD} \cdot S_0 + HOLD \cdot S_1\\
S_0^+ &=& \overline{HOLD} \cdot \overline{\mbox{$S_1$}} + HOLD \cdot S_0
\end{eqnarray*}
\end{minipage}\hspace{.25in}%
\begin{minipage}{1.05in}
\epsfig{file=part3/figs/holdS2.eps,width=1.05in}
\end{minipage}\hspace{.25in}\begin{minipage}{1.05in}
\epsfig{file=part3/figs/holdS1.eps,width=1.05in}
\end{minipage}\hspace{.25in}\begin{minipage}{1.05in}
\epsfig{file=part3/figs/holdS0.eps,width=1.05in}
\end{minipage}

Notice the similarity between the equations for $S_1^+S_0^+$ and the 
equations for a \mbox{2-to-1}~mux: when $HOLD=1$, the counter retains 
its state, and when $HOLD=0$, it counts.

\vfill

\pagebreak

An implementation appears below.
%
By using semantic meaning in our choice of representation---in
particular the use of $S_2$ to record whether
the counter is currently halted ($S_2=1$) or counting ($S_2=0$)---we
have enabled ourselves to 
separate out the logic for deciding whether to advance the counter
fairly cleanly from the logic for advancing the counter itself.
Only the $HOLD$ bit in the diagram is used to determine
whether or not the counter should advance in the current cycle.

Let's check that the implementation matches our original design.
%
Start by verifying that the $HOLD$ variable is calculated correctly,
$HOLD=H+S_2\overline{G}$,
then look back at the K-map for $S_2^+$ in the low-level design to
verify that the expression we used does indeed match.\\

\centerline{\epsfig{file=part3/figs/sscount.eps,width=5.1in}}\vspace{12pt}

Next, check the mux abstraction.
%
When $HOLD=1$, the next-state logic for~$S_1^+$ and~$S_0^+$ 
reduces to $S_1^+=S_1$ and $S_0^+=S_0$;
in other words, the counter stops counting and simply stays in its 
current state.  When $HOLD=0$, these equations become
$S_1^+=S_0$ and $S_0^+=\overline{\mbox{$S_1$}}$, which produces the repeating
sequence for $S_1S_0$ of $00$, $01$, $11$, $10$, as desired.
You may want to look back at our two-bit Gray code counter design
to compare the next-state equations.

We can now verify that the implementation produces the correct transition
behavior.  In the counting states, $S_2=0$, and the $HOLD$ value simplifies
to $HOLD=H$.  Until we push the ``halt'' button, $S_2$ remains~$0$, and
and the counter continues to count in the correct sequence.
When $H=1$, $HOLD=1$, and the counter stops at its current value
($S_2^+S_1^+S_0^+=1S_1S_0$, 
which is shorthand for $S_2^+=1$, $S_1^+=S_1$, and $S_0^+=S_0$).

In any of the halted states, $S_2=1$, and we can reduce $HOLD$ to
$HOLD=H+\overline{G}$.  Here, so long as we press the ``halt'' button
or do not press the ``go'' button, the counter stays in its current
state, because $HOLD=1$.  If we release ``halt'' and press ``go,''
we have $HOLD=0$, and the counter resumes counting
($S_2^+S_1^+S_0^+=0S_0\overline{\mbox{$S_1$}}$,
which is shorthand for $S_2^+=0$, $S_1^+=S_0$, and 
$S_0^+=\overline{\mbox{$S_1$}}$).
%
We have now verified the implementation.

What if you wanted to build a three-bit Gray code counter with the same
controls for starting and stopping?  You could go back to basics and struggle 
with six-variable \mbox{K-maps}.  Or you could simply copy the~$HOLD$ 
mechanism from the two-bit design above, insert muxes between the next 
state logic and the flip-flops of the three-bit Gray code counter that 
we designed earlier, and control the muxes with the $HOLD$ bit.  
Abstraction is a powerful tool.\\

\pagebreak

\subsubsection{Impact of the State Representation}

What happens if we choose a bad representation?  For the same FSM---the
two-bit Gray code counter with start and stop inputs---the 
table below shows a poorly chosen mapping from states to internal 
state representation.
%
Below the table is a diagram of an implementation using that
representation.
%
Verifying that the implementation's behavior
is correct is left as an exercise for the determined reader.\\

\centerline{
\begin{tabular}{|c|c|c|c|c|}\cline{1-2}\cline{4-5}
state& $S_2S_1S_0$& \protect{\mbox{\hspace{.25in}}}& state& $S_2S_1S_0$ \\ \cline{1-2}\cline{4-5}
{\st COUNT~A}& 000& & {\st HALT~A}& 111 \\
{\st COUNT~B}& 101& & {\st HALT~B}& 110 \\
{\st COUNT~C}& 011& & {\st HALT~C}& 100 \\
{\st COUNT~D}& 010& & {\st HALT~D}& 001 \\ \cline{1-2}\cline{4-5}
\end{tabular}
}\vspace{10pt}

\centerline{\epsfig{file=part3/figs/sscomplex.eps,width=6.5in}}


\vfill

\pagebreak

