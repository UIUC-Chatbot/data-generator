\classtitle

\subsection{Extending Keyless Entry with a Timeout}

This set of notes builds on the keyless entry control FSM that we
designed earlier.  In particular, we use a counter to make the alarm
time out, turning itself off after a fixed amount of time.  The goal
of this extension is to illustrate how we can make use of components
such as registers and counters as building blocks for our FSMs
without fully expanding the design to explicitly illustrate all
possible states.

%\subsubsection{Physical Design, Sensors, and Timing}

\begin{minipage}{2.05in}
To begin, let's review the FSM that we designed earlier for keyless 
entry.  The state transition diagram for our design is replicated to 
the right.
%
The four states are labeled with state bits and output bits,
$S_1S_0/DRA$, where $D$~indicates that the driver's door should
be unlocked, $R$~indicates that the rest of the doors should be
unlocked, and $A$~indicates that the alarm should be on.
%
Transition arcs in the diagram are labeled with concise versions of 
the inputs~$ULP$ (using don't cares), where $U$~represents an unlock
button, $L$~represents a lock button, and $P$~represents a panic 
button.\mpline

In this design, once a user presses the panic button~$P$, the alarm
sounds until the user presses the\linebreak\mpdone
\end{minipage}\hspace{0.25in}%
\begin{minipage}{4.2in}
\epsfig{file=part3/figs/ke-trans-diag-brief.eps,width=4.2in}
\end{minipage}

lock button~$L$ to turn it off.
Instead of sounding the alarm indefinitely, we might want to turn
the alarm off after a fixed amount of time.  In other words, after 
the system has been in the ALARM state for, say, thirty or sixty seconds,
we might want to move back to the LOCKED state even if the user has
not pushed the lock button.  The blue annotation in the diagram indicates
the arc that we must adjust.  But thirty or sixty seconds is a large 
number of clock cycles, and our FSM must keep track of the time.
Do we need to draw all of the states?

Instead of following the design process that we outlined earlier, let's
think about how we can modify our existing design to incorporate the new
functionality.  In order to keep track of time, we use a binary counter.
%
Let's say that we want our timeout to be~$T$ cycles.
%
When we enter the alarm state, we want to set the counter's value 
to~$T-1$, then let the counter count down until it reaches~0, at 
which point a timeout occurs.
%
To load the initial value, our counter should have a parallel load 
capability that sets the counter value when input $LD=1$.  When $LD=0$,
the counter counts down. 
%
The counter also has an output~$Z$ that indicates that the counter's 
value is currently zero, which we can use to indicate a timeout on
the alarm.
%
You should be able to build such a counter based on what you have learned
earlier in the class.  Here, we will assume that we can just make use of
it.

How many bits do we need in our counter?  The answer depends on~$T$.
If we add the counter to our design, the state of the counter is 
technically part of the state of our FSM, but we can treat it
somewhat abstractly.  For example, we only plan to make use of the 
counter value in the ALARM state, so we ignore the counter bits in the
three other states.  In other words, $S_1S_0=10$ means that the system 
is in the LOCKED state regardless of the counter's value.

\begin{minipage}{4.5in}
We expand the ALARM state into~$T$ separate states based on the value
of the counter.  As shown to the right, we name the states ALARM(1) 
through ALARM(T).  All of these alarm states use $S_1S_0=01$, but
they can be differentiated using a ``timer'' (the counter value).\mpline

We need to make design decisions about how the arcs entering and
leaving the ALARM state in our original design should be used once
we have incorporated the timeout.  As a first step, we decide that
all arcs entering ALARM from other states now enter ALARM(1).  
Similarly, if the user presses the panic button~$P$ in any of the
ALARM(t) states, the system returns to ALARM(1).  Effectively, 
pressing the panic button resets the timer.\mpline

The only arc leaving the ALARM state goes to the LOCKED state on 
$ULP=x10$.  We replicate this arc for all ALARM(t) states: the
user can push the lock button at any time to silence the alarm.\mpline

Finally, the self-loop back to the ALARM state on $ULP=x00$ becomes
the countdown arcs in our expanded states, taking ALARM(t) to ALARM(t+1),
and ALARM(T) to LOCKED.\mpline

Now that we have a complete specification for the extended design, we
can implement it.  We want to reuse our original design as much as possible,
but we have three new features that must be considered.  First, when
we enter the ALARM(1) state, we need to set the counter value to~$T-1$.
Second, we need the counter value to count downward while in the ALARM
state.  Finally, we need to move back to the LOCKED state when a timeout
occurs---that is, when the counter reaches zero.
\end{minipage}\hspace{0.25in}%
\begin{minipage}{1.75in}
\epsfig{file=part3/figs/ke-alarm-expansion.eps,width=1.75in}
\end{minipage}

The first problem is fairly easy.  Our counter supports parallel load,
and the only value that we need to load is~$T-1$, so we apply the constant
bit pattern for~$T-1$ to the load inputs and raise the~$LD$ input whenever
we enter the ALARM(1) state.  In our original design, we chose to enter
the ALARM state whenever the user pressed~$P$, regardless of the other
buttons.  Hence we can connect~$P$ directly to our counter's~$LD$ input.

The second problem is handled by the counter's countdown functionality.
In the ALARM(t) states, the counter will count down each cycle, moving
the system from ALARM(t) to ALARM(t+1).

The last problem is slightly trickier, since we need to change~$S_1S_0$.
Notice that $S_1S_0=01$ for the ALARM state and $S_1S_0=00$ for the LOCKED
state.  Thus, we need only force~$S_0$ to~0 when a timeout occurs.
We can use a single \mbox{2-to-1} multiplexer for this purpose.  The
``0'' input of the mux comes from the original $S_0^+$ logic, and the
``1'' input is a constant~0.  All other state logic remains unchanged.  
When does a timeout occur? 
%
First, we must be in the ALARM(T) state, so~$S_1S_0=01$ and the 
counter's~$Z$ output is raised.  Second, the input combination must
be~$ULP=xx0$---notice that both~$ULP=x00$ and $ULP=x10$ return to 
LOCKED from ALARM(T).  A single, four-input AND gate thus suffices to
obtain the timeout signal, $\bar{S_1}S_0Z\bar{P}$, which we connect 
to the select input of the mux between the~$S_0^+$ logic and the
$S_0$~flip-flop.

The extension thus requires only a counter, a mux, and a gate, as shown below.\\

\centerline{\epsfig{file=part3/figs/ke-alarm-exp-impl.eps,width=2.65in}}

\vfill

\pagebreak

