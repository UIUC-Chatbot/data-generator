\classtitle

\subsection{Summary of Part 4 of the Course}

With the exception of control unit design strategies and redundancy 
and coding, most of the material in this part of the course is drawn from
Patt and Patel Chapters~4 through~7.  You may also want to read Patt and 
Patel's Appendix~C for details of their control unit design.

In this short summary, we give you lists at several levels of difficulty
of what we expect you to be able to do as a result of the last few weeks
of studying (reading, listening, doing homework, discussing your
understanding with your classmates, and so forth).

We'll start with the easy stuff.  You should recognize all of these terms
and be able to explain what they mean.  
% For the specific circuits, you 
% should be able to draw them and explain how they work.
%(You may skip the *'d terms in Fall~2012.)

\vfill

\begin{minipage}[t]{2.6in}
\begin{list}{$\bullet$}{\setlength{\itemsep}{0pt}\setlength{\parskip}{0pt}%
\setlength{\topsep}{0pt}\setlength{\partopsep}{0pt}\setlength{\parsep}{0pt}}

\item{von Neumann elements}
\begin{list}{-}{\setlength{\itemsep}{0pt}\setlength{\parskip}{0pt}%
\setlength{\topsep}{0pt}\setlength{\partopsep}{0pt}\setlength{\parsep}{0pt}}
\item{program counter (PC)}
\item{instruction register (IR)}
\item{memory address register (MAR)}
\item{memory data register (MDR)}
\item{processor datapath}
\item{bus}
\item{control signal}
\item{instruction processing}
\end{list}

\item{Instruction Set Architecture (ISA)}
\begin{list}{-}{\setlength{\itemsep}{0pt}\setlength{\parskip}{0pt}%
\setlength{\topsep}{0pt}\setlength{\partopsep}{0pt}\setlength{\parsep}{0pt}}
\item{instruction encoding}
\item{field (in an encoded instruction)}
\item{operation code (opcode)}
\end{list}

\item{assemblers and assembly code}
\begin{list}{-}{\setlength{\itemsep}{0pt}\setlength{\parskip}{0pt}%
\setlength{\topsep}{0pt}\setlength{\partopsep}{0pt}\setlength{\parsep}{0pt}}
\item{opcode mnemonic\\ (such as ADD, JMP)}
\item{two-pass process}
\item{label}
\item{symbol table}
\item{pseudo-op / directive}
\end{list}

\end{list}
\end{minipage}\hspace{.75in}
\begin{minipage}[t]{3in}
\begin{list}{$\bullet$}{\setlength{\itemsep}{0pt}\setlength{\parskip}{0pt}%
\setlength{\topsep}{0pt}\setlength{\partopsep}{0pt}\setlength{\parsep}{0pt}}

\item{systematic decomposition}
\begin{list}{-}{\setlength{\itemsep}{0pt}\setlength{\parskip}{0pt}%
\setlength{\topsep}{0pt}\setlength{\partopsep}{0pt}\setlength{\parsep}{0pt}}
\item{sequential}
\item{conditional}
\item{iterative}
\end{list}

% no documentation, and advanced topics ... no testing
%
% \item{logic design optimization}
% \begin{list}{-}{\setlength{\itemsep}{0pt}\setlength{\parskip}{0pt}%
% \setlength{\topsep}{0pt}\setlength{\partopsep}{0pt}\setlength{\parsep}{0pt}}
% \item{bit-sliced (including multiple\\ bits per slice)}
% \item{serialized}
% \item{pipelined logic}
% \item{tree-based}
% \end{list}

\item{control unit design strategies}
\begin{list}{-}{\setlength{\itemsep}{0pt}\setlength{\parskip}{0pt}%
\setlength{\topsep}{0pt}\setlength{\partopsep}{0pt}\setlength{\parsep}{0pt}}
\item{control word / microinstruction}
\item{sequencing / microsequencing}
\item{hardwired control}
\begin{list}{-}{\setlength{\itemsep}{0pt}\setlength{\parskip}{0pt}%
\setlength{\topsep}{0pt}\setlength{\partopsep}{0pt}\setlength{\parsep}{0pt}}
\item{single-cycle}
\item{multi-cycle}
\end{list}
\item{microprogrammed control}
% \item{pipelining (of instruction processing)}
\end{list}

\item{error detection and correction
\begin{list}{--}{\setlength{\itemsep}{0pt}\setlength{\parskip}{0pt}%
\setlength{\topsep}{0pt}\setlength{\partopsep}{0pt}\setlength{\parsep}{0pt}}
\item code/sparse representation
\item code word
\item bit error
\item odd/even parity bit
\item Hamming distance between code words
\item Hamming distance of a code
\item Hamming code
\item SEC-DED
\end{list}
}

\end{list}

\end{minipage}

\vfill
\vfill
\vfill
\vfill

\pagebreak

We expect you to be able to exercise the following skills:

\begin{list}{$\bullet$}{\setlength{\itemsep}{0pt}\setlength{\parskip}{0pt}%
\setlength{\topsep}{0pt}\setlength{\partopsep}{0pt}\setlength{\parsep}{0pt}}

% FIXME ... should write something about critical path, but expecting
% them to do much with it isn't reasonable
%
% \item{Implement arbitrary Boolean logic, and be able to reason about
% alternative designs in terms of their critical path delays and number
% of gates required to implement.}

\item{Map RTL (register transfer language) operations into control words
for a given processor datapath.}

\item{Systematically decompose a (simple enough) problem to the level 
of \mbox{LC-3} instructions.}

\item{Encode \mbox{LC-3} instructions into machine code.}

\item{Read and understand programs written in \mbox{LC-3} assembly/machine code.}

\item{Test and debug a small program in \mbox{LC-3} assembly/machine code.}

\item{Be able to calculate the Hamming distance of a code/representation.}

\item{Know the relationships between Hamming distance and the abilities
to detect and to correct bit errors.}

\end{list}

We expect that you will understand the concepts and ideas to the extent
that you can do the following:

\begin{list}{$\bullet$}{\setlength{\itemsep}{0pt}\setlength{\parskip}{0pt}%
\setlength{\topsep}{0pt}\setlength{\partopsep}{0pt}\setlength{\parsep}{0pt}}

\item{Explain the role of different types of instructions in allowing
a programmer to express a computation.}

% FIXME: ISA design is beyond the scope of this course; just include
% for interest, if at all
%
% \item{Explain the tradeoffs in different addressing modes so as to motivate
% inclusion of multiple addressing modes in an ISA.}

\item{Explain the importance of the three types of subdivisions in systematic
decomposition (sequential, conditional, and iterative).}

\item{Explain the process of transforming assembly code into machine code
(that is, explain how an assembler works, including describing the use of
the symbol table).}

\item{Be able to use parity for error detection, and Hamming codes for
error correction.}

\end{list}

At the highest level, 
we hope that, while you do not have direct substantial experience in 
this regard from our class (and should not expect to be tested on these
skills), that you will nonetheless be able to begin
to do the following when designing combinational logic:

\begin{list}{$\bullet$}{\setlength{\itemsep}{0pt}\setlength{\parskip}{0pt}%
\setlength{\topsep}{0pt}\setlength{\partopsep}{0pt}\setlength{\parsep}{0pt}}

\item{Design and compare implementations using gates, decoders, muxes, and/or memories 
as appropriate, and including reasoning about the relevant design tradeoffs 
in terms of area and delay.}

\item{Design and compare implementation as a bit-sliced, serial, pipelined, or tree-based 
design, again including reasoning about the relevant design tradeoffs in
terms of area and delay.}

\item{Design and compare implementations of processor control units
using both hardwired and microprogrammed strategies,
and again including reasoning about the relevant design tradeoffs in
terms of area and delay.}

\item{Understand basic tradeoffs in the sparsity of code words with error 
detection and correction capabilities.}

\end{list}

\pagebreak

%\mbox{~~~}  % blank 3rd page
%
%\vfill
%
%\pagebreak
