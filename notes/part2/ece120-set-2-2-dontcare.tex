\classtitle

\subsection{Boolean Properties and Don't Care Simplification}

This set of notes begins with a brief illustration of a few properties
of Boolean logic, which may be of use to you in manipulating algebraic
expressions and in identifying equivalent logic functions without resorting 
to truth tables.
%
We then discuss the value of underspecifying a logic function so as
to allow for selection of the simplest possible implementation.
%
This technique must be used carefully to avoid incorrect behavior,
so we illustrate the possibility of misuse with an example, then
talk about several ways of solving the example correctly.
%
We conclude by generalizing the ideas in the example to several
important application areas and talking about related problems.\\

\subsubsection{Logic Properties}

Table~\ref{tab:bool} (on the next page)
lists a number of properties of Boolean logic.
Most of these are easy to derive from our earlier definitions, but
a few may be surprising to you.  In particular, in the algebra of
real numbers, multiplication
distributes over addition, but addition does not distribute over
multiplication.  For example, $3\times{(4+7)}=(3\times{4})+(3\times{7})$,
but $3+(4\times{7})\not=(3+4)\times{(3+7)}$.  In Boolean algebra,
both operators distribute over one another, as indicated in
Table~\ref{tab:bool}.  The consensus properties may also be 
nonintuitive.  Drawing a \mbox{K-map} may help you understand the
consensus property on the right side of the table.  For the 
consensus variant on the left side of the table,
consider that since either~$A$ or~$\overline{A}$ must be~0,
either~$B$ or~$C$ or both must be~1 for the first two factors on the left
to be~1 when ANDed together.  But in that case, the third factor 
is also~1, and is thus redundant.

As mentioned previously, Boolean
algebra has an elegant symmetry known as a duality, in which any
logic statement (an expression or an equation) is related to a
second logic statement.
%
To calculate the {\bf dual form} of a Boolean expression or equation, 
replace~0 with~1, replace~1 with~0, 
replace~AND with~OR, and
replace~OR with~AND.
%
{\em Variables are not changed when finding the dual form.}
%
The dual form of a dual form is the original logic statement.
%
Be careful when calculating a dual form: our convention for ordering 
arithmetic operations is broken by the exchange, so you may want
to add explicit parentheses before calculating the dual.  For
example, the dual of~$AB+C$ is not~$A+BC$.
Rather, the dual of~$AB+C$ is~$(A+B)C$.
%
{\em Add parentheses as necessary when calculating a dual form to ensure
that the order of operations does not change.}

Duality has several useful practical applications.  First, the 
{\bf principle of duality} states that any theorem or identity has 
the same truth value in dual form (we do not prove the principle
here).
%
The rows of Table~\ref{tab:bool} are organized according to this
principle: each row contains two equations that are the duals 
of one another.  
% However, we rename variables freely to
% simplify the dual form.  For example, taking the dual of~$1+A=1$,
% we obtain~$0\cdot{\overline{A}}=0$.  But we created variable~$A$ 
% solely for the purpose of expressing the property, so we might
% instead have called the complementary value~$A$.  We rename
% (using the same name!) to obtain~$0\cdot{A}=0$, as shown in the table.\\
%
Second, the dual form is useful when designing certain types of logic,
such as the networks of transistors connecting the output of a CMOS
gate to high voltage and ground.  If you look at the gate designs in
the textbook (and particularly those in the exercises),
you will notice that these networks are duals.  
%
A function/expression is not a theorem nor an identity,
thus the principle of duality does not apply to the dual of an
expression.
%
However, if you treat the value~0 as ``true,'' the dual form of an
expression has the same truth values as the original (operating with
value~1 as ``true'').
%
Finally, you can calculate the complement of a Boolean function
(any expression) by calculating the dual form and then complementing
each variable.\\

\subsubsection{Choosing the Best Function}

When we specify how something works using a human language, we
leave out details.  Sometimes we do so deliberately, assuming that a
reader or listener can provide the details themselves: ``Take me to the 
airport!''
rather than ``Please bend your right arm at the elbow and shift your
right upper arm forward so as to place your hand near the ignition key.
Next, ...''  

\begin{table}
\centerline{
\begin{tabular}{|lll|}\hline
$1+A=1$& $0\cdot{A}=0$&\\
$1\cdot{A}=A$& $0+A=A$&\\
$A+A=A$& $A\cdot{A}=A$&\\
$A\cdot{\overline{A}}=0$& $A+\overline{A}=1$&\\
$\overline{A+B}=\overline{A}~\overline{B}$&
$\overline{AB}=\overline{A}+\overline{B}$& DeMorgan's laws\\
$(A+B)C=AC+BC$&
$A~B+C=(A+C)(B+C)$&distribution\\
$(A+B)(\overline{A}+C)(B+C)=(A+B)(\overline{A}+C)$&
$A~B+\overline{A}~C+B~C=A~B+\overline{A}~C$& consensus\\\hline
\end{tabular}
}
\caption{Boolean logic properties.  The two columns are dual forms of
one another.}
\label{tab:bool}
\end{table}

You know the basic technique for implementing a Boolean function
using {\bf combinational logic}: use a \mbox{K-map} to identify a
reasonable SOP or POS form, draw the resulting design, and perhaps
convert to NAND/NOR gates.

\pagebreak

When we develop combinational logic designs, we may also choose to
leave some aspects unspecified.  In particular, the value of a
Boolean logic function to be implemented may not matter for some
input combinations.  If we express the function as a truth table,
we may choose to mark the function's value for some input combinations 
as ``{\bf don't care},'' which is written as ``x'' (no quotes).

What is the benefit of using ``don't care'' values?  
%
Using ``don't care'' values allows you to choose from among several
possible logic functions, all of which produce the desired results
(as well as some combination of~0s and~1s in place of the ``don't
care'' values).  
%
Each input combination marked as ``don't care'' doubles the number
of functions that can be chosen to implement the design, often enabling 
the logic needed for implementation to be simpler.

\begin{minipage}{5.25in}
For example, the \mbox{K-map} to the right specifies a function~$F(A,B,C)$
with two ``don't care'' entries.  
%
If you are asked to design combinational logic for this function,
you can
choose any values for the two ``don't care'' entries.  When identifying
prime implicants, each~``x'' can either be a~0 or a~1.
\end{minipage}\hspace{.25in}%
\begin{minipage}{1in}
\epsfig{file=part2/figs/dcfirstex.eps,width=1in}
\end{minipage}

\begin{minipage}{5.25in}
Depending on the choices made for the~x's, we obtain one of 
the following four functions:
%
\begin{eqnarray*}
F&=&\overline{A}~B+B~C\\
F&=&\overline{A}~B+B~C+A~\overline{B}~\overline{C}\\
F&=&B\\
F&=&B+A~\overline{C}\\
\end{eqnarray*}
\end{minipage}\hspace{.25in}%
\begin{minipage}{1in}
\epsfig{file=part2/figs/dcfirstpick.eps,width=1in}
\end{minipage}

Given this set of choices, a designer typically chooses the third: $F=B$,
which corresponds to the \mbox{K-map} shown to the right of the 
equations.  The design then 
produces~$F=1$ when~$A=1$, $B=1$, and~$C=0$ ($ABC=110$), 
and produces~$F=0$ when~$A=1$, $B=0$, and~$C=0$ ($ABC=100$).
These differences are marked with shading and green italics in the new
\mbox{K-map}.  No implementation ever produces an ``x.''\\

\subsubsection{Caring about Don't Cares}

What can go wrong?
%
In the context of a digital system, unspecified details
may or may not be important.  However, {\em any implementation of 
a specification 
implies decisions} about these details, so decisions should only be left 
unspecified if any of the possible answers is indeed acceptable.

As a concrete example, let's design logic to control an ice cream 
dispenser.  The dispenser has two flavors, lychee and mango, but
also allows us to create a blend of the two flavors.
For each of the two flavors, our logic must output two bits
to control the amount of ice cream that comes out of the dispenser.
The two-bit $C_L[1:0]$ output of our logic must specify the number 
of half-servings of lychee ice cream as a binary number, and
the two-bit $C_M[1:0]$ output must specify the number of 
half-servings of mango ice cream.  Thus, for either flavor,
$00$ indicates none of that flavor,
$01$ indicates one-half of a serving, and 
$10$ indicates a full serving.

Inputs to our logic will consist of three buttons: an~$L$~button to
request a serving of lychee ice cream, a~$B$~button to request a
blend---half a serving of each flavor, and an~$M$~button to request
a serving of mango ice cream.  Each button produces a~1 
when pressed and a~0 when not pressed.

\pagebreak

Let's start with the assumption that the user only presses one button
at a time.  In this case, we can treat input combinations in which
more than one button is pressed as ``don't care'' values in the truth
tables for the outputs.  K-maps for all four output bits appear below.
The x's indicate ``don't care'' values.\\

\centerline{\epsfig{file=part2/figs/CLhigh.eps,width=1.00in}\hspace{.25in}\epsfig{file=part2/figs/CLlow.eps,width=1.00in}\hspace{.25in}\epsfig{file=part2/figs/CMhigh.eps,width=1.05in}\hspace{.25in}\epsfig{file=part2/figs/CMlow.eps,width=1.00in}}

When we calculate the logic function for an output, each ``don't care''
value can be treated as either~0 or~1, whichever is more convenient
in terms of creating the logic.  In the case of $C_M[1]$, for 
example, we can treat the three x's in the ellipse as 1s, treat the~x
outside of the ellipse as a 0, and simply
use $M$ (the implicant represented by the ellipse) for $C_M[1]$.  The
other three output bits are left as an exercise, although the result 
appears momentarily.

\begin{minipage}{4.2in}
The implementation at right takes full advantage of the ``don't care''
parts of our specification.  In this case, we require no logic at all;
we need merely connect the inputs to the correct outputs.  Let's verify
the operation.  We have four cases to consider.  First, if none of the 
buttons are pushed ($LBM=000$), we get no ice cream, as desired ($C_M=00$ 
and $C_L=00$).  Second, if we request lychee ice cream ($LBM=100$), the
outputs are $C_L=10$ and $C_M=00$, so we get a full serving of lychee
and no mango.  Third, if we request a blend ($LBM=010$), the outputs
are $C_L=01$ and $C_M=01$, giving us half a serving of each flavor.
Finally, if we request mango ice cream ($LBM=001$), we get no lychee
but a full serving of mango.
\end{minipage}\hspace{.25in}%
\begin{minipage}{1.35in}
\epsfig{file=part2/figs/icbasic.eps,width=2.05in}
\end{minipage}

The K-maps for this implementation appear below.  Each of
the ``don't care'' x's from the original design has been replaced with
either a~0 or a~1 and highlighted with shading and green italics.
Any implementation produces 
either~0 or~1 for every output bit for every possible input combination.\\

\centerline{\epsfig{file=part2/figs/CLhigh-basic.eps,width=1.00in}\hspace{.25in}\epsfig{file=part2/figs/CLlow-basic.eps,width=1.00in}\hspace{.25in}\epsfig{file=part2/figs/CMhigh-basic.eps,width=1.00in}\hspace{.25in}\epsfig{file=part2/figs/CMlow-basic.eps,width=1.00in}}

As you can see, leveraging ``don't care'' output bits can sometimes
significantly simplify our logic.  In the case of this example, we were
able to completely eliminate any need for gates!  Unfortunately, 
the resulting implementation may sometimes produce unexpected results. 
%
Based on the implementation, what happens if a user presses more
than one button?  The ice cream cup overflows!

Let's see why.
%
Consider the case $LBM=101$, in which we've pressed both the lychee and
mango buttons.  Here $C_L=10$ and $C_M=10$, so our dispenser releases a 
full serving of each flavor, or two servings total.  Pressing other 
combinations may have other repercussions as well.
Consider pressing lychee and blend ($LBM=110$).  The outputs are then
$C_L=11$ and $C_M=01$.  Hopefully the dispenser simply gives us one
and a half servings of lychee and a half serving of mango.  However,
if the person who designed the dispenser assumed that no one would ever
ask for more than one serving, something worse might happen.  In other
words, giving an input of $C_L=11$ to the ice cream dispenser may lead to
other unexpected behavior if its designer decided that that input 
pattern was a ``don't care.''

The root of the problem is that {\em while we don't care about the value of
any particular output marked ``x'' for any particular input combination,
we do actually care about the relationship between the outputs}.  

What can we do?  When in doubt, it is safest to make 
choices and to add the new decisions to the specification rather than 
leaving output values specified as ``don't care.''
%
For our ice cream dispenser logic, rather than leaving the outputs 
unspecified whenever a user presses more than one button, we could 
choose an acceptable outcome for each input combination and 
replace the x's with 0s and 1s.  We might, for example, decide to
produce lychee ice cream whenever the lychee button is pressed, regardless
of other buttons~($LBM=1xx$, which means that we don't care about the
inputs~$B$ and~$M$, so $LBM=100$, $LBM=101$, $LBM=110$, or~$LBM=111$).  
That decision alone covers three of the
four unspecified input patterns.  We might also decide that when the 
blend and mango buttons are pushed together (but without the lychee
button, LBM=011), our logic produces a blend.  
%
The resulting K-maps are shown below, again with shading and green italics 
identifying
the combinations in which our original design specified ``don't care.''\\

\centerline{\epsfig{file=part2/figs/CLhigh-priority.eps,width=1.00in}\hspace{.25in}\epsfig{file=part2/figs/CLlow-priority.eps,width=1.00in}\hspace{.25in}\epsfig{file=part2/figs/CMhigh-priority.eps,width=1.00in}\hspace{.25in}\epsfig{file=part2/figs/CMlow-priority.eps,width=1.00in}}\vspace{6pt}

\begin{minipage}{2.45in}
The logic in the dashed box to the right implements the set of choices
just discussed, and matches the K-maps above.
%
Based on our additional choices, this implementation enforces
a strict priority scheme on the user's button presses.
%
If a user requests lychee, they can also press either or
both of the other buttons with no effect.  The lychee button has
priority.  Similarly, if the user does not press lychee, but press-\linebreak
\end{minipage}\hspace{.25in}%
\begin{minipage}{3.8in}
\epsfig{file=part2/figs/icpriority.eps,width=3.8in}\vspace{12pt}
\end{minipage}\mpdone

es the blend button, pressing the mango button at the same time has no
effect.  Choosing mango requires that no other buttons be pressed.
We have thus chosen a prioritization order for the buttons and imposed 
this order on the design.

We can view this same implementation in another way.
%
Note the one-to-one correspondence between inputs (on the left) and 
outputs (on the right) for the dashed box.
%
This logic takes the user's button presses and chooses at most one of
the buttons to pass along to our original controller implementation 
(to the right of the dashed box).
%
In other words, rather than thinking of the logic in the dashed
box as implementing a specific set of decisions, we can think of the
logic as cleaning up the inputs to ensure that only valid
combinations are passed to our original implementation.
%
Once the inputs are cleaned up, the original implementation is 
acceptable, because input combinations containing more than a 
single~1 are in fact impossible.

Strict prioritization is one useful way to clean up our inputs.
%
In general, we can design logic to map each
of the four undesirable input patterns into one of the permissible 
combinations (the
four that we specified explicitly in our original design, with $LBM$ 
in the set $\{000,001,010,100\}$).  Selecting a prioritization scheme
is just one approach for making these choices in a way
that is easy for a user to understand and 
is fairly easy to implement.

\begin{minipage}{2.45in}
A second simple approach is to ignore illegal
combinations by mapping them into the ``no buttons pressed'' 
input pattern.
%
Such an implementation appears to the right, laid out to show that
one can again view the logic in the dashed box either as cleaning up 
the inputs (by mentally grouping the logic with the inputs) or as a specific 
set of choices for our ``don't care'' output values (by grouping the 
logic with the outputs).
%
In either case, the logic shown 
enforces our as-\linebreak
\end{minipage}\hspace{.25in}%
\begin{minipage}{3.8in}
\epsfig{file=part2/figs/icconserve.eps,width=3.8in}\vspace{12pt}
\end{minipage}\mpdone

sumptions in a fairly conservative way:
if a user presses more than one button, the logic squashes all button
presses.  Only a single 1~value at a time can pass through to 
the wires on the right of the figure.

\pagebreak

For completeness, the K-maps corresponding to this implementation are given
here.\\

\centerline{\epsfig{file=part2/figs/CLhigh-conserve.eps,width=1.00in}\hspace{.25in}\epsfig{file=part2/figs/CLlow-conserve.eps,width=1.00in}\hspace{.25in}\epsfig{file=part2/figs/CMhigh-conserve.eps,width=1.00in}\hspace{.25in}\epsfig{file=part2/figs/CMlow-conserve.eps,width=1.00in}}\vspace{12pt}


\subsubsection{Generalizations and Applications*}

The approaches that we illustrated to clean up the input signals to
our design have application in many areas.  The ideas in this 
section are drawn from the field and are sometimes the subjects of 
later classes, but {\em are not exam material for our class.}

Prioritization of distinct
inputs is used to arbitrate between devices attached to a
processor.  Processors typically execute much more quickly than do devices.
When a device needs attention, the device signals the processor
by changing the voltage on an interrupt line (the name comes from the
idea that the device interrupts the processor's current activity, such
as running a user program).  However, more than
one device may need the attention of the processor simultaneously, so
a priority encoder is used to impose a strict order on the devices and
to tell the processor about their needs one at a time.
%
If you want to learn more about this application, take ECE391.

When components are designed together, assuming that some input patterns
do not occur is common practice, since such assumptions can dramatically
reduce the number of gates required, improve performance, reduce power
consumption, and so forth.  As a side effect, when we want to test a
chip to make sure that no defects or other problems prevent the chip
from operating correctly, we have to be careful so as not to ``test''
bit patterns that should never occur in practice.  Making up random bit
patterns is easy, but can produce bad results or even destroy the chip
if some parts of the design have assumed that a combination produced 
randomly can never occur.  To avoid these problems, designers add extra
logic that changes the disallowed patterns into allowed patterns, just
as we did with our design.  The use of random bit patterns is common in
Built-In Self Test (BIST), and so the process of inserting extra logic
to avoid problems is called BIST hardening.  BIST hardening can add
\mbox{10-20\%} additional logic to a design.
%
Our graduate class on digital system testing, ECE543, covers this
material, but has not been offered recently.

\vfill

\pagebreak

