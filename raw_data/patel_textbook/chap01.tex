\documentclass{patt}
\usepackage[section]{placeins}
\graphicspath{{../art/ch01/},{../art/designelements/},{../art/designelements/UnNumberedArt/}}

\setcounter{chapter}{0}
\setcounter{page}{1}
\setcounter{part}{1}

 \makeatletter
 \def\@makechapterhead#1{%
  \begingroup
  \parindent \z@%
  \vspace*{-8.5\p@}%
  \begin{picture}(0,0)
    \put(432,-577){\includegraphics[width=0.723333in,height=9.52667in]{PattChp.eps}}
    \put(428,-577){\rule{1\p@}{9.52667in}}
    \put(381.5,-181){\vbox{%
        \includegraphics{PattIcon1\ifnum\value{part}=4
            a\else b\fi.eps}\vspace{1pc}\par
        \includegraphics{PattIcon2\ifnum\value{part}=4
            a\else b\fi.eps}\vspace{1pc}\par
        \includegraphics{PattIcon3\ifnum\value{part}=4
            a\else b\fi.eps}\vspace{1pc}\par
        \includegraphics{PattIcon4\ifnum\value{part}=4
            a\else b\fi.eps}\vspace{1pc}\par
        \includegraphics{PattIcon5\ifnum\value{part}=4
            a\else b\fi.eps}\vspace{1pc}}}
  \end{picture}%
  \settowidth{\chapternumberwidth}{\fontsize{12}{12}\selectfont\industriasolid\trackonefifty{chapter}}%
  \hfill\parbox{\chapternumberwidth}{%
    \centering\industriasolid
    \centerline{\hss\fontsize{12}{12}\selectfont\trackonefifty{chapter}\hss}\par\vspace{1.5pc}
    \centerline{\sans\fontsize{72}{12}\selectfont\colour\thechapter}}
  \par
  \vspace{6.48pc}
  {\industriasolid\fontsize{30}{32}\selectfont\trackten{#1}\par}%
  \vspace{1.5pc}%
  {\colour\rule{36.4pc}{4\p@}}
  \vspace{4pt}
  \endgroup}
\makeatother

\begin{document}

\chapter{Welcome Aboard}\label{chapt:welcome}

\vspace{21pt}

%1.1
\section{What We Will Try to Do}

Welcome to {\em From Bits and Gates to C and Beyond}. Our intent
is to introduce you over the next xxx pages to the world
of computing. As we do so, we have one objective above all others:
to show you very clearly that there is no magic to computing. The
computer is a deterministic system---every time we hit it over the
head in the same way and in the same place (provided, of course,
it was in the same starting condition), we get the same response.
The computer is not an electronic genius; on the contrary, if
anything, it is an electronic idiot, doing exactly what we tell it
to do. It has no mind of its own.

What appears to be a very complex organism is really just a
very large, systematically interconnected collection of very simple parts.
Our job throughout this book is to introduce you to those very
simple parts, and, step-by-step, build the interconnected
structure that you know by the name {\em computer}. Like a
house, we will start at the bottom, construct the foundation
first, and then go on to add layer after layer, as we get
closer and closer to what most people know as a full-blown
computer. Each time we add a layer, we will explain what
we are doing, tying the new ideas to the underlying fabric.
Our goal is that when we are done, you will be able to write
programs in a computer language such as C, using the
sophisticated features of that language, and understand
what is going on underneath, inside the computer.

\pagebreak

%1.2
\section{How We Will Get There}

We will start (in Chapter~2) by first showing that any information processed by
the computer is represented by a sequence of 0s and 1s.  That is, we will 
encode all information as sequences of 0s and 1s. For example, one encoding 
of the letter $a$ that is commonly used is the sequence 01100001.  One encoding
of the decimal number {\em 35} is the sequence 00100011.  We will see how to 
perform operations on such encoded information.

Once we are comfortable with information represented as codes
made up of 0s and 1s and operations (addition, for example)
being performed on these representations, we will begin the
process of showing how a computer works.  Starting in Chapter~3, we will
note that the computer is a piece of electronic equipment and, as such,
consists of electronic parts operated by voltages, 
and interconnected by wires.
Every wire in the computer, at every moment in time, is
either at a high voltage or a low voltage. For our representation of 0s 
and 1s, we do not specify exactly how high. 
We only care whether there is or is not a large enough voltage relative 
to 0 volts to identify it as a $1$.  That is, the absence or
presence of a reasonable voltage relative to 0 volts is what determines 
whether it represents the value 0 or the value 1.
\looseness=1

In Chapter~3, we will
see how the transistors that make up today's microprocessor (the heart of 
the modern computer) works.  We will further see how those transistors 
are combined into larger structures that perform operations, such as addition,
and into structures that allow us to save information for later
use.  In Chapter~4, we will combine these larger structures into
the Von Neumann machine, a basic model that describes how a
computer works.  We will also begin to study a simple
computer, the LC-3.  We will continue our study of the LC-3 in
Chapter~5.  {\em LC-3} stands for Little Computer 3.  We actually started
with LC-1 but needed two more shots at it before (we think)
we got it right!  The LC-3 has all the important characteristics of
the microprocessors that you may have already heard of, for
example, the Intel 8088, which was used in the first IBM PCs
back in 1981.  Or the Motorola 68000, which was used in the
Macintosh, vintage 1984.  Or the Pentium IV, one of the
high-performance microprocessors of choice for the PC in the
year 2003. Or today's laptop and desktop microprocessors, the Intel Core 
processors -- I3, I5 and I7.   Or even the ARM microprocessors that are 
used in most smartphones today.  That is,
the LC-3 has all the important characteristics of these "real"
microprocessors without being so complicated that it gets in
the way of your understanding.

Once we understand how the LC-3 works, the next step is to
program it, first in its own language (Chapter~5 and Chapter~6), and
then in a language called {\em assembly language} that is a little
bit easier for humans to work with (Chapter~7).  Chapter~8 introduces 
representations of information more complex than a simple number -- stacks, 
queues, and character strings, and shows how to implement them. 
Chapter~9 deals with the problem of getting information into (input)
and out of (output) the LC-3.  Chapter~9 also deals with services provided 
to a computer user by the operating system.  We conclude the first half of 
the book (Chapter~10) with an extensive example, the simulation of a 
calculator, an app on most smart phones today.

In the second half of the book (Chapters~11--21), we turn our attention to 
high-level programming concepts, which we introduce via the C and C++ 
programming languages.  High level languages enable programmers to more 
effectively develop complex software, by abstracting away the details of the 
underlying hardware.  C and C++ in particular offer a rich set of programmer 
friendly constructs, but are close enough to the hardware so that we can 
examine how code is tranformed to execute on the layers below.  
Our goal is to enable you to write short, simple programs using the core 
parts of these programming languages, all the while being able to comprehend
the transformations required for your code to execute on the underlying 
hardware.

We'll start with basic topics in C such as variables and operators 
(Chapter~12), control structures (Chapter~13), and functions (Chapter~14).  
We'll see that these are straightforward extensions of concepts introduced 
in the first half of the textbook.  We then move on to programming concepts 
in Chapters~15--19 that will enable us to create more powerful pieces of 
code: Testing and Debugging (Chapter~15), Pointers and Arrays in C
(Chapter~16), Recursion (Chapter~17), Input and Output in C (Chapter~18), and
Data Structures in C (Chapter 19).

Chapters 20 and 21 are devoted to C++, which we present as an evolution of 
the C programming language.   Because the C++ language was intially defined 
as a superset of C, many of the concepts covered in Chapters~11--19 directly 
map onto the C++ language.  We will introduce some of the core notions in 
C++ that have helped establish C++ as one of the most popular languages for 
developing real-world software.  Chapter~20 is our Introduction to C++,  
and Chapter~21 is an introduction to Objects, a powerful programming 
construct in the C++ language.

In almost all cases, we try to tie
high-level C and C++ constructs to the underlying LC-3, so that you
will understand what you demand of the computer when you use
a particular construct in a C or C++ program.

\vspace{-16pt}
%1.3
\section{Two Recurring Themes}

Two themes permeate this book that we as professors previously
took for granted, assuming that everyone recognized their value and
regularly emphasized them to students of engineering and
computer science.  However, it has become clear to us that from the git-go,
we need to make these points explicit.  So, we state
them here up front.  The two themes are (a) the notion of abstraction
and (b) the importance of not separating in your mind the notions of
hardware and software.  Their value to your development as an effective
engineer or computer scientist goes well beyond your understanding of
how a computer works and how to program it.

The notion of abstraction is central to all that you will learn and expect
to use in practicing your craft, whether it be in mathematics, physics,
any aspect of engineering, or business. It is hard to think of any body
of knowledge where the notion of abstraction is not critical.

The misguided hardware/software separation is directly related to your 
continuing study of computers and your work with them.  

We will discuss each in turn.

\vspace{-6pt}
%1.3.1
\subsection{The Notion of Abstraction}

The use of abstraction is all around us.  When we get in a taxi and
tell the driver, ``Take me to the airport,'' we are using abstraction.
If we had to, we could probably direct the driver each step of the way:
``Go down this street ten blocks, and make a left turn.''  And, when the
driver got there, ``Now take this street five blocks and make a right turn.''
And on and on.  You know the details, but it is a lot quicker to just
tell the driver to take you to the airport.

Even the statement ``Go down this street ten blocks\,$\ldots$'' can be broken
down further with instructions on using the accelerator, the steering wheel,
watching out for other vehicles, pedestrians, etc.

Abstraction is a technique for establishing a simpler way for a person to
interact with a system, removing the details that are unnecessary for the 
person to interact effectively with that system.  Our ability to abstract is 
very much a productivity enhancer.  It allows us to deal with a situation 
at a higher level, focusing
on the essential aspects, while keeping the component ideas in the background.
It allows us to be more efficient in our use of time and brain activity.
It allows us to not get bogged down in the detail when everything about the
detail is working just fine.

There is an underlying assumption to this, however: {\em when everything about 
the detail is just fine}.  What if everything about the detail is not just
fine? Then, to be successful, our ability to abstract must be combined with our
ability to {\em un}-abstract.  Some people use the word {\em deconstruct}---the
ability to go from the abstraction back to its component parts.


Two stories come to mind.

The first involves a trip through Arizona the first author made a long time ago
in the hottest part of the summer.  At the time he was living in Palo Alto,
California, where the temperature tends to be mild almost always.  He knew 
enough to take the car to a mechanic before making the trip, and tell him to 
check the cooling system.  That was the abstraction: cooling system.  What he
 had not mastered was that the capability of a cooling system for Palo Alto, 
California is not the same as the capability of a cooling system for the 
summer deserts of Arizona.  The result: two days in Deer Lodge, Arizona 
(population 3), waiting for a head gasket to be shipped in.

The second story (perhaps apocryphal) is supposed to have happened during
the infancy of electric power generation.  General Electric Co. was having
trouble with one of its huge electric power generators and did not know what
to do.  On the front of the generator were lots of dials containing lots of
information, and lots of screws that could be rotated clockwise or
counterclockwise as the operator wished.  Something on the other side of the
wall of dials and screws was malfunctioning and no one knew what to do. 
As the story goes, they called in one of the early giants in
the electric power industry.  He looked at the dials and listened to the
noises for a minute, then took a small screwdriver from his pocket
and rotated one screw 35 degrees counterclockwise.  The problem
immediately went away.  He submitted a bill for \$1{,}000 (a lot of money in
those days) without any elaboration.  The controller found the bill for
two minutes' work a little unsettling, and asked for further clarification.
Back came the new bill:

\begin{Verbatim}[fontsize=\fontsize{9}{10}]
     Turning a screw 35 degrees counterclockwise:  $  0.75
     Knowing which screw to turn and by how much:   999.25
\end{Verbatim}
%$

In both stories the message is the same.  It is more efficient to
think of entities as abstractions.  One does not want to get bogged
down in details unnecessarily.  And as long as nothing untoward
happens, we are OK.  If there had been no trip to Arizona,
the abstraction ``cooling system'' would have been sufficient.  If the
electric power generator never malfunctioned, there would have been no
need for the power engineering guru's deeper understanding.

As we will see, modern computers are comprised of transistors. These 
transistors are combined to form logic "gates" -- an abstraction that lets us 
think in terms of 0s and 1s instead of the varying voltages on the 
transistors.  A logic circuit is a further abstraction of a combination of 
gates.  When one designs a logic circuit out of gates, it is much more
efficient to not have to think about the internals of each gate.  To
do so would slow down the process of designing the logic circuit.  One
wants to think of the gate as a component.  But if there is a problem
with getting the logic circuit to work, it is often helpful to look at
the internal structure of the gate and see if something about its
functioning is causing the problem.

When one designs a sophisticated computer application program, whether
it be a new spreadsheet program, word processing system, or computer
game, one wants to think of each of the components one is using as an
abstraction.  If one spent time thinking about the details of each 
component when it was not necessary, the distraction could easily
prevent the total job from ever getting finished.  But when there is a
problem putting the components together, it is often useful to examine
carefully the details of each component in order to uncover the
problem.

The ability to abstract is a most important skill.  In our
view, one should try to keep the level of abstraction as high as possible,
consistent with getting everything to work effectively.  Our approach in this
book is to continually raise the level of abstraction.  We describe logic
gates in terms of transistors.  Once we understand the abstraction of gates,
we no longer think in terms of transistors.  Then we build larger structures
out of gates.  Once we understand these larger abstractions, we no longer think
in terms of gates.

\paragraph{The Bottom Line}

Abstractions allow us to be much more efficient in dealing
with all kinds of situations.  It is also true that one can be effective
without understanding what is below the abstraction as long as everything
behaves nicely.  So, one should not pooh-pooh the notion of abstraction.  On the
contrary, one should celebrate it since it allows us to be more efficient.

In fact, if we never have to combine a component with anything else into
a larger system, and if nothing can go wrong with the component, then it is
perfectly fine to understand this component only at the level of its
abstraction.

\looseness=1
But if we have to combine multiple components into a larger system, we
should be careful not to allow their abstractions to be the deepest level
of our understanding.  If we don't know the components below the level of their
abstractions, then we are at the mercy of them working together without our
intervention.  If they don't work together, and we are unable to go below the
level of abstraction, we are stuck.  And that is the state we should take
care not to find ourselves~in.

%1.3.2
\subsection{Hardware versus Software}

\looseness=-1
Many computer scientists and engineers refer to themselves as hardware
people or software people.  By hardware, they generally mean the
physical computer and all the specifications associated with it.  By
software, they generally mean the programs, whether operating systems
like Android, ChromeOS, Linux or Windows, or database systems like Access, MongoDB, Oracle or DB-terrific, or application programs like Facebook, Chrome, Excel or Word.  The implication is that
the person knows a whole lot about one of these two things and
precious little about the other.  Usually, there is the further
implication that it is OK to be an expert at one of these (hardware OR
software) and clueless about the other.  It is as if there were a big
wall between the hardware (the computer and how it actually works) and
the software (the programs that direct the computer to do their bidding), and
that one should be content to remain on one side of that wall or
the~other.

The power of abstraction allows us to "usually" operate at a level where we
do not have to think about the underlying layers all the time.  This is a
good thing. It enables us to be more productive.  But if we are clueless
about the underlying layers, then we are not able to take advantage of the
nuances of those underlying layers when it is very important to be able to. 

That is not to say that you must work at the lower level of abstraction
and not take advantage of the productivity enhancements that abstraction
provides.  On the contrary, you are encouraged to work at the highest level
of abstraction available to you.  But in doing so, if you are able to, at
the same time, keep in mind the underlying levels, you will find yourself
able to do a much better job. 

As you approach your study and practice of computing, we urge you to take 
the approach that hardware and software are names for components of
two parts of a computing system that work best when they are designed by
people who take into account the capabilities and limitations of both.

Microprocessor designers who understand the needs of the programs that will
execute on the microprocessor they are designing can design much more
effective microprocessors than those who don't.  For example, Intel, AMD, 
ARM and other major producers of microprocessors recognized a
few years ago that a large fraction of future programs would contain video
clips as part of e-mail, video games, and full-length movies.  They
recognized that it would be important for such programs to execute 
efficiently.  The result: most microprocessors today contain special 
hardware capability to process those video clips.  Intel defined 
additional instructions, initially called their MMX instruction set,
and developed special hardware for it.  Motorola, IBM and Apple did 
essentially the same thing, resulting in the AltiVec instruction set and
special hardware to support it.  

A similar story can be told about software designers.
The designer of a large computer program who understands the capabilities
and limitations of the hardware that will carry out the tasks of that program
can design the program so it executes more efficiently than the designer 
who does not 
understand the nature of the hardware.  One important task that almost all
large software systems need to carry out is called sorting, where a number of
items have to be arranged in some order.  The words in a dictionary are
arranged in alphabetical order.  Students in a class are often graded based 
on a numerical order, according to their scores on the final exam.  There is 
a large number of fundamentally different programs one can write to arrange a 
collection of items in order.  Donald Knuth, one of the top computer 
scientists in the world, devoted 391 pages to the task in 
{\em The Art of Computer Programming}, vol.~3.  Which sorting program works 
best is often very dependent on how much the software designer is aware of 
the underlying characteristics of the hardware.

\vspace{-12pt}
\paragraph{The Bottom Line}

We believe that whether your inclinations are in the direction of a computer
hardware career or a computer software career, you will be much
more capable if you master both.  This book is about getting you started on 
the path to mastering both hardware and software.  Although we sometimes 
ignore making the point explicitly when we are in the trenches of working 
through a concept, it really is the case that each sheds light on the other.

When you study data types, a software concept, in C (Chapter~12), you
will understand how the finite word length of the computer, a hardware
concept, affects our notion of data types.

When you study functions in C (Chapter~14), you will be able to tie
the {\em rules} of calling a function with the
hardware implementation that makes those rules necessary.

When you study recursion, a powerful algorithmic device (initially in 
Chapter~8 and more extensively in
Chapter~17), you will be able to tie it to the hardware.  If you take the 
time to do that, you will better understand when the additional time to 
execute a procedure recursively is worth it.

When you study pointer variables in C (in Chapter~16), your knowledge of
computer memory will provide a deeper understanding of what pointers provide,
and very importantly, when they should be used, and when 
they should be avoided.

When you study data structures in C (in Chapter~19), your knowledge of
computer memory will help you better understand
what must be done to manipulate the actual structures in memory efficiently.

We realize that most of the terms in the preceding five short paragraphs
may not be familiar to you {\em yet}.  That is OK; you can reread this page
at the end of
the semester.  What is important to know right now is that there are important
topics in the software that are very deeply interwoven with topics in the
hardware.  Our contention is that mastering either is easier if you pay
attention to both.

Most importantly, most computing problems yield better solutions when the 
problem solver has the capability of both at his or her disposal.

%1.4
\section{A Computer System}

We have used the word {\em computer} more than two dozen times in the
preceding pages, and although we did not say so explicitly,
we used it to mean a system consisting of the software (i.e., computer programs)
that directs 
and specifies the processing of information and the hardware that performs the 
actual processing of information in response to what the software asks the
hardware to do.  When we say ``performing the actual processing,'' we mean 
doing the actual additions, multiplications, and so forth in the hardware 
that are necessary to get the job done. A more precise term for this
hardware is a {\em central processing unit} (CPU), or simply a
{\em processor} or {\em microprocessor}.  This textbook is primarily about 
the processor and the programs that are executed by the processor.

%1.4.1
\subsection{A (very) little history for a (lot) better perspective}

Before we get into the detail of how the processor and the software
associated with it work, we should take a moment and note the enormous and
unparalleled leaps of performance that the computing industry has made in the
relatively short time computers have been around.  After all, it wasn't until
the 1940s that the first computers showed their faces.  One of the first 
computers was the ENIAC (the Electronic Numerical Integrator and Calculator), 
a general purpose electronic computer that could be reprogrammed for different 
tasks.  It was designed  and built in 1943-1945 at the University of 
Pennsylvania by Presper Eckert and his colleagues.  It contained more than 
17,000 vacuum tubes.  It was approximately 8 feet high, more than 100 feet wide,
and about 3 feet deep (about 300 square feet of floor space).  It weighed 
30 tons, and required 140 KWatts to operate.  Figure~\ref{fig:eniac} shows 
three operators programming the ENIAC by plugging and unplugging cables 
and switches.

%Figure 1.1
\begin{figure}[b]
\vspace{2pt}
\centerline{\includegraphics[width=2.9in,height=3.09in]{eniac.eps}}
\caption{The ENIAC, designed and built at Unversity of Pennsylvania, 1943-45}
\label{fig:eniac}
\end{figure}


\FloatBarrier
About 40 years and many computer companies and computers later, in the early 
1980s, the Burroughs A series was born.  One of the dozen or so 18 inch boards 
that comprise that machine is shown in 
Figure~\ref{fig:cpu_board}.  Each board contained 50 or more integrated 
circuit packages.  Instead of 300 square feet, it took up around 50-60 square 
feet, instead of 30 tons, it weighed about 1 ton, and instead of 140 Kwatts,
it required approximately 25 Kwatts to operate.

%Figure 1.2
\begin{figure}
\centerline{\includegraphics[width=4.20333in,height=3.75in]{pat67509_0101.eps}}
\caption{A processor board, vintage 1980s (Courtesy of Emilio Salgueiro, Unisys Corporation.)}
\label{fig:cpu_board}
\end{figure}

\FloatBarrier

Fast forward another 30 or so years and we find many of today's computers 
on desk tops (Figure~\ref{fig:desktop}), laptops (Figure~\ref{fig:laptop}) 
and most recently in smartphones (Figure~\ref{fig:smartphone}).  Their 
relative weights and energy requirements have decreased enormously, and the 
speed at which they process information has increased also enormously.  
We estimate that the computing power in a smart phone today (that is, how fast
we can compute with a smart phone) is more than four million times 
the computing power of the ENIAC!

%Figure 1.3
\begin{figure}
\vspace{2pt}
\centerline{\includegraphics[width=2.9in,height=3.09in]{desktop.eps}}
\caption{A desktop computer.}
\label{fig:desktop}
\end{figure}


%Figure 1.4
\begin{figure}
\centerline{\includegraphics[width=3.74in,height=2.31333in]{laptop.eps}}
\caption{A laptop.}
\label{fig:laptop}
\vspace{12pt}
\end{figure}


%Figure 1.5
\begin{figure}
\centerline{\includegraphics[width=3.74in,height=2.31333in]{smartphone.eps}}
\caption{A smartphone.}
\label{fig:smartphone}
\vspace{12pt}
\end{figure}


\FloatBarrier
The integrated circuit packages that comprise modern digital computers have
also seen phenomenal improvement.  An example of one of today's microprocessors
is shown in Figure~\ref{fig:microprocessor}.  The first microprocessor, 
the Intel 4004 in 1971,
contained 2300 transistors, and operated at 106 KHz.  By 1992, those
numbers had jumped to 3.1 million transistors at a frequency of 66 MHz on the
Intel Pentium microprocessor, an increase in both parameters of a factor of
about 1000.  
Today's microprocessors contain upwards of 5 billion transistors and can 
operate at upwards of 4 GHz, another increase in both parameters of about a 
factor of 1000.

This factor of one million since 1971 in both the number of transistors and 
the frequency that the microprocessor operates at has had very important 
implications.  The fact that each operation can be 
performed in one millionth of the time it took in 1971 means the 
microprocessor can do one million things today in the time it took to do one 
thing in 1971.  The fact that there are more than a million times as many 
transistors on a chip means we can do a lot 
more things at the same time today than we could in 1971.  

The result of all this is we have today computers that seem able to
understand the languages people speak -- English, Spanish, Chinese, for
example.  We have computers that seem able to recognize faces.  Many see
this as the magic of artificial intelligence.  We will see as we get into 
the details of how a computer works that much of what appears to be magic 
is really due to how blazingly fast very simple mindless operations (many 
at the same time) can be carried out.

%Figure 1.6
\begin{figure}
\centerline{\includegraphics[width=3.74in,height=2.31333in]{microprocessor.eps}}
\caption{A microprocessor.}
\label{fig:microprocessor}
\vspace{12pt}
\end{figure}

\FloatBarrier

%1.4.2
\subsection{The parts of a computer system}

When most people use the word computer, they
usually mean more than just the processor (i.e., CPU) that is in charge of 
doing what the software directs.  They usually mean the
collection of parts that in combination form their {\em computer
system} \index{computer system}.  Today that computer system is often a laptop
(see Figure~\ref{fig:laptop}), augmented with many additional devices.

A computer system generally includes, in addition to the processor, a keyboard
for typing commands, a mouse or keypad or joystick for positioning on menu 
entries, a monitor for displaying information that the computer system has
produced, memory for temporarily storing information, disks and USB memory 
sticks of one sort or another for storing information for a very long
time, even after the computer has been turned off, connections to other 
devices such as a printer for obtaining paper copies of that information,
and the collection of programs (the software) that the user wishes to~execute.

All these items help the computer user do
his or her job.  Without a printer, for example, the user would
have to copy by hand what is displayed on the monitor.  Without a
mouse, keypad, or joystick, the user would have to type each command, 
rather than simply position the mouse, keypad, or joystick.

So, as we begin our journey, which focuses on the CPU which occupies a small
fraction of 1 square inch of silicon and the software that makes the CPU do
our bidding, we note that the computer systems we use contain a lot of 
additional components.

%1.5
\vspace{-6pt}
\section{Two Very Important Ideas}
\label{sec:two_ideas}

Before we leave this first chapter, there are two
very important ideas that we would like you to understand,
ideas that are at the core of what computing is all about.

\begin{altdescription}
  \item[Idea 1:]  All computers (the biggest and the smallest, the fastest and
  the slowest, the most expensive and the cheapest) are capable of
  computing exactly the same things if they are given enough time and
  enough memory.  That is, anything a fast computer can do, a slow
  computer can do also.  The slow computer just does it more slowly.
  A more expensive computer cannot figure out something that a cheaper
  computer is unable to figure out as long as the cheaper computer can
  access enough memory.  (You may have to go to the store to buy more memory
  whenever it runs out of memory in order to keep increasing memory.)
  {\bfseries All} computers can do {\bfseries exactly} the same things.  Some
  computers can do things faster, but none can do {\bfseries more} than any\break
  other.

\item[Idea 2:] We describe our problems in English or some other language
    spoken by people.  Yet the problems are solved by electrons running
    around inside the computer.  It is necessary to transform our problem
    from the language of humans to the voltages that influence the flow
    of electrons.  This transformation is really a sequence of systematic
    transformations, developed and improved over the last 70 years, which
    combine to give the computer the ability to carry out what appears to
    be some very complicated tasks.  In reality, these tasks are simple
    and straightforward.
\end{altdescription}

The rest of this chapter is devoted to discussing these two ideas.

%1.6
\vspace{-6pt}
\section{Computers as Universal Computational Devices}

It may seem strange that an introductory textbook begins by describing
how computers work.  After all, mechanical engineering students begin
by studying physics, not how car engines work.  Chemical engineering
students begin by studying chemistry, not oil refineries.  Why should
computing students begin by studying computers?

The answer is that computers are different.
To learn the fundamental principles
of computing, you must study computers or machines that can do what
computers can do.  The reason for this has to do with the notion that
computers are
{\em universal computational devices}.  Let's see what that means.



\looseness=-1
Before modern computers, there were many kinds of
calculating machines.  Some were {\em analog machines}---machines that
produced an answer by measuring some physical quantity such as
distance or voltage.  For example, a slide rule is an analog machine
that multiplies numbers by sliding one logarithmically graded ruler
next to another.  The user can read a logarithmic ``distance'' on the
second ruler.  Some early analog adding machines worked by dropping
weights on a scale.  The difficulty with analog machines is that it is
very hard to increase their accuracy.

\looseness=-1
This is why {\em digital machines}---machines that perform
computations by manipulating a fixed finite set of digits or
letters---came to dominate computing.  You are familiar with the
distinction between analog and digital watches.  An analog watch has
hour and minute hands, and perhaps a second hand.  It gives the time
by the positions of its hands, which are really angular measures.
Digital watches give the time in digits.  You can increase accuracy
just by adding more digits.  For example, if it is important for you
to measure time in hundredths of a second, you can buy a watch that
gives a reading like 10:35.16 rather than just 10:35.  How would you
get an analog watch that would give you an accurate reading to one
one-hundredth of a second?  You could do it, but it would take a
mighty long second hand!  When we talk about computers in this book,
we will always mean digital machines.

Before modern digital computers, the most common
digital machines in the West were adding machines.
In other parts of the world another digital machine, the abacus,
was common.
Digital adding machines were mechanical or electromechanical devices
that could perform a specific kind of computation: adding integers.
There were also digital machines that could multiply integers.
There were digital machines that could put a stack of cards
with punched names in alphabetical order.  The main limitation
of all these machines is that they could do only
one specific kind of computation.
If you owned only an adding machine and wanted
to multiply two integers, you had some pencil-and-paper work to do.

This is why computers are different.  You can tell a computer
how to add numbers.  You can tell it how to multiply.  You can
tell it how to alphabetize a list or perform any
computation you like.
When you think of a new kind of computation, you
do not have to buy or design a new computer.
You just give the old computer a new set of instructions (or program)
to carry out the new computation.  This is why we say the computer
is a {\em universal computational device}.  Computer
scientists believe that
{\em anything that can be computed, can be computed by a computer}
provided it has enough time and enough memory.
When we study computers, we
study the fundamentals of all computing.
We learn what computation is and what can be computed.

\looseness=1
The idea of a
universal computational device is due to Alan Turing.
\index{Turing, Alan}
Turing proposed in 1937 that all computations could be
carried out by a particular kind of machine, which is now called
a Turing machine.  He gave a mathematical description of this
kind of machine, but did not actually build one.
Digital computers were not operating until several years later.  
Turing was more interested in solving a \nobreak philosophical
problem: defining computation.  He began by looking at the kinds
of actions that people perform when they compute; these
include making marks on paper, \nobreak writing symbols according
to certain rules when other symbols are
present, and so on.  He abstracted these actions and specified a mechanism
that could carry them out.  He gave some examples
of the kinds of things that these machines could do.  One
Turing machine could add two integers;  another
\index{Turing machine}
could multiply two integers.

Figure~\ref{fig:turing_machine} shows what we call ``black box''
\index{black box model}
models of Turing machines that add and multiply.  In each case,
the operation to be performed is described in the box.  The data
on which to operate is shown as inputs to the box.  The result
of~the operation is shown as output from the box.  A black
box model provides no information as to exactly how the
operation is performed, and indeed, there are many ways to
add or multiply two numbers.

%Figure 1.7
\begin{figure}[t]
\centerline{\includegraphics{pat67509_0104.eps}}
\caption{Black box models of Turing machines}
\label{fig:turing_machine}
\end{figure}

Turing proposed that every computation
can be performed by some Turing machine.  We call this
{\em Turing's thesis}.  Although Turing's thesis has never
been proved, there does exist a lot of evidence to suggest it is true.
We know, for example, that various enhancements one can make to Turing machines
do not result in machines that can compute more.

Perhaps the best argument to support Turing's thesis was provided by Turing
himself in his original paper.
He said that one way to try to construct a machine more powerful
than any particular Turing machine was to make a machine $U$ that could
simulate {\em all} Turing machines.  You would simply describe to
$U$ the particular Turing machine you wanted it to
simulate, say a machine to add two integers, give $U$ the input
data, and $U$ would compute the appropriate
output, in this case the sum of the inputs.
Turing then showed that there was, in fact, a~Turing machine that could do this, so even this attempt
to find something that could not be computed by
Turing machines failed.

Figure~\ref{fig:utm}  further illustrates the point.  Suppose you 
wanted to
compute \mbox{$g\cdot(e+f)$.} You would simply provide to $U$
descriptions of the Turing machines to add and to multiply,
and the three inputs, $e$, $f$, and $g$.  $U$ would do the
rest.

%Figure 1.8
\begin{figure}[hb]
\centerline{\includegraphics{pat67509_0105.eps}}
\caption{Black box model of a universal Turing machine}
\label{fig:utm}

\end{figure}

In specifying $U$, Turing had provided us with a deep insight:
He had given us the first description of what computers do.
In fact, both a computer (with as much memory as
it wants) and a universal Turing machine can compute exactly
\index{universal Turing machine}
the same things. In both cases you give the machine a
description of a computation  and the data it needs,
and the machine computes the appropriate answer.
Computers and universal Turing
machines can compute anything that can be computed because
they are {\em programmable}.

This is the reason that a big or expensive computer cannot do
more than a small, cheap computer.  More money
may buy you a faster computer, a monitor with higher resolution,
or a nice sound system.  But if you have a small, cheap computer,
you already have a universal computational device.

\vspace{-12pt}
%1.7
\section{How Do We Get the Electrons to Do the Work?}

%Figure 1.9
\begin{figure}
\vspace{-6pt}
\centerline{\includegraphics[scale=0.8]{pat67509_0106.eps}}
\caption{Levels of transformation}
\label{fig:levels}
\end{figure}

Figure~\ref{fig:levels} shows the process we must go through to get the 
electrons (which actually do the work) to do our bidding.  We call the 
steps\vadjust{\vfill\pagebreak}
of this process the ``Levels of Transformation.''  As we will see, at
\index{levels of transformation}
each level we have choices.  If we ignore any of the levels, our
ability to make the best use of our computing system can be very adversely
affected.

\FloatBarrier
%1.7.1
\subsection{The Statement of the Problem}

We describe the problems we wish to solve in a ``natural
\nobreak language.''  Natural languages are languages that people speak, like
English, French, Japanese, Italian, and so on.
They have evolved over centuries in accordance
with their usage.  They are fraught with a lot of things unacceptable
for providing instructions to a computer.  Most important of these
unacceptable attributes is ambiguity.  Natural language is filled with
ambiguity. To infer the meaning of a sentence, a listener is often helped by
the tone of voice of the speaker, or at the very least, the context of
the sentence.

An example of ambiguity in English is the sentence, ``Time flies like an
arrow.''  At least three interpretations are possible, depending on whether
(1) one~is noticing how fast time passes, (2) one is at a track meet for
insects, or (3) one is writing a letter to the Dear Abby of Insectville.
In the first case, a simile; one is comparing the speed of time passing
to the speed of an arrow that
has been released.  In the second case, one is telling the
timekeeper to do his/her job much like an arrow would.  In the
third case, one is relating that a particular group of flies (time flies,
as opposed to fruit flies) are all in love with the same arrow.

Such ambiguity would be unacceptable in instructions provided to a computer.
The computer, electronic idiot that it is, can only do as it is told.  To
tell it to do something where there are multiple interpretations would
cause the computer to not know which interpretation to follow.

%1.7.2
\subsection{The Algorithm}

The first step in the sequence of transformations is to transform the
natural language description of the problem to an algorithm, and in so doing,
\index{algorithm}
get rid of the objectionable characteristics of the natural language.  
An algorithm is a
step-by-step procedure that is guaranteed to terminate, such that each step
is precisely stated and can be carried out by the computer.  There
are terms to describe each of these properties.

We use the term {\em definiteness} to describe the notion that each step is
\index{definiteness}
precisely stated.  A recipe for excellent pancakes that instructs the
preparer to ``stir until lumpy'' lacks definiteness, since the notion of
lumpiness is not precise.

We use the term {\em effective computability} to describe the notion that
\index{effective computability}
each step can be carried out by a computer.  A procedure that instructs
the computer to ``take the largest prime number'' lacks effective computability,
since there is no largest prime number.

We use the term  {\em finiteness} to describe the notion that the procedure
terminates.
\index{finiteness}

For every problem there are usually many different algorithms for
solving that problem.  One algorithm may require the fewest number of
steps.  Another algorithm may allow some steps to be performed concurrently.
A computer that allows more than one thing to be done at a time
can often solve the problem in less time, even though it is
likely that the total number of steps to be performed has increased.

\enlargethispage{-1pc}

%1.7.3
\subsection{The Program}

The next step is to transform the algorithm into a computer program in one
of the programming languages that are available.  Programming languages are
``mechanical languages.''  That is, unlike natural languages, mechanical
languages did
not evolve through human discourse.  Rather, they were invented for use in
specifying a sequence of instructions to a computer.  Therefore,
mechanical languages do not suffer from failings such as ambiguity that would
make them unacceptable for specifying a computer program.

There are more than 1{,}000 programming languages.  Some have been
designed for use with particular applications, such as Fortran for solving
scientific calculations and COBOL for solving business data-processing 
problems.  In the second half of this book, we will use C and C++, languages 
that were designed for manipulating low-level hardware structures.

Other languages are useful for still other purposes.  Prolog is the language
of choice for many applications that require the design of an expert system.
LISP was for years the language of choice of a substantial number of
people working on problems dealing with artificial intelligence.  Pascal is
a language invented as a vehicle for teaching beginning students how to
program.

There are two kinds of programming languages, high-level languages
and low-level
\index{high-level language}
languages.  High-level languages are at a distance (a high level) from
the underlying computer.  At their best, they are independent of
the computer on which the programs will execute.  We say the language
is ``machine independent.''
All the languages mentioned thus far are high-level languages.
Low-level languages are tied to the computer on which the programs
will execute.  There is generally one such low-level language
for each computer.  That language is called the {\em assembly
language} for that computer.
\index{assembly language}

%1.7.4
\subsection{The ISA}

The next step is to translate the program into the instruction set of 
the particular computer that will be used to carry out the work of the 
program.  The instruction set architecture (ISA)
\index{Instruction Set Architecture}
\index{ISA}
is the complete specification of
the interface between programs that have been written and the underlying
computer hardware that must carry out the work of those programs.

An analogy that may be helpful in understanding the concept of an ISA is
provided by the automobile.  Corresponding to a computer program, 
represented as a sequence of 0s and 1s in the case of the computer, is 
the human sitting in the driver's seat of a car.  Corresponding to the 
microprocessor hardware is the car, itself.  The "ISA" of the automobile is 
the specification of everything 
the human needs to know to tell the automobile what to do, and everything 
the automobile needs to know to carry out the tasks specified by the human 
driver.  For example, one element of the automobile's "ISA" is the pedal 
on the floor known as the "brake, and its function.  The human knows that 
if he/she steps on the brake the car will stop.  The automobile knows that 
if it feels pressure from the human on that pedal, the hardware of the 
automobile must engage those elements necessary to stop the car.  The full 
"ISA" of the car includes the specification of the other pedals, the 
steering wheel, the ignition key, the gears, windshield wipers, etc.  
For each, the "ISA" specifies (a) what the human has to do to tell the 
automobile what he/she wants done, and (b) correspondingly, what the 
automobile will interpret those actions to mean so it (the automobile) 
can carry out the specified task.

The ISA of a computer serves the same purpose as the "ISA" of an automobile,
except instead of the driver and the car, the ISA of a computer specifies 
the interface between the computer program directing the computer hardware
and the hardware carrying out those directions.  For example, the set of
instructions that the computer can carry out; that is, what operations the
computer can perform and where to get the data that is needed to perform those
operations.  The 
term {\em opcode} is used to decribe the operation.  The term {\em operand} 
is used to describe individual data values.  The ISA specifies the
\index{operand}
\index{data type}
acceptable representations for operands.  They are called
{\em data types.} A {\em data type} is
a representation of an operand such that the computer can perform operations
on that representation.  The ISA specifies the mechanisms that the computer 
can use to figure out where the operands are located.
These mechanisms are called {\em addressing modes}.\index{addressing mode}

The number of opcodes, data types, and addressing modes specified by an ISA
vary among different ISAs. Some ISAs have as few as a half dozen opcodes,
whereas others have as many as several hundred.  Some ISAs have only one 
data type,
while others have more than a dozen.  Some ISAs have one or two addressing
modes, whereas others have more than 20.  The x86, the ISA used in the PC,  
has more than 200 opcodes, more than a dozen data types, and more than 
two dozen addressing modes.

The ISA also specifies the number of unique locations that comprise the
computer's memory and the number of individual 0s and 1s that are
contained in each location.

Many ISAs are in use today.  The most widely-known example is the x86, 
introduced by Intel Corporation in 1979 and currently also manufactured 
by AMD and other companies.  Other ISAs and the companies responsible for 
them include ARM and THUMB (ARM), POWER and z/Architecture (IBM), 
and SPARC (Oracle).

The translation from a high-level language (such as C) to the ISA of the
computer on which the program will execute (such as x86) is usually done by a
translating program called a {\em compiler}.  To translate from a program 
written in C to the x86 ISA, one would need \index{compiler}
a C to x86 compiler.  For each high-level language and each desired
target ISA, one must provide a corresponding compiler.

The translation from the unique assembly language of a computer
to its ISA is done by an assembler.
\index{assembler}

\enlargethispage{-\baselineskip}

%1.7.5
\subsection{The Microarchitecture}

The next step is the implementation of the ISA, referred to as 
its {\em microarchitecture}.  The automobile analogy that we used in our 
discussion of the ISA is also useful in showing the relationship between 
an ISA and a microarchitecture that implements that ISA.  The automobile's 
"ISA" describes what the driver needs to know as he/she sits inside the 
automobile to make the automobile carry out the driver's wishes.  All 
automobiles have the same ISA.  If there are three pedals on the floor, 
it does not matter what manufacturer produced the car, the middle one is 
always the brake.  The one on the right is always the accelerator, and the 
more it is depressed, the faster the car will move.  Because there is only 
one ISA for automobiles, one does not need one driver's license for Buicks 
and a different driver's license for Hondas. 

The microarchitecture (or implementation) of the automobile's ISA, on the 
other hand, is about what goes on underneath the hood.  Here all automobile 
makes and models can be different, depending on what cost/performance 
trade-offs the automobile designer made before the car was manufactured.  
Some automobiles come with disc brakes, others (in the past, at least) with 
drums.  Some automobiles have eight cylinders, others run on six cylinders, 
and still others have only four.  Some are turbocharged, some are not.  
Some automobiles can travel 60 miles on one gallon of gasoline, others are
lucky to travel from gas station to the next without running out of gas.
Some automobiles cost 6,000 US dollars, others cost 200,000 US dollars.
In each case, the \nobreak ``microarchitecture'' of the specific automobile 
is a result of the automobile designers' decisions regarding the tradeoffs 
of cost and performance.  The fact that the "microarchitecture" of every model 
or make is different is a good reason to take one's Honda, when it is 
malfunctioning, to a Honda repair person, and not to a Buick repair person.

In the previous section we identified ISAs of several computer manufacturers, 
including the x86 (Intel), the PowerPC (IBM and Motorola), and THUMB (ARM).
Each has been implemented by many different 
microarchitectures. \index{miroarchitecture} 
For example, the x86's original implementation in 1979 was the 8086, 
followed by the 80286, 80386, and 80486 in the 1980s.  More recently, in 2001, 
Intel introduced the Pentium IV microprocessor.  Even more recently, in 2015, 
Intel introduced Skylake.  Each of these x86 microprocessors has its own 
microarchitecture.

The story is the same for the PowerPC ISA, with more than a dozen 
different microprocessors, each having its own microarchitecture.  

Each microarchitecture is an opportunity for computer designers to make 
different trade-offs between the cost of the microprocessor, the performance 
that the microprocessor will provide, and the energy that is required to power 
the microprocessor.  Computer design is always an exercise in trade-offs, 
as the designer opts for higher (or lower) performance, more (or less) energy 
required, at greater (or lesser) cost.

%1.7.6
\subsection{The Logic Circuit}

The next step is to implement each element of the microarchitecture out of simple
logic circuits.  Here, also, there are choices, as the logic designer decides how
\index{logic circuit}
to best make the trade-offs between cost and performance.  So, for example, 
even for an operation as simple as addition, there are several choices of logic circuits to perform
the operation at differing speeds and corresponding costs.

%1.7.7
\subsection{The Devices}

Finally, each basic logic circuit is implemented in accordance with the requirements
of the particular device technology used.  So, CMOS circuits are different from
\index{CMOS}
NMOS circuits, which are different, in turn, from gallium arsenide\break circuits.

\paragraph{The Bottom Line}

In summary, from the natural language description of a problem to the electrons 
that actually solve the problem by moving from one voltage potential to 
another, many transformations need to be performed.  If we could speak 
electron, or if the electrons could understand English, perhaps we could just 
walk up to the computer and get the electrons to do our bidding.  Since we 
can't speak electron and they can't speak English, the best we can do is this
systematic sequence of transformations.  At each level of transformation, 
there are choices as to how to proceed.  Our handling of those choices 
determines the resulting cost and performance of our computer.

\looseness=1
In this book, we describe each of these transformations.  We show how
transistors combine to form logic circuits, how logic circuits combine to form
the microarchitecture, and how the microarchitecture implements 
a particular ISA.  In our case, the ISA is the LC-3.  We complete the process 
by going from the
English-language description of a problem to a C or C++ program that solves 
the problem, and we show how that C or C++ program is translated 
(i.e., compiled) to the ISA of the~LC-3.

We hope you enjoy the ride.

\begin{exercises}

\item[1.1]
[1] Explain the first of the two important ideas stated in Section~1.5.

\item[1.2]
[1] Can a higher-level programming language instruct a computer to
compute more than a lower-level programming language?

\item[1.3]
[1] What difficulty with analog computers encourages computer designers
to use digital designs?

\item[1.4]
[1] Name one characteristic of natural languages that prevents them
from being used as programming languages.

\item[1.5]
[5] Say we had a ``black box,'' which takes two numbers as input and
outputs their sum.  See Figure~1.7a. Say
we had another box capable of multiplying two numbers together.
See Figure~1.7b.  We can connect these boxes
together to calculate $p\times(m+n)$.  See
Figure~1.7c. Assume we have an unlimited number of these boxes.
Show how to connect them together to calculate:

\begin{enumerate}
        \item[a.] $ax+b$

        \item[b.] The average of the four input numbers
$w$, $x$, $y$, and $z$

        \item[c.] $a^2+2ab+b^2$ (Can you do it with one
add box and one multiply box?)
\end{enumerate}

%Figure 1.7
\begin{figure}[b]
\centerline{\includegraphics{pat67509_0107.eps}}
\caption{``Black boxes'' capable of (a) addition, (b) multiplication, and
  (c)~a combination of addition and multiplication
\label{ex_fig:black_boxes}}
\end{figure}

\item[1.6]
[1] Write a statement in a natural language and offer two different
interpre\-tations of that statement.

\item[1.7]
[3] The discussion of abstraction in Section~1.3.1 noted that one does not
need to understand the makeup of the components as long as ``everything about
the detail is just fine.''  The case was made that when everything is not fine,
one must be able to deconstruct the components, or be at the mercy of the
abstractions.  In the taxi example, suppose
you did not understand the component, that is, you had no clue how to
get to the airport.  Using the notion of abstraction, you simply tell the
driver,\break
``Take me to the airport.''  Explain when this is a productivity enhancer, and
when it could result in very negative consequences.

\item[1.8]
[5] John said, ``I saw the man in the park with a telescope.'' What
did he mean? How many reasonable interpretations can you provide
for this statement? List them. What property does this sentence
demonstrate that makes it unacceptable as a statement in a program.

\item[1.9]
[1] Are natural languages capable of expressing algorithms?

\item[1.10]
[1] Name three characteristics of algorithms. Briefly explain each
of these three characteristics.

\item[1.11]
[4] For each characteristic of an algorithm, give an example of a
procedure that does not have the characteristic, and is
therefore not an algorithm.

\item[1.12]
[5] Are items $a$ through $e$ in the following list algorithms? If not, what
qualities required of algorithms do they lack?

\begin{enumerate}[b.]
        \item[a.]
        Add the first row of the following matrix to another
row whose first column contains a nonzero entry. ({\em Reminder}:
Columns run vertically; rows run horizontally.)
$$
        \left[
        \begin{matrix}
        1 & 2 & 0 & 4 \\
        0 & 3 & 2 & 4 \\
        2 & 3 & 10 & 22 \\
        12 & 4 & 3 & 4
        \end{matrix}
        \right]
$$

\item[b.]
In order to show that there are as many prime numbers as there are
natural numbers, match each prime number with a natural number in
the following manner. Create pairs of prime and natural numbers
by matching the first prime number with 1 (which is the first
natural number) and the second prime number with 2, the third
with 3, and so forth. If, in the end, it turns out that each
prime number can be paired with each natural number, then it
is shown that there are as many prime numbers as natural
numbers.

\item[c.]
Suppose you're given two vectors each with 20 elements
and asked to perform the
following operation. Take the first element of the first
vector and multiply it by the first element of the second
vector.  Do the same to the second elements, and so forth.
Add all the individual products together to derive the
dot product.

\item[d.]
Lynne and Calvin are trying to decided who will take the dog
for a walk.  Lynne suggests that they flip a coin and pulls
a quarter out of her pocket.  Calvin does not trust Lynne
and suspects that the quarter may be weighted (meaning that
it might favor a particular outcome when tossed) and suggests
the following procedure to fairly determine who will walk the
dog.

\begin{enumerate}
\item[\rm1.] Flip the quarter twice.

\item[\rm2.] If the outcome is heads on the first flip and tails on the
second, then I will walk the dog.

\item[\rm3.] If the outcome is tails on the first flip, and heads on the
second, then you will walk the dog.

\item[\rm4.] If both outcomes are tails or both outcomes are heads,
then we flip twice again.
\end{enumerate}

Is Calvin's technique an algorithm?


\item [e.]
Given a number, perform the following steps in order:
\begin{enumerate}
\item[\rm1.] Multiply it by four
\item[\rm2.] Add four
\item[\rm3.] Divide by two
\item[\rm4.] Subtract two
\item[\rm5.] Divide by two
\item[\rm6.] Subtract one
\item[\rm7.] At this point, add one to a counter to keep track of the fact
that you performed steps 1 through 6. Then test the result you got
when you subtracted one. If 0, write down the number of times you performed
steps 1 through 6 and stop. If not 0, starting with the result of subtracting
1, perform the above 7 steps again.
\end{enumerate}
\end{enumerate}

\smallskip

\item[1.13]
[4] Two computers, A and B, are identical except for the fact that A
has a subtract instruction and B does not. Both have add instructions.
Both have instructions that can take a value and produce the negative
of that value. Which computer is able to solve more problems, A or B?
Prove your result.

\smallskip

\item[1.14]
[4] Suppose we wish to put a set of names in alphabetical order. We
call the act of doing so {\em sorting}. One algorithm that can
accomplish that is called the bubble sort. We could then program
our bubble sort algorithm in C, and compile the C program to
execute on an x86 ISA. The x86 ISA can be implemented with an
Intel Pentium IV microarchitecture. Let us call the sequence
``Bubble Sort, C program, x86 ISA, Pentium IV microarchitecture''
one {\em transformation process}.

\hspace*{18pt}Assume we have available four sorting algorithms and can program
in C, C++, Pascal, Fortran, and COBOL. We have available compilers
that can translate from each of these to either x86 or SPARC, and
we have available three different microarchitectures for x86 and
three different microarchitectures for SPARC.

\begin{enumerate}
        \item[a.] How many transformation processes are possible?
        \item[b.] Write three examples of transformation processes.
        \item[c.] How many transformation processes are possible if
instead of three different microarchitectures for x86 and three
different microarchitectures for SPARC, there were two for x86
and four for SPARC?
\end{enumerate}

\smallskip

\item[1.15]
[7] Identify one advantage of programming in a higher-level language compared to
a lower-level language. Identify one disadvantage.

\smallskip

\item[1.16]
[1] Name at least three things specified by an ISA.

\smallskip

\item[1.17]
[1] Briefly describe the difference between an ISA and a microarchitecture.

\item[1.18]
[4] How many ISAs are normally implemented by a single
microarchitecture? Conversely, how many microarchitectures
could exist for a single ISA?

\item[1.19]
[1] List the levels of transformation and name an example for each level.

\item[1.20]
[4] The levels of transformation in Figure~1.6 are often referred to
as levels of abstraction. Is that a reasonable characterization? If yes,
give an example. If no, why not?

\item[1.21]
[7] Say you go to the store and buy some word processing software.
What form is the software actually in?  Is it in a high-level
programming language? Is it in assembly language? Is it in
the ISA of the computer on which you'll run it? Justify
your answer.

\item[1.22] 
[5] Suppose you were given a task at one of the transformation
  levels shown in Figure~1.6, and required to tranform it to the level
  just below. At which level would it be most difficult to perform the
  transformation to the next lower level? Why?

\item[1.23] 
[5] Why is an ISA unlikely to change between successive
  generations of microarchitectures that implement it?  For example,
  why would Intel want to make certain that the ISA implemented by the
  Pentium III is the same as the one implemented by the Pentium II?  {\em
    Hint:} When you upgrade your computer (or buy one with a newer
  CPU), do you need to throw out all your old software?
\end{exercises}
\end{document}

\end{document}
