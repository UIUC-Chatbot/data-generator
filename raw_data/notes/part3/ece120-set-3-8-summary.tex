\classtitle

\subsection{Summary of Part 3 of the Course}


%In this short summary, we 
%give you lists at several levels of difficulty 
%of what we expect you to be able to do as a result of the last few weeks 
%of studying (reading, listening, doing homework, discussing your 
%understanding with your classmates, and so forth).

Students often find this part of the course more challenging than the
earlier parts of the course.
%
In addition to these notes, you should read Chapters~4 and~5 of the 
Patt and Patel textbook, which cover the von Neumann
model, instruction processing, and ISAs.  

%Students typically find that the homeworks in this part of the course
%require more time than did those in earlier parts of the course.
%Problems on the exam will be similar in nature but designed to require 
%less actual time to solve (assuming that you have been doing the homeworks).  

%We'll start with the easy stuff.  

You should recognize all of these terms and be able
to explain what they mean.  For the specific circuits, you should be able 
to draw them and explain how they work.  Actually, we don't care whether 
you can draw something from memory---a mux, for example---provided that 
you know what a mux does and can derive a gate diagram correctly for one 
in a few minutes.  Higher-level skills are much more valuable.

\begin{minipage}[t]{2.75in}
\begin{list}{$\bullet$}{\setlength{\itemsep}{0pt}\setlength{\parskip}{0pt}%
\setlength{\topsep}{0pt}\setlength{\partopsep}{0pt}\setlength{\parsep}{0pt}}

\item{digital systems terms
\begin{list}{--}{\setlength{\itemsep}{0pt}\setlength{\parskip}{0pt}%
\setlength{\topsep}{0pt}\setlength{\partopsep}{0pt}\setlength{\parsep}{0pt}}
\item module
\item fan-in
\item fan-out
\item machine models: Moore and Mealy
\end{list}
}

\item{simple state machines
\begin{list}{--}{\setlength{\itemsep}{0pt}\setlength{\parskip}{0pt}%
\setlength{\topsep}{0pt}\setlength{\partopsep}{0pt}\setlength{\parsep}{0pt}}
\item synchronous counter
\item ripple counter
\item serialization (of bit-sliced design)
\end{list}
}

\item{finite state machines (FSMs)
\begin{list}{--}{\setlength{\itemsep}{0pt}\setlength{\parskip}{0pt}%
\setlength{\topsep}{0pt}\setlength{\partopsep}{0pt}\setlength{\parsep}{0pt}}
\item states and state representation
\item transition rule
\item self-loop
\item next state (+) notation
\item meaning of don't care in input \\ combination
\item meaning of don't care in output
\item unused states and initialization
\item completeness (with regard to \\ FSM specification)
\item list of (abstract) states
\item next-state table/state transition table/state table
\item state transition diagram/transition \\ diagram/state diagram
\end{list}
}

\item{memory
\begin{list}{--}{\setlength{\itemsep}{0pt}\setlength{\parskip}{0pt}%
\setlength{\topsep}{0pt}\setlength{\partopsep}{0pt}\setlength{\parsep}{0pt}}
\item number of addresses
\item addressability
\item read/write logic
\item serial/random access memory (RAM)
\item volatile/non-volatile (N-V)
\item static/dynamic RAM (SRAM/DRAM)
\item SRAM cell
\item DRAM cell
\item design as a collection of cells
\item coincident selection
%\item bit lines and sense amplifiers
\end{list}
}

\end{list}
\end{minipage}\hfill%
\begin{minipage}[t]{3.1in}
\begin{list}{$\bullet$}{\setlength{\itemsep}{0pt}\setlength{\parskip}{0pt}%
\setlength{\topsep}{0pt}\setlength{\partopsep}{0pt}\setlength{\parsep}{0pt}}

\item{von Neumann model
\begin{list}{--}{\setlength{\itemsep}{0pt}\setlength{\parskip}{0pt}%
\setlength{\topsep}{0pt}\setlength{\partopsep}{0pt}\setlength{\parsep}{0pt}}
\item{processing unit
\begin{list}{--}{\setlength{\itemsep}{0pt}\setlength{\parskip}{0pt}%
\setlength{\topsep}{0pt}\setlength{\partopsep}{0pt}\setlength{\parsep}{0pt}}
\item register file
\item arithmetic logic unit (ALU)
\item word size
\end{list}
}
\item{control unit
\begin{list}{--}{\setlength{\itemsep}{0pt}\setlength{\parskip}{0pt}%
\setlength{\topsep}{0pt}\setlength{\partopsep}{0pt}\setlength{\parsep}{0pt}}
\item program counter (PC)
\item instruction register (IR)
\item implementation as FSM
\end{list}
}
\item input and output units
\item{memory
\begin{list}{--}{\setlength{\itemsep}{0pt}\setlength{\parskip}{0pt}%
\setlength{\topsep}{0pt}\setlength{\partopsep}{0pt}\setlength{\parsep}{0pt}}
\item memory address register (MAR)
\item memory data register (MDR)
\end{list}
}
\item{processor datapath}
\item{bus}
\item{control signal}
\end{list}
}

\item{tri-state buffer
\begin{list}{--}{\setlength{\itemsep}{0pt}\setlength{\parskip}{0pt}%
\setlength{\topsep}{0pt}\setlength{\partopsep}{0pt}\setlength{\parsep}{0pt}}
\item meaning of Z/hi-Z output
\item use in distributed mux
\end{list}
}

\item{instruction processing}
\begin{list}{-}{\setlength{\itemsep}{0pt}\setlength{\parskip}{0pt}%
\setlength{\topsep}{0pt}\setlength{\partopsep}{0pt}\setlength{\parsep}{0pt}}
\item{fetch}
\item{decode}
\item{execute}
\item{register transfer language (RTL)}
\end{list}

\item{Instruction Set Architecture (ISA)}
\begin{list}{-}{\setlength{\itemsep}{0pt}\setlength{\parskip}{0pt}%
\setlength{\topsep}{0pt}\setlength{\partopsep}{0pt}\setlength{\parsep}{0pt}}
\item{instruction encoding}
\item{field (of an encoded instruction)}
\item{operation code (opcode)}
\item{types of instructions}
\begin{list}{-}{\setlength{\itemsep}{0pt}\setlength{\parskip}{0pt}%
\setlength{\topsep}{0pt}\setlength{\partopsep}{0pt}\setlength{\parsep}{0pt}}
\item{operations}
\item{data movement}
\item{control flow}
\end{list}
\item{addressing modes}
\begin{list}{-}{\setlength{\itemsep}{0pt}\setlength{\parskip}{0pt}%
\setlength{\topsep}{0pt}\setlength{\partopsep}{0pt}\setlength{\parsep}{0pt}}
\item{immediate}
\item{register}
\item{PC-relative}
\item{indirect}
\item{base + offset}
\end{list}
\end{list}


\end{list}
\end{minipage}

\vfill

\pagebreak

We expect you to be able to exercise the following skills:

\begin{list}{$\bullet$}{\setlength{\itemsep}{0pt}\setlength{\parskip}{0pt}%
\setlength{\topsep}{0pt}\setlength{\partopsep}{0pt}\setlength{\parsep}{0pt}}

\item{Transform a bit-sliced design into a serial design, and explain the 
tradeoffs involved in terms of area and time required to compute a result.}
\item{Based on a transition diagram, implement a synchronous counter from 
flip-flops and logic gates.}
\item{Implement a binary ripple counter (but not necessarily a more general 
type of ripple counter) from flip-flops and logic gates.}
\item{Given an FSM implemented as digital logic, analyze the FSM to produce 
a state transition diagram.}
\item{Design an FSM to meet an abstract specification for a task, including 
production of specified output signals, and possibly including selection 
of appropriate inputs.}
\item{Complete the specification of an FSM by ensuring that each state 
includes a transition rule for every possible input combination.}
\item{Compose memory chips into larger memory systems, using additional
decoders when necessary.}
\item{Encode \mbox{LC-3} instructions into machine code.}
\item{Read and understand programs written in \mbox{LC-3} assembly/machine code.}
\end{list}

At a higher level, we expect that you understand the concepts and ideas 
sufficiently well to do the following:

\begin{list}{$\bullet$}{\setlength{\itemsep}{0pt}\setlength{\parskip}{0pt}%
\setlength{\topsep}{0pt}\setlength{\partopsep}{0pt}\setlength{\parsep}{0pt}}

\item{Abstract design symmetries from an FSM specification in order to 
simplify the implementation.}
\item{Make use of a high-level state design, possibly with many sub-states 
in each high-level state, to simplify the implementation.}
\item{Use counters to insert time-based transitions between states (such 
as timeouts).}
\item{Implement an FSM using logic components such as registers, 
counters, comparators, and adders as building blocks.}
\item{Explain the basic organization of a computer's microarchitecture
as well as the role played by elements of a von Neumann design in the
processing of instructions.}
\item{Identify the stages of processing an instruction (such as fetch,
decode, getting operands, execution, and writing back results) in a 
processor control unit state machine diagram.}
\end{list}

And, at the highest level, we expect that you will be able to do the following:

\begin{list}{$\bullet$}{\setlength{\itemsep}{0pt}\setlength{\parskip}{0pt}%
\setlength{\topsep}{0pt}\setlength{\partopsep}{0pt}\setlength{\parsep}{0pt}}

\item{Explain the difference between the Moore and Mealy machine models, 
as well as why you might find each of them useful when designing an FSM.}
\item{Understand the need for initialization of an FSM, be able to analyze 
and identify potential problems arising from lack of initialization, and 
be able to extend an implementation to include initialization to an 
appropriate state when necessary.}
\item{Understand how the choice of internal state bits for an FSM can 
affect the complexity of the implementation of next-state and output 
logic, and be able to select a reasonable state assignment.}
\item{Identify and fix design flaws in simple FSMs by analyzing an existing 
implementation, comparing it with the specification, and removing any 
differences by making any necessary changes to the implementation.}

\end{list}

\pagebreak

\mbox{~~~} % empty 3rd page

\vfill

\pagebreak

