\classtitle

\subsection{Summary of Part 1 of the Course}

This short summary provides a list of both terms that we expect you to
know and and skills that we expect you to have after our first few weeks
together.  The first part of the course is shorter than the other three
parts, so the amount of material is necessarily less.
%
These notes supplement the Patt and Patel textbook, so you will also 
need to read and understand the relevant chapters (see the syllabus)
in order to master this material completely.

According to educational theory, the difficulty of learning depends on 
the type of task involved.  Remembering new terminology is relatively 
easy, while applying the ideas underlying design decisions shown by 
example to new problems posed as human tasks is relatively hard.
%
In this short summary, we give you lists at several levels of difficulty
of what we expect you to be able to do as a result of the last few weeks
of studying (reading, listening, doing homework, discussing your
understanding with your classmates, and so forth).

This time, we'll list the skills first and leave the easy stuff for the 
next page.
%
We expect you to be able to exercise the following skills:

\begin{list}{$\bullet$}{\setlength{\itemsep}{0pt}\setlength{\parskip}{0pt}%
\setlength{\topsep}{0pt}\setlength{\partopsep}{0pt}\setlength{\parsep}{0pt}}

\item{Represent decimal numbers with unsigned, 2's complement, and IEEE
floating-point representations, and be able to calculate the decimal value
represented by a bit pattern in any of these representations.}

\item{Be able to negate a number represented in the 2's complement
representation.}

\item{Perform simple arithmetic by hand on unsigned and 2's complement
representations, and identify when overflow occurs.}

\item{Be able to write a truth table for a Boolean expression.}

\item{Be able to write a Boolean expression as a sum of minterms.}

% MOVED TO PART 4
%
% \item{Be able to calculate the Hamming distance of a code/representation.}
% 
% \item{Know the relationships between Hamming distance and the abilities
% to detect and to correct bit errors.}

\item{Know how to declare and initialize C~variables with one of the 
primitive data types.}

\end{list}

At a more abstract level, we expect you to be able to:

\begin{list}{$\bullet$}{\setlength{\itemsep}{0pt}\setlength{\parskip}{0pt}%
\setlength{\topsep}{0pt}\setlength{\partopsep}{0pt}\setlength{\parsep}{0pt}}

\item{Understand the value of using a common mathematical basis, such
as modular arithmetic, in defining multiple representations (such as
unsigned and 2's complement).}

\item{Write Boolean expressions for the overflow conditions
on both unsigned and 2's complement addition.}

% MOVED TO PART 4
%
% \item{Be able to use parity for error detection, and Hamming codes for
% error correction.}

\item{Be able to write single {\tfix if} statements and {\tfix for} loops
in~C in order to perform computation.}

\item{Be able to use {\tfix scanf} and {\tfix printf} for basic input and 
output in~C.}

\end{list}

And, at the highest level, we expect that you will be able to reason about
and analyze problems in the following ways:

\begin{list}{$\bullet$}{\setlength{\itemsep}{0pt}\setlength{\parskip}{0pt}%
\setlength{\topsep}{0pt}\setlength{\partopsep}{0pt}\setlength{\parsep}{0pt}}

\item{Understand the tradeoffs between integer
%  FIXME?     not covered by book nor notes currently 
%, fixed-point,    
and floating-point representations for numbers.}

\item{Understand logical completeness and be able to prove or disprove
logical completeness for sets of logic functions.}

% PARTIALLY MOVED TO PART 4
%
% \item{Understand the properties necessary in a representation, and understand
% basic tradeoffs in the sparsity of code words with error detection and
% correction capabilities.}
\item{Understand the properties necessary in a representation: no ambiguity
in meaning for any bit pattern, and agreement in advance on the meanings of 
all bit patterns.}

\item{Analyze a simple, single-function C~program and be able to explain its purpose.}

\end{list}

\vfill

\pagebreak

You should recognize all of these terms
and be able to explain what they mean.
%(the parentheses give page numbers,
%or ``P\&P'' for Patt \& Patel).
%
%Actually, we don't care whether you can draw something from memory---a full
%adder, for example---provided that you know what a full adder does and can
%derive a gate diagram correctly for one in a few minutes.  Higher-level
%skills are much more valuable.  (You may skip the *'d terms in Fall~2012.)
%
Note that we are not saying that you should, for example, be able to 
write down the ASCII~representation from memory.  In that example, 
knowing that it is a~\mbox{7-bit} representation used for English
text is sufficient.  You can always look up the detailed definition 
in practice.

\begin{minipage}[t]{3.0in}
\begin{list}{$\bullet$}{\setlength{\itemsep}{0pt}\setlength{\parskip}{0pt}%
\setlength{\topsep}{0pt}\setlength{\partopsep}{0pt}\setlength{\parsep}{0pt}}

\item{universal computational devices / \\ computing machines %(\pageref{one:hm:ucd})
\begin{list}{--}{\setlength{\itemsep}{0pt}\setlength{\parskip}{0pt}%
\setlength{\topsep}{0pt}\setlength{\partopsep}{0pt}\setlength{\parsep}{0pt}}
\item undecidable %(\pageref{one:hm:undecidable})
\item the halting problem %(\pageref{one:hm:haltingprob})
\end{list}
}

% pre-Fall 2015 version
%
% \item{Turing machines
% \begin{list}{--}{\setlength{\itemsep}{0pt}\setlength{\parskip}{0pt}%
% \setlength{\topsep}{0pt}\setlength{\partopsep}{0pt}\setlength{\parsep}{0pt}}
% \item universal computational device/\\ computing machine
% \item intractable/undecidable
% \item the halting problem
% \end{list}
% }

\item{information storage in computers
\begin{list}{--}{\setlength{\itemsep}{0pt}\setlength{\parskip}{0pt}%
\setlength{\topsep}{0pt}\setlength{\partopsep}{0pt}\setlength{\parsep}{0pt}}
\item bits %(\pageref{one:tcr:bits})
\item representation %(P\&P)
\item data type %(\pageref{one:tcr:bits})
\item unsigned representation %(\pageref{one:tcr:unsigned})
\item 2's complement representation
%
% FIXME?  not covered by book nor notes currently
% \item fixed-point representation
%
\item IEEE floating-point representation
\item ASCII representation
%\item equivalence classes
\end{list}
}

\item{operations on bits
\begin{list}{--}{\setlength{\itemsep}{0pt}\setlength{\parskip}{0pt}%
\setlength{\topsep}{0pt}\setlength{\partopsep}{0pt}\setlength{\parsep}{0pt}}
\item 1's complement operation
\item carry (from addition)
\item overflow (on any operation) %(\pageref{one:tcr:overflow})
\item Boolean logic and algebra
\item logic functions/gates
\item truth table
\item AND/conjunction
\item OR/disjunction
\item NOT/logical complement/\\ (logical) negation/inverter
\item XOR
\item logical completeness
\item minterm
\end{list}
}

\item{mathematical terms
\begin{list}{--}{\setlength{\itemsep}{0pt}\setlength{\parskip}{0pt}%
\setlength{\topsep}{0pt}\setlength{\partopsep}{0pt}\setlength{\parsep}{0pt}}
\item modular arithmetic
\item implication
\item contrapositive
\item proof approaches: by construction,\\ by contradiction, by induction
\item without loss of generality (w.l.o.g.)
\end{list}
}

% MOVED TO PART 4
%
% \item{error detection and correction
% \begin{list}{--}{\setlength{\itemsep}{0pt}\setlength{\parskip}{0pt}%
% \setlength{\topsep}{0pt}\setlength{\partopsep}{0pt}\setlength{\parsep}{0pt}}
% \item code/sparse representation
% \item code word
% \item bit error
% \item odd/even parity bit
% \item Hamming distance between code words
% \item Hamming distance of a code
% \item Hamming code
% \item SEC-DED
% \end{list}
% }

\end{list}
\end{minipage}\hfill%
\begin{minipage}[t]{2.85in}
\begin{list}{$\bullet$}{\setlength{\itemsep}{0pt}\setlength{\parskip}{0pt}%
\setlength{\topsep}{0pt}\setlength{\partopsep}{0pt}\setlength{\parsep}{0pt}}

\item{high-level language concepts
\begin{list}{--}{\setlength{\itemsep}{0pt}\setlength{\parskip}{0pt}%
\setlength{\topsep}{0pt}\setlength{\partopsep}{0pt}\setlength{\parsep}{0pt}}
\item syntax
\item{variables
\begin{list}{--}{\setlength{\itemsep}{0pt}\setlength{\parskip}{0pt}%
\setlength{\topsep}{0pt}\setlength{\partopsep}{0pt}\setlength{\parsep}{0pt}}
\item declaration
\item primitive data types
\item symbolic name/identifier
\item initialization
\end{list}
}
% FIXME?  really not necessary for them
%\item strongly typed languages
\item expression
\item statement
\end{list}
}

\item{C operators
\begin{list}{--}{\setlength{\itemsep}{0pt}\setlength{\parskip}{0pt}%
\setlength{\topsep}{0pt}\setlength{\partopsep}{0pt}\setlength{\parsep}{0pt}}
\item operands
\item arithmetic
\item bitwise
\item comparison/relational
\item assignment
\item address
\item arithmetic shift
\item logical shift
\item precedence
\end{list}
}

\item{functions in C
\begin{list}{--}{\setlength{\itemsep}{0pt}\setlength{\parskip}{0pt}%
\setlength{\topsep}{0pt}\setlength{\partopsep}{0pt}\setlength{\parsep}{0pt}}
\item {\tfix main}
\item function call
\item arguments
\item {{\tfix printf} and {\tfix scanf}
\begin{list}{--}{\setlength{\itemsep}{0pt}\setlength{\parskip}{0pt}%
\setlength{\topsep}{0pt}\setlength{\partopsep}{0pt}\setlength{\parsep}{0pt}}
\item format string
\item escape character
\end{list}
}
\item {\tfix sizeof} %(built-in)
\end{list}
}

\item{transforming tasks into programs
\begin{list}{--}{\setlength{\itemsep}{0pt}\setlength{\parskip}{0pt}%
\setlength{\topsep}{0pt}\setlength{\partopsep}{0pt}\setlength{\parsep}{0pt}}
\item flow chart
\item sequential construct
\item conditional construct
\item iterative construct/iteration/loop
\item loop body
\end{list}
}

\item{C statements
\begin{list}{--}{\setlength{\itemsep}{0pt}\setlength{\parskip}{0pt}%
\setlength{\topsep}{0pt}\setlength{\partopsep}{0pt}\setlength{\parsep}{0pt}}
\item statement:\\ null, simple, compound
\item {\tfix if} statement
\item {\tfix for} loop
\item {\tfix return} statement
\end{list}
}

% THESE ARE NOT REQUIRED TOPICS
%
% \item{execution of C programs
% \begin{list}{--}{\setlength{\itemsep}{0pt}\setlength{\parskip}{0pt}%
% \setlength{\topsep}{0pt}\setlength{\partopsep}{0pt}\setlength{\parsep}{0pt}}
% \item compiler/interpreter
% \item source code
% \item header files
% \item assembly code
% \item instructions
% \item executable image
% \end{list}
% }
% 
% \item{the C preprocessor
% \begin{list}{--}{\setlength{\itemsep}{0pt}\setlength{\parskip}{0pt}%
% \setlength{\topsep}{0pt}\setlength{\partopsep}{0pt}\setlength{\parsep}{0pt}}
% \item \#include directive
% \item \#define directive
% \end{list}
% }

\end{list}
\end{minipage}

\pagebreak

\mbox{~~~}  % blank 3rd page

\vfill

\pagebreak

