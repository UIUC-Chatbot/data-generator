\classtitle

\subsection{Using Abstraction to Simplify Problems}

In this set of notes, we illustrate the use of abstraction to simplify
problems, then introduce a component called a multiplexer that allows
selection among multiple inputs.
%
We begin by showing how two specific 
examples---integer subtraction and identification of letters
in ASCII---can be implemented using logic functions that we have already
developed.  We also introduce a conceptual technique for
breaking functions into smaller pieces, which allows us to solve
several simpler problems and then to compose a full solution from 
these partial solutions.

Together with the idea of bit-sliced designs that we introduced earlier,
these techniques help to simplify the process of designing logic that
operates correctly.  The techniques can, of course, lead to 
less efficient designs,
but {\em correctness is always more important than performance}.
%
The potential loss of efficiency is often acceptable for three reasons.
%
First, as we mentioned earlier, computer-aided design tools for 
optimizing logic functions are fairly effective, and in many cases
produce better results than human engineers (except in the rare cases 
in which the human effort required to beat the tools is worthwhile).
%
Second, as you know from the design of the~\mbox{2's complement} 
representation, we may be able to reuse specific pieces of hardware if
we think carefully about how we define our problems and representations.
%
Finally, many tasks today are executed in software, which is designed
to leverage the fairly general logic available via an instruction set
architecture.  A programmer cannot easily add new logic to a user's
processor.  As a result, the hardware used to execute a
function typically is not optimized for that function.
%
The approaches shown in this set of notes illustrate how abstraction
can be used to design logic.\\

\subsubsection{Subtraction}

Our discussion of arithmetic implementation has focused so far on 
addition.  What about other operations, such as subtraction, multiplication,
and division?  The latter two require more work, and we will not
discuss them in detail until later in our class (if at all).

Subtraction, however, can be performed almost trivially using logic that
we have already designed.
%
Let's say that we want to calculate the difference~$D$ between 
two~\mbox{$N$-bit} numbers~$A$ and~$B$.  In particular, we want 
to find~$D=A-B$.  For now, think of~$A$, $B$, and~$D$ 
as~2's~complement values.
%
Recall how we defined the~2's~complement representation: 
the~\mbox{$N$-bit} pattern that we use to represent~$-B$ is the 
same as the base~2 bit pattern for~$(2^N-B)$, so we can use an adder if we
first calculate the bit pattern for~$-B$, then add the resulting
pattern to~$A$.
As you know, our~\mbox{$N$-bit} adder always produces a result that
is correct modulo~$2^N$, so the result of such an operation,
$D=2^N+A-B$, is correct so long as the subtraction does not overflow.

\begin{minipage}{4.7in}
How can we calculate~$2^N-B$?  The same way that we do by hand!
Calculate the~1's~complement, $(2^N-1)-B$, then add~1.
%
The diagram to the right shows how we can use the~\mbox{$N$-bit} adder
that we designed in Notes Set~2.3 to build an~\mbox{$N$-bit} subtracter.
%
New elements appear in blue in the figure---the rest of the logic
is just an adder.  The box labeled ``1's comp.'' calculates 
the~\mbox{1's complement} of the value~$B$, which together with the
carry in value of~1 correspond to calculating~$-B$.  What's in the
``1's comp.'' box?  One inverter per bit in~$B$.  That's all 
we need to calculate the~1's~complement.
%
You might now ask: does this approach also work for unsigned numbers?
The answer is yes, absolutely.  However, the overflow conditions for
both~2's complement and unsigned subtraction are different than the
overflow condition for either type of addition.  What does the carry
out of our adder signify, for example?  The answer may not be 
immediately obvious.
\end{minipage}\hspace{.25in}%
\begin{minipage}{1.55in}
\epsfig{file=part2/figs/subtracter.eps,width=1.55in}
\end{minipage}

Let's start with the overflow condition for unsigned subtraction.
%
Overflow means that we cannot represent the result.  With an~\mbox{$N$-bit}
unsigned number, we have~$A-B\not\in[0,2^N-1]$.  Obviously, the difference
cannot be larger than the upper limit, since~$A$ is representable and
we are subtracting a non-negative (unsigned) value.  We can thus assume
that overflow occurs only when $A-B<0$.  In other words, when~$A<B$.

\pagebreak

To calculate the unsigned subtraction overflow condition in terms of the
bits, recall that our adder is calculating~$2^N+A-B$.  The carry out
represents the~$2^N$ term.  When $A\geq{B}$, the result of the adder
is at least~$2^N$, and we see a carry out, $C_{out}=1$.  However, when
$A<B$, the result of the adder is less than~$2^N$, and we see no carry
out,~$C_{out}=0$.  {\em Overflow for unsigned subtraction is thus inverted 
from overflow for unsigned addition}: a carry out of~0 indicates an 
overflow for subtraction.

What about overflow for~2's~complement subtraction?  We can use arguments
similar to those that we used to reason about overflow of~2's~complement
addition to prove that subtraction of one negative number from a second
negative number can never overflow.  Nor can subtraction of a non-negative
number from a second non-negative number overflow.

If~$A\geq{0}$ and~$B<0$, the subtraction overflows iff~$A-B\geq{2^{N-1}}$.
Again using similar arguments as before, we can prove that the difference~$D$
appears to be negative in the case of overflow, so the product
$\overline{A_{N-1}}~B_{N-1}~D_{N-1}$ evaluates to~1 when this type of
overflow occurs (these variables represent the most significant bits of the 
two operands and the difference; in the case of~2's complement, they are
also the sign bits).
%
Similarly, if~$A<0$ and~$B\geq{0}$, we have overflow when~$A-B<-2^{N-1}$.
Here we can prove that~$D\geq{0}$ on overflow, so 
$A_{N-1}~\overline{B_{N-1}}~\overline{D_{N-1}}$ evaluates to~1.

Our overflow condition for~\mbox{$N$-bit}~2's~complement subtraction
is thus given by the following:
%
\begin{eqnarray*}
\overline{A_{N-1}}~B_{N-1}~D_{N-1}+A_{N-1}~\overline{B_{N-1}}~\overline{D_{N-1}}
\end{eqnarray*}
%
If we calculate all four overflow conditions---unsigned and~2's~complement,
addition and subtraction---and provide some way to choose whether or not to 
complement~$B$ and to control the~$C_{in}$ input, we can use the same hardware
for addition and subtraction of either type.\\

\subsubsection{Checking ASCII for Upper-case Letters}

Let's now consider how we can check whether or not an ASCII character is
an upper-case letter.  Let's call the~\mbox{7-bit}
letter~$C=C_6C_5C_4C_3C_2C_1C_0$ and the function that we want to
calculate~$U(C)$.  The function~$U$ should equal~1 whenever~$C$ represents
an upper-case letter, and should equal~0 whenever~$C$ does not.

In ASCII, the~\mbox{7-bit} patterns from~0x41 through~0x5A correspond
to the letters~A through~Z in alphabetic order.  Perhaps you want to draw 
a~\mbox{7-input} \mbox{K-map}?  Get a few large sheets of paper!
%
Instead, imagine that we've written the full~\mbox{128-row} truth
table.  Let's break the truth table into pieces.  Each piece will
correspond to one specific pattern of the three high bits~$C_6C_5C_4$,
and each piece will have~16 entries for the four low bits~$C_3C_2C_1C_0$.
The truth tables for high bits 000, 001, 010, 011, 110, and 111
are easy: the function is exactly~0.  The other two truth tables appear 
on the left below.  We've called the two functions~$T_4$ and~$T_5$, where
the subscripts correspond to the binary value of the three high bits of~$C$.\\

\centerline{
\begin{minipage}{1.95in}
\begin{tabular}{cccc|cc}
$C_3$& $C_2$& $C_1$& $C_0$& $T_4$& $T_5$\\ \hline
0& 0& 0& 0& 0& 1\\
0& 0& 0& 1& 1& 1\\
0& 0& 1& 0& 1& 1\\
0& 0& 1& 1& 1& 1\\ \hline
0& 1& 0& 0& 1& 1\\
0& 1& 0& 1& 1& 1\\
0& 1& 1& 0& 1& 1\\
0& 1& 1& 1& 1& 1\\ \hline
1& 0& 0& 0& 1& 1\\
1& 0& 0& 1& 1& 1\\
1& 0& 1& 0& 1& 1\\
1& 0& 1& 1& 1& 0\\ \hline
1& 1& 0& 0& 1& 0\\
1& 1& 0& 1& 1& 0\\
1& 1& 1& 0& 1& 0\\
1& 1& 1& 1& 1& 0\\
\end{tabular}
\end{minipage}\hspace{.5in}%
\begin{minipage}{1.15in}
\epsfig{file=part2/figs/ascii-t4.eps,width=1.15in}\vspace{36pt}
\epsfig{file=part2/figs/ascii-t5.eps,width=1.15in}
\end{minipage}\hspace{.5in}%
\begin{minipage}{1.8in}
\begin{eqnarray*}
T_4&=&C_3+C_2+C_1+C_0\\ \\ \\ \\ \\ \\ \\
T_5&=&\overline{C_3}+\overline{C_2}~\overline{C_1}+\overline{C_2}~\overline{C_0}
\end{eqnarray*}
\end{minipage}}

As shown to the right of the truth tables, we can then draw simpler
\mbox{K-maps} for~$T_4$ and~$T_5$, and can solve the \mbox{K-maps}
to find equations for each, as shown to the right (check that you get
the same answers).

How do we merge these results to form our final expression for~$U$?
%
We AND each of the term functions~($T_4$ and~$T_5$) with the 
appropriate minterm for the
high bits of~$C$, then OR the results together, as shown here:

\begin{eqnarray*}
U&=&C_6~\overline{C_5}~\overline{C_4}~T_4+C_6~\overline{C_5}~C_4~T_5\\
\\
&=&C_6~\overline{C_5}~\overline{C_4}~(C_3+C_2+C_1+C_0)+
C_6~\overline{C_5}~C_4~(\overline{C_3}+\overline{C_2}~\overline{C_1}+\overline{C_2}~\overline{C_0})
\end{eqnarray*}

Rather than trying to optimize by hand, we can at this point let the CAD
tools take over, confident that we have the right function to identify
an upper-case ASCII letter.

Breaking the truth table into pieces and using simple logic to reconnect
the pieces is one way to make use of abstraction when solving complex
logic problems.  
%
In fact, recruiters for some companies often ask
questions that involve using specific logic elements as building blocks
to implement other functions.  Knowing that you can implement
a truth table one piece at a time will help you to solve this type of
problem.

Let's think about other ways to tackle the problem of calculating~$U$.
In Notes Sets~2.3 and~2.4, we developed adders and comparators.  Can
we make use of these components as building blocks to check whether~$C$ 
represents an
upper-case letter?  Yes, of course we can: by comparing~$C$ with the
ends of the range of upper-case letters, we can check whether or not~$C$
falls in that range.

The idea is illustrated on the left below using two~\mbox{7-bit} comparators
constructed as discussed in Notes Set~2.4.
The comparators are the black parts of the drawing, while the blue parts
represent our extensions to calculate~$U$.  Each comparator is given
the value~$C$ as one input.  The second value to the comparators is 
either the letter~A~(0x41) or the letter~Z~(0x5A).  The meaning of 
the~\mbox{2-bit} input to and result of each comparator is given in the 
table on the right below.  The inputs on the right of each comparator
are set to~0 to ensure that equality is produced if~$C$ matches
the second input~($B$).
%
One output from each comparator is then routed to
a NOR gate to calculate~$U$.  Let's consider how this combination works.
The left comparator compares~$C$ with the letter~A~(0x41).  If~$C\geq$~0x41,
the comparator produces~$Z_0=0$.  In this case, we may have a letter.
On the other hand, if~$C<$~0x41, the comparator produces~$Z_0=1$, and
the NOR gate outputs~$U=0$, since we do not have a letter in this case.
%
The right comparator compares~$C$ with the letter~Z~(0x5A).  If~$C\leq$~0x5A,
the comparator produces~$Z_1=0$.  In this case, we may have a letter.
On the other hand, if~$C>$~0x5A, the comparator produces~$Z_1=1$, and
the NOR gate outputs~$U=0$, since we do not have a letter in this case.
%
Only when~0x41~$\leq{C}\leq$~0x5A does~$U=1$, as desired.\\ 

\vfill

\centerline{
\begin{minipage}{3.6in}
\epsfig{file=part2/figs/ascii-cmp-based.eps,width=3.6in}
\end{minipage}\hspace{.75in}%
\begin{minipage}{1.35in}
\begin{tabular}{cc|l}
$Z_1$& $Z_0$& meaning\\ \hline
0& 0& $A=B$\\
0& 1& $A<B$\\
1& 0& $A>B$\\
1& 1& not used\\
\end{tabular}
\end{minipage}}\vspace{12pt}

\vfill
\vfill

\pagebreak

\begin{minipage}{3.5in}
What if we have only~\mbox{8-bit} adders available for our use, such as
those developed in Notes Set~2.3?  Can we still calculate~$U$?  Yes.
The diagram shown to the right illustrates the approach, again with black
for the adders and blue for our extensions.  Here we are 
actually using the adders as subtracters, but calculating the~1's~complements
of the constant values by hand.  The ``zero extend'' box simply adds
a leading~0 to our~\mbox{7-bit} ASCII letter.  The left adder 
subtracts the letter~A from~$C$: if no carry is produced, we know that
$C<$~0x41 and thus~$C$ does not represent an upper-case letter, and~$U=0$.
%
Similarly, the right adder subtracts~0x5B (the letter~Z plus one)
from~$C$.  If a carry is produced, we know that~$C\geq$~0x5B, and 
thus~$C$ does not represent an upper-case letter, and~$U=0$.
%
With the right combination of carries (1 from the left and 0 from the
right), we obtain~$U=1$.
\end{minipage}\hspace{.25in}%
\begin{minipage}{2.75in}
\epsfig{file=part2/figs/ascii-add-based.eps,width=2.75in}
\end{minipage}

Looking carefully at this solution, however, you might be struck by the
fact that we are calculating two sums and then discarding them.  Surely
such an approach is inefficient?

We offer two answers.  First, given the design shown above, a good CAD tool
recognizes that the sum outputs of the adders are not being used,
and does not generate logic to calculate them.  The logic for the two 
carry bits used to calculate~$U$ can then be optimized.  Second, the design
shown, including the calculation of the sums, is similar in efficiency
to what happens at the rate of about~$10^{15}$ times per second, 
24~hours a day, seven days a week,
inside processors in data centers processing HTML, XML, and other types
of human-readable Internet traffic.  Abstraction is a powerful tool.

Later in our class, you will learn how to control logical connections
between hardware blocks so that you can make use of the same hardware
for adding, subtracting, checking for upper-case letters, and so forth.\\

\subsubsection{Checking ASCII for Lower-case Letters}

Having developed several approaches for checking for an upper-case letter,
the task of checking for a lower-case letter should be straightforward.
In ASCII, lower-case letters are represented by the~\mbox{7-bit} patterns 
from~0x61 through~0x7A.
%
One can now easily see how more abstract designs make solving similar
tasks easier.  If we have designed our upper-case checker 
with \mbox{7-variable} K-maps,
we must start again with new K-maps for the lower-case checker.  If
instead we have taken the approach of designing logic for the upper and 
lower bits of the ASCII character, we can reuse most of that logic,
since the functions $T_4$ and $T_5$ are identical when checking for
a lower-case character.  Recalling the algebraic form of~$U(C)$, we can
then write a function~$L(C)$ (a lower-case checker) as shown on the
left below.

\begin{minipage}{2.65in}
\begin{eqnarray*}
U&=&C_6~\overline{C_5}~\overline{C_4}~T_4+C_6~\overline{C_5}~C_4~T_5\\
\\
L&=&C_6~C_5~\overline{C_4}~T_4+C_6~C_5~C_4~T_5\\
\\
&=&C_6~C_5~\overline{C_4}~(C_3+C_2+C_1+C_0)+\\
&&C_6~C_5~C_4~(\overline{C_3}+\overline{C_2}~\overline{C_1}+\overline{C_2}~\overline{C_0})\\
\end{eqnarray*}
\end{minipage}\hspace{0.25in}%
\begin{minipage}{3.6in}
\epsfig{file=part2/figs/ascii-cmp-lower.eps,width=3.6in}
\end{minipage}

Finally, if we have used a design based on comparators or adders, the
design of a lower-case checker becomes trivial: simply change the numbers
that we input to these components, as shown in the figure on the right above
for the comparator-based design.  The only changes from the upper-case checker
design are the inputs to the comparators and the output produced, 
highlighted with blue text in the figure.

\pagebreak

\subsubsection{The Multiplexer}

Using the more abstract designs for checking ranges of ASCII characters,
we can go a step further and create a checker for both upper- and lower-case
letters.  To do so, we add another input~$S$ that allows us to select the 
function that we want---either the upper-case checker~$U(C)$ or the 
lower-case checker~$L(C)$.

For this purpose, we make use of a logic block called a {\bf multiplexer},
or {\bf mux}.
Multiplexers are an important abstraction for digital logic.  In 
general, a multiplexer allows us to use one digital signal to 
select which of several others is forwarded to an output.

\begin{minipage}{3.4in}
The simplest form of the multiplexer is the 2-to-1~multiplexer shown to 
the right.   The logic diagram illustrates how the mux works.  The block 
has two inputs from the left and one from the top.  The top input allows 
us to choose which of the left inputs is forwarded to the output.  
When the input~$S=0$, the upper AND gate outputs~0, and the lower AND gate
outputs the value of~$D_0$.  The OR gate then produces~$Q=0+D_0=D_0$.
Similarly, when input~$S=1$, the upper AND gate outputs~$D_1$, and the
lower AND gate outputs~0.  In this case, the OR gate produces~$Q=D_1+0=D_1$.
\end{minipage}\hspace{.25in}%
\begin{minipage}{2.85in}
\epsfig{file=part2/figs/mux.eps,width=2.85in}\vspace{12pt}
\end{minipage}

The symbolic form of the mux is a trapezoid with data inputs on the 
larger side, an output on the smaller side, and a select input on the
angled part of the trapezoid.  The labels inside the trapezoid indicate 
the value of the select input~$S$ for which the adjacent data signal, 
$D_1$ or $D_0$, is copied to the output~$Q$.

We can generalize multiplexers in two ways.  First, we can extend the 
single select input to a group of select inputs.  An~\mbox{$N$-bit}
select input allows selection from amongst~$2^N$ inputs.  A~\mbox{4-to-1} 
multiplexer is shown below, for example.  The logic diagram on the left
shows how the \mbox{4-to-1~mux} operates.  For any combination of $S_1S_0$,
three of the AND gates produce~0, and the fourth outputs the $D$~input
corresponding to the interpretation of~$S$ as an unsigned number.
Given three zeroes and one~$D$~input, the OR gate thus reproduces one of 
the~$D$'s.  When $S_1S_0=10$, for example, the third AND gate copies~$D_2$,
and~$Q=D_2$.\\

\centerline{\epsfig{file=part2/figs/mux4-to-1.eps,width=5.60in}}

As shown in the middle figure, a \mbox{4-to-1}~mux can also be built from
three~\mbox{2-to-1} muxes.  Finally, the symbolic form of a \mbox{4-to-1}~mux 
appears on the right in the figure.

\pagebreak

The second way in which we can generalize multiplexers is by using
several multiplexers of the same type and using the same signals for 
selection.  For example, we might use a single select bit~$T$ to choose 
between any number of paired inputs.  Denote input pair by~$i$~$D_1^i$ 
and~$D_0^i$.  For each pair, we have an output~$Q_i$.  
%
When~$T=0$, $Q_i=D_0^i$ for each value of~$i$.
%
And, when~$T=1$, $Q_i=D_1^i$ for each value of~$i$.
%
Each value of~$i$ requires a~\mbox{2-to-1} mux with its select input
driven by the global select signal~$T$.

Returning to the example of the upper- and lower-case checker, we
can make use of two groups of seven~\mbox{2-to-1} muxes, all controlled by
a single bit select signal~$S$, to choose between the inputs needed
for an upper-case checker and those needed for a lower-case checker.

Specific configurations of multiplexers are often referred to
as {\bf $N$-to-$M$ multiplexers}.  Here the value~$N$ refers to the
number of inputs, and~$M$ refers to the number of outputs.  The
number of select bits can then be calculated as~$\log_2(N/M)$---$N/M$ 
is generally a power of two---and one way to build such a 
multiplexer is to use~$M$ copies of
an ~$(N/M)$-to-1 multiplexer.

Let's extend our upper- and lower-case checker to check
for four different ranges of ASCII characters, as shown below.
This design uses two \mbox{28-to-7} muxes to create a single
checker for the four ranges.  Each of the muxes in the figure logically 
represents seven \mbox{4-to-1} muxes.\\

\centerline{\epsfig{file=part2/figs/ascii-four-range.eps,width=3.75in}}\vspace{12pt}

\begin{minipage}{2.8in}
The table to the right describes the behavior of the checker.
%
When the select input~$S$ is set to~$00$, the left mux selects the value~0x00,
and the right mux selects the value~0x1F, which checks whether the ASCII
character represented by~$C$ is a control character.
%
When the select input~$S=01$, the muxes produce the values needed to check
whether~$C$ is an upper-case letter.
%
Similarly, when the select input~$S=10$,\linebreak
\end{minipage}\hspace{0.25in}%
\begin{minipage}{3.45in}
\begin{tabular}{c|c|c|c}
& left& right& \\
& comparator& comparator& \\
$S_1S_0$& input& input& $R(C)$ produced \\ \hline
00& 0x00& 0x1F& control character?\\
01& 0x41& 0x5A& upper-case letter?\\
10& 0x61& 0x7A& lower-case letter?\\
11& 0x30& 0x39& numeric digit?\\ 
\end{tabular}\\
\end{minipage}\mpdone

the muxes produce the values 
needed to check whether~$C$ is a lower-case letter.
%
Finally, when the select input~$S=11$, the left mux selects the value~0x30,
and the right mux selects the value~0x39, which checks whether the ASCII
character represented by~$C$ is a digit (0~to~9).



\pagebreak
