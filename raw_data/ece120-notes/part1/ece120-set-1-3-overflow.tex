\classtitle

\subsection{Overflow Conditions}

This set of notes discusses the overflow conditions for unsigned and
2's complement addition.  For both types, we formally prove that
the conditions that we state are correct.  Many of our faculty want our
students to learn to construct formal proofs, so we plan to begin
exposing you to this process in our classes.
%
Prof. Lumetta is a fan of Prof. George Polya's educational theories
with regard to proof techniques, and in particular the idea that one
builds up a repertoire of approaches by seeing the approaches used 
in practice.\\

% , but teaching proof
% techniques effectively is challenging, particularly when exercises
% are of the form, ``Prove that $<$insert true theorem$>$ is true'' rather
% than the more open-ended form that we typically encounter in research,
% ``Prove that $<$insert some conjecture$>$ is true, or find a 
% counterexample.''

\subsubsection{Implication and Mathematical Notation}

Some of you may not have been exposed to basics of mathematical logic, so
let's start with a brief introduction to implication.  We'll use 
variables~$p$ and~$q$ to represent statements that can be either true
or false.  For example,~$p$ might represent the statement, ``Jan is
an ECE student,'' while~$q$ might represent the statement, ``Jan
works hard.''  The {\bf logical complement} or {\bf negation} of 
a statement~$p$,
written for example as ``not~$p$,'' has the opposite truth value:
if~$p$ is true, not~$p$ is false, and if~$p$ is false, not~$p$ is
true.

An {\bf implication} is a logical relationship between two statements.
The implication itself is also a logical statement, and may be true or
false.  In English, for example, we might say, ``If $p$, $q$.''
In mathematics, the same implication is usually written as either 
``$q$~if~$p$'' or ``$p\rightarrow{q}$,'' and the latter is read 
as, ``$p$ implies $q$.''  
%
Using our example values for~$p$ and~$q$, we can see that
$p\rightarrow{q}$ is true: ``Jan is an ECE student'' does
in fact imply that ``Jan works hard!''

The implication $p\rightarrow{q}$ 
is only considered false if~$p$ is true and~$q$ is false.
In all other cases, the implication is true.
This definition can be a little confusing at first, so let's use
another example to see why.
%
Let ~$p$ represent the statement
``Entity X is a flying pig,'' and let~$q$ represent 
the statement, ``Entity X obeys air traffic control regulations.''
%
Here the implication $p\rightarrow{q}$ is again true: 
flying pigs do not exist, so~$p$ is false, and thus 
``$p\rightarrow{q}$'' is true---for any value of statement~$q$!

Given an implication ``$p\rightarrow{q}$,'' we say that the {\bf
converse} of the implication is the statement 
``$q\rightarrow{p}$,'' which is also an implication.
In mathematics, the converse of 
$p\rightarrow{q}$
is sometimes written
as ``$q$~only~if~$p$.''  The converse of an implication may or may not have
the same truth value as the implication itself.  Finally,
we frequently use the shorthand notation, ``$p$~if~and~only~if~$q$,''
(or, even shorter, ``$p$~iff~$q$'') to mean 
``$p\rightarrow{q}$ {\em and}
$q\rightarrow{p}$.'' This last statement is true only when both
implications are true.\\

\subsubsection{Overflow for Unsigned Addition}

Let's say that we add two~\mbox{$N$-bit} unsigned numbers,~$A$
and~$B$.  The~\mbox{$N$-bit} unsigned representation 
can represent integers in the range~$[0,2^N-1]$.
%
Recall that we say that the addition operation has 
overflowed if the number represented by the~\mbox{$N$-bit} pattern
produced for the sum does not actually represent the number~$A+B$.

For clarity, let's name the bits of~$A$ by writing the number
as $a_{N-1}a_{N-2}...a_1a_0$.  Similarly, let's write~$B$ as
$b_{N-1}b_{N-2}...b_1b_0$.  Name the sum $C=A+B$.  The sum that
comes out of the add unit has only~$N$ bits, but recall that
we claimed in class that the overflow condition for unsigned 
addition is given by the {\bf carry} out of the most significant
bit.  So let's write the sum as 
$c_{N}c_{N-1}c_{N-2}...c_1c_0$, realizing that~$c_N$ is the
carry out and not actually part of the sum produced by the 
add unit.

{\bf Theorem:}
%
Addition of two~\mbox{$N$-bit} unsigned numbers
$A=a_{N-1}a_{N-2}...a_1a_0$
and
$B=b_{N-1}b_{N-2}...b_1b_0$
to produce sum
$C=A+B=c_{N}c_{N-1}c_{N-2}...c_1c_0$,
overflows if and only if
the carry out~$c_N$ of the addition is a~1~bit.

\pagebreak

{\bf Proof:}
%
Let's start with the ``if'' direction.  In other words,~$c_N=1$ implies
overflow.  Recall that unsigned addition is the same as base~2 addition,
except that we discard bits beyond~$c_{N-1}$ from the sum~$C$.
The bit~$c_N$ has place value~$2^N$, so, when~$c_N=1$ we can write that 
the correct sum~$C\geq{2^N}$.  But no value that large can be represented
using the~\mbox{$N$-bit} unsigned representation, so we have an overflow.

The other direction (``only if'') is slightly more complex: we need to
show that overflow implies that~$c_N=1$.  We use a range-based argument
for this purpose.  Overflow means that the sum~$C$ is outside the
representable range~$[0,2^N-1]$.  Adding two non-negative numbers cannot
produce a negative number, so the sum can't be smaller than~0.  Overflow 
thus implies that $C\geq{2^N}$.

Does that argument complete the proof?  No, because some numbers, such 
as $2^{N+1}$, are larger than $2^N$, but do not have a~1~bit in the $N$th
position when written in binary.  We need to make use of the constraints
on~$A$ and~$B$ implied by the possible range of the representation.

In particular, given that~$A$ and~$B$ are represented as~\mbox{$N$-bit}
unsigned values, we can write
%
\begin{eqnarray*}
0 \leq & A & \leq 2^N - 1\\
0 \leq & B & \leq 2^N - 1
\end{eqnarray*}
%
We add these two inequalities and replace~$A+B$ with~$C$ to obtain
%
\begin{eqnarray*}
0 \leq & C & \leq 2^{N + 1} - 2
\end{eqnarray*}
%
Combining the new inequality with the one implied by the overflow 
condition, we obtain
%
\begin{eqnarray*}
2^N \leq & C & \leq 2^{N + 1} - 2
\end{eqnarray*}
%
All of the numbers in the range allowed by this inequality have~$c_N=1$,
completing our proof.\\

\subsubsection{Overflow for 2's Complement Addition}

Understanding overflow for~2's complement addition is somewhat trickier,
which is why the problem is a good one for you to think about on your
own first.
%
Our operands,~$A$ and~$B$, are now two~\mbox{$N$-bit}~2's complement numbers.
The~\mbox{$N$-bit}~2's complement representation 
can represent integers in the range~$[-2^{N-1},2^{N-1}-1]$.
%
Let's start by ruling out a case that we can show never leads to overflow.

{\bf Lemma:} 
%
Addition of two~\mbox{$N$-bit} 2's complement numbers~$A$ and~$B$
does not overflow if one of the numbers is negative and the other is
not.

{\bf Proof:}
%
We again make use of the constraints implied by the fact that~$A$ and~$B$
are represented as~\mbox{$N$-bit} 2's complement values.  We can assume
{\bf without loss of generality}\footnote{This common mathematical phrasing
means that we are using a problem symmetry to cut down the length of the
proof discussion.  In this case, the names~$A$ and~$B$ aren't particularly
important, since addition is commutative~($A+B=B+A$).  Thus the proof
for the case in which~$A$ is negative (and~$B$ is not) is identical to the
case in which~$B$ is negative (and~$A$ is not), except that all of the 
names are swapped.  The term ``without loss of generality'' means that
we consider the proof complete even with additional assumptions, in
our case that~$A<0$ and~$B\geq{0}$.}, or {\bf w.l.o.g.}, 
that~$A<0$ and~$B\geq{0}$.

Combining these constraints with the range representable 
by~\mbox{$N$-bit}~2's complement, we obtain
%
\begin{eqnarray*}
-2^{N-1} \leq & A & < 0\\
0 \leq & B & < 2^{N-1}
\end{eqnarray*}
%
We add these two inequalities and replace~$A+B$ with~$C$ to obtain
%
\begin{eqnarray*}
-2^{N-1} \leq & C & < 2^{N-1}
\end{eqnarray*}
%
But anything in the range specified by this inequality can be represented
with~{$N$-bit}~2's complement, and thus the addition does not overflow.\\


We are now ready to state our main theorem.  For convenience, 
let's use different names for the actual sum~$C=A+B$ and the sum~$S$
returned from the add unit.  We define~$S$ as the number represented by
the bit pattern produced by the add unit.  When overflow 
occurs,~$S\not=C$, but we always have~$(S=C)~\mbox{mod}~2^N$.

{\bf Theorem:} 
%
Addition of two~\mbox{$N$-bit} 2's complement numbers~$A$ and~$B$
overflows if and only if one of the following conditions holds:
\begin{enumerate}
\item{$A<0$ and $B<0$ and $S\geq{0}$}
\item{$A\geq{0}$ and $B\geq{0}$ and $S<0$}
\end{enumerate}

{\bf Proof:}
%
We once again start with the ``if'' direction.  That is, if condition~1 
or condition~2 holds, we have an overflow.  The proofs are straightforward.
Given condition~1, we can add the two inequalities $A<0$ and $B<0$ to 
obtain $C=A+B<0$.  But $S\geq{0}$, so clearly $S\not=C$, thus overflow 
has occurred.

Similarly, if condition~2 holds, we can add the inequalities~$A\geq{0}$
and~$B\geq{0}$ to obtain $C=A+B\geq{0}$.  Here we have~$S<0$, so again
$S\not=C$, and we have an overflow.

We must now prove the ``only if'' direction, showing that any overflow
implies either condition~1 or condition~2.  By the 
{\bf contrapositive}\footnote{If we have a statement of the form
($p$~implies~$q$), its contrapositive is the 
statement~(not~$q$~implies~not~$p$).
Both statements have the same truth value.  In this case, we can turn
our Lemma around as stated.} of our
Lemma, we know that if an overflow occurs, either both operands are 
negative, or they are both positive.

%\vfill
%
%\pagebreak

Let's start with the case in which both operands are negative, so~$A<0$
and~$B<0$, and thus the real sum~$C<0$ as well.  Given that~$A$ and~$B$
are represented as~\mbox{$N$-bit}~2's complement, they must fall in
the representable range, so we can write
%
\begin{eqnarray*}
-2^{N-1} \leq & A & < 0\\
-2^{N-1} \leq & B & < 0
\end{eqnarray*}
%
We add these two inequalities and replace~$A+B$ with~$C$ to obtain
%
\begin{eqnarray*}
-2^N \leq & C & < 0
\end{eqnarray*}
%
Given that an overflow has occurred, $C$ must fall outside of the 
representable range.  Given that $C<0$, it cannot be larger than the
largest possible number representable using~\mbox{$N$-bit}~2's
complement, so we can write
%
\begin{eqnarray*}
-2^N \leq & C & < -2^{N-1}
\end{eqnarray*}
%
We now add~$2^N$ to each part to obtain
%
\begin{eqnarray*}
0 \leq & C + 2^N & < 2^{N-1}
\end{eqnarray*}
%
This range of integers falls within the representable range 
for~\mbox{$N$-bit}~2's complement, so we can replace the middle
expression with~$S$ (equal to~$C$ modulo~$2^N$) to find that
%
\begin{eqnarray*}
0 \leq & S & < 2^{N-1}
\end{eqnarray*}
%
Thus, if we have an overflow and both~$A<0$ and~$B<0$, the resulting
sum~$S\geq{0}$, and condition~1 holds.

The proof for the case in which we observe an overflow when 
both operands are non-negative ($A\geq{0}$ and $B\geq{0}$)
is similar, and leads to condition~2.  We again begin with
inequalities for~$A$ and~$B$:
%
\begin{eqnarray*}
0 \leq & A & < 2^{N-1}\\
0 \leq & B & < 2^{N-1}
\end{eqnarray*}
%
We add these two inequalities and replace~$A+B$ with~$C$ to obtain
%
\begin{eqnarray*}
0 \leq & C < & 2^N
\end{eqnarray*}
%
Given that an overflow has occurred, $C$ must fall outside of the 
representable range.  Given that $C\geq{}0$, it cannot be smaller than the
smallest possible number representable using~\mbox{$N$-bit}~2's
complement, so we can write
%
\begin{eqnarray*}
2^{N-1} \leq & C & < 2^N
\end{eqnarray*}
%
We now subtract~$2^N$ to each part to obtain
%
\begin{eqnarray*}
-2^{N-1} \leq & C - 2^N & < 0
\end{eqnarray*}
%
This range of integers falls within the representable range 
for~\mbox{$N$-bit}~2's complement, so we can replace the middle
expression with~$S$ (equal to~$C$ modulo~$2^N$) to find that
%
\begin{eqnarray*}
-2^{N-1} \leq & S & < 0
\end{eqnarray*}
%
Thus, if we have an overflow and both~$A\geq{0}$ and~$B\geq{0}$, the resulting
sum~$S<0$, and condition~2 holds.

Thus overflow implies either condition~1 or condition~2, completing our
proof.

\vfill

\pagebreak

