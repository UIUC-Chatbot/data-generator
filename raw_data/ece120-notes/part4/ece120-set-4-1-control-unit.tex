\classtitle

\subsection{Control Unit Design}

Appendix~C of the Patt and Patel textbook describes a microarchitecture for
the LC-3 ISA, including a control unit implementation.
%
In this set of notes, we introduce a few concepts and strategies for
control unit design, using the textbook's \mbox{LC-3} microarchitecture 
to help illustrate them.  Several figures from the textbook are
reproduced with permission in these notes as an aid to understanding.

The control unit of a computer based on the von Neumann model can be viewed
as an FSM that fetches instructions from memory and executes them.  Many
possible implementations exist both for the control unit itself and for the
resources that it controls, the other components in the von Neumann model,
which we collectively call the {\bf datapath}.
%
In this set of notes, we discuss two strategies for structured control
unit design and introduce the idea of using memories to encode logic 
functions.

Let's begin by recalling that the control unit consists of three parts: a 
high-level FSM that controls instruction processing, a program counter~(PC)
register that holds the address of the next instruction to be executed,
and an instruction register~(IR) that holds the current instruction as
it executes.

Other von Neumann components provide inputs to the control unit.  The
memory unit, for example, contains the instructions and data on which 
the program executes.  
%
The processing unit contains a register file and condition 
codes (N, Z, and P for the \mbox{LC-3} ISA).  
%
The outputs of the control unit are signals that control operation of the
datapath: the processing unit, the memory, and the I/O interfaces.
%
The basic problem that we must solve, then, for control unit design, is to 
map instruction processing and the state of the FSM (including the PC and
the IR) into appropriate sequences of {\bf control signals} for the datapath.\\

\subsubsection{LC-3 Datapath Control Signals}

As we have skipped over the implementation details of interrupts and 
privilege of the \mbox{LC-3} in our class, let's consider a 
microarchitecture and datapath without those capabilities.  
%
The figure on the next page (Patt and Patel Figure~C.3) shows an \mbox{LC-3}
datapath and control signals without support for interrupts and privilege.

\begin{minipage}{1.7in}
Some of the datapath control signals mentioned
in the textbook are no longer necessary in the simplified design.
%
Let's discuss the signals that remain and give some examples of how they
are used.  A list appears to the right.\mpline

First, we have a set of seven \mbox{1-bit} control signals (starting 
with ``LD.'') that specifies whether registers in the datapath load 
new values.\mpline

Next, there are four~\mbox{1-bit} signals (starting with ``Gate'') for 
tri-state buffers that control access to the bus.  These four implement 
a distributed mux for the bus.  Only one value can\linebreak
\end{minipage}\hspace{0.25in}%
\begin{minipage}{4.55in}
\begin{tabular}{rl}
signal& meaning\\ \hline
LD.MAR& load new value into memory address register\\
LD.MDR& load new value into memory data register\\
LD.IR& load new value into instruction register\\
LD.BEN& load new value into branch enable register\\
LD.REG& load new value into register file\\
LD.CC& load new values into condition code registers (N,Z,P)\\
LD.PC& load new value into program counter\\ \hline
GatePC& write program counter value onto bus\\
GateMDR& write memory data register onto bus\\
GateALU& write arithmetic logic unit result onto bus\\
GateMARMUX& write memory address register mux output onto bus\\ \hline
PCMUX& select value to write to program counter (2 bits)\\
DRMUX& select value to write to destination register (2 bits)\\
SR1MUX& select register to read from register file (2 bits)\\
ADDR1MUX& select register component of address (1 bit)\\
ADDR2MUX& select offset component of address (2 bits)\\
MARMUX& select type of address generation (1 bit)\\ \hline
ALUK& select arithmetic logic unit operation (2 bits)\\ \hline
MIO.EN& enable memory\\ \hline
R.W& read or write from memory\\ \\
\end{tabular}
\end{minipage}\mpdone

appear on the bus in 
any cycle, so at most one of these signals can be~1; others must all 
be~0 to avoid creating a short.

\pagebreak

\epsfig{file=part4/patt-patel-appendix-C/APPC03.eps,width=6.5in}

\pagebreak

The third group (ending with ``MUX'') of signals
controls multiplexers in the datapath.
The number of bits for each depends on the number of inputs to the mux;
the total number of signals is~10.
The last two groups of signals control the ALU and the memory, 
requiring a total of~4 more signals.  The total of all groups is thus
25~control signals for the datapath without support for privilege
and interrupts.\\

\subsubsection{Example Control Word: ADD}

Before we begin to discuss control unit design in more detail, let's
work through a couple of examples of implementing specific RTL with
the control signals available.  The figure below (Patt and Patel 
Figure~C.2) shows a state machine for the \mbox{LC-3} ISA (again 
without detail on interrupts and privilege).\\

\centerline{\epsfig{file=part4/patt-patel-appendix-C/APPC02.eps,width=6.25in}}

Consider now the state that implements the ADD instruction---state
number~1 in the figure on the previous page, just below and to the 
left of the decode state.
%
The RTL for the state is: DR~$\leftarrow$~SR~+~OP2,~set~CC.

We can think of the 25~control signals that implement the desired RTL
as a {\bf control word} for the datapath. 
%
Let's begin with the register load control signals.  The RTL writes
to two types of registers: the register file and the condition codes.
To accomplish these simultaneous writes, LD.REG and LD.CC must be high.
No other registers in the datapath should change, so the other five
LD signals---LD.MAR, LD.MDR, LD.IR, LD.BEN, and LD.PC---should be low.

What about the bus?  In order to write the result of the add operation
into the register file, the control signals must allow the ALU to write
its result on to the bus.  So we need GateALU=1.  And the other Gate
signals---GatePC, GateMDR, and GateMARMUX---must all be~0.  The 
condition codes are also calculated from the value on the bus, but
they are calculated on the same value as is written to the register
file (by the definition of the \mbox{LC-3} ISA).
%
If the RTL for an FSM state implicitly requires more than one value 
to appear on the bus in the same cycle, that state is impossible to 
implement using the given datapath.  Either the datapath or the state 
machine must be changed in such a case.  The textbook's design has been 
fairly thoroughly tested and debugged.

\begin{minipage}{1.8in}
The earlier figure of the datapath does not 
show all of the muxes.  The remaining muxes appear in the figure 
to the right (Patt and Patel Figure~C.6).\mpline

Some of the muxes in the datapath must be used to enable the
addition needed for ADD to occur.  The\linebreak
DRMUX must select its IR[11:9] 
input in order to write to the destination register specified by the 
ADD instruction.  Similarly, the SR1MUX must select its IR[8:6] input 
in order to\linebreak
\end{minipage}\hspace{0.25in}%
\begin{minipage}{4.45in}
\epsfig{file=part4/patt-patel-appendix-C/APPC06.eps,width=4.45in}\\
\end{minipage}\mpdone

pass the first source register specified by the ADD to the ALU as input~A
(see the datapath figure).  SR2MUX is always controlled by the 
mode bit IR[5], so the control unit does not need to generate anything
(note that this signal was not in the list given earlier).
%
The rest of the muxes in the datapath---PCMUX, ADDR1MUX, ADDR2MUX, and
MARMUX---do not matter, and the signals controlling them are don't cares.
For example, since the PC does not change, the output of the PCMUX
is simply discarded, thus which input the PCMUX forwards to its output
cannot matter.

The ALU must perform an addition, so we must set the operation type
ALUK appropriately.  And memory should not be enabled (MIO.EN=0),
in which case the read/write control for memory, R.W, is a don't
care.  These 25~signal values together (including seven don't cares)
implement the RTL for the single state of execution for an 
ADD instruction.\\


\subsubsection{Example Control Word: LDR}

As a second example, consider the first state in the sequence that 
implements the LDR instruction---state number~6 in the figure on the 
previous page.
%
The RTL for the state is: MAR~$\leftarrow$~BaseR~+~off6, but BaseR is
abbreviated to ``B'' in the state diagram.

What is the control word for this state?
%
Let's again begin with the register load control signals.  Only the
MAR is written by the RTL, so we need LD.MAR=1 and the other load
signals all equal to~0.

The address (BaseR~+~off6) is generated by the address adder, then 
passes through the MARMUX to the bus, from which it can be written into
the MAR.  To allow the MARMUX to write to the bus, we set GateMARMUX
high and set the other three Gate control signals low.

More of the muxes are needed for this state's RTL than we needed for
the ADD execution state's RTL.
%
The SR1MUX must again select its IR[8:6] input, this time in order to 
pass the BaseR specified by the instruction to ADDR1MUX.  ADDR1MUX
must then select the output of the register file in order to pass the 
BaseR to the address adder.  The other input of the address adder should
be off6, which corresponds to the sign-\linebreak
extended version of IR[5:0].
ADDR2MUX must select this input to pass to the address adder. 
Finally, the MARMUX must select the output of the address adder.
The PCMUX and DRMUX do not matter for this case, and can be left as
don't cares.  Neither the PC nor any register in the register file
is written.

The output of the ALU is not used, so which operation it performs is 
irrelevant, and the ALUK controls are also don't cares.  Memory is
also not used, so again we set MIO.EN=0.  And, as before, the R.W
control for memory is a don't care.
%
These 25~signal values together (again including seven don't cares)
implement the RTL for the first state of LDR execution.\\


\subsubsection{Hardwired Control}

\begin{minipage}{3.75in}
Now we are ready to think about how control signals can be generated.
%
As illustrated to the right, instruction processing consists of two 
steps repeated infinitely: fetch an instruction, then execute the 
instruction.\mpline

Let's say that we choose a fixed number of cycles for each of these two
steps.  We can then control our system with a\linebreak
\end{minipage}\hspace{0.25in}%
\begin{minipage}{2.5in}
\epsfig{file=part4/figs/inst-loop.eps,width=2.5in}\\
\end{minipage}\mpdone

counter, using the counter's value and the IR to generate the
control signals through combinational
logic.  The PC is used only as data and has little or no direct effect on 
how the system fetches and processes instructions.\footnote{Some ISAs do 
split the address space into privileged and non-privileged regions, but 
we ignore that possibility here.}
%
This approach in general is called {\bf hardwired control}.

How many cycles do we need for instruction fetch?  How many cycles do
we need for instruction processing?  The answers depend on the factors:
the complexity of the ISA, and the complexity of the datapath.

Given a simple ISA, we can design a datapath that
is powerful enough to process any instruction in a single cycle.  
The control unit for such a design is
an example of {\bf single-cycle, hardwired control}.  While this
approach simplifies the control unit, the cycle time
for the clock is limited by the slowest instruction.
The clock must also be slow enough to let memory 
operations complete in single cycle, both for instruction fetch and
for instructions that require a memory access.
%
Such a low clock rate is usually not acceptable.

More generally, we can use a simpler datapath and break both instruction
fetch and instruction processing into multiple steps.  Using the datapath
in the figure from Patt and Patel, for example, instruction fetch requires 
three steps, and the number of steps for instruction processing depends 
on the type of instruction being processed.  The state diagram shown
earlier illustrates the steps for each opcode.

Although the datapath is not powerful enough to complete instructions
in a single cycle, we can build a {\bf multi-cycle, hardwired control} 
unit (one based on combinational logic).  
%
In fact, this type of control unit is not much more complex than the 
single-cycle version.  For the control unit's FSM, we can use a binary 
counter to enumerate first the steps of fetch, then the steps
of processing.  The counter value along with the IR~register can 
drive combinational logic to produce control signals.  And, to avoid
processing at the speed of the slowest instruction (the opcode that
requires the largest number of steps; LDI and STI in the \mbox{LC-3}
state diagram), we can add a reset signal to the counter to force it 
back to instruction fetch.  The FSM counter reset signal is simply 
another control signal.  Finally, we can add one more signal that 
pauses the counter while waiting for a memory operation to complete.  
The system clock can then run at the speed of the logic rather than 
at the speed of memory.
%
The control unit implementation discussed in Appendix~C of Patt and Patel 
is not hardwired, but it does make use of a memory ready signal to 
achieve this decoupling between the clock speed of the processor and the
access time of the memory.

\begin{minipage}{1.75in}
The figure to the right illustrates a general 
multi-cycle, hardwired control unit.  The three blocks on
the left are the control unit state.  The combinational logic in
the middle uses the control unit state along with some
datapath status bits to compute the control signals for the datapath
and the extra controls needed for the FSM counter, IR, and PC.
The datapath appears to the right in the figure.
\end{minipage}\hspace{0.25in}%
\begin{minipage}{4.5in}
\epsfig{file=part4/figs/hardwired.eps,width=4.5in}\\
\end{minipage}

How complex is the combinational logic?  As mentioned earlier, we 
assume that the~PC does not directly affect control.  But we still 
have 16~bits of IR, the FSM counter state, and the datapath status
signals.  Perhaps we need 24-variable \mbox{K-maps}?  
%
Here's where engineering and human design come to 
the rescue: by careful design of the ISA and the encoding, the
authors have made many of the datapath control signals for the \mbox{LC-3}
ISA quite simple.
For example, the register that appears on the register file's SR2
output is always specified by IR[2:0].  The SR1 output 
requires a mux, but the choices are limited to IR[11:9] and IR[8:6]
(and R6 in the design with support for interrupts).
Similarly, the destination register in the register file
is always R7 or IR[11:9] (or, again, R6 when supporting interrupts).
%
The control signals for an \mbox{LC-3}
datapath depend almost entirely on the state of the control unit 
FSM (counter bits in a hardwired design) and the opcode IR[15:12].
%
The control signals are thus reduced to fairly simple functions.

Let's imagine building a hardwired control unit for the \mbox{LC-3}.
%
Let's start by being more precise about the number of inputs to
the combinational logic.
%
Although most decisions are based on the opcode, the datapath
and state diagram shown earlier for the \mbox{LC-3} ISA do have
one instance of using another instruction bit to determine behavior.
Specifically, the JSR instruction has two modes, and the control
unit uses IR[11] to choose between them.  So we need to have
five bits of IR instead of four as input to our logic.

How many datapath status signals are needed?
%
When the control unit accesses memory, it must wait until the 
memory finishes the access, as indicated by a memory ready signal~R.
And the control unit must implement the conditional part of conditional
branches, for which it uses the datapath's branch enable signal~BEN.
%
These two datapath status signals suffice for our design.

How many bits do we need for the counter?
Instruction fetch requires three cycles: one to move
the PC to the MAR and increment the PC, a second to read from
memory into MDR, and a third to move the instruction bits across
the bus from MDR into IR.  Instruction decoding in a hardwired design
is implicit and 
requires no cycles: since all of our control signals can depend on
the IR, we do not need a cycle to change the FSM state to reflect the opcode.
Looking at the \mbox{LC-3} state diagram, we see
that processing an instruction requires at most five cycles.  In
total, at
most eight steps are needed to fetch and process any \mbox{LC-3} 
instruction, so we can use a~\mbox{3-bit} binary counter.

We thus have a total of ten bits of input: IR[15:11], R, BEN, and
a \mbox{3-bit} counter.
%
Adding the RESET and PAUSE controls for our FSM counter to the 
25~control signals listed earlier, we need to find~27 functions
on 10~variables.
%
That's still a lot of big \mbox{K-maps} to solve.  Is there an
easier way?\\


\subsubsection{Using a Memory for Logic Functions}

Consider a case in which you need to compute many functions on a small 
number of bits, such as we just described for the multi-cycle, hardwired
control unit.  One strategy is to use a memory (possibly a read-only memory).
A \mbox{$2^m\times{N}$} memory can be viewed as computing~$N$ arbitrary
functions on~$m$ variables.  The functions to be computed are specified
by filling in the bits of the memory.  So long as the value of~$m$ is fairly
small, the memory (especially SRAM) can be fast.

Synthesis tools (or hard work) can, of course, produce smaller designs
that use fewer gates.  Actually, tools may be able to optimize a fixed
design expressed as read-only memory, too.  But designing the functions
with a memory makes them easier to modify later.  If we make a mistake,
for example, in computing one of the functions, we need only change a bit
or two in the memory instead of solving equations and reoptimizing and
replacing logic gates.  We can also extend our design if we have space
remaining (that is, if the functions are undefined for some combinations 
of the~$m$ inputs).  The
Cray~T3D supercomputer, for example, used a similar approach to add
new instructions to the Alpha processors on which it was based.

This strategy is effective in many contexts, so let's briefly discuss two 
analogous cases.  In software, a memory becomes a lookup table.  Before
handheld calculators, lookup tables were used by humans to compute 
transcendental functions such as sines, cosines, logarithms.  Computer 
graphics hardware and software used a similar approach for transcendental
functions in order to reduce cost and improve 
speed.  Functions such as counting 1~bits in a word are useful for 
processor scheduling and networking, but not all ISAs provide this type
of instruction.  In such cases, lookup tables in software are often the
best solution.

In programmable hardware such as Field Programmable Gate Arrays (FPGAs),
lookup tables (called LUTs in this context) have played an important role
in implementing arbitrary logic functions.
%
The FPGA is the modern form of the programmable logic array~(PLA)
mentioned in the textbook, and will be your main tool 
for developing digital hardware in ECE385.
%
For many years, FPGAs served as a hardware prototyping platform, but
many companies today ship their first round products using 
designs mapped to FPGAs.  Why?  Chips are more and more expensive
to design, and mistakes are costly to fix.  In contrast, while
companies pay more to buy an FPGA than to produce a chip (after the 
first chip!), errors in the design can usually be fixed by sending 
customers a new version through the Internet.

Let's return to our \mbox{LC-3} example.
%
Instead of solving the \mbox{K-maps}, we can use a small memory:
$2^{10}\times{27}$~bits (27,648~bits total).
%
We just need calculate the bits, put
them into the memory, and use the memory to produce the control signals.
%
The ``address'' input to the memory are the same 10~bits that we
needed for our combinational logic: IR[15:11], R, BEN, and the FSM counter.
The data outputs of the memory are the control signals and the
RESET and PAUSE inputs to the FSM counter.  And we're done.

We can do a little better, though.  The datapath in the textbook was
designed to work with the textbook's control unit.  If we add a little
logic, we can significantly simplify our memory-based, hardwired implementation.
For example, we only need to pause the FSM counter when waiting for memory.
If we can produce a control signal that indicates a need to wait for
memory, say \mbox{WAIT-MEM}, we can use a couple of gates to compute the
FSM counter's PAUSE signal as \mbox{WAIT-MEM}~AND~(NOT~R).  Making this change
shrinks our memory to~$2^9\times{27}$~bits.  The extra two control
signals in this case are RESET and \mbox{WAIT-MEM}.

Next, look at how BEN is used in the state diagram:
the only use is to terminate the
processing of branch instructions when no branch should occur (when 
BEN=0).  We can fold that functionality into the FSM counter's RESET
signal by producing a branch reset signal, \mbox{BR-RESET}, to reset
the counter to end a branch and a second signal, \mbox{INST-DONE},
when an instruction is done.  The RESET input for the FSM counter
is then (\mbox{BR-RESET}~AND~(NOT~BEN))~OR~\mbox{INST-DONE}.
%
And our memory further shrinks to~$2^8\times{28}$~bits, where the extra
three control signals are \mbox{WAIT-MEM}, \mbox{BR-RESET}, and 
\mbox{INST-DONE}.

Finally, recall that the only need for IR[11] is to implement the two
forms of JSR.  But we can add wires to connect SR1 to PCMUX's fourth 
input, then control the PCMUX output selection using IR[11] when appropriate
(using another control signal).  With this extension, we can implement
both forms with a single state, writing to both~R7 and~PC in the same
cycle.
%
Our final memory can then be~$2^7\times{29}$~bits (3,712~bits total),
which is less than one-seventh the number of bits that we needed before 
modifying the datapath.\\


\pagebreak

\subsubsection{Microprogrammed Control}

We are now ready to discuss the second approach to control unit design.
Take another look at the state diagram for the the \mbox{LC-3} ISA.  Does it 
remind you of anything?  Like a flowchart, it has relatively few arcs 
leaving each state---usually only one or two.

What if we treat the state diagram as a program?  We can use a small
memory to hold {\bf microinstructions} (another name for control words) 
and use the FSM state number as the memory address.  
%
Without support for interrupts or privilege, and with the datapath 
extension for JSR mentioned for hardwired control, the \mbox{LC-3} state 
machine requires fewer than 32~states.
%
The datapath has 25~control signals, but we need one more for the
datapath extension for JSR.
%
We thus start with \mbox{5-bit} state number (in a register)
and a~\mbox{$2^5\times{26}$~bit} memory, which we call
our control ROM (read-only memory) to distinguish
it from the big, slow, von Neumann memory.
%
Each cycle, the {\bf microprogrammed control} unit applies the FSM state 
number to the control ROM (no IR bits, 
just the state number), gets back a set of control signals, and
uses them to drive the datapath.

\begin{minipage}{4.25in}
To write our microprogram, we need to calculate the control signals
for each microinstruction and put them in the control ROM, but we also
need to have a way to decide which microinstruction should execute 
next.  We call the latter problem {\bf sequencing} or microsequencing.\mpline

Notice that most of the time there's no choice: we have only {\em one}
next microinstruction.  One simple approach is then to add the address
(the \mbox{5-bit} state ID) of the next microinstruction to the control ROM.
Instead of 26~bits per FSM state, we now have 31~bits per FSM state.\mpline

Sometimes we do need to have two possible next states.  When waiting
for memory (the von Neumann memory, not the control ROM) to
finish an access, for example, we want our FSM to stay
in the same state, then move to the next state when the access completes.
Let's add a second address to each microinstruction, and add
a branch control signal for the microprogram to decide whether we
should use the first address or the second for the next
microinstruction.  This design, using a~$2^5\times{36}$~bit memory
(1,152 bits total), appears to the right.
\end{minipage}\hspace{0.25in}%
\begin{minipage}{2in}
\epsfig{file=part4/figs/microprogrammed-no-decode.eps,width=2in}
\end{minipage}

\begin{minipage}{3.95in}
The microprogram branch control signal is a Boolean logic expression
based on the memory ready signal~R and 
IR[11].  We can implement it with a state ID comparison and
a mux, as shown to the right.  For the branch instruction execution state, 
the mux selects input~1, BEN.  For all other states, the mux selects 
input~0, R.  When an FSM state has only a single next state, we set both 
IDs in the control ROM to the ID for that state, so the value of~R 
has no effect.
\end{minipage}\hspace{0.25in}%
\begin{minipage}{2.3in}
\epsfig{file=part4/figs/microbranch.eps,width=2.3in}\\
\end{minipage}

% We can simplify the implementation of the microprogram branch control
% by setting both next-state addresses to the same value when a state
% does not branch.  Then the branch control becomes a don't care for that
% state.\mpline

\begin{minipage}{4.25in}
What's missing?  Decode!  We do have one FSM state in which we need to be
able to branch to one of sixteen possible next states, one for each
opcode.  Let's just add another mux and choose the state IDs for starting
to process each opcode in an easy way, as shown to the right with
extensions highlighted in blue.  The textbook assigns the first state 
for processing each opcode the IDs IR[15:12] preceeded by two~0s (the
textbook's design requires \mbox{6-bit} state IDs).  We adopt the same
strategy.  For example, the first FSM state for processing an ADD is 00001, 
and the first state for a TRAP is 01111.  In each case, the opcode 
encoding specifies the last four bits.\mpline

Now we can pick the remaining state IDs arbitrarily and fill in the 
control ROM with control 
signals that implement each state's RTL and the two possible next states.
Transitions from the decode state are handled by the extra mux.\mpline

The microprogrammed control unit implementation in Appendix~C of 
Patt and Patel is similar to the one that we have developed here,
but is slightly more complex so that it can handle interrupts
and privilege.\\ \\ \\ 
\end{minipage}\hspace{0.25in}%
\begin{minipage}{2in}
\epsfig{file=part4/figs/microprogrammed.eps,width=2in}
\end{minipage}

\vfill

\pagebreak
