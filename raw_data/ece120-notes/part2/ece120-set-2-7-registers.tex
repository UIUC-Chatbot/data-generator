\classtitle

\subsection{Registers}

This set of notes introduces registers, an abstraction used for 
storage of groups of bits in digital systems.  We introduce some
terminology used to describe aspects of register design and
illustrate the idea of a shift register.  The registers shown here
are important abstractions for digital system design.\\
%
% {\em In the Fall 2012 offering of our course, we will cover this
% material on the third midterm.}
%

\subsubsection{Registers}

\begin{minipage}{2.85in}
A {\bf register} is a storage element composed from one or more
flip-flops operating on a common clock.
%
In addition to the flip-flops,
most registers include logic to control the bits stored by the register.
%
For example, D~flip-flops 
copy their inputs at the rising edge of each clock cycle,
discarding whatever bits they have stored before the rising edge
(in the previous clock cycle).
%
To enable a flip-flop to retain its value, we might try to hide the 
rising edge of the clock from the flip-flop, as shown to the right.\mpline

The $LOAD$ input controls the clock signals through a method known as
{\bf clock gating}.\linebreak
\end{minipage}\hspace{.25in}%
\begin{minipage}{3.4in}
\centerline{\epsfig{file=part2/figs/lec16-1a.eps,width=2.45in}}\vspace{18pt}
\epsfig{file=part2/figs/lec16-1b.eps,width=3.4in}\vspace{12pt}
\end{minipage}\mpdone

When $LOAD$ is high, the circuit reduces to a
regular D~flip-flop.  When $LOAD$ is low, the flip-flop clock 
input,~$c$, is held high, and the flip-flop stores its 
current value.
%
The problems with clock gating are twofold.  First, adding logic to
the clock path introduces clock skew, which may cause timing problems
later in the development process (or, worse, in future projects that
%
use your circuits as components).  Second, in the design shown above,
the $LOAD$ signal
can only be lowered while the clock is high to prevent spurious rising
edges from causing incorrect behavior, as shown in the timing diagram.

\begin{minipage}{4.25in}
A better approach is to use a mux and a feedback loop from the 
flip-flop's output, as shown in the figure to the right.  
%
When $LOAD$ is low, the mux selects the feedback line, and the 
register reloads its current value.  
%
When $LOAD$ is high, the mux selects the $IN$ input, and the register 
loads a new value.  
%
The result is similar to a gated D~latch with distinct write enable 
and clock lines.
\end{minipage}\hspace{.25in}%
\begin{minipage}{2in}
\epsfig{file=part2/figs/lec16-2.eps,width=2in}
\end{minipage}\vspace{4pt}

\begin{minipage}{1.25in}
We can use this extended flip-flop as a bit slice for a multi-bit register.
%
A four-bit register of this type
is shown to the right.  Four data \mbox{lines---one} for each
\mbox{bit---enter} the registers from the top of the figure.  
When~$LOAD$ is low, the logic copies each flip-flop's value back to its
input,\linebreak
\end{minipage}\hspace{.25in}%
\begin{minipage}{5in}
\epsfig{file=part2/figs/lec16-3.eps,width=5in}
\end{minipage}\mpdone

and the~$IN$ input lines are ignored.  When~$LOAD$ is high,
the muxes forward each~$IN$ line to the corresponding flip-flop's~$D$
input, allowing the register to load the new~\mbox{4-bit} value.
The use of one input line per bit to load a multi-bit
register in a single cycle is termed a {\bf parallel load}.\\

\subsubsection{Shift Registers}

\begin{minipage}{1.25in}
Certain types of registers include logic to manipulate data held
within the register.  A {\bf shift register} is an important example
of this\linebreak
\end{minipage}\hspace{.25in}%
\begin{minipage}{5in}
\epsfig{file=part2/figs/lec16-4.eps,width=5in}\vspace{12pt}
\end{minipage}\mpdone

type.  The simplest shift register is a series of D~flip-flops,
with the output of each attached to the input of the next, as shown to the
right above.  In the circuit shown, a serial input $SI$ accepts a single bit 
of data per cycle and delivers the bit four cycles later to a serial 
output $SO$.  Shift registers serve many purposes in modern systems, from the
obvious uses of providing a fixed delay and performing bit shifts for
processor arithmetic to rate matching between components and reducing
the pin count on programmable logic devices such as field programmable
gate arrays (FPGAs), the modern form of the programmable logic array
mentioned in the textbook.

An example helps to illustrate the rate matching problem: 
historical I/O buses used fairly slow clocks, as they had to
drive signals and be arbitrated over relatively long distances.
The Peripheral Control
Interconnect (PCI) standard, for example, provided for~33 and~66~MHz
bus speeds.  To provide adequate data rates, such buses use many wires
in parallel, either~32 or~64 in the case of PCI.  In contrast, a
Gigabit Ethernet (local area network) signal travelling over a fiber
is clocked at 1.25~GHz, but sends only one bit per cycle.  Several
layers of shift registers sit between the fiber and the I/O bus to
mediate between the slow, highly parallel signals that travel over the
I/O bus and the fast, serial signals that travel over the 
fiber.  The latest variant of PCI, PCIe (e for ``express''),
uses serial lines at much higher clock rates.

Returning to the figure above, imagine that the outputs $Q_i$ feed
into logic clocked at $1/4^{th}$ the rate of the shift register 
(and suitably synchronized).  Every four cycles, the flip-flops fill
up with another four bits, at which point the outputs are read in
parallel.  The shift register shown can thus serve to transform serial
data to \mbox{4-bit-parallel} data at one-quarter the clock speed.
Unlike the registers discussed earlier, the shift register above does
not support parallel load, which prevents it from transforming a slow,
parallel stream of data into a high-speed serial stream.  The use of
{\bf serial load} requires $N$ cycles for an \mbox{N-bit}
register, but can reduce the number of wires needed to support the
operation of the shift register.  How would you add support for
parallel load?  How many additional inputs would be necessary?

The shift register above also shifts continuously, and cannot store a 
value.  A set of muxes, analogous to those that we used to control 
register loading, can be applied to control shifting, as shown 
below.\\

\centerline{\epsfig{file=part2/figs/lec16-5.eps,width=5.3in}}

Using a \mbox{4-to-1}~mux, we can construct a shift
register with additional functionality.  The bit slice at the top
of the next page allows us to build a {\bf bidirectional shift register} with 
parallel load capability and the ability to retain its value indefinitely.
The two-bit control input~$C$ uses a representation that
we have chosen for the four operations supported by our shift register, 
as shown in the table below the bit slice design.

\begin{minipage}{3.95in}
The bit slice allows us to build~\mbox{$N$-bit} shift registers by
replicating the slice and adding a fixed amount of ``{\bf glue logic}.''
For example, the figure below represents a~\mbox{4-bit} bidirectional 
shift register constructed in this way.  The mux
used for the $SO$ output logic is the glue logic needed in addition
to the four bit slices.
%
At each rising clock edge, the action specified by~$C_1C_0$ is taken.  
When $C_1C_0=00$, the
register holds its current value, with the register
value appearing on
$Q[3:0]$ and each flip-flop feeding its output back into its input.
For $C_1C_0=01$, the shift register shifts left: the serial input,
$SI$, is fed into flip-flop~0, and $Q_3$ is passed to the serial
output, $SO$.  Similarly, when $C_1C_0=11$, the shift register shifts
right: $SI$ is fed into flip-flop~3, and $Q_0$ is passed to $SO$.
Finally, the case $C_1C_0=10$ causes all flip-flops to accept new
values from $IN[3:0]$, effecting a parallel load.\\
\end{minipage}\hspace{.25in}%
\begin{minipage}{2.3in}
\epsfig{file=part2/figs/lec16-6.eps,width=2.3in}\vspace{6pt}
\begin{tabular}{c|c}
$C_1C_0$& meaning\\ \hline
00& retain current value\\
01& shift left (low to high)\\
10& load new value (from $IN$)\\
11& shift right (high to low)\\
\end{tabular}
\end{minipage}\vspace{-4pt}

\centerline{\epsfig{file=part2/figs/lec16-7.eps,width=5.2in}}

Several specialized shift operations are used to support data
manipulation in modern processors~(CPUs).  Essentially, these
specializations dictate the glue logic for a shift
register as well as the serial input value.  The simplest is a {\bf
logical shift}, for which $SI$ is hardwired to~$0$: incoming
bits are always~$0$.  A {\bf cyclic shift} takes $SO$ and feeds it
back into $SI$, forming a circle of register bits through which the
data bits cycle.

Finally, an {\bf arithmetic shift} treats the shift register contents
as a number in 2's~complement form.  For non-negative numbers and left
shifts, an arithmetic shift is the same as a logical
shift.  When a negative number is arithmetically shifted to
the right, however, the sign bit is retained, resulting in a function
similar to division by two.  The difference lies in the rounding
direction.  Division by two rounds towards zero in most 
processors: $-5/2$~gives~$-2$.
Arithmetic shift right rounds away from zero for negative numbers (and
towards zero for positive numbers): $-5>>1$~gives~$-3$.  We transform our
previous shift register into one capable of arithmetic shifts by
eliminating the serial input and feeding the most significant bit,
which represents the sign in 2's~complement form, back into itself for
right shifts, as shown below.  The bit shifted in for left shifts
has been hardwired to~0.\vspace{4pt}

\centerline{\epsfig{file=part2/figs/lec16-8.eps,width=5.2in}}

\pagebreak

