\classtitle

\subsection{Summary of Part 2 of the Course}

These notes supplement the Patt and Patel textbook, so you will also 
need to read and understand the relevant chapters (see the syllabus)
in order to master this material completely.

The difficulty of learning depends on the type of task involved.
Remembering new terminology is relatively easy, while applying
the ideas underlying design decisions shown by example to new problems 
posed as human tasks is relatively hard.
%
In this short summary, we give you lists at several levels of difficulty 
of what we expect you to be able to do as a result of the last few weeks 
of studying (reading, listening, doing homework, discussing your 
understanding with your classmates, and so forth).

We'll start with the skills, and leave the easy stuff for the next page.
%
We expect you to be able to exercise the following skills:

\begin{list}{$\bullet$}{\setlength{\itemsep}{0pt}\setlength{\parskip}{0pt}%
\setlength{\topsep}{0pt}\setlength{\partopsep}{0pt}\setlength{\parsep}{0pt}}

\item{Design a CMOS gate for a simple Boolean function from n-type 
and p-type transistors.}

\item{Apply DeMorgan's laws repeatedly to simplify the form of
the complement of a Boolean expression.}

\item{Use a K-map to find a reasonable expression for a Boolean function (for
example, in POS or SOP form with the minimal number of terms).}

\item{More generally, translate Boolean logic functions among 
concise algebraic, truth table, K-map, and canonical (minterm/maxterm) forms.}

\end{list}

When designing combinational logic, we expect you to be able to apply
the following design strategies:

\begin{list}{$\bullet$}{\setlength{\itemsep}{0pt}\setlength{\parskip}{0pt}%
\setlength{\topsep}{0pt}\setlength{\partopsep}{0pt}\setlength{\parsep}{0pt}}

\item{Make use of human algorithms 
(for example, multiplication from addition).}

\item{Determine whether a bit-sliced approach is applicable, and, if so,
make use of one.}

\item{Break truth tables into parts so as to solve each part of a function 
separately.}

\item{Make use of known abstractions (adders, comparators, muxes, or other
abstractions available to you) to simplify the problem.}

\end{list}

And, at the highest level, we expect that you will be able to do the following:

\begin{list}{$\bullet$}{\setlength{\itemsep}{0pt}\setlength{\parskip}{0pt}%
\setlength{\topsep}{0pt}\setlength{\partopsep}{0pt}\setlength{\parsep}{0pt}}

\item{Understand and be able to reason at a high-level about circuit design
tradeoffs between area/cost and performance (and to know that power is also 
important, but we haven't given you any quantification methods).}

\item{Understand the tradeoffs typically made to develop bit-sliced 
designs---typically, bit-sliced designs are simpler but bigger and 
slower---and how one can develop variants between the extremes of
the bit-sliced approach and optimization of functions specific
to an~\mbox{$N$-bit} design.}

\item{Understand the pitfalls of marking a function's value as ``don't care'' 
for some input combinations, and recognize that implementations do not 
produce ``don't care.''}

\item{Understand the tradeoffs involved in selecting a representation for
communicating information between elements in a design, such as the bit 
slices in a bit-sliced design.}

\item{Explain the operation of a latch or a flip-flop, particularly in 
terms of the bistable states used to hold a bit.}

\item{Understand and be able to articulate the value of the clocked 
synchronous design abstraction.}

\end{list}

\vfill

\pagebreak

You should recognize all of these terms
and be able to explain what they mean.  For the specific circuits, you 
should be able to draw them and explain how they work.
%
Actually, we don't care whether you can draw something from memory---a full
adder, for example---provided that you know what a full adder does and can
derive a gate diagram correctly for one in a few minutes.  Higher-level
skills are much more valuable.  

\begin{minipage}[t]{2.8in}
\begin{list}{$\bullet$}{\setlength{\itemsep}{0pt}\setlength{\parskip}{0pt}%
\setlength{\topsep}{0pt}\setlength{\partopsep}{0pt}\setlength{\parsep}{0pt}}

\item{Boolean functions and logic gates}
\begin{list}{-}{\setlength{\itemsep}{0pt}\setlength{\parskip}{0pt}%
\setlength{\topsep}{0pt}\setlength{\partopsep}{0pt}\setlength{\parsep}{0pt}}
\item{NOT/inverter}
\item{AND}
\item{OR}
\item{XOR}
\item{NAND}
\item{NOR}
\item{XNOR}
\item{majority function}
\end{list}

\item{specific logic circuits}
\begin{list}{-}{\setlength{\itemsep}{0pt}\setlength{\parskip}{0pt}%
\setlength{\topsep}{0pt}\setlength{\partopsep}{0pt}\setlength{\parsep}{0pt}}
\item{full adder}
%\item{half adder}
\item{ripple carry adder}
\item N-to-M multiplexer (mux)
\item N-to-2N decoder
\item{\mbox{$\bar{\mbox{R}}$-$\bar{\mbox{S}}$~latch}}
\item{\mbox{R-S~latch}}
\item{gated D latch}
%
% SSL altered term 3 Dec 21 
%
%\item{master-slave implementation of a positive edge-triggered D~flip-flop}
\item{dual-latch implementation of a positive edge-triggered D~flip-flop}
\item{(bidirectional) shift register}
\item{register supporting parallel load}
\end{list}

\item{design metrics}
\begin{list}{-}{\setlength{\itemsep}{0pt}\setlength{\parskip}{0pt}%
\setlength{\topsep}{0pt}\setlength{\partopsep}{0pt}\setlength{\parsep}{0pt}}
\item{metric}
\item{optimal}
\item{heuristic}
\item{constraints}
\item{power, area/cost, performance}
\item{computer-aided design (CAD) tools}
\item{gate delay}
\end{list}

\item{general math concepts}
\begin{list}{-}{\setlength{\itemsep}{0pt}\setlength{\parskip}{0pt}%
\setlength{\topsep}{0pt}\setlength{\partopsep}{0pt}\setlength{\parsep}{0pt}}
\item{canonical form}
%\item{domain of a function}
\item{\mbox{$N$-dimensional} hypercube}
\end{list}

\item{tools for solving logic problems}
\begin{list}{-}{\setlength{\itemsep}{0pt}\setlength{\parskip}{0pt}%
\setlength{\topsep}{0pt}\setlength{\partopsep}{0pt}\setlength{\parsep}{0pt}}
\item{truth table}
\item{Karnaugh map (K-map)}
\item{implicant}
\item{prime implicant}
\item{bit-slicing}
\item{timing diagram}
\end{list}

\end{list}
\end{minipage}\hspace{.70in}%
\begin{minipage}[t]{3in}
\begin{list}{$\bullet$}{\setlength{\itemsep}{0pt}\setlength{\parskip}{0pt}%
\setlength{\topsep}{0pt}\setlength{\partopsep}{0pt}\setlength{\parsep}{0pt}}

\item{device technology}
\begin{list}{-}{\setlength{\itemsep}{0pt}\setlength{\parskip}{0pt}%
\setlength{\topsep}{0pt}\setlength{\partopsep}{0pt}\setlength{\parsep}{0pt}}
\item{complementary metal-oxide\\ semiconductor (CMOS)}
\item{field effect transistor (FET)}
\item{transistor gate, source, drain}
\end{list}

\item{Boolean logic terms}
\begin{list}{-}{\setlength{\itemsep}{0pt}\setlength{\parskip}{0pt}%
\setlength{\topsep}{0pt}\setlength{\partopsep}{0pt}\setlength{\parsep}{0pt}}
\item{literal}
\item{algebraic properties}
\item{dual form, principle of duality}
\item{sum, product}
\item{minterm, maxterm}
\item{sum-of-products (SOP)}
\item{product-of-sums (POS)}
\item{canonical sum/SOP form}
\item{canonical product/POS form}
\item{logical equivalence}
\end{list}

\item{digital systems terms}
\begin{list}{-}{\setlength{\itemsep}{0pt}\setlength{\parskip}{0pt}%
\setlength{\topsep}{0pt}\setlength{\partopsep}{0pt}\setlength{\parsep}{0pt}}
\item{word size}
\item{\mbox{$N$-bit} Gray code}
\item{combinational/combinatorial logic}
\begin{list}{-}{\setlength{\itemsep}{0pt}\setlength{\parskip}{0pt}%
\setlength{\topsep}{0pt}\setlength{\partopsep}{0pt}\setlength{\parsep}{0pt}}
\item{two-level logic}
\item{``don't care'' outputs (x's)}
\end{list}
\item{sequential logic}
\begin{list}{-}{\setlength{\itemsep}{0pt}\setlength{\parskip}{0pt}%
\setlength{\topsep}{0pt}\setlength{\partopsep}{0pt}\setlength{\parsep}{0pt}}
\item{state}
\item{active low input}
\item{set a bit (to 1)}
\item{reset a bit (to 0)}
%
% SSL altered term 3 Dec 21 
%
%\item{master-slave implementation}
\item{dual-latch implementation}
\item{positive edge-triggered}
\end{list}
\item{clock signal}
\begin{list}{-}{\setlength{\itemsep}{0pt}\setlength{\parskip}{0pt}%
\setlength{\topsep}{0pt}\setlength{\partopsep}{0pt}\setlength{\parsep}{0pt}}
\item{square wave}
\item{rising/positive clock edge}
\item{falling/negative clock edge}
\item{clock gating}
\end{list}
\item{clocked synchronous sequential circuit}
\item{parallel/serial load of register}
% FIXME?  too informal to ask them to remember it
% \item{glue logic}
\item{logical/arithmetic/cyclic shift}
\end{list}


\end{list}

\end{minipage}\hspace{.25in}

\pagebreak

\mbox{~~~} % blank 3rd page



\vfill

\pagebreak

