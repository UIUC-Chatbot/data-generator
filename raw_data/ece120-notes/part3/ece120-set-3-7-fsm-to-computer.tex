\classtitle

\subsection{From FSM to Computer}

The FSM designs we have explored so far have started with a human-based
design process in which someone writes down the desired behavior in
terms of states, inputs, outputs, and transitions.  Such an approach
makes it easier to build a digital FSM, since the abstraction used
corresponds almost directly to the implementation.

As an alternative, one can start by mapping the desired task into a
high-level programming language, then using components such as registers,
counters, and memories to implement the variables needed.  In this approach,
the control structure of the code maps into a high-level FSM design.
Of course, in order to implement our FSM with digital logic, we eventually
still need to map down to bits and gates.

In this set of notes, we show how one can transform a piece of code
written in a high-level language into an FSM.  This process is meant to
help you understand how we can design an FSM that executes simple
pieces of a flow chart such as assignments, {\tfix if} statements, and 
loops.  Later, we generalize this concept and build an FSM that allows
the pieces to be executed to be specified after the FSM is built---in 
other words, the FSM executes a program specified by bits stored in 
memory.  This more general model, as you might have already guessed, 
is a computer.  \\

\subsubsection{Specifying the Problem}

Let's begin by specifying the problem that we want to solve.
Say that we want to find the minimum value in a set of~10 integers.
Using the~C programming language, we can write the following fragment of 
code:\\

\centerline{\begin{minipage}{4in}
{\fix
\begin{tabbing}
aaaa\=aaaa\=\kill
int \>values[10];~~~~/* 10 integers--filled in by other code */\\
int \>idx;\\
int \>min\\
\\
min = values[0];\\
for (idx = 1; 10 > idx; idx = idx + 1) \{\\
\>  if (min > values[idx]) \{\\
\>  \>  min = values[idx];\\
\>  \}\\
\}\\
/* The minimum value from array is now in min. */
\end{tabbing}
}
\end{minipage}}

The code uses array notation, which we have not used previously in our 
class, so let's first discuss the meaning of the code.

The code uses three variables.
%
The variable {\tfix values} represents the~10 values in our set.
The suffix ``[10]'' after the variable name tells the compiler that
we want an array of~10 integers~({\tfix int}) indexed from~0 to~9.
These integers can be treated as~10 separate variables, but can be
accessed using the single name ``{\tfix values}'' along with an index
(again, from~0 to~9 in this case).
%
The variable {\tfix idx} holds a loop index that we use to examine each
of the values one by one in order to find the minimum value in the set.
%
Finally, the variable {\tfix min} holds the smallest known value as 
the program examines each of the values in the set.

The program body consists of two statements.  
%
We assume that some other piece of code---one not shown here---has 
initialized the~10 values in our set before the code above executes.
%
The first statement initializes the
minimum known value ({\tfix min}) to the value stored at index~0 in the 
array~({\tfix values[0]}).
The second statement is a loop in which the variable {\tfix index} 
takes on values from~1 to~9.  For each value, an {\tfix if} statement
compares the current
known minimum with the value stored in the array at index given by the
{\tfix idx} variable.  If the stored value is smaller, the current known 
value (again, {\tfix min}) is updated to reflect the program's
having found a smaller value.  When the loop finishes all nine iterations,
the variable {\tfix min} holds the smallest value among the set of~10 
integers stored in the {\tfix values} array.

\begin{minipage}{2.47in}
As a first step towards designing an FSM to implement the code, we transform
the code into a flow chart, as shown to the right.  The program again begins
with initialization, which appears in the second column of the flow chart.  
The loop in the program translates to the third column of the flow chart, 
and the {\tfix if} statement to the middle comparison and update 
of {\tfix min}.\mpline

Our goal is now to design an FSM to implement the flow chart.  In order
to do so, we want to leverage the same kind of abstraction that we used
earlier, when extending our keyless entry system with a timer.  Although the
timer's value was technically also\linebreak\mpdone
\end{minipage}\hspace{.25in}%
\begin{minipage}{3.78in}
\centerline{\epsfig{file=part3/figs/part3-min-flow-chart.eps,width=3.78in}}
\end{minipage}

part of the FSM's state, we treated it
as data and integrated it into our next-state decisions in only a couple
of cases.

For our minimum value problem, we have two sources of data.  First, an
external program supplies data in the form of a set of~10 integers.  If
we assume~\mbox{32-bit} integers, these data technically form 320~input bits!
Second, as with the keyless entry system timer, we have data used internally
by our FSM, such as the loop index and the current minimum value.  These
are technically state bits.  For both types of data, we treat them
abstractly as values rather than thinking of them individually as bits,
allowing us to develop our FSM at a high-level and then to implement it 
using the components that we have developed earlier in our course.\\

\subsubsection{Choosing Components and Identifying States}

Now we are ready to design an FSM that implements the flow chart.
What components do we need, other than our state logic?
We use registers and counters to implement the variables {\tfix idx}
and {\tfix min} in the program.
For the array {\tfix values}, we use a~\mbox{16$\times$32-bit} 
memory.\footnote{We technically only need a~\mbox{10$\times$32-bit} 
memory, but we round up the size of the address space to reflect more
realistic memory designs; one can always optimize later.}
We need a comparator to implement the test for the {\tfix if} statement.
We choose to use a serial comparator, which allows us to illustrate again
how one logical high-level state can be subdivided into many actual states.
To operate the serial comparator, we make use of two shift registers that 
present the comparator with one bit per cycle on each input, and a counter
to keep track of the comparator's progress.

How do we identify high-level states from our flow chart?  Although
the flow chart attempts to break down the program into `simple' steps,
one step of a flow chart may sometimes require more than one state
in an FSM.  Similarly, one FSM state may be able to implement several
steps in a flow chart, if those steps can be performed simultaneously.
Our design illustrates both possibilities.

How we map flow chart elements into FSM states also depends to some 
degree on what components we use, which is why we began with some discussion
of components.  In practice, one can go back and forth between the two, 
adjusting components to better match the high-level states, and adjusting 
states to better match the desired components.

Finally, note that we are only concerned with high-level states, so we do 
not need to provide details (yet) down to the level of individual clock 
cycles, but we do want to define high-level states that can be implemented
in a fixed number of cycles, or at least a controllable number of cycles.
If we cannot specify clearly when transitions occur from an FSM state, we
may not be able to implement the state.

\pagebreak

Now let's go through the flow chart and identify states.  Initialization of
{\tfix min} and {\tfix idx} need not occur serially, and the result of the
first comparison between {\tfix idx} and the constant~10 is known in advance,
so we can merge all three operations into a single state, which we 
call {\tfix INIT}.

We can also merge the updates of {\tfix min} and {\tfix idx} into a second
FSM state, which we call {\tfix COPY}.  However, the update to {\tfix min} 
occurs only when the comparison ({\tfix min > value[idx]}) is true.  
We can use logic to predicate execution of the update.  In other words, we 
can use the output of the comparator, which is available after the comparator 
has finished comparing the two values (in a high-level FSM state that we 
have yet to define), to determine whether or not the register holding 
{\tfix min} loads a new value in the {\tfix COPY} state.

Our model of use for this FSM involves external logic filling the memory
(the array of integer values), executing the FSM ``code,'' and then
checking the answer.  To support this use model, we create a FSM state 
called {\tfix WAIT} for cycles in which the FSM has no work to do.
Later, we also make use of an external input signal {\tfix START} 
to start the FSM
execution.  The {\tfix WAIT} state logically corresponds to the ``START'' 
bubble in the flow chart.

\begin{minipage}{2.29in}
Only the test for the {\tfix if} statement remains.  Using a serial
comparator to compare two~\mbox{32-bit} values requires~32~cycles.
However, we need an additional cycle to move values into our shift 
registers so that the comparator can see the first bit.  Thus our
single comparison operation breaks into two high-level states.  In the
first state, which we call {\tfix PREP}, we copy {\tfix min} to one
of the shift registers, copy {\tfix values[idx]} to the other shift
register, and reset the counter that measures the cycles needed for
our serial comparator.  We then move to a second high-level state,
which we call {\tfix COMPARE}, in which we feed one bit per cycle
from each shift register to the serial comparator.  The {\tfix COMPARE} 
state\linebreak\mpdone
\end{minipage}\hspace{.25in}%
\begin{minipage}{3.96in}
\centerline{\epsfig{file=part3/figs/part3-min-flow-chart-states.eps,width=3.96in}}
\end{minipage}

executes for~32~cycles, after which the comparator
produces the one-bit answer that we need, and we can move to the
{\tfix COPY} state.  The association between the flow chart and the
high-level FSM states is illustrated in the figure shown to the right
above.

\begin{minipage}{3.25in}
We can now also draw an abstract state diagram for our FSM, as shown
to the right.  The FSM begins in the {\tfix WAIT} state.  After external
logic fills the {\tfix values} array, it signals the FSM to begin by
raising the {\tfix START} signal.  The FSM transitions into the 
{\tfix INIT} state, and in the next cycle into the {\tfix PREP} state.
From {\tfix PREP}, the FSM always moves to {\tfix COMPARE}, where it
remains for 32~cycles while the serial comparator executes a comparison.
After {\tfix COMPARE}, the FSM moves to the {\tfix COPY}\linebreak\mpdone
\end{minipage}\hspace{.25in}%
\begin{minipage}{3in}
\centerline{\epsfig{file=part3/figs/part3-min-state-diag.eps,width=3in}}
\end{minipage}

state, where
it remains for one cycle.  The transition from {\tfix COPY} depends on
how many loop iterations have executed.  If more loop iterations remain,
the FSM moves to {\tfix PREP} to execute the next iteration.  If the
loop is done, the FSM returns to {\tfix WAIT} to allow external logic
to read the result of the computation.\\

\pagebreak

\subsubsection{Laying Out Components}

\begin{minipage}{2.41in}
Our high-level FSM design tells us what our components need to be able to
do in any given cycle.  For example, when we load new values into the shift
registers that provide bits to the serial comparator, we always copy 
{\tfix min} into one shift register and {\tfix values[idx]} into the second.
Using this information, we can put together our components and simplify our
design by fixing the way in which bits flow between them.\mpline

The figure at the right shows how we can organize our components.
Again, in practice, one goes back and forth thinking about states,
components, and flow from state to state.  In these notes, we 
present only a completed design.\mpline

Let's take a detailed look at each of the components.
%
At the upper left of the figure is a \mbox{4-bit} binary counter called
{\tfix IDX} to hold the {\tfix idx}
variable.  The counter can be reset to~0 using the {\tfix RST} input.
Otherwise, the {\tfix CNT} input controls whether or not the counter
increments its value.  With this counter design, we can force {\tfix idx} 
to~0 in the {\tfix WAIT} state and then count upwards in the {\tfix INIT}
and {\tfix COPY} states.
\end{minipage}\hspace{.25in}%
\begin{minipage}{3.84in}
\centerline{\epsfig{file=part3/figs/part3-min-components.eps,width=3.84in}}
\end{minipage}

A memory labeled {\tfix VALUES} to hold the array {\tfix values} appears 
in the upper right of
the figure.  The read/write control for the memory is hardwired to~1~(read)
in the figure, and the data input lines are unattached.  To integrate 
with other logic that can operate our FSM, we need to add more 
control logic to allow writing into the memory and to attach the data
inputs to something that provides the data bits.  The address input of
the memory comes always from the {\tfix IDX} counter value; in other words,
whenever we access this memory by making use of the data output lines,
we read {\tfix values[idx]}.

In the middle left of the figure is a \mbox{32-bit} register for the 
{\tfix min} variable.  It has a control input {\tfix LD} that 
determines whether or not it loads a new value at the end of the clock
cycle.  If a new value is loaded, the new value always corresponds to
the output of the {\tfix VALUES} memory, {\tfix values[idx]}.  Recall
that {\tfix min} always changes in the {\tfix INIT} state, and may change
in the {\tfix COPY} state.  But the new value stored in {\tfix min}
is always {\tfix values[idx]}. 
Note also that when the FSM completes its task, the result of the 
computation is left in the {\tfix MIN} register for external logic to
read (connections for this purpose are not shown in the figure).

Continuing downward in the figure, we see two right shift registers
labeled~{\tfix A} and~{\tfix B}.  Each has a control input {\tfix LD}
that enables a parallel load.  Register~{\tfix A} loads from register
{\tfix MIN}, and register~{\tfix B} loads from the memory data output
({\tfix values[idx]}).  These loads are needed in the {\tfix PREP}
state of our FSM.  When {\tfix LD} is low, the shift registers
simply shifts to the right.  The serial output {\tfix SO} makes the
least significant bit of each shift register available.  Shifting
is necessary to feed the serial comparator in the {\tfix COMPARE} state.

Below register~{\tfix A} is a \mbox{5-bit} binary counter called {\tfix CNT}.
The counter is used to control the serial comparator in the {\tfix COMPARE}
state.  A reset input {\tfix RST} allows it to be forced to~0 in the 
{\tfix PREP} state.  When the counter value is exactly zero, the 
output~{\tfix Z} is high.

\pagebreak

The last major component is the serial comparator, which is based on the
design developed in Notes~Set~3.1.  The two bits to be compared in 
a cycle come from shift registers~{\tfix A} and~{\tfix B}.  The first
bit indicator comes from the zero indicator of counter~{\tfix CNT}.
The comparator actually produces two outputs ({\tfix Z1} and {\tfix Z0}),
but the meaning of the
{\tfix Z1} output by itself is {\tfix A~$>$~B}.  In the diagram,
this signal has been labeled~{\tfix THEN}.

There are two additional elements in the figure that we have yet to discuss.
Each simply compares the value in a register with a fixed constant and 
produces a~\mbox{1-bit} signal.  When the FSM finishes an iteration of
the loop in the {\tfix COPY} state, it must check the loop condition
({\tfix 10~$>$~idx}) and move either to the {\tfix PREP} state or, when
the loop finishes, to the {\tfix WAIT} state to let the external logic
read the answer from the {\tfix MIN} register.  The loop is done when the
current iteration count is nine, so we compare {\tfix IDX} with nine to
produce the {\tfix DONE} signal.  The other constant comparison is between
the counter {\tfix CNT} and the value~31 to produce the {\tfix LAST} signal,
which indicates that the serial comparator is on its last cycle of 
comparison.  In the cycle after {\tfix LAST} is high, the {\tfix THEN} 
output of the comparator indicates whether or not {\tfix A~$>$~B}.\\

\subsubsection{Control and Data}

One can think of the components and the interconnections between them
as enabling the movement of data between registers, while the high-level 
FSM controls which data move from register to register in each cycle.  
%
With this model in mind, we call the components and interconnections
for our design the {\bf datapath}---a term that we will see again when
we examine the parts of a computer in the coming weeks.
%
The datapath requires several inputs to control the operation of the
components---these we can treat as outputs of the FSM.
These signals allow the FSM to control the motion of data in the 
datapath, so we call them {\bf control signals}.
%
Similarly, the datapath produces several outputs that we can treat
as inputs to the FSM.
%
The tables below summarize the control signals (left) and outputs (right)
from the datapath for our FSM.

\begin{minipage}[t]{3in}
\begin{tabular}{|c|l|}\hline
datapath& \\
input& \multicolumn{1}{|c|}{meaning}\\ \hline
{\tfix IDX.RST}& reset {\tfix IDX} counter to 0\\
{\tfix IDX.CNT}& increment {\tfix IDX} counter\\
{\tfix MIN.LD}& load new value into {\tfix MIN} register\\
{\tfix A.LD}& load new value into shift register {\tfix A}\\
{\tfix B.LD}& load new value into shift register {\tfix B}\\
{\tfix CNT.RST}& reset {\tfix CNT} counter\\ \hline
\end{tabular}
\end{minipage}\hspace{0.15in}%
\begin{minipage}[t]{3.35in}
\begin{tabular}{|c|l|c|}\hline
datapath& &\\
output& \multicolumn{1}{|c|}{meaning}& based on\\ \hline
{\tfix DONE}& last loop iteration finished& {\tfix IDX~=~9}\\
{\tfix LAST}& serial comparator executing& {\tfix CNT~=~31}\\
& ~~~last cycle&\\
{\tfix THEN}& {\tfix if} statement condition true& {\tfix A~$>$~B}\\ \hline
\multicolumn{3}{c}{}\\
\multicolumn{3}{c}{}\\
\end{tabular}
\end{minipage}\\

Using the datapath controls signals and outputs, we can now write a more
formal state transition table for the FSM, as shown below.
%
The ``actions'' column of the table
lists the changes to register and counter values
that are made in each of the FSM states.  The notation used to represent
the actions is called {\bf register transfer language} ({\bf RTL}).
The meaning of an individual action is similar to the meaning of the 
corresponding statement from our C~code or from the flow chart.
For example, in the {\tfix WAIT} state, ``{\tfix IDX~$\leftarrow$~0}''
means the same thing as ``{\tfix idx = 0;}''.  In particular, both mean
that the value currently stored in the {\tfix IDX} counter is overwritten 
with the number~0 (all~0 bits).


\begin{minipage}{1.25in}
The meaning of RTL is slightly different from the usual interpretation of
high-level programming languages, however, in terms of when the actions
happen.  A list of C statements is generally executed one at a time.
%
In contrast, the entire list of RTL actions\linebreak\mpdone
\end{minipage}\hspace{.25in}%
\begin{minipage}{5in}
\begin{tabular}{|c|l|c|c|}\hline
state& \multicolumn{1}{|c|}{actions (simultaneous)}& condition& next state\\ \hline
{\tfix WAIT}& {\tfix IDX}~$\leftarrow$~0 (to read {\tfix VALUES[0]} in {\tfix INIT})& {\tfix START}& {\tfix INIT}\\ 
&& $\overline{\mbox{\tfix START}}$& {\tfix WAIT}\\ \hline
{\tfix INIT}& {\tfix MIN}~$\leftarrow$~{\tfix VALUES[IDX]} ({\tfix IDX} is 0 in this state)& (always)& {\tfix PREP}\\ 
& {\tfix IDX}~$\leftarrow$~{\tfix IDX} + 1& &\\ \hline
{\tfix PREP}& {\tfix A}~$\leftarrow$~{\tfix MIN}& (always)& {\tfix COMPARE}\\
& {\tfix B}~$\leftarrow$~{\tfix VALUES[IDX]}& &\\
& {\tfix CNT}~$\leftarrow$~0& &\\ \hline
{\tfix COMPARE}& run serial comparator& {\tfix LAST}& {\tfix COPY}\\
&& $\overline{\mbox{\tfix LAST}}$& {\tfix COMPARE}\\ \hline
{\tfix COPY}& {\tfix THEN}: {\tfix MIN}~$\leftarrow$~{\tfix VALUES[IDX]}& {\tfix DONE}& {\tfix WAIT}\\ 
& {\tfix IDX}~$\leftarrow$~{\tfix IDX} + 1& $\overline{\mbox{\tfix DONE}}$& {\tfix PREP}\\ \hline
\end{tabular}
\end{minipage}

for an FSM state is executed
simultaneously, at the end of the clock cycle.  As you know, an FSM moves 
from its current state into a new state at the end of every clock cycle,
so actions during different cycles usually are associated with different 
states.
We can, however, change the value in more than one register at the end 
of the same clock cycle, so we can execute more than one RTL action in
the same state, so long as the actions do not exceed the capabilities
of our datapath (the components must be able to support the simultaneous 
execution of the actions).  Some care must be taken with states that 
execute for more than one cycle to ensure that repeating the 
RTL actions is appropriate.  In our design, only the {\tfix WAIT} and
{\tfix COMPARE} states execute for more than one cycle.  The {\tfix WAIT}
state resets the {\tfix IDX} counter repeatedly, which causes no problems.
The {\tfix COMPARE} statement has no RTL actions---all of the shifting,
comparison, and counting activity needed to do its work occurs within 
the datapath itself.

One additional piece of RTL syntax needs explanation.  In the {\tfix COPY}
state, the first action begins with ``{\tfix THEN}:,'' which means that the 
prefixed RTL action occurs only when the {\tfix THEN} signal is high.
Recall that the {\tfix THEN} signal indicates that the comparator has
found {\tfix A~$>$~B}, so the equivalent C~code is ``{\tfix if (A~$>$~B) 
\{min~=~values[idx]\}}''.\\

\subsubsection{State Representation and Logic Expressions}

Let's think about the representation for the FSM states.  The FSM has 
five states, so we could use as few as three flip-flops.  Instead, we choose
to use a {\bf one-hot encoding}, in which any valid bit pattern has exactly
one~1~bit.  In other words, we use five flip-flops instead of three,
and our states are represented with the bit patterns 10000, 01000, 00100,
00010, and 00001.

The table below shows the mapping from each high-level state to 
both the five-bit encoding for the state as well as the six control signals 
needed for the datapath.  For each state, the values of the control signals
can be found by examining the actions necessary in that state.\\

\centerline{
\begin{tabular}{|c|c|cccccc|}\hline
state& {\tfix S$_4$S$_3$S$_2$S$_1$S$_0$}& {\tfix IDX.RST}& {\tfix IDX.CNT}& {\tfix MIN.LD}& {\tfix A.LD}& {\tfix B.LD}& {\tfix CNT.RST}\\ \hline
{\tfix WAIT}& {\tfix 1~0~0~0~0}& 1& 0& 0& 0& 0& 0\\
{\tfix INIT}& {\tfix 0~1~0~0~0}& 0& 1& 1& 0& 0& 0\\
{\tfix PREP}& {\tfix 0~0~1~0~0}& 0& 0& 0& 1& 1& 1\\
{\tfix COMPARE}& {\tfix 0~0~0~1~0}& 0& 0& 0& 0& 0& 0\\
{\tfix COPY}& {\tfix 0~0~0~0~1}& 0& 1& {\tfix THEN}& 0& 0& 0\\ \hline
\multicolumn{8}{c}{}\\
\end{tabular}
}

The {\tfix WAIT} state needs to set {\tfix IDX} to~0 but need not affect 
other register or counter values, so {\tfix WAIT} produces a~1 only for
{\tfix IDX.RST}.  The {\tfix INIT} state needs to load {\tfix values[0]} into 
the {\tfix MIN} register while simultaneously incrementing the {\tfix IDX}
counter (from~0 to~1), so {\tfix INIT} produces~1s for {\tfix IDX.CNT}
and {\tfix MIN.LD}.  The {\tfix PREP} state loads both shift registers
and resets the counter {\tfix CNT} by producing~1s for {\tfix A.LD},
{\tfix B.LD}, and {\tfix CNT.RST}.  The {\tfix COMPARE} state does not
change any register values, so it produces all~0s.  Finally, the {\tfix COPY}
state increments the {\tfix IDX} counter while simultaneously loading a
new value into the {\tfix MIN} register.  The {\tfix COPY} state produces~1
for {\tfix IDX.CNT}, but must use the signal {\tfix THEN} coming from the
datapath to decide whether or not {\tfix MIN} is loaded.

\begin{minipage}{2.15in}
\vspace{8pt}
The advantage of a one-hot encoding becomes obvious when we write
equations for the six control signals and the next-state logic, as shown
to the right.  
%
Implementing the logic to complete our design now requires only a handful 
of small logic gates.
\end{minipage}\hspace{0.4in}%
\begin{minipage}{1.7in}
\begin{eqnarray*}
{\tfix IDX.RST} & = & {\tfix S}_4\\
{\tfix IDX.CNT} & = & {\tfix S}_3 + {\tfix S}_0\\
{\tfix MIN.LD} & = & {\tfix S}_3 + {\tfix S}_0 \cdot {\tfix THEN}\\
{\tfix A.LD} & = & {\tfix S}_2\\
{\tfix B.LD} & = & {\tfix S}_2\\
{\tfix CNT.RST} & = & {\tfix S}_2
\end{eqnarray*}
\end{minipage}\hspace{0.4in}%
\begin{minipage}{1.85in}
\begin{eqnarray*}
{\tfix S}_4^+ & = & {\tfix S}_4 \cdot \overline{\tfix START} + {\tfix S}_0 \cdot {\tfix DONE}\\
{\tfix S}_3^+ & = & {\tfix S}_4 \cdot {\tfix START}\\
{\tfix S}_2^+ & = & {\tfix S}_3 + {\tfix S}_0 \cdot \overline{\tfix DONE}\\
{\tfix S}_1^+ & = & {\tfix S}_2 + {\tfix S}_1 \cdot \overline{\tfix LAST}\\
{\tfix S}_0^+ & = & {\tfix S}_1 \cdot {\tfix LAST}\\
\end{eqnarray*}
\end{minipage}

Notice that the terms in each control signal can be read directly from 
the rows of the state table and OR'd together.  The terms in each of the
next-state equations represent the incoming arcs for the corresponding
state.  For example, the {\tfix WAIT} state has one self-loop (the first
term) and a transition arc coming from the {\tfix COPY} state when the
loop is done.
%
These expressions complete our design.

\vfill

\pagebreak

