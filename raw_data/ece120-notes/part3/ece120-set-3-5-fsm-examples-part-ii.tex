\classtitle

\subsection{Finite State Machine Design Examples, Part II}

This set of notes provides several additional examples of FSM design.
We first design an FSM to control a vending machine, introducing
encoders and decoders as components that help us to implement our
design.  We then design a game controller for a logic puzzle
implemented as a children's game.  Finally, we analyze a digital FSM
designed to control the stoplights at the intersection of two roads.\\


\subsubsection{Design of a Vending Machine}

For the next example, we design an FSM to control a simple vending machine.  
The machine accepts \mbox{U.S.~coins}\footnote{Most countries have small 
bills or coins in demoninations suitable for vending machine prices, so think 
about some other currency if you prefer.} as payment and offers a choice
of three items for sale.

What states does such an FSM need?  The FSM needs to keep track of how
much money has been inserted in order to decide whether a user can 
purchase one of the items.  That information alone is enough for the
simplest machine, but let's create a machine with adjustable item
prices.  We can use registers to hold the item prices, which 
we denote~$P_1$, $P_2$, and~$P_3$.

Technically, the item prices are also part of the internal state of the 
FSM.  However,  we leave out discussion (and, indeed, methods) for setting
the item prices, so no state with a given combination of prices has any 
transition to a state with a different set of item prices.
In other words, any given combination of item prices induces a subset 
of states that operate independently of the subset induced by a distinct 
combination of item prices.  By abstracting away the prices in this way,
we can focus on a general design that allows the owner of the machine
to set the prices dynamically.

\begin{minipage}{4.1in}
Our machine will not accept pennies, so let's have the FSM keep track of
how much money has been inserted as a multiple of 5~cents (one nickel).
The table to the right shows five types of coins, their value in 
dollars, and their value in terms of nickels.\mpline

The most expensive item in the machine might cost a dollar or two, so
the FSM must track at least~20 or~40 nickels of value.  Let's\linebreak
\end{minipage}\hspace{0.25in}%
\begin{minipage}{2.15in}
\begin{tabular}{l|c|c}
\multicolumn{1}{c|}{coin type}& value& \# of nickels\\ \hline
nickel&      \$0.05& 1\\
dime&        \$0.10& 2\\
quarter&     \$0.25& 5\\
half dollar& \$0.50& 10\\
dollar&      \$1.00& 20
\end{tabular}\vspace{12pt}
\end{minipage}\mpdone

decide to
use six bits to record the number of nickels, which allows the machine
to keep track of up to~\$3.15 (63~nickels).  We call the abstract
states {\bf STATE00} through {\bf STATE63}, and refer to a state with
an inserted value of $N$~nickels as {\bf STATE${<}N{>}$}.

Let's now create a next-state table, as shown at the top of the next page.
The user can insert one of the five coin types, or can pick one of the 
three items.  What should happen if the user inserts more money than the 
FSM can track?  Let's make the FSM reject such coins.  Similarly, if the 
user tries to buy an item without inserting enough money first, the FSM 
must reject the request.  For each of the possible input events, we add a 
condition to separate the FSM states that allow the input event to 
be processed as the user desires from those states that do not.  For example,
if the user inserts a quarter, those states with $N<59$ transition to
states with value $N+5$ and accept the quarter.  Those states with
$N\geq{59}$ reject the coin and remain in {\bf STATE${<}N{>}$}.

\centerline{\begin{minipage}{5.6in}
\begin{tabular}{c|l|c|l|c|c}
&&& \multicolumn{3}{|c}{final state}\\
&&& \multicolumn{1}{|c}{}& \multicolumn{1}{c}{accept}& release\\ 
initial state& \multicolumn{1}{|c|}{input event}& condition& \multicolumn{1}{|c}{state}& \multicolumn{1}{c}{coin}& product\\ \hline
{\bf STATE${<}N{>}$}& no input& always& {\bf STATE${<}N{>}$}& ---& none\\
{\bf STATE${<}N{>}$}& nickel inserted& $N<63$& {\bf STATE${<}N+1{>}$}& yes& none\\
{\bf STATE${<}N{>}$}& nickel inserted& $N=63$& {\bf STATE${<}N{>}$}& no& none\\
{\bf STATE${<}N{>}$}& dime inserted& $N<62$& {\bf STATE${<}N+2{>}$}& yes& none\\
{\bf STATE${<}N{>}$}& dime inserted& $N\geq{62}$& {\bf STATE${<}N{>}$}& no& none\\
{\bf STATE${<}N{>}$}& quarter inserted& $N<59$& {\bf STATE${<}N+5{>}$}& yes& none\\
{\bf STATE${<}N{>}$}& quarter inserted& $N\geq{59}$& {\bf STATE${<}N{>}$}& no& none\\
{\bf STATE${<}N{>}$}& half dollar inserted& $N<54$& {\bf STATE${<}N+10{>}$}& yes& none\\
{\bf STATE${<}N{>}$}& half dollar inserted& $N\geq{54}$& {\bf STATE${<}N{>}$}& no& none\\
{\bf STATE${<}N{>}$}& dollar inserted& $N<44$& {\bf STATE${<}N+20{>}$}& yes& none\\
{\bf STATE${<}N{>}$}& dollar inserted& $N\geq{44}$& {\bf STATE${<}N{>}$}& no& none\\
{\bf STATE${<}N{>}$}& item~1 selected& $N\geq{P_1}$& {\bf STATE${<}N-P_1{>}$}& ---& 1\\
{\bf STATE${<}N{>}$}& item~1 selected& $N<P_1$& {\bf STATE${<}N{>}$}& ---& none\\
{\bf STATE${<}N{>}$}& item~2 selected& $N\geq{P_2}$& {\bf STATE${<}N-P_2{>}$}& ---& 2\\
{\bf STATE${<}N{>}$}& item~2 selected& $N<P_2$& {\bf STATE${<}N{>}$}& ---& none\\
{\bf STATE${<}N{>}$}& item~3 selected& $N\geq{P_3}$& {\bf STATE${<}N-P_3{>}$}& ---& 3\\
{\bf STATE${<}N{>}$}& item~3 selected& $N<P_3$& {\bf STATE${<}N{>}$}& ---& none\\
\end{tabular}\vspace{12pt}
\end{minipage}}

We can now begin to formalize the I/O for our machine.  Inputs include 
insertion of coins and selection of items for purchase.  Outputs include
a signal to accept or reject an inserted coin as well as signals to release
each of the three items.

\begin{minipage}{4.8in}
For input to the FSM, we assume that a coin inserted in any given cycle 
is classified and delivered to our FSM using the three-bit representation 
shown to the right.
%
For item selection, we assume that the user has access to three buttons,
$B_1$, $B_2$, and $B_3$, that indicate a desire to purchase the 
corresponding item.\mpline

For output, the FSM must produce a signal~$A$ indicating whether a coin
should be accepted.  To control the release of items that have been purchased,
the FSM must produce the signals~$R_1$, $R_2$, and $R_3$, corresponding
to the re-\linebreak
\end{minipage}\hspace{0.25in}%
\begin{minipage}{1.45in}
\begin{tabular}{l|c}
\multicolumn{1}{c|}{coin type}& $C_2C_1C_0$\\ \hline
none&        110\\
nickel&      010\\
dime&        000\\
quarter&     011\\
half dollar& 001\\
dollar&      111\\
\end{tabular}\vspace{12pt}
\end{minipage}\mpdone

lease of each item.  Since outputs in our class depend only on
state, we extend the internal state of the FSM to include bits for each of
these output signals.  The output signals go high in the cycle after
the inputs that generate them.  Thus, for example, the accept signal~$A$
corresponds to a coin inserted in the previous cycle, even if a second
coin is inserted in the current cycle.  This meaning must be made clear to
whomever builds the mechanical system to return coins.

Now we are ready to complete the specification.  How many states does the
FSM have?  With six bits to record money inserted and four bits to 
drive output signals, we have a total of 1,024~($2^{10}$) states!
Six different coin inputs are possible, and the selection buttons allow
eight possible combinations, giving 48 transitions from each state.
Fortunately, we can use the meaning of the bits to greatly simplify
our analysis.

First, note that the current state of the coin accept bit and item
release bits---the four bits of FSM state that control the outputs---have
no effect on the next state of the FSM.  Thus, we can consider only the
current amount of money in a given state when thinking about the 
transitions from the state.  As you have seen, we can further abstract
the states using the number~$N$, the number of nickels currently held by 
the vending machine.

We must still consider all 48 possible transitions from {\bf STATE${<}N{>}$}.
Looking back at our abstract next-state table, notice that we had only
eight types of input events (not counting ``no input'').  If we
strictly prioritize these eight possible events, we can safely ignore
combinations.  Recall that we adopted a similar strategy for several 
earlier designs, including the ice cream dispenser in Notes Set~2.2 and
the keyless entry system developed in Notes Set~3.1.3.

We choose to prioritize purchases over new coin insertions, and to 
prioritize item~3 over item~2 over item~1.  These prioritizations
are strict in the sense that if the user presses~$B_3$, both other
buttons are ignored, and any coin inserted is rejected, regardless of
whether or not the user can actually purchase item~3 (the machine
may not contain enough money to cover the item price).
%
With the choice of strict prioritization, all transitions from all
states become well-defined.  We apply the transition rules in order of
decreasing priority,
with conditions, and with \mbox{don't-cares} for lower-priority inputs. 
For example, for any of the 16~{\bf STATE50}'s (remember that the four
current output bits do not affect transitions), the table below lists all
possible transitions assuming that $P_3=60$, $P_2=10$, and $P_1=35$.\\

\centerline{
\begin{tabular}{ccccc|ccccc}
&&&&& \multicolumn{5}{|c}{next state}\\
initial state& $B_3$& $B_2$& $B_1$& $C_2C_1C_0$& state& $A$& $R_3$& $R_2$& $R_1$\\ \hline
{\bf STATE50}& 1&x&x& xxx& {\bf STATE50}& 0& 0&0&0\\
{\bf STATE50}& 0&1&x& xxx& {\bf STATE40}& 0& 0&1&0\\
{\bf STATE50}& 0&0&1& xxx& {\bf STATE15}& 0& 0&0&1\\
{\bf STATE50}& 0&0&0& 010& {\bf STATE51}& 1& 0&0&0\\
{\bf STATE50}& 0&0&0& 000& {\bf STATE52}& 1& 0&0&0\\
{\bf STATE50}& 0&0&0& 011& {\bf STATE55}& 1& 0&0&0\\
{\bf STATE50}& 0&0&0& 001& {\bf STATE60}& 1& 0&0&0\\
{\bf STATE50}& 0&0&0& 111& {\bf STATE50}& 0& 0&0&0\\
{\bf STATE50}& 0&0&0& 110& {\bf STATE50}& 0& 0&0&0
\end{tabular}
}
Next, we need to choose a state representation.  But this task is 
essentially done: each output bit ($A$, $R_1$, $R_2$, and $R_3$) is
represented with one bit in the internal representation, and the
remaining six bits record the number of nickels held by the vending
machine using an unsigned representation.

The choice of a numeric representation for the money held is important,
as it allows us to use an adder to compute the money held in
the next state.\\

\subsubsection{Encoders and Decoders}

Since we chose to prioritize purchases, let's begin by building logic
to perform state transitions for purchases.  Our first task is to
implement prioritization among the three selection buttons.  For this
purpose, we construct a \mbox{4-input} {\bf priority encoder}, which 
generates a signal~$P$ whenever any of its four input lines is active
and encodes the index of the highest active input as a two-bit unsigned
number~$S$.   A truth table for our priority encoder appears on the 
left below, with \mbox{K-maps} for each of the output bits on the right.

\begin{minipage}{1.95in}
\begin{tabular}{cccc|cc}\\
$B_3$& $B_2$& $B_1$& $B_0$& $P$& $S$\\ \hline
1&x&x&x& 1& 11\\
0&1&x&x& 1& 10\\
0&0&1&x& 1& 01\\
0&0&0&1& 1& 00\\
0&0&0&0& 0& xx\\
\end{tabular}\vspace{12pt}
\end{minipage}\hspace{0.25in}\begin{minipage}{1.05in}
\epsfig{file=part3/figs/prioencP.eps,width=1.05in}
\end{minipage}\hspace{0.25in}\begin{minipage}{1.05in}
\epsfig{file=part3/figs/prioencS1.eps,width=1.05in}
\end{minipage}\hspace{0.25in}\begin{minipage}{1.05in}
\epsfig{file=part3/figs/prioencS0.eps,width=1.05in}
\end{minipage}

\begin{minipage}{3.6in}
From the \mbox{K-maps}, we extract the following equations:
\begin{eqnarray*}
P &=& B_3 + B_2 + B_1 + B_0\\
S_1 &=& B_3 + B_2\\
S_0 &=& B_3 + \overline{B_2}B_1
\end{eqnarray*}
which allow us to implement our encoder as shown to the right.\mpline

If we connect our buttons~$B_1$, $B_2$, and $B_3$ to the priority 
encoder (and feed~$0$ into the fourth input), it produces a signal~$P$ 
indicating that the user is trying to make a purchase and a two-bit
signal~$S$ indicating which item the user wants.
\end{minipage}\hspace{0.25in}\begin{minipage}{2.65in}
\epsfig{file=part3/figs/prioencimpl.eps,width=2.65in}
\end{minipage}

\begin{minipage}{3.2in}
We also need to build logic to control the item release outputs~$R_1$, $R_2$,
and~$R_3$.  An item should be released only when it has been selected 
(as indicated by the priority encoder signal~$S$) and the vending machine
has enough money.  For now, let's leave aside calculation of the item 
release signal, which we call~$R$, and focus on how we can produce the
correct values of~$R_1$, $R_2$, and~$R_3$ from~$S$ and~$R$.\mpline

The component to the right is a {\bf decoder} with an enable input.  A 
decoder takes an input signal---typically one coded as a binary number---and 
produces one output for each possible value of the signal.  You may
notice the similarity with the structure of a mux: when the decoder
is enabled ($EN=1$), each of the AND gates produces\linebreak
\end{minipage}\hspace{0.25in}%
\begin{minipage}{3.05in}
\epsfig{file=part3/figs/decoder4.eps,width=3.05in}\vspace{12pt}
\end{minipage}\mpdone

one minterm of the input signal~$S$.  In
the mux, each of the inputs is then included in
one minterm's AND gate, and the outputs of all AND gates are ORd together.
In the decoder, the AND gate outputs are the outputs of the decoder.
Thus, when enabled, the decoder produces exactly one~1~bit on its outputs.
When not enabled ($EN=0$), the decoder produces all~0~bits.

We use a decoder to generate the release signals for the vending machine
by connecting the signal~$S$ produced by the 
priority encoder to the decoder's~$S$ input and connecting the item
release signal~$R$ to the decoder's~$EN$ input.  The outputs~$D_1$,
$D_2$, and~$D_3$ then correspond to the individual item release 
signals~$R_1$, $R_2$, and~$R_3$ for our vending machine.

\subsubsection{Vending Machine Implementation}

\begin{minipage}{3.15in}
We are now ready to implement the FSM to handle purchases, as shown to the 
right.  The current number of nickels, $N$, is stored in a register in the
center of the diagram.  Each cycle, $N$ is fed into a \mbox{6-bit} adder,
which subtracts the price of any purchase requested in that cycle. 
%
Recall that we chose to record item prices in registers.  We avoid the 
need to negate prices before adding them by storing the negated prices in
our registers.  Thus, the value of register PRICE1 is~$-P_1$, the
the value of register PRICE2 is~$-P_2$, and the
the value of register PRICE3 is~$-P_3$.
%
The priority encoder's~$S$ signal is then used to select the value of 
one of these three registers (using a \mbox{24-to-6} mux) as the second
input to the adder.\mpline

We use the adder to execute a subtraction, so the carry out~$C_{out}$ 
is~1 whenever the value of~$N$ is at least as great as the amount 
being subtracted.  In that case, the purchase is successful.  The AND
gate on the left calculates the signal~$R$ indicating a successful purchase,
which is then used to select the next value of~$N$ using the \mbox{12-to-6}
mux below the adder.  
%
When no item selection buttons are pushed, $P$ and thus $R$ are both~0, 
and the mux below the adder keeps~$N$ unchanged in the next cycle.  
Similarly, if~$P=1$ but~$N$ is
insufficient,~$C_{out}$ and thus~$R$ are both~0, and again~$N$ does
not change.  Only when~$P=1$ and~$C_{out}=1$ is the purchase successful,
in which case the price is subtracted from~$N$ in the next cycle.
\end{minipage}\hspace{0.25in}%
\begin{minipage}{3.1in}
\epsfig{file=part3/figs/vendingvers1.eps,width=3.1in}\vspace{12pt}
\end{minipage}

The signal~$R$ is also used to enable a decoder that generates the three
individual item release outputs.  The correct output is generated based on
the decoded~$S$ signal from the priority encoder, and all three output
bits are latched into registers to release the purchased item in the next 
cycle.

One minor note on the design so far: by hardwiring $C_{in}$ to~0, we created 
a problem for items that cost nothing (0~nickels): in that case, $C_{out}$ is
always~0.  We could instead store
$-P_1-1$ in PRICE1 (and so forth) and feed~$P$ in to~$C_{in}$, but maybe
it's better not to allow free items.

\begin{minipage}{3.9in}
How can we support coin insertion?  Let's use the same adder to add
each inserted coin's value to~$N$.  The table at the right shows
the value of each coin as a \mbox{5-bit} unsigned number of nickels.
Using this table, we can fill in \mbox{K-maps} for each bit of~$V$, as
shown below.  Notice that we have marked the two undefined bit patterns
for the coin type~$C$ as don't cares in the \mbox{K-maps}.
\end{minipage}\hspace{0.25in}%
\begin{minipage}{2.35in}
\begin{tabular}{l|c|c}
\multicolumn{1}{c|}{coin type}& $C_2C_1C_0$& $V_4V_3V_2V_1V_0$\\ \hline
none&        110& 00000\\
nickel&      010& 00001\\
dime&        000& 00010\\
quarter&     011& 00101\\
half dollar& 001& 01010\\
dollar&      111& 10100
\end{tabular}\vspace{12pt}
\end{minipage}

\begin{minipage}{1.00in}
\epsfig{file=part3/figs/valueV4.eps,width=1.00in}
\end{minipage}\hspace{0.25in}\begin{minipage}{1.00in}
\epsfig{file=part3/figs/valueV3.eps,width=1.00in}
\end{minipage}\hspace{0.25in}\begin{minipage}{1.00in}
\epsfig{file=part3/figs/valueV2.eps,width=1.00in}
\end{minipage}\hspace{0.25in}\begin{minipage}{1.00in}
\epsfig{file=part3/figs/valueV1.eps,width=1.00in}
\end{minipage}\hspace{0.25in}\begin{minipage}{1.00in}
\epsfig{file=part3/figs/valueV0.eps,width=1.00in}
\end{minipage}


\begin{minipage}{3.75in}
Solving the \mbox{K-maps} gives the following equations, which we
implement as shown to the right.\mpline

\begin{eqnarray*}
V_4 &=& C_2C_0\\
V_3 &=& \overline{C_1}C_0\\
V_2 &=& C_1C_0\\
V_1 &=& \overline{C_1}\\
V_0 &=& \overline{C_2}C_1
\end{eqnarray*}
\end{minipage}\hspace{0.25in}%
\begin{minipage}{2.5in}
\epsfig{file=part3/figs/valueimpl.eps,width=2.5in}
\end{minipage}

\begin{minipage}{3.15in}
Now we can extend the design to handle coin insertion, as shown to the right
with new elements highlighted in blue.  The output of the coin value 
calculator is extended with a leading~0 and then fed into a \mbox{12-to-6} mux.
When a purchase is requested, $P=1$ and the mux forwards the item price to 
the adder---recall that we chose to give purchases priority over coin
insertion.  When no purchase is requested, the value of any coin inserted
(or~0 when no coin is inserted) is passed to the adder.\mpline

Two new gates have been added on the lower left.  First, let's verify that
purchases work as before.  When a purchase is requested, $P=1$, so the
NOR gate outputs~0, and the OR gate simple forwards~$R$ to control the
mux that decides whether the purchase was successful, just as in our
original design.\mpline

When no purchase is made ($P=0$, and $R=0$), the adder adds the value of 
any inserted coin to~$N$.  If the addition overflows, $C_{out}=1$, and
the output of the NOR gate is~0.  Note that the NOR gate output is stored
as the output~$A$ in the next cycle, so a coin that causes overflow in the
amount of money stored is rejected.  The OR gate also outputs~0, and~$N$
remains unchanged.  If the addition does not overflow, the NOR gate
outputs a~1, the coin is accepted ($A=1$ in the next cycle), and 
the mux allows the sum~$N+V$ to be written back as the new value of~$N$.
\end{minipage}\hspace{0.25in}%
\begin{minipage}{3.1in}
\epsfig{file=part3/figs/vendingvers2.eps,width=3.1in}
\end{minipage}

The tables at the top of the next page define all of the state variables,
inputs, outputs, and internal signals used in the design, and list the
number of bits for each variable.

\pagebreak

\begin{minipage}{3.7in}
\begin{tabular}{|ccl|}\hline
\multicolumn{3}{|r|}{{\bf FSM state}}\\
PRICE1& 6& negated price of item 1 ($-P_1$)\\
PRICE2& 6& negated price of item 2 ($-P_2$)\\
PRICE3& 6& negated price of item 3 ($-P_3$)\\
$N$& 6& value of money in machine (in nickels)\\
$A$& 1& stored value of accept coin output\\
$R_1$& 1& stored value of release item 1 output\\
$R_2$& 1& stored value of release item 2 output\\
$R_3$& 1& stored value of release item 3 output\\ \hline
\multicolumn{3}{|r|}{{\bf internal signals}}\\
$V$& 5& inserted coin value in nickels\\
$P$& 1& purchase requested (from priority encoder)\\
$S$& 2& item \# requested (from priority encoder)\\
$R$& 1& release item (purchase approved)\\ \hline
\end{tabular}
\end{minipage}\hspace{0.15in}\begin{minipage}{2.65in}
\begin{tabular}{|ccl|}\hline
\multicolumn{3}{|r|}{{\bf inputs}}\\
$B_1$& 1& item 1 selected for purchase\\
$B_2$& 1& item 2 selected for purchase\\
$B_3$& 1& item 3 selected for purchase\\
$C$& 3& coin inserted (see earlier table\\ 
& & for meaning)\\ \hline
\multicolumn{3}{|r|}{{\bf outputs}}\\
$A$& 1& accept inserted coin (last cycle)\\
$R_1$& 1& release item 1\\
$R_2$& 1& release item 2\\
$R_3$& 1& release item 3\\
\multicolumn{3}{|l|}{{\em Note that outputs correspond one-to-one}}\\
\multicolumn{3}{|l|}{{\em with four bits of FSM state.}}\\ \hline
\end{tabular}
\end{minipage}\vspace{12pt}


\subsubsection{Design of a Game Controller}

For the next example, imagine that you are part of a team building a
game for children to play at Engineering Open House.
%
The game revolves around an old logic problem in which a farmer must
cross a river in order to reach the market.  The farmer is traveling
to the market to sell a fox, a goose, and some corn.  The farmer has
a boat, but the boat is only large enough to carry the fox, the goose,
or the corn along with the farmer.  The farmer knows that if he leaves
the fox alone with the goose, the fox will eat the goose.  Similarly,
if the farmer leaves the goose alone with the corn, the goose will 
eat the corn.
%
How can the farmer cross the river?

Your team decides to build a board illustrating the problem with
a river from top to bottom and lights illustrating the positions of 
the farmer (always with the boat), the fox, the goose, and the
corn on either the left bank or the right bank of the river.
Everything starts on the left bank, and the children can play 
the game until they win by getting everything to the right bank or
until they make a mistake.
%
As the ECE major on your team, you get to design the FSM!

%\item{choose a state representation}\label{step-repn}
%\item{calculate logic expressions}\label{step-logic}
%\item{implement with flip-flops and gates}\label{step-gates}

Since the four entities (farmer, fox, goose, and corn) can be only
on one bank or the other, we can use one bit to represent the location
of each entity.  Rather than giving the states names, let's just
call a state $FXGC$.  The value of~$F$ represents the location of the farmer,
either on the left bank~($F=0$) or the right bank~($F=1$).  Using
the same representation (0~for the left bank, 1~for the right bank),
the value of~$X$ represents the location of the fox, $G$~represents the
location of the goose, and~$C$ represents the location of the corn.

\begin{minipage}{2.75in}
We can now put together an abstract next-state table, as shown to
the right.  Once the player wins or loses, let's have the game indicate
their final status and stop accepting requests to have the farmer cross
the river.  We can use a reset button to force the game back into the
original state for the next player.\mpline

Note that we have included conditions for some of the input events, as 
we did previously\linebreak
\end{minipage}\hspace{0.25in}%
\begin{minipage}{3.5in}
\begin{tabular}{c|l|c|c}
initial state& \multicolumn{1}{|c|}{input event}& condition& final state\\ \hline
$FXGC$& no input& always& $FXGC$\\
$FXGC$& reset & always& $0000$\\
$FXGC$& cross alone& always& $\bar{F}XGC$\\
$FXGC$& cross with fox& $F=X$& $\bar{F}\bar{X}GC$\\
$FXGC$& cross with goose& $F=G$& $\bar{F}X\bar{G}C$\\
$FXGC$& cross with corn& $F=C$& $\bar{F}XG\bar{C}$\\
\end{tabular}\vspace{12pt}
\end{minipage}\mpdone

with the vending machine design.
%
The conditions here require that the farmer be on the same bank as any
entity that the player wants the farmer to carry across the river.

Next, we specify the I/O interface. 
%
For input, the game has five buttons.  A reset button~$R$ forces the
FSM back into the initial state.  The other four buttons cause the
farmer to cross the river: $B_F$~crosses alone, $B_X$~with the fox,
$B_G$~with the goose, and $B_C$~with the corn.

For output, we need position indicators for the four entities, but let's
assume that we can simply output the current state $FXGC$ and have
appropriate images or lights appear on the correct banks of the 
river.  We also need two more indicators: $W$~for reaching the winning
state, and~$L$ for reaching a losing state.

Now we are ready to complete the specification.  We could use a strict
prioritization of input events, as we did with earlier examples.  Instead,
in order to vary the designs a bit, we use a strict prioritization among 
allowed inputs.  The reset button~$R$ has the highest priority, followed
by~$B_F$, $B_C$, $B_G$, and finally~$B_X$.  However, only those buttons 
that result in an allowed move are considered when selecting one button 
among several pressed in a single clock cycle.

\begin{minipage}{2.9in}
As an example, consider the 
state~$FXGC=0101$.  The farmer is not on the same bank as the fox, nor
as the corn, so the~$B_X$ and~$B_C$ buttons are ignored, leading to the
next-state table to the right.  Notice that~$B_G$ is accepted even if~$B_C$
is pressed because the farmer is not on the same\linebreak
\end{minipage}\hspace{0.25in}%
\begin{minipage}{3.35in}
\begin{tabular}{cccccc|c}
$FXGC$& $R$& $B_F$& $B_X$& $B_G$& $B_C$& $F^+X^+G^+C^+$\\ \hline
0101& 1& x& x& x& x& 0000\\
0101& 0& 1& x& x& x& 1101\\
0101& 0& 0& x& 1& x& 1111\\
0101& 0& 0& x& 0& x& 0101\\
\end{tabular}\vspace{12pt}
\end{minipage}\mpdone

bank as the corn.
As shown later, this approach to prioritization
of inputs is equally simple in terms of implementation.  

\begin{minipage}{2.9in}
Recall that we want to stop the game when the player wins or loses.
In these states, only the reset button is accepted.  For example, the 
state~$FXGC=0110$ is a losing state because\linebreak
\end{minipage}\hspace{0.25in}%
\begin{minipage}{3.35in}
\begin{tabular}{cccccc|c}
$FXGC$& $R$& $B_F$& $B_X$& $B_G$& $B_C$& $F^+X^+G^+C^+$\\ \hline
0110& 1& x& x& x& x& 0000\\
0110& 0& x& x& x& x& 0110\\
\end{tabular}\vspace{12pt}
\end{minipage}\mpdone

the farmer
has left the fox with the goose on the opposite side of the river.
In this case, the player can reset the game, but other buttons
are ignored.

As we have already chosen a representation, we now move on to implement
the FSM.  Let's begin by calculating the next state ignoring the
reset button and the winning and losing states, as shown in the logic 
diagram below.\\

\centerline{\epsfig{file=part3/figs/farmervers1.eps,width=5.9in}}\vspace{12pt}

The left column of XOR gates determines whether the farmer is on the same
bank as the corn (top gate), goose (middle gate), and fox (bottom gate).
The output of each gate is then used to mask out the corresponding button:
only when the farmer is on the same bank are these buttons considered.
The adjusted button values are then fed into a priority encoder, which
selects the highest priority input event according to the scheme that
we outlined earlier (from highest to lowest, $B_F$, $B_C$, $B_G$, and
$B_X$, ignoring the reset button).  

The output of the priority encoder is then used to drive another column
of XOR gates on the right in order to calculate the next state.
If any of the allowed buttons is pressed, the priority encoder 
outputs~$P=1$, and the farmer's bank is changed.  If~$B_C$ is allowed
and selected by the priority encoder (only when~$B_F$ is not pressed),
both the farmer and the corn's banks are flipped.  The goose and the fox
are handled in the same way.

\begin{minipage}{1.60in}
Next, let's build a component to produce the win and lose signals.
The one winning state is~$FXGC=1111$, so we simply need an AND gate.
For the lose signal~$L$, we can fill in a \mbox{K-map} and derive an 
expression, as shown to the right, then implement as shown in the logic
diagram to the far right.  For the \mbox{K-map}, remember that the player
loses whenever the fox and the goose are on the same side of the river,
but opposite from the farmer, or whenever the goose and the corn are on
the same side of the river, but opposite from the farmer.
\end{minipage}\hspace{0.25in}%
\begin{minipage}{1.50in}
\centerline{\epsfig{file=part3/figs/farmerL.eps,width=1.05in}}
\begin{eqnarray*}
L &=& F \bar{X} \bar{G} + \bar{F} X G +\\
&& F \bar{G} \bar{C} + \bar{F} G C
\end{eqnarray*}
\end{minipage}\hspace{0.25in}%
\begin{minipage}{2.90in}
\epsfig{file=part3/figs/farmerwinlose.eps,width=2.9in}
\end{minipage}

\begin{minipage}{2.8in}
Finally, we complete our design by integrating the next-state
calculation and the win-lose calculation with a couple of muxes, as
shown to the right.
%
The lower mux controls the final value of the next state: note that
it selects between the constant state $FXGC=0000$ when the reset button~$R$
is pressed and the output of the upper mux when~$R=0$.
%
The upper mux is controlled by $W+L$, and retains the current state
whenever either signal is~1.  In other words, once the player has won
or lost, the upper mux prevents further state changes until the reset
button is pressed.
%
When~$R$, $W$, and~$L$ are all~0, the next state is calculated
according to whatever buttons have been pressed.
\end{minipage}\hspace{0.25in}%
\begin{minipage}{3.45in}
\epsfig{file=part3/figs/farmerhighlevel.eps,width=3.45in}
\end{minipage}\vspace{12pt}



\subsubsection{Analysis of a Stoplight Controller}

In this example, we begin with a digital FSM design and analyze it to
understand how it works and to verify that its behavior is appropriate.
%
The FSM that we analyze has been designed to control the stoplights
at the intersection of two roads.  For naming purposes, we assume that one
of the roads runs East and West~(EW), and the second runs North and 
South~(NS).  

The stoplight controller has two inputs, each of which 
senses vehicles approaching from either direction on one of the two roads.
The input~$V^{EW}=1$ when a vehicle approaches from either the East or the
West, and the input~$V^{NS}=1$ when a vehicle approaches from either the
North or the South.  These inputs are also active when vehicles are stopped
waiting at the corresponding lights.
%
Another three inputs, $A$, $B$, and~$C$, control the timing behavior of the
system; we do not discuss them here except as variables.

\begin{minipage}{5.15in}
The outputs of the controller consist of two~\mbox{2-bit} values, 
$L^{EW}$ and $L^{NS}$, that specify the light colors for the two roads.
In particular, $L^{EW}$ controls the lights facing East and West, and
$L^{NS}$ controls the lights facing North and South.  The meaning of these
outputs is given in the table to the right.
\end{minipage}\hspace{0.25in}%
\begin{minipage}{1.1in}
\begin{tabular}{c|c}
$L$& light color\\ \hline
0x& red\\
10& yellow\\
11& green\\
\end{tabular}
\end{minipage}\vspace{12pt}

Let's think about the basic operation of the controller.
%
For safety reasons, the controller must ensure that the lights on one
or both roads are red at all times.  
%
Similarly, if a road has a green light, the controller should 
show a yellow light before showing a red light to give drivers some
warning and allow them to slow down.
%
Finally, for fairness, the controller should alternate green lights
between the two roads.

Now take a look at the logic diagram below.
%
The state of the FSM has been split into two pieces: a \mbox{3-bit} 
register~$S$ and a \mbox{6-bit} timer.  The timer is simply a binary 
counter that counts downward and produces an output of~$Z=1$ when it 
reaches~0.  Notice that the register~$S$ only takes a new value
when the timer reaches~0, and that the~$Z$ signal from the timer
also forces a new value to be loaded into the timer in the next 
cycle.  We can thus think of transitions in the FSM on a cycle by 
cycle basis as consisting of two types.  The first type simply
counts downward for a number of cycles while holding the register~$S$
constant, while the second changes the value of~$S$ and sets the
timer in order to maintain the new value of~$S$ 
for some number of cycles.\\

\centerline{\epsfig{file=part3/figs/stoplightvers1.eps,width=5in}}\vspace{12pt}

%3.45

\begin{minipage}{2.55in}
Let's look at the next-state logic for~$S$, which feeds into the~$IN$
inputs on the \mbox{3-bit} register ($S_2^+=IN_2$ and so forth).  Notice 
that none of the inputs to the FSM directly affect these values.  The
states of~$S$ thus act like a counter.  By examining the connections,
we can derive equations for the next state and draw a transition
diagram, as shown to the right.
%
As the figure shows, there are six states in the loop defined by the 
next-state logic, with the two remaining states converging into the
loop after a single cycle.\mpline

Let's now examine the outputs for each
state in order to understand how the stoplight sequencing
works.
%
We derive equations for the outputs that control the lights, as shown
to the right, then calculate values and colors for each
state, as shown to the far right.  For completeness, the table 
includes the states outside of the desired loop.  The 
lights are all red in both of these states, which is necessary for safety.
\end{minipage}\hspace{0.25in}%
\begin{minipage}{1.05in}
\begin{eqnarray*}
\\
S_2^+ &=& \overline{S_2} + S_0\\
S_1^+ &=& \overline{S_2} \oplus S_1\\
S_0^+ &=& \overline{S_2}
\end{eqnarray*}\vspace{60pt}

\begin{eqnarray*}
L_1^{EW} &=& S_2 S_1\\
L_0^{EW} &=& S_0\\
L_1^{NS} &=& S_2 \overline{S_1}\\
L_0^{NS} &=& S_0\\
\end{eqnarray*}
\end{minipage}\hspace{0.25in}%
\begin{minipage}{2.4in}
\centerline{\epsfig{file=part3/figs/stoplightstatevers1.eps,width=1.5in}}\vspace{24pt}
\begin{tabular}{c|cc|cc}
&&& EW& NS\\
&&& light& light\\
$S$& $L^{EW}$& $L^{NS}$& color& color\\ \hline
000& 00& 00&    red&    red\\
111& 11& 01&  green&    red\\
110& 10& 00& yellow&    red\\
010& 00& 00&    red&    red\\
101& 01& 11&    red&  green\\
100& 00& 10&    red& yellow\\ \hline
001& 01& 01&    red&    red\\
011& 01& 01&    red&    red\\
\end{tabular}\vspace{12pt}
\end{minipage}


Now let's think about how the timer works.  As we already noted, the
timer value is set whenever~$S$ enters a new state, but it can also be
set under other conditions---in particular, by the signal~$F$ calculated
at the bottom of the FSM logic diagram.  

\begin{minipage}{3.4in}
For now, assume that $F=0$.  In this case, the timer is set only when
the state~$S$ changes, and we can find the duration of each state by
analyzing the muxes.  The bottom mux selects~$A$ when~$S_2=0$, and 
selects the output of the top mux when~$S_2=1$.  The top mux selects~$B$
when~$S_0=1$, and selects~$C$ when~$S_0=0$.  Combining these results,
we can calculate the duration of the next states of~$S$ when~$F=0$, 
as shown in the table to the right.  We can then combine the next
state duration with our previous calculation of the state sequencing 
(also the order in the table) to obtain the durations of each state, also
shown in the rightmost column of the table.
\end{minipage}\hspace{0.25in}%
\begin{minipage}{2.85in}
\begin{tabular}{c|cc|cc}
& EW& NS& next& current\\
& light& light& state& state\\
$S$& color& color& duration& duration\\ \hline
000&    red&    red& $A$& $C$\\
111&  green&    red& $B$& $A$\\
110& yellow&    red& $C$& $B$\\
010&    red&    red& $A$& $C$\\
101&    red&  green& $B$& $A$\\
100&    red& yellow& $C$& $B$\\ \hline
001&    red&    red& $A$& ---\\
011&    red&    red& $A$& ---\\
\end{tabular}\vspace{12pt}
\end{minipage}

What does~$F$ do?  Analyzing the gates that produce it gives 
$F=S_1S_0\overline{V^{NS}}+\overline{S_1}S_0\overline{V^{EW}}$.  If we 
ignore the two states outside of the main loop for~$S$, the first term 
is~1 only when the lights are green on the East and West roads and the 
detector for the North and South roads indicates that no vehicles are 
approaching.  Similarly, the second term is~1 only when the lights are 
green on the North and South roads and the detector for the East and 
West roads indicates that no vehicles are approaching.

What happens when~$F=1$?  First, the OR gate feeding into the timer's
$LD$ input produces a~1, meaning that the timer loads a new value
instead of counting down.  Second, the OR gate controlling the lower
mux selects the~$A$ input.  In other words, the timer is reset to~$A$
cycles, corresponding to the initial value for the green light states.
In other words, the light stays green until vehicles approach on 
the other road, plus~$A$ more cycles.

Unfortunately, the signal~$F$ may also be~1 in the unused states of~$S$,
in which case the lights on both roads may remain red even though cars
are waiting on one of the roads.  To avoid this behavior, we must be 
sure to initialize the state~$S$ to one of the six states in the
desired loop.

\pagebreak

